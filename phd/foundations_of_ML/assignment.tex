\documentclass[a4paper]{article}
\usepackage{fullpage}

\usepackage[utf8]{inputenc}

\usepackage{graphicx, url}

\usepackage{amsmath, amsfonts, amssymb, amsthm}
\usepackage{mathptmx}

\newcommand{\Real}{\mathbb{R}}
\newcommand{\Cplx}{\mathbb{C}}
\newcommand{\ex}{\mathbb{E}}
\newcommand{\pr}{\mathbb{P}}
\newcommand{\argmin}{\mathop{\text{argmin}}}
\newcommand{\argmax}{\mathop{\text{argmax}}}
\newcommand{\tr}{\text{tr}}
\newcommand{\one}{\mathbf{1}}
\newcommand{\nil}{\mathbf{0}}
\newcommand{\Rcal}{{\mathcal{R}}}
\newcommand{\Lcal}{\mathcal{L}}
\newcommand{\Hcal}{\mathcal{H}}
\newcommand{\Fcal}{\mathcal{F}}

\title{Foundations Of Machine Learning: assignment}
\author{Nazarov Ivan}
\begin{document}

\section{Assignment \#1} % (fold)
\label{sec:assignment_1}

\subsection{Problem A} % (fold)
\label{sub:problem_a}

\subsubsection{Task 1} % (fold)
\label{ssub:task_1}

Let $f:(0, +\infty)\mapsto\Real$ be a function with an inverse $f^{-1}$, and let
$X$ be an rv, and suppose $\pr(X > t)\leq f(t)$ for all $t>0$. Then for any $\delta>0$,
one has
\[ \pr\bigl(X>f^{-1}(\delta)\bigr) \leq \delta \,, \]
since $f$ is invertible. Therefore for al $\delta>0$
\[ 1-\delta \leq \pr\bigl(X \leq f^{-1}(\delta)\bigr) \,. \]

% subsubsection task_1 (end)

\subsubsection{Task 2} % (fold)
\label{ssub:task_2}

Let $X$ be a non-negative rv taking integer values. Consider it expectation:
\[ \ex X = \sum_{n\geq 0} n \pr(X=n) \,. \]
However $\{X=n\} = \{X\geq n\}\setminus \{X\geq n+1\}$, whence
\[ \ex X = \sum_{n\geq 0} n \bigl(\pr(X\geq n) - \pr(X\geq n+1)\bigr) \,. \]
Now since the terms in the summation are non-negative it is possible to reorder
so that
\begin{align*}
	\ex X
    &= \sum_{n\geq 0} n \bigl(\pr(X\geq n) - \pr(X\geq n+1)\bigr) \\
    &= \sum_{n\geq 0} n \pr(X\geq n)
      - \sum_{n\geq 0} n \pr(X\geq n+1) \\
    &= \sum_{n\geq 0} n \pr(X\geq n)
      - \sum_{n\geq 0} (n+1) \pr(X\geq n+1)
      + \sum_{n\geq 0} \pr(X\geq n+1) \\
    &= \pr(X\geq 0) + \sum_{n\geq 1} \pr(X\geq n) \,,
\end{align*}
whence the result follows.

% subsubsection task_2 (end)

% subsection problem_a (end)

\subsection{Problem B} % (fold)
\label{sub:problem_b}

Let $D$ be a distribution over $X$, and $F:X\mapsto\{-1,+1\}$ be the true deterministic
labelling function. WE a re interested in finding a good approximation of the label
bias of $(D, f)$, i.e. $P_+$ defined by
\[ p_+ = \pr_{x\sim D}\bigl(f(x) = +1)\bigr) \,. \]

Let $S=(x_i)_{i=1}^m$ be an iid sample drawn from $D$. Since the random variable
$f(X)$ for $X\sim D$ is binary, it can have only one kind on distribution -- Bernoulli.
Thus one can use an ML estimator: since $(f(x_i))_{i=1}^m$ is an iid sample as well,
the log-likelihood of labels of the sample $S$ is
\[ \mathcal{L} = \sum_{i=1}^m 1_{\{f(x_i)=+1\}} \log p_+ + 1_{\{f(x_i)=-1\}} \log(1-p_+) \,, \]
whence the MLE of $p_+$ is given by: $\hat{p}_+(S) = m^{-1} \sum_{i=1}^m 1_{\{f(x_i)=+1\}}$.
This is an unbiased estimator of $p_+$:
\[ \ex_{S\sim D^m} \hat{p}_+(S)
  = m^{-1} \sum_{i=1}^m \ex_{x_i\sim D} 1_{\{f(x_i)=+1\}} 
  = \ex_{x\sim D} 1_{f(x)=+1} = p_+ \,. \]
Now $1_{\{f(x_i)=+1\}}$ are iid $0-1$ random variables with expectation $p_+$, which
means that by the two-sided Hoeffding inequality for any $\epsilon>0$ one has:
\[ \pr_{S\sim D^m}\bigl(
    \bigl| \sum_{i=1}^m 1_{\{f(x_i)=+1\}} - m p_+ \bigr| > m\epsilon
  \bigr) \leq 2 \text{exp}\biggl\{\frac{-2m^2\epsilon^2}{m}\biggr\}
  \,, \]
whence $\pr_{S\sim D^m}( |\hat{p}_+(S) - p_+| > \epsilon ) \leq 2 e^{-2m\epsilon^2}$.
Therefore for any $\delta>0$ it is true that
\[ |\hat{p}_+(S) - p_+| \leq \sqrt{\frac{\log\frac{2}{\delta}}{2m}} \,, \]
with probability at least $1-\delta$ over $S\sim D^m$.
Furthermore, since the upper bound on the probability is exponential in $m$,
the first Borell-Cantelli lemma implies that for all $\epsilon>0$
\[ \pr_{S\sim D_\infty} \bigl(
  |\hat{p}_+(S_{:m}) - p_+| > \epsilon
    \text{ i.o. over } m \bigr) = 0 \,, \]
where $D_\infty$ is a distribution over infinite iid sequences in $X$. Therefore
one has $\hat{p}_+(S_{:m}) \to p_+$ almost surely.

% subsection problem_b (end)

\subsection{Problem D} % (fold)
\label{sub:problem_d}
Recall that the empirical loss $\hat{R}_S(h)$ of $h$ is given by
\[ \hat{R}_S(h) = \ex_{(x,y)\sim S} L(h(x), y) \,, \]
where $(x,y)\sim S$ means empirical distribution over the sample data $S=(z_i)_{i=1}^m$.
The true loss of $h$ is $R(h) = \ex_{(x,y)\sim D} L(h(x), y)$. Consider the $0-1$
loss $L(h(x), y) = 1_{\{h(x)\neq y\}}$.

Consider nested hypotheses sets $(H_k)_{k\geq1}$ with $H_k\subset H_{k+1}$ of hypotheses
taking values in $\{-1,+1\}$, and put $H=\cup_{k\geq1} H_k$. Furthermore for any
$h\in H$ let $k: H\mapsto\mathbb{N}$ be defined by $k(h) = \inf\{k\geq1\,:\, h\in H_k\}$
-- the index of the first hypothesis set $H_k$ that houses $h$. For any $k\geq1$
define the objective function $F_k$ by
\[ F_k(h) = \hat{R}_S(h) + \mathcal{R}_m(H_k) + \sqrt{\frac{\log k}{m}} \,. \]

\subsubsection*{step 1} % (fold)
\label{ssub:step_1}

First, by definition of the Rademacher complexity the inclusion $H_k\subset H_{k+1}$
implies that
\[ \mathcal{R}_m(H_k) \leq \mathcal{R}_m(H_{k+1}) \,, \]
which in turn implies that for any $h\in H_k$ it is true that $k\geq k(h)$ one has
\[ \mathcal{R}_m(H_k) + \sqrt{\frac{\log k}{m}}
  \geq \mathcal{R}_m(H_{k(h)}) + \sqrt{\frac{\log {k(h)}}{m}} \,. \]
Therefore $F_k(h) \geq F_{k(h)}(h)$, and for all $k\geq1$ and $h\in H_k$ it is true
that
\[ R(h) - F_k(h) \leq R(h) - F_{k(h)}(h) \leq \sup_{h\in H} R(h) - F_{k(h)}(h) \,. \]
Conversely, for any $f\in H$
\[ \sup_{k\geq1} \sup_{h\in H_k} R(h) - F_k(h)
    \geq \sup_{h\in H_{k(f)}} R(h) - F_{k(f)}(h)
    \geq R(f) - F_{k(f)}(f)
  \,, \]
which ultimately implies that
\[ \sup_{k\geq1}\sup_{h\in H_k} R(h) - F_k(h)
    = \sup_{h\in H} R(h) - F_{k(h)}(h)
  \,. \]

% subsubsection* step_1 (end)

\subsubsection*{step 2} % (fold)
\label{ssub:step_2}

Now, the main theorem on the one-sided Rademacher complexity bound (slide~15) states,
that for a hypothesis class $H_k$ one has for all $\epsilon>0$
\[ \pr_{S\sim D^m}\Bigl( \{S\,:\,
    \sup_{h\in H_k} R(h) - \hat{R}_S(h) > \mathcal{R}_m(H_k) + \epsilon\}
  \Bigr) \leq \text{exp}\bigl\{-2m\epsilon^2\bigr\}
  \,. \]
If $\epsilon_k = \epsilon + \sqrt{\frac{\log k}{m}}$, then for all $k\geq 1$ the main
theorem implies that
\[ \pr_{S\sim D^m}\Bigl( \{S\,:\,
    \sup_{h\in H_k} R(h) - \hat{R}_S(h) - \mathcal{R}_m(H_k) - \sqrt{\frac{\log k}{m}} > \epsilon\}
  \Bigr) \leq \text{exp}\bigl\{-2m\epsilon_k^2\bigr\}
  \,. \]
However for any $k\geq 1$ it is true that $\epsilon_k^2 \geq \epsilon^2 + \frac{\log k}{m}$,
whence
\[ \text{exp}\bigl\{-2m\epsilon_k^2\bigr\}
  \leq k^{-2} \text{exp}\bigl\{-2m\epsilon^2\bigr\}
  \,,\]
By the Union Bound one has
\[ \pr_{S\sim D^m}\Bigl(
  \bigcup_{k\geq 1} \{S\,:\, \sup_{h\in H_k} R(h) - F_k(h) > \epsilon\}
  \Bigr) \leq e^{-2m\epsilon^2} \sum_{k\geq 1} k^{-2}
  = \frac{\pi^2}{6} e^{-2m\epsilon^2}
  \,. \]
Now, step 1 and the definition of the least upper bound imply that
\begin{align*}
  \bigcup_{k\geq 1} \{S\,:\, \sup_{h\in H_k} R(h) - F_k(h) > \epsilon\}
    &= \{S\,:\, \sup_{k\geq 1} \sup_{h\in H_k} R(h) - F_k(h) > \epsilon\}\\
    &= \{S\,:\, \sup_{h\in H} R(h) - F_{k(h)}(h) > \epsilon\} \,,
\end{align*}
whence for any $\epsilon>0$
\[ \pr_{S\sim D^m}\Bigl(
  \sup_{h\in H} R(h) - F_{k(h)}(h) > \epsilon
  \Bigr) \leq 2 e^{-2m\epsilon^2}
  \,. \]

% subsubsection* step_2 (end)

\subsubsection*{step 3} % (fold)
\label{ssub:step_3}

The SRM-optimal hypothesis $f^*$ is given by the minimizer of the objective:
\[ f^* = \argmin_{h\in H_k, k\geq1} F_k(h) \,, \]
and the best (lowest risk) hypothesis $h^*$ is given by
\[ h^* = \argmin_{h\in H} R(h) \,. \]
The hypothesis $f^*$ is observable and feasible, whereas the $h^*$ is unknown.

\noindent Fix some $\delta>0$. By the uniform concentration result from step 2 one
has
\[ \sup_{h\in H} R(h) - F_{k(h)}(h) \leq \sqrt{\frac{\log\frac{4}{\delta}}{2m}}
  \,,\]
with probability at least $1-\frac{\delta}{2}$ over $S\sim D^m$. However the SRM
$f^*$ hypothesis is such that $F_{k(f^*)}(f^*) \leq F_{k(h)}(h)$ for all $h\in H$,
since by step 1
\[\argmin_{h\in H_k, k\geq1} F_k(h) = \argmin_{h\in H} F_{k(h)}(h) \,. \]
Therefore with the same probability one has
\[ R(f^*) \leq \hat{R}_S(h^*) + \mathcal{R}_m(H_{k(h^*)})
        + \sqrt{\frac{\log k(h^*)}{m}} + \sqrt{\frac{\log\frac{4}{\delta}}{2m}}
  \,. \]

\noindent Now, the one-sided Rademacher complexity bound for $H_{k(h^*)}$ implies
that
\[ \hat{R}_S(h^*) \leq R(h^*) + \mathcal{R}_m(H_{k(h^*)})
  + \sqrt{\frac{\log\frac{2}{\delta}}{2m}} \,, \]
with probability at least $1-\frac{\delta}{2}$ over $S\sim D^m$.
Applying the union bound yields this result: for all $\delta>0$
\[ R(f^*) \leq R(h^*) + 2\mathcal{R}_m(H_{k(h^*)})
  + \sqrt{\frac{\log k(h^*)}{m}}
  + \sqrt{\frac{\log\frac{4}{\delta}}{2m}}
  + \sqrt{\frac{\log\frac{2}{\delta}}{2m}} \,, \]
with probability at least $1-\delta$ over $S\sim D^m$.

% subsubsection* step_3 (end)

\subsubsection*{step 4} % (fold)
\label{ssub:step_4}

Since $x\mapsto\sqrt{x}$ and $x\mapsto \log x$ are concave one has
\begin{align*}
  \sqrt{\frac{\log\frac{4}{\delta}}{2m}} + \sqrt{\frac{\log\frac{2}{\delta}}{2m}}
  &= 2 \Biggl( \frac{1}{2}\sqrt{\frac{\log\frac{4}{\delta}}{2m}}
      + \frac{1}{2} \sqrt{\frac{\log\frac{2}{\delta}}{2m}} \Biggr)
  \leq 2 \sqrt{\frac{\frac{1}{2} \log\frac{4}{\delta}+\frac{1}{2} \log\frac{2}{\delta}}{2m}} \\
  &\leq 2 \sqrt{\frac{\log\bigl(\frac{1}{2}\frac{4}{\delta} + \frac{1}{2} \frac{2}{\delta}\bigr)}{2m}}
  = 2 \sqrt{\frac{\log\frac{3}{\delta}}{2m}}
  = \sqrt{\frac{2\log\frac{3}{\delta}}{m}} \,.
\end{align*}
Therefore we end up with the following result: for any $\delta>0$
\[ R(f^*) \leq R(h^*) + 2\mathcal{R}_m(H_{k(h^*)})
  + \sqrt{\frac{\log k(h^*)}{m}}
  + \sqrt{\frac{2\log\frac{3}{\delta}}{m}}
  \,, \]
with probability at least $1-\delta$ over $S\sim D^m$.

% subsubsection* step_4 (end)

\subsubsection{step 5} % (fold)
\label{ssub:step_5}

Suppose we are told which hypothesis set $H_n$ the best hypothesis $h^*$ belongs
to. Then, obviously, the SRM would degenerate into ERM for the objective $F_n(h)$.

% subsubsection step_5 (end)

% subsection problem_d (end)

% section assignment_1 (end)


\end{document}
