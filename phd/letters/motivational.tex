\documentclass[14pt]{letter}
\usepackage{graphicx}% http://ctan.org/pkg/graphicx
\setlength{\parindent}{0pt}% Remove paragraph indent

\begin{document}

\vspace{1.5cm}

\begin{minipage}{0.5\linewidth}
To PhD admission committee, \par
CDISE CREI, Skoltech \par
Moscow, Russia \par
\end{minipage} \par\bigskip

Dear Committee members, \par\bigskip

I would like to begin this letter by briefly writing about how I came into the field of Data Science, then move on to the outlines of my academic work, and finally conclude with my objectives as a Ph.D. student at CDISE CREI at Skoltech. \par\medskip

During the last year of my undergraduate studies at the Department of Economics at HSE (2003-2007) I specialized in Methods of Mathematical Analysis of Economics, which included deeper studies of Econometrics, Time Series analysis, convex optimization and mathematical modelling in economics. I chose this specialization because I was very interested in statistics and mathematics. While studying for an Economist, I became interested in numerical algorithms and continued developing my passion for programming and computer science, which I have been interested in since high school. \par\medskip

Two years after my graduation I started working at the Institute for Financial Studies, where I worked for 3.5 years. There I actively participated in economic and financial research projects that involved statistical analysis, numeric algorithms and required programming. Despite these skills not being particularly useful for an economist, I still continued to hone applied mathematical and programming skills and apply them to different tasks that came up occasionally while working at the IFS. \par\medskip

In second half of 2012 I moved on to a more IT-centric job. There I was worked on a statistical package for large-scale market simulation for a financial regulator. I was the lead developer of the statistical modelling and simulation module of the software that the team was developing. After the software was successfully developed, I switched to a different project, which was more oriented at data mining, migration and warehousing. It also required familiarity in finance and extensive knowledge in SQL and database structure. While working on this it I got the taste of what Big Data really is from a processing perspective: I was prototyping and optimizing large data flows and reports over a database of accounting transactions of a major bank. \par\medskip

Throughout my career, I was feeling that simple data manipulation and mundane business analytics was not my cup of tea. I was always interested in statistics, computer science and mathematics, which is why in early 2013 I decided to start preparing for enrolment into the master's programme in Data Science at HSE. It was my understanding, that Data Science, and especially Machine Learning, was at the nexus of these fields, a fusion of disciplines that I was very excited to be a part of. \par\medskip

The first year of master's programme was very revealing for me: I learnt new exciting machine learning and numerical algorithms, and theoretical foundations thereof, I further improved my programming skills and deepened knowledge in complex algorithms. My course project was about the application of a so-called ``crossing tree'' to study of self-similarity of in stochastic processes with stationary increments. This tree captures the behaviour of a process at different resolutions by keeping track of the first hitting times of consecutively coarser lattices. The result if the project was a simulation study of a conjecture concerning certain types of self-similar processes. Geoffrey Decrouez, Ph.D., who supervised my project, and I are planning to submit a conference paper in 2016-2017, outlining the results. \par\medskip

During the final year at the master's programme, feeling the need for more practical experience with ML and DM, I applied for an intern position at Laboratory 10, lead by Evgeniy Burnaev, Ph.D., at IITP. There I obtained invaluable experience in applications of ML to engineering and got more in-depth understanding of software development cycle. I also participated in algorithmically and mathematically intensive projects concerning online statistical learning in finance, and long-term forecasting. \par\medskip

Evgeniy became the scientific supervisor for my master thesis, which I successfully defended in 2016. My thesis, titled ``conformal methods in multidimensional linear models and anomaly detection'', was concerned with developing and testing a distribution-free algorithm that could yield confidence sets for predictions by kernel ridge regression. The approach I used, called ``conformalization'', was suggested by Volodimir Vovk in 2005, basically boils down to constructing a hypothesis test for a point-prediction against the empirical measure of some non-conformity score and then inverting the test to get a set of most likely target values. A conference paper describing the results was submitted to the upcoming 2016 ICMLA conference. \par\medskip

The Ph. D. programme in Applied Mathematics at CDISE CREI at Skoltech, would allow me to further deepen my knowledge in Statistics, Statistical Learning theory and other theoretical foundations of Machine Learning. I am eager to work under supervision of Maxim Fedorov, D.Sc, and Evgeniy Burnaev, Ph. D., who are outstanding researchers, leaders and specialists. I am convinced that their supervision will provide constant influx of challenging real-world industrial problems, a vast set of opportunities for collaboration with the leading researchers in the field, and enable me to gain new and much better understanding of theoretical foundations of ML, mathematics and computer science. \par\medskip

I believe that my passion for statistics, mathematics, and algorithms, coupled with my experience in solving applied problems make me a worthy candidate for this programme. Ultimately, with the help of facilities and expertise offered at Skoltech I would like to become a top-grade specialist in ML, capable of developing brand new complex learning algorithms, or enhancing already existing ones, providing rigorous theoretical basis thereof. At the same time I want to have more extensive experience in solving diverse applied ML problems, and be able to quickly analyse the resource requirements and outline a scope of jobs, tools, both theoretical and practical, necessary for the problem solution. \par\medskip

Sincerely,
Nazarov Ivan, \par
Junior Researcher, \par
Laboratory 10, \par
the Institute for Information Transmission Problems, \par
Russian Academy of Sciences, Moscow \par

e-mail: ivan.nazarov@iitp.ru, ivannnnz@gmail.com \par
cell phone: +7 915 3172225
\hfill\today
\hfill\par

\end{document}
