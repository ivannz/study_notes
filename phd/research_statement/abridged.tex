\documentclass[a4paper]{extarticle}
\usepackage[T2A]{fontenc}

\usepackage[utf8]{inputenc}
\usepackage{graphicx}

\usepackage{amsmath, amsfonts, amssymb, amsthm}

\usepackage[english, russian]{babel}

\newcommand{\eng}[1]{\foreignlanguage{english}{#1}}
\newcommand{\rus}[1]{\foreignlanguage{russian}{#1}}

\title{Abridged research statement}
\author{Nazarov I.N.}

\begin{document}
\maketitle

\noindent \textbf{\eng{Eng}:}\selectlanguage{english}
Many data-intense applied fields such as maintenance of complex systems, structural
integrity monitoring, diagnostic medicine, intrusion detection, and anti-fraud solutions
in banking and many others, require that it be possible to decide if the new data
diverges in some sense from the previously seen observations. Thus, I plan to focus
my PhD research primarily on developing a general framework for anomaly detection
which would provide a rigorous definition of an anomaly, a notion of detection precision
and a method for determining the amount of observations needed for the required precision.
My secondary goal is to design efficient procedures for distribution-free confidence
measures for various ML algorithms, and provide necessary theoretical basis for
their coverage guarantees and efficiency. The recent results of my numerical study
of conformal confidence intervals for kernel ridge regression have been accepted
to ICMLA 2016.\par \bigskip

\noindent \textbf{\eng{Rus}:}\selectlanguage{russian}
Во множестве прикладных задач, таких как обслуживание многокомпонентных систем,
мониторинг структурной целостности зданий, диагностическая медицина, обнаружение
вторжений, анти-фрод решения для банковской сферы и прочее, тем или иным образом
возникает проблема обнаружения аномальных отклонений новых от ранее наблюдавшихся
или типичных данных. В рамках исследовательской работы в аспирантуре я планирую,
во-первых, разработать общий подход к обнаружению аномалий, предложить методы оценки
надёжности детекторов аномалий и исследовать методы, позволяющие определить минимально
необходимый объём наблюдений для достижения указанного уровня надёжности.
Во-вторых, я намереваюсь спроектировать и теоретически обосновать вычислительно
эффективные процедуры для вычисления мер предсказательной надёжности разнообразных
классических алгоритмов машинного обучения. Результаты проведённого симуляционного
исследования конформных процедур для ядерной гребневой регрессии будут представлены
на конференции \eng{ICMLA} 2016.
\end{document}
