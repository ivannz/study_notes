\documentclass{beamer}
\usepackage[utf8x]{inputenc}
\usepackage[english, russian]{babel}
\newcommand{\eng}[1]{\foreignlanguage{english}{#1}}
\newcommand{\rus}[1]{\foreignlanguage{russian}{#1}}
\selectlanguage{russian}
\mode<presentation>
{
  \usetheme{default}      % or try Darmstadt, Madrid, Warsaw, ...
  \usecolortheme{default} % or try albatross, beaver, crane, ...
  \usefonttheme{default}  % or try serif, structurebold, ...
  \setbeamertemplate{navigation symbols}{}
  \setbeamertemplate{caption}[numbered]
} 

%\usepackage{fullpage}

\usepackage{graphicx, url}

\usepackage{amsmath, amsfonts, xfrac}
\usepackage{mathtools}

\usepackage{lipsum}

\newcommand{\obj}[1]{\left\{ #1 \right \}}
\newcommand{\clo}[1]{\left [ #1 \right ]}
\newcommand{\clop}[1]{\left [ #1 \right )}
\newcommand{\ploc}[1]{\left ( #1 \right ]}

\newcommand{\brac}[1]{\left ( #1 \right )}
\newcommand{\crab}[1]{\left ] #1 \right [}
\newcommand{\induc}[1]{\left . #1 \right \vert}
\newcommand{\abs}[1]{\left | #1 \right |}
\newcommand{\nrm}[1]{\left\| #1 \right \|}
\newcommand{\brkt}[1]{\left\langle #1 \right\rangle}

\newcommand{\floor}[1]{\left\lfloor #1 \right\rfloor}

\newcommand{\defn}{\mathop{\overset{\Delta}{=}}\nolimits}

\begin{document}
\selectlanguage{russian}
\title[Предпочтение при прочих равных в АФП]{Моделирование отношения предпочтения при прочих равных в Анализе Формальных Понятий\\[0.5cm] Modeling Ceteris Paribus Preferences in Formal Concept Analysis}
\author{\eng{Sergei Obiedkov}}
\institute{\eng{National Research University Higher School of Economics, Moscow, Russia}}
\date{\eng{11th International Conference, ICFCA 2013, May 2013}}

%% Title slide
\frame{\titlepage}

%% Table of Contents
\begin{frame}
  \frametitle{\rus{Содержание}}
  \tableofcontents
\end{frame}

\selectlanguage{russian}
\section{Введение} % (fold)
\label{sec:introduction}

\begin{frame}\selectlanguage{russian}\frametitle{Определения}
  %% Описание основного оъекта исследования в статье
  %%%% Формальный контекст
  \begin{block}{Формальный контекст}
    $\mathbb{K} \defn \brac{G, M, I}$, где $G$ -- объекты (сущности), $M$ -- атрибуты (свойства), $I \subseteq G\times M$ -- бинарное отношение принадлежности свойства сущности.
  \end{block}
  %%%% Операторы вывода
  \begin{block}{Операторы вывода в $\mathbb{K}$}
  $A\subseteq G$, $B\subseteq M$
    \begin{align*}
      A^\uparrow &\defn \obj{\induc{m\in M}\,\forall a\; a\in A \rightarrow (a,m)\in I }\\
      B^\downarrow &\defn \obj{\induc{g\in G}\,\forall b\; b\in B \rightarrow (g,b)\in I }
    \end{align*}
  \end{block}
\end{frame}

\begin{frame}\selectlanguage{russian}\frametitle{Определения}
  %%%% Контекст предпочтений
  \begin{block}{Контекст предпочтений}
    $\mathbb{P}\defn \brac{G, M, I, \leq}$, где $\brac{G, M, I}$ -- формальный контекст, $\leq$ -- отношение предпочтения на множестве объектов $G$
  \end{block}
  %%%% Отношение предпочтения на G
  \begin{block}{Отношение предпочтения}
    $\brac{G, \leq}$ -- предпорядок на множестве объектов, $\leq$ -- рефлексивно и транзитивное отношение.
  \end{block}
  %%%% Обычно предпочтения это линейный предпорядок
\end{frame}

\begin{frame}\selectlanguage{russian}\frametitle{Предпочтения наборов объектов}
  %%%% Достаточно предпорядка $\brac{G, \leq}$
  \begin{block}{Универсальное предпочтение}
    Объекты $B\subseteq G$ универсально предпочитаются объектам $A\subseteq G$, $A\trianglelefteq_\forall B$, если\[ \forall a\in A\, \forall b\in B\, a\leq b\]
  \end{block}
  \begin{block}{Экзистенциальное предпочтение}
    Объекты $B\subseteq G$ экзистенциально предпочитаются объектам $A\subseteq G$, $A\trianglelefteq_\exists B$, если\[ \forall a\in A\, \exists b\in B\, a\leq b\]
  \end{block}
%  \begin{block}
%    может не быть рефлексивности: $A \not\trianglelefteq_\forall A$, если $A = \obj{x,y}$ и $x\not \leq y$
%    для любого $A,B\subseteq M$ выполнено $A\trianglelefteq_\forall \emptyset$ и $\emptyset\trianglelefteq_\forall B$
%    не антирефлексивно: $\obj{x} \trianglelefteq_\forall \obj{x}$
%  \end{block}
\end{frame}

\begin{frame}\selectlanguage{russian}\frametitle{Предпочтения наборов свойств}
  %%%% К предпорядку $\brac{G, \leq}$ необходим контекст \brac{G, M, I}
  \begin{block}{Определение}

    Свойства $B\subseteq M$ универсально предпочитаются набору $A\subseteq M$, $A\preceq_\forall B$, если $A^\downarrow \trianglelefteq_\forall B^\downarrow$.

    Экзистенциальное предпочтение на наборах свойств определяются схожим образом: $A\preceq_\exists B$ $\Leftrightarrow$ $A^\downarrow \trianglelefteq_\exists B^\downarrow$
  \end{block}

  \begin{itemize}
    \item Наборы атрибутов сравниваются по \emph{всем} объекты, которые ими обладают
  \end{itemize}

  \eng{Obiedkov S., (2012): Modeling Preferences over Attribute Sets in Formal Concept Analysis, ICFCA 2012, LNAI 7278, pp. 227-243, 2012}
\end{frame}

% section introduction (end)

\section{Предпочтение при прочих равных} % (fold)
\label{sec:ceteris_paribus}

\begin{frame}\selectlanguage{russian}\frametitle{Предпочтение при некоторых прочих равных}
%%%% Параметризованное предпочтение
  Пусть $\mathbb{P}\defn \brac{G, M, I, \leq}$ -- контекст предпочтений,  $A,B,C\subseteq M$.
  \begin{block}{Определение (2, стр. 190)}
    Набор свойств $B$ предпочитается набору $A$ относительно атрибутов $C$ в контексте $\mathbb{P}$, $\mathbb{P} \models A\preceq_C B$, если для любых $a\in A^\downarrow$ и $b\in B^\downarrow$ выполнена импликация \[ a^\uparrow \cap C = b^\uparrow \cap C \rightarrow a \leq b\]
  \end{block}
  \begin{itemize}
    \item Сравнение объектов, обладающих всеми свойствами, но имеющих ``схожие'' атрибуты
    \item $A\preceq_\forall B$ эквивалентно $A\preceq_\emptyset B$
  \end{itemize}
\end{frame}

\begin{frame}\selectlanguage{russian}\frametitle{Предпочтение \eng{Ceteris Paribus}}
  \begin{block}{Определение (7, стр. 196)}
    Набор атрибутов $B$ предпочитается набору $A$ при прочих равных в контексте $\mathbb{P}$, $A\preceq B$, если $B$ предпочитается $A$ относительно атрибутов $M\setminus (A\cup B)$.
  \end{block}
  \begin{block}{Теорема 2 (стр. 197)}
    В контексте $\mathbb{P}$ выполнено $A\preceq_C B$ тогда и только тогда, когда $\mathbb{P}\models A\cup D\preceq B\cup E$ для любых $D, E\subseteq M$ таких, что $D\cup C = B\cup C$ и $E\cup C = A\cup C$.
  \end{block}
%%%% Смысл этих вот объединений
  \begin{itemize}
    \item Достаточно рассматривать $D,E\subseteq M$ такие что $D\cup E\subseteq C$ (Следствие 1 (стр. 198))
  \end{itemize}
\end{frame}

% section ceteris_paribus (end)

\section{Выявление предпочтений} % (fold)
\label{sec:preference_mining}

\begin{frame}\selectlanguage{russian}\frametitle{Связь предпочтений при прочих равных и импликаций}
  \begin{itemize}
    \item Выявление таких предпочтений сводится к задаче поиска импликаций
  \end{itemize}
  \begin{block}{Определение (3, стр. 191)}
    Преобразованием $\mathbb{P}$ называется контекст $\mathbb{K}^\mathbb{P}_\sim = \brac{ G\times G, M\times\obj{1,2,3}\cup \obj{\leq}, I_\sim}$ такой, что \begin{align*}
      (g_1, g_2) I_\sim (m, 1) &\Leftrightarrow m\in g_1^\uparrow\\
      (g_1, g_2) I_\sim (m, 2) &\Leftrightarrow m\in g_2^\uparrow\\
      (g_1, g_2) I_\sim (m, 3) &\Leftrightarrow g_1^\uparrow \cap \obj{m} = g_2^\uparrow \cap \obj{m}\\
      (g_1, g_2) I_\sim \leq &\Leftrightarrow g_1 \leq g_2
    \end{align*}
  \end{block}
  %%%% Рассказать немного о смысле происходящего
  %%%% связь с моделированием функциональных связей
\end{frame}

\begin{frame}\selectlanguage{russian}
  %%%% Оговорить slight abuse of notation
  \begin{block}{Определение (4, стр. 193)}
    Переводом отношения предпочтения $A\preceq_C B$ из контекста $\mathbb{P}$ в контекст $\mathbb{K}^\mathbb{P}_\sim$, $T_\sim\brac{A\preceq_C B}$, называется импликация \[A\times \obj{1} \cup B\times \obj{2} \cup C\times \obj{3} \rightarrow \obj{\leq}\] в контексте $\mathbb{K}^\mathbb{P}_\sim$
  \end{block}
  %%%% Доказательство
  \begin{block}{Утверждение (1, стр. 193)}
    В контексте $\mathbb{P}$ справедливо предпочтение $A\preceq_C B$ тогда и только тогда, когда в контексте $\mathbb{K}^\mathbb{P}_\sim$ выполнена импликация $T_\sim\brac{A\preceq_C B}$.
  \end{block}
\end{frame}

% section preference_mining (end)

\section{Сложность выводимости предочтений } % (fold)
\label{sec:complexity_issues}

\begin{frame}\selectlanguage{russian}\frametitle{Выводимость предпочтений при прочих равных}
  \begin{block}{Определение (5, стр. 193)}
    Предпочтение $A\preceq_C B$ есть следствие набора предпочтений $\Pi$ если для любого контекста $\mathbb{P}$, для которого выполнено $\mathbb{P}\models \Pi$, справедливо $\mathbb{P}\models A\preceq_C B$.
  \end{block}
  \begin{block}{Задача 1 (стр. 194)}
    Для заданного набора $\Pi$ предпочтений при прочих равных и предпочтения $A\preceq_C B$ над $M$, определить верно ли $\Pi \models A\preceq_C B$
  \end{block}
  \begin{block}{Утверждение 2 (стр. 194)}
    Задача определения $\Pi \models \emptyset\preceq_\emptyset \emptyset$ принадлежит классу coNP сложных задач
  \end{block}
%% не может быть решён за полиномиальное время если P\neq NP
\end{frame}


\begin{frame}\selectlanguage{russian}
  Утверждение 2 доказывается редукцией задачи к следующей coNP-полной проблеме
  \begin{block}{Задача 3 (стр. 195)}
    Для некоторой пропозиционной формулы $\phi$ в конъюнктивной нормальной форме решить, является ли он невыполнимой при любом значении входных переменных.
  \end{block}
  \begin{block}{Редкуция}
    Пусть $M$ -- множество всех переменных $\phi$. В множество $\Pi_\emptyset$ входят следующие предпочтения:
    \begin{itemize}
      \item $\emptyset \preceq_{\obj{m}} \emptyset$ для каждого $m\in M$
      \item $P \preceq_\emptyset N$ для каждого дизъюнкта в $\phi$, где $P$ -- переменные входящие в дизъюнкт без отрицания, $N$ -- с отрицанием.
    \end{itemize}
    Формула $\phi$ невыполнима тогда и только тогда, когда $\Pi_\emptyset\models \emptyset \preceq_\emptyset \emptyset$.
  \end{block}
\end{frame}

\begin{frame}\selectlanguage{russian}
  \begin{block}{Теорема 2 (стр. 196)}
    Задача 1 является coNP-полной задачей 
  \end{block}
  Действительно
  \begin{itemize}
    \item задача 2 есть частный случай задачи 1
    \item утверждение 2 устанавливает что задача 2 coNP сложная
    \item для доказательства того, что $\Pi\not\models A\preceq_C B$, достаточно предъявить контекст $\mathbb{P}$, такой что $\mathbb{P}\models \Pi$, но $\mathbb{P}\not\models A\preceq_C B$. Проверка этих условий выполняется за полиномиальное время. При этом вполне достаточно предъявления подконтекста контекста $\mathbb{P}$ всего с двумя объектами, из-за которых не выполняется предпочтение.
  \end{itemize}
\end{frame}

% section complexity_issues (end)

\end{document}

