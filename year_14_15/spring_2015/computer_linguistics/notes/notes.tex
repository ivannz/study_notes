\documentclass[a4paper]{article}
\usepackage[utf8]{inputenc}

\usepackage{graphicx, url}

\usepackage{amsmath, amsfonts, amssymb, amsthm}
\usepackage{xfrac, mathptmx}

\newcommand{\obj}[1]{{\left\{ #1 \right \}}}
\newcommand{\clo}[1]{{\left [ #1 \right ]}}
\newcommand{\clop}[1]{{\left [ #1 \right )}}
\newcommand{\ploc}[1]{{\left ( #1 \right ]}}

\newcommand{\brac}[1]{{\left ( #1 \right )}}
\newcommand{\induc}[1]{{\left . #1 \right \vert}}
\newcommand{\abs}[1]{{\left | #1 \right |}}
\newcommand{\nrm}[1]{{\left\| #1 \right \|}}
\newcommand{\brkt}[1]{{\left\langle #1 \right\rangle}}
\newcommand{\floor}[1]{{\left\lfloor #1 \right\rfloor}}

\newcommand{\Real}{\mathbb{R}}
\newcommand{\Cplx}{\mathbb{C}}
\newcommand{\Pwr}{\mathcal{P}}

\newcommand{\defn}{\mathop{\overset{\Delta}{=}}\nolimits}

\usepackage[russian, english]{babel}
\newcommand{\eng}[1]{\foreignlanguage{english}{#1}}
\newcommand{\rus}[1]{\foreignlanguage{russian}{#1}}

\title{Computer lingusitics}
\author{Nazarov Ivan, \rus{101мНОД(ИССА)}\\the DataScience Collective}
\begin{document}
\selectlanguage{english}
\maketitle

\selectlanguage{russian}
\section{Lecture 1} % (fold)
\label{sec:lecture_1}
\eng{2015-01-12: Introduction}
\begin{enumerate}
	\item Практические мелкие задачи -- поверхностный синтаксический анализ
	\item Работоспособное приложение
	\item Теория
\end{enumerate}
\eng{Natural Language Processing}
\eng{Computaional linguistics}
Общая лингвистика \begin{itemize}
	\item Синтаксис, синтактика
	\item Семантика
	\item Прагматика -- естественный язык со своими особенностями развивался из соображений удобства решения пракитических задач гоминидов.
\end{itemize}	
Теория формальных языков, Иерархия грамматик Хомского (\eng{Noam Chomsky})

\eng{Quatitative (statistical) linguistics}

Проблема существования языковой универсали (инвариант):
к сожалению среди всех языков универсалью может лищь считаться существование гласных и согласных.
Среди европейских языков -- существование частей речи.

Семиотика -- теория знаковых систем.
Треугольник Фреге
\begin{description}
	\item[\eng{Signifier}] Смысловой символ
	\item[\eng{Signified}] Представление в сознании
	\item[\eng{Referent}] Целевой предмет или явление
\end{description}

Базовые единицы:
\begin{itemize}
	\item фонемы/графемы
	\item лексемы
\end{itemize}

Уровни:
\begin{itemize}
	\item Синтаксичечкий -- предложение
		словосочетания $\to$ сверхфразовые единства
	\item Графематический -- графемы
	\item Морфологический -- слова
	\item Семантический -- элементарная единица ``сема''
	\item Дискурсивный -- \eng{Coherent connected set of sentences}
\end{itemize}

Невозможность взаимо однозначного отображения лексемы в смысл
\begin{description}
	\item[Полисемия] многозначность языковой единицы
	\item[Синонимия] совпадение единиц по смыслу
	\item[Омонимия] совпадение единиц по форме, существенное различие по смыслу.
	Бывает лексическая и морфологическая.
\end{description}

\begin{itemize}
	\item Семантические сети
	\item Синтез нового текста -- языковая компетенция
	\item векторная модлеь текста (\eng{bag of words})
\end{itemize}

% section lecture_1 (end)


\end{document}

