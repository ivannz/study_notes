\documentclass[a4paper]{article}
\usepackage[utf8]{inputenc}

\usepackage{graphicx, url}

\usepackage{amsmath, amsfonts, amssymb, amsthm}
\usepackage{xfrac, mathptmx}

\newcommand{\obj}[1]{{\left\{ #1 \right \}}}
\newcommand{\clo}[1]{{\left [ #1 \right ]}}
\newcommand{\clop}[1]{{\left [ #1 \right )}}
\newcommand{\ploc}[1]{{\left ( #1 \right ]}}

\newcommand{\brac}[1]{{\left ( #1 \right )}}
\newcommand{\induc}[1]{{\left . #1 \right \vert}}
\newcommand{\abs}[1]{{\left | #1 \right |}}
\newcommand{\nrm}[1]{{\left\| #1 \right \|}}
\newcommand{\brkt}[1]{{\left\langle #1 \right\rangle}}
\newcommand{\floor}[1]{{\left\lfloor #1 \right\rfloor}}

\newcommand{\Real}{\mathbb{R}}
\newcommand{\Cplx}{\mathbb{C}}
\newcommand{\Pwr}{\mathcal{P}}

\newcommand{\defn}{\mathop{\overset{\Delta}{=}}\nolimits}

\usepackage[english, russian]{babel}
\newcommand{\eng}[1]{\foreignlanguage{english}{#1}}
\newcommand{\rus}[1]{\foreignlanguage{russian}{#1}}

\title{Integral operators in finance}
\author{Nazarov Ivan, \rus{101мНОД(ИССА)}\\the DataScience Collective}
\begin{document}
\selectlanguage{english}
\maketitle


Fundamentals of variational calculus

time evolution of the the density functions, derived from the optimization problems.

Applications of theory to managing financial portfolio.

\rus{Владимир Рубенович, +7 (903) 155-27-97 к515 }

emails:
econ2014magistr@gmail.com
hsefondy@gmail.com
frin1315@gmail.com
hse.datasci.2014@gmail.com
apeconomics2014@gmail.com


The main goal is Forecasting the time evolution probability densities.

An example:
Binary options: touch- and leave-options strike when stock price hits or leaves a specified set.
	Terms one minute base time scale
	$Q$ -- amount of money
	If the rate crosses a given level within one minute, then we get a premium, otherwise loose everything

the table of intervals-premia
	 1 -- 60\% -- break even probability 62.5\% (why?)
	 5 -- 68\%
	15 -- 70\%
	30 -- 75\%
	60 -- 80\%

The first quiestions are \begin{itemize}
	\item Is it a deterministic process (use runs-test
	\item what is the probability of up/down movements
\end{itemize}

$P_t\sim D_t$ and the current level is $Q_0$.
The distribution $D_t$ could be quite asymmetric and multi-modal.
The price process is never stable and is highly volatile.
Loses are asymmetric and harmful.

\begin{enumerate}
	\item An empirical distribution constructed on some sliding window can be used for immediate forecasting.
	\item Forecast the financial variable $Q_{t+dt}$ given $Q_0 = Q(t_0)$. This is complex.
	What is being forecast? Prices of returns?
\end{enumerate}

Consider the following partial differential equation (Fokker-Planck equations)

\[\frac{\partial}{\partial t} p(x,t) = -\frac{\partial}{\partial x} \brac{ \mu(x) p(x,t) } + \frac{\partial^2}{\partial x^2} \brac{ \sigma(x) p(x,t) } \]

The analyst may never know what is more efficient: a stable in the short term distribution function or an evolving one?

What is $x$ in the equation? Kolmogorov's equation is a decomposition of the change in $x$ and $t$ and the
\[x+dx, t+dt \models \int w(x,dx) p(x, dx, dt )\ldots \]
Like the truncated Taylor's expansion in the Ito's formula.

This form of the Fokker-Planck equation is just a convention.
Or use a random variable with all central moments equal to zero.
Suppose $dx \sim \mathcal{N}$ then the odd moments are zero.


Subjective perception of the density with shifting local centre.

% We do finance here not mathematics!
% Either forecast or lost.

Independent identically normally distributed increments yes they give a second order equation, but not the normal variate itself.

Euler-Lagrange equation. \[\frac{d }{dt}F_x - F_{\dot{x}} = 0\]

First derivative in the variational problem.

Derivation of the Fokker-Planck equation.

How to make $p(x,t)$ evolve in time? \begin{description}
	\item[Differential] Use differential operators. This approach requires the knowledge of parameterts governing the instantaneous diffusion.\hfill\\
	the solution of the Fokker-Planck equation is determined by two functions $\mu(x)$ and $\sigma(x)$

	The instantaneous drift term ($\frac{d}{dt} x = \mu(x)$).
	If $\mu(x) = m (k-x)$ -- mean reverting process, stabilizes over the long run (Ornstein-Uhlenbeck process)
	The instantaneous volatility $\sigma(x) = \sigma_0^2$ -- constant. Given the Gaussian distribution.

	But the default $\mu(x)$ and $\sigma(x)$ are not good enough. And in addition we don't know them.

	How to obtain the best approximations of $\mu$ and $\sigma$? How to chose the one 
	\begin{enumerate}
		\item Use the basics of stochastic filed theory
		\item Construct the function of instantaneous diffusion as eigenvalues of some Fredholm integral operator.
	\end{enumerate}

	Non parametric estimation of $\mu$ and $\sigma$	(c.f. Florensen (?) )
	% Diffusion DIFFUSION!!!!!

	\item[Integral] Act as if the $\mu$ and $\sigma$ are unknown (the drift and the spread). \hfill \\
	Do not equation explicit expressions of the drift and spread.

	Will attempt to construct the source-wise representation (\rus{исотокобразное представление}).
	Fredholm's equation.
	
\end{description}

How are the densities obtained?

The operator equation in some function space.
\[L_x^\brkt{2}\clo{p} = T_t^\brkt{1}\clo{P}\]
$L_x^\brkt{n}$ -- differential operator of order $n$ with respect to the variable $x$.

\[p(x,t) = \sum_{n\geq 1} c_n u_n(x) e^{-\lambda_n t}\]


Basis of functions $\brac{u_i(\cdot)}_{i\geq 1}$ is orthonormal is it is complete and
\[\int\limits_a^b u_j(x) u_j(x) w(x)dx = \delta_{ij}\]
Like in linear algebra but with functions!!!

For some $f(x)$  it is true that $f(x) = \sum_i c_i u_i(x)$ (if $u_j$ is complete) \[
c_i = \int_a^b f(x)u_i(x)w(x)dx
\]
Orthonormality allows some bound on approximation error for truncated series.

Use the $L^2$ norm in our functional space.

\[\nrm{f-g}_2 \defn \brac{\int \abs{f-g}^2 w(x) dx}^\frac{1}{2}\]

% If you don't kick the hedgehog, it won't fly.
% Construct the evolution of the exchange rate consensus forecasts available in Bloomberg.
% Central bank - interbank liquidity relations determines the exchange rate: REPO deals closure

% Use MathCAD!!!

Literature:
\begin{itemize}
	\item Quantitative Finance
	\begin{enumerate}
		\item 
	\end{enumerate}
\end{itemize}

Choose the process and simulate
Choose a financial task
Complete two laboratory tasks.

$\sigma(x)$ -- variance not standard deviation!!!

\begin{align*}
	\omega(x) &= N e^{\int \frac{ - \mu(x) + \frac{d}{dx} \sigma(x)}{\sigma(x)}}\\
\end{align*}

$\omega(x)$ -- the source density function. Construct an operator, which would make $\omega$ evolve in time.
Basis functions $u_j(x)$ are also the solutions of the second order equation: eigenfunctions. 

\[
  \sigma(x) \frac{d^2}{dx^2} \omega \\
+ \brac{ 2\frac{d}{dx}\sigma(x) - \mu(x) } \frac{d}{dx} \omega \\
+ \brac{ \frac{d^2}{dx^2}\sigma(x) - \frac{d}{dx}\mu(x) } \omega \\
= 0
\]

The basis functions are the solutions of this equation -- eigenfunctions.
And this basis is complete (for functions on the same domain).

A simplified Fokker-Planck equation with $ \frac{\partial}{\partial t} p(x,t) \equiv 0$


Strum-Liuville equation:
\[L_x^\brkt{2}\clo{f} = 0\]

Cauchy problem: eigenfunctions of the operator. The spectral decomposition determine the effects of the operator.
\[L_x^\brkt{2}\clo{u_i(x)} = \lambda u_i(x)\]

In simple cases the spectral values and eigenfunctions can be obtained by separation of variables or Rodrigues.

Polynomial basis.
The functions $\mu(x)$ and $\sigma(x)$ are not arbitrary when Rodrigues' method is applied.

Bochner's theorem generalizes the Rodrigues's method


%% Home work
The Fokker-Planck equation is
\[\frac{\partial}{\partial t} p(x,t) = - \frac{\partial}{\partial x} \brac{ \mu(x,t) p(x,t) } + \frac{\partial^2}{\partial x^2} \brac{ \sigma(x,t) p(x,t) }\]

Another form of the same equation is
\begin{align*}
	\frac{\partial}{\partial t} p(x,t) &= \sigma(x,t) \frac{\partial^2}{\partial^2 x} p(x,t) \\
	&+ \brac{ 2 \frac{\partial}{\partial x} \sigma(x,t) - \mu(x,t) } \frac{\partial}{\partial x} p(x,t) \\
	&+ \brac{ \frac{\partial^2}{\partial x^2} \sigma(x,t) - \frac{\partial}{\partial x} \mu(x,t) } p(x,t) 
\end{align*}

Assume $\mu$ and $\sigma$ are stationary (time-independent), i.e. \[\mu(x,t) = \mu(x) \text{ and }\sigma(x,t) = \sigma(x)\]
and consider $p(x,t)$ of the form $X(x) \cdot T(t)$ -- with variables separated -- where the functions $X$ and $T$ are non-zero.
We are working in the Hilbert space of square integrable twice differentiable real functions on some interval of $\Real$ with the $L^2$ norm.
Then \begin{align*}
	X(x) \frac{\partial}{\partial t} T(t) &= T(t) \sigma(x) \frac{\partial^2}{\partial^2 x} X(x)\\
	&+ T(t) \brac{ 2 \frac{\partial}{\partial x} \sigma(x) - \mu(x) } \frac{\partial}{\partial x} X(x)\\
	&+ T(t) \brac{ \frac{\partial^2}{\partial x^2} \sigma(x) - \frac{\partial}{\partial x} \mu(x) } X(x)
\end{align*}
After minor simplification we get \begin{align*}
	\frac{1}{T(t)} \frac{\partial}{\partial t} T(t) &= \frac{1}{X(x)}\left \{ \sigma(x) \frac{\partial^2}{\partial^2 x} X(x) + \brac{ 2 \frac{\partial}{\partial x} \sigma(x) - \mu(x) } \frac{\partial}{\partial x} X(x) \right.\\
	&\left . + \brac{ \frac{\partial^2}{\partial x^2} \sigma(x) - \frac{\partial}{\partial x} \mu(x) } X(x) \right \}
\end{align*}

Since the left hand side depends on $t$ alone whilst the right-hand side only on $x$,
it must therefore be true that both sides are constant, say $\lambda\in \Real$. Thus
\[ T(t) = C_1 e^{\lambda t} \] and $X(\cdot)$ satisfies
\[\lambda X(x) = \sigma(x) \frac{\partial^2}{\partial^2 x} X(x) + \brac{ 2 \frac{\partial}{\partial x} \sigma(x) - \mu(x) } \frac{\partial}{\partial x} X(x) + \brac{ \frac{\partial^2}{\partial x^2} \sigma(x) - \frac{\partial}{\partial x} \mu(x) }\]
which means that $X$ is the eigenfunction of the following second order differential operator
\[L^\brkt{2}_x \defn \sigma(x) \frac{\partial^2}{\partial^2 x} + \brac{ 2 \frac{\partial}{\partial x} \sigma(x) - \mu(x) } \frac{\partial}{\partial x} + \brac{ \frac{\partial^2}{\partial x^2} \sigma(x) - \frac{\partial}{\partial x} \mu(x) } \]
whereas $T$ is the eigenfunction of $L^\brkt{1}_t \defn \frac{\partial}{\partial t}$ with the same eigenvalue.



\end{document}
