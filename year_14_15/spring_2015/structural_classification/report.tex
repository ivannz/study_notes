\documentclass[a4paper,12pt]{extreport}
\usepackage[utf8]{inputenc}
\usepackage{fullpage}
\usepackage{graphicx, url}

\usepackage{amsmath, amsfonts}
\usepackage{mathtools}

\newcommand{\Real}{\mathbb{R}}
\newcommand{\Cplx}{\mathbb{C}}

\newcommand{\Ccal}{\mathcal{C}}
\newcommand{\Ncal}{\mathcal{N}}
\newcommand{\Dcal}{\mathcal{D}}
\newcommand{\Lcal}{\mathcal{L}}

\newcommand{\ex}{{\mathbb{E}}}
\newcommand{\pr}{{\mathbb{P}}}
\newcommand{\var}{{\text{var}}}

\newcommand{\defn}{\overset{\Delta}{=}}

\newcommand{\re}{\operatorname{Re}\nolimits}
\newcommand{\im}{\operatorname{Im}\nolimits}

\usepackage[english, russian]{babel}
\newcommand{\eng}[1]{\foreignlanguage{english}{#1}}
\newcommand{\rus}[1]{\foreignlanguage{russian}{#1}}

\title{\rus{Основы теории измерений в задачах получения и обработки экспертных оценок}}
\author{\rus{Назаров Иван,} \rus{101мНОД(ИССА)}}

\begin{document}
\selectlanguage{russian}
\maketitle

\section*{Отношения на множествах} % (fold)
\label{sec:binary_relations}

% 1. Основные понятия теории измерений: основные свойства используемых отношений, примеры отношений, матричный способ представления отношений.

Отношение $R$ на множестве $M$ есть подмножство декартова приведения $M\times M$.
Любое отношение можно эквивалентным образом представить при помощи отображения
$r:M\times M\to \{0,1\}$ задаваемого следующим образом: $r(x,y) = 1$ если $(x,y)\in R$
и $0$ в противном случае. В свою очередь, любая бинарная функция на $M\times M$
задаёт отношение. В случае конечного множества $M$ отношение $R$ обычно задаётся
матрицей, отражающей значения, принимаемые функций $r$.

Среди основных свойств отношений модно выделить: \begin{itemize}
	\item \textbf{рефлексивность} (если $(x,x)\in R$ для любого $x\in M$) и двойственная
	ей, но не обратная, \textbf{антирефлексивность} (если $(x,x)\notin R$ для любого $x\in M$);
	\item аналогично \textbf{симметричность} (если $(x,y)\in R$, то $(x,y)\in R$) и
	\textbf{асимметричность} (если $(x,y)\in R$, то $(x,y)\notin R$);
	\item \textbf{антисимметричность} (если $(x,y),(y,x)\in R$, то $x=y$);
	\item \textbf{транзитивность} ($(x,y)\in R$ и $(y,z)\in R$ влечёт $(x,z)\in R$);
	\item \textbf{полнота} (для всех $x,y\in M$ справедливо $(x,y)\in R$ или $(y,x)\in R$
	или и то и другое);
\end{itemize}

Каждое в отдельности свойства представляет слишком широкий класс отношений, поэтому
обычно рассматривают комбинации свойств. Например: \begin{itemize}
	\item если отношение $R$ симметрично и рефлексивно, то $R$ -- отношения
	\textbf{толерантности};
	\item рефлексивное, антисимметричное и транзитивное отношение -- \textbf{частичный порядок};
	\item полное отношение частичного порядка -- \textbf{линейный порядок};
	\item если $R$ обладает рефлексивностью, симметричностью и транзитивностью,
	то $R$ -- отношение \textbf{эквивалентности}.
\end{itemize}

% Примеры:
% \begin{itemize}
% 	\item отношение ``больше либо равно'' на множестве вещественных чисел, $(\Real, \geq)$,
% 	-- линейный порядок;
% 	\item отношение ``вложение по подмножеству'' на абстрактном $M$, $(\mathcal{P}(M), \subseteq)$,
% 	где $\mathcal{P}(M)$ -- множество всех пожмножеств $M$, -- частичный порядок;
% 	\item отношение ``быть похожим'' -- толерантность;
% 	\item отношение ``равенство по модулю'' на множестве целых чисел -- эквивалентность.
% \end{itemize}

% section* binary_relations (end)

\section*{Шкалы измерений} % (fold)
\label{sec:_scales_}

Перейдём к важной теме численного описания мнений и суждений экспертов при помощи
шкалирования. Шкала -- это система чисел или иных элементов, принятых для оценки
или измерения каких-либо величин. Формально, это отображение $\phi:M\to X$ множества
объектов $M$ во пространство значений $X$.

Тип и свойства шкалы $\phi$ определяется характеристиками её пространства-образа $X$.
Сред них: \begin{itemize}
	\item описание шкалы, характеризующее элементы $X$: либо вещественные или натуральные
	числа, либо абстрактные сущности;
	\item наличие отношений на множестве $X$, таких как порядок или эквивалентность.
	В случае частичного порядка ключевым является существование наибольших и
	наименьших элементов;
	\item начальная точка -- элемента задающий базовый уровень, от которого
	определяются соотношения между элементами шкалы;
	\item возможность определения меры степени сходства или различия (расстояния)
	между элементами.
\end{itemize}

Согласно этим основным характеристикам выделяются четыре основных вида шкал: номинальная,
абсолютная, порядковая (ординальная, ранговая) и интервальная.

\subsection*{Номиналная шкала} % (fold)
\label{sub:nominal_scale}

Номинальная шкала используется для описания принадлежности объекта определённому
классу. Формально, множество-образ $X$ отображения $\phi$ содержит метки непересекающихся
классов, присваиваемых элементам $M$ этим отображением. В случае если метки классов
известны, и отображение $\phi$ точно известно, то шкалу называют \textbf{категоризованной}.

Простейшим случаем номинальной шкалы является дихотомическая шкала, которая состоит
только из двух классов. Очевидно, что для номинальной шкалы арифметические операции
бессмысленны. Основное преимущество номинальной шкалы в том, что она позволяет
определить частоты вхождений объектов в определённые классы.

% subsection* nominal_scale (end)
\subsection*{Порядковая} % (fold)
\label{sub:ordered}

Шкалы данного типа используются для отнесения объектов к определенному классу
в соответствии со степенью выраженности некоторого свойства. Формально на
множестве-образе $X$ шкалы задаётся линейный порядок, позволяющий произвести
ранжирование объектов. Именно поэтому, подобные шкалы используются при определении
экспертных оценок, а также при определении предпочтений потребителей, установлении
рейтинга того или иного кандидата, измерении полезности, и т.д.

Числа в этой шкале отражают только лишь порядок следования объектов и не позволяют
получить количественную оценку степени предпочтительности одного объекта другому.
Это вызвано тем, что в шкалу порядка не вводится расстояние как характеристика шкалы.

% subsection* ordered (end)

\subsection*{Интервальная шкала} % (fold)
\label{sub:interval_scale}

Интервальная шкала используется для отражения величины различия между объектами:
числа в пространстве значений $X$ не только упорядочены по рангам, но и разделены
определенными интервалами. Особенность этой шкалы состоит в том, что нулевая точка
выбирается произвольным образом. Результаты измерений по шкале интервалов можно
обрабатывать всеми математическими методами, кроме вычисления отношений. Данные
шкалы интервалов дают ответ на вопрос ``на сколько больше?'', но не позволяют
утверждать, что одно значение измеренной величины во столько-то раз больше или
меньше другого.

\noindent Типичным примером интервальной шкалы является календарное время.

% subsection* interval_scale (end)

\subsection*{Шкала отношений} % (fold)
\label{sub:relative_scale}

Данный тип шкал во многом схож с интервальным, однако в пространстве значений $X$
шкалы $\phi$ строго определено положение нулевой точки (эталона). Благодаря этому
шкала отношений не накладывает никаких ограничений на арифметические операции,
используемые для обработки результатов наблюдений. По шкале отношений измеряют и
те величины, которые образуются как разности чисел, отсчитанных по шкале интервалов.
При использовании шкалы отношений измерение какой-либо величины сводится к
экспериментальному определению отношения этой величины к эталону, используемому в
качестве единицы.

Например, измеряя длину объекта, мы узнаем, во сколько раз эта длина больше длины
эталона (метровой линейки), фактически играющего роль единицы длины.

% subsection* relative_scale (end)

% section* _scales_ (end)

% \section*{Примеры шкал в задачах экспертизы и анализа экспертных оценок} % (fold)
% \label{sec:typcal_scales}

% \subsection*{Шкала Лайкерта} % (fold)
% \label{sub:Likert_scale}

% Эта психометрическая порядковая шкала, разработанная в 1932 г. Лайкертом, часто
% используется в опросниках и анкетных исследованиях для оценки степени своего
% согласия или несогласия с некоторым суждением. Обычно используется пять градаций:
% \begin{enumerate}
% 	\item ``полностью не согласен'';
% 	\item ``не согласен'';
% 	\item ``затрудняюсь ответить'';
% 	\item ``согласен'';
% 	\item ``полностью согласен''.
% \end{enumerate}

% subsection* Likert_scale (end)

%  \subsection*{Семантический дифференциал} % (fold)
%  \label{sub:semantic_differential}
 
% Семантический дифференциал -- метод построения индивидуальных или групповых
% семантических пространств. Координатами объекта в семантическом пространстве
% служат его оценки по ряду биполярных градуированных оценочных шкал, противоположные
% полюса которых заданы с помощью вербальных антонимов.
% % Наиболее часто данная шкала
% % используется в виде одной из трех базисных семибалльных шкал:
% % \begin{itemize}
% % 	\item плохой -- хороший;
% % 	\item сильный --  слабый;
% % 	\item пассивный --  активный.
% % \end{itemize} 

%  % subsection* semantic_differential (end)

% \subsection*{Шкала Стейпеля} % (fold)
% \label{sub:staples_scale}

% Шкала Стейпеля -- симметричная, обычно десятибалльная, шкала: от $-5$ до $+5$. В отличие
% от предыдущих двух шкал здесь нет нейтральной точки. Респондента просят сказать, в
% какой мере относится или не относится к объекту та или иная характеристика. Если она
% полностью относится к объекту, выбирается значение $+5$, если наоборот, то $-5$.

% % subsection* staples_scale (end)

% \subsection*{Шкала Харрингтона} % (fold)
% \label{sub:Harringtons_scale}

% Нередко при оценивании проектов возникает необходимость в использовании критериев,
% оценки по которым могут быть получены лишь с помощью специально разрабатываемых
% вербально-числовых шкал. Смысл вербально-числовых шкал в том, что они позволяют
% измерить степень интенсивности критериального свойства, имеющего субъективный характер.
% В состав вербально-числовых шкал входят, как правило, содержательное описание градаций
% шкалы и числовые значения, соответствующие каждой из градаций шкалы. 

% % subsection* Harringtons_scale (end)

% section* typcal_scales (end)

\section*{Экспертные оценки, используемые в задачах экспертизы} % (fold)
\label{sec:expert_grades}
%% Слайд 40-50
% Экспертные оценки, используемые в задачах экспертизы: типы оценок, методы получения количественных и качественных экспертных оценок, методы получения интегральной  (результирующей) характеристики проведенной экспертизы (набора экспертных оценок).

Экспертные оценки бывают следующих типов: вербальные оценки, группировки,
парные сравнения, множественные сравнения, ранжировки, векторы предпочтений,
баллы, интервальные оценки, точечные оценки, многоточечные оценки, функциональные
оценки.

Методы получения качественных экспертных оценок: экспертная классификация,
метод парных сравнений, метод множественных сравнений, ранжирование альтернативных
вариантов, гиперупорядочение, \textbf{метод векторов предпочтений} и дискретные
экспертные кривые.

Существует множество методов получения количественных экспертных оценок:
непосредственная количественная оценка, метод средней точки, метод \textbf{Черчмена-Акофа},
метод \textbf{фон Неймана-Моргенштерна}, прямая оценка вероятности событий, метод отношений,
метод равноценной корзины, метод переменного интервала и метод фиксированного интервала.

\subsection*{Метод Черчмена-Акофа} % (fold)
\label{sub:Churchman_Akoff_method}

Метод Черчмена--Акоффа используется при количественной оценке сравнительной
предпочтительности альтернативных вариантов и допускает корректировку оценок,
даваемых экспертами.

В методе предполагается, что оценки альтернативных вариантов $(a_i)_{i=1}^n$ есть
неотрицательные числа $(f(a_i))_{i=1}^n$, где $f$ -- некоторое отображение, чьи
свойства представлены ниже: \begin{itemize}
	\item если альтернатива $a_i$ строго предпочтительнее $a_j$, $i\neq j$, то $f(a_i)>f(a_j)$;
	\item если $a_i$ и  $a_j$ равноценны, то из оценки равны: $f(a_i)=f(a_j)$;
	\item оценка одновременной реализации альтернативных вариантов $a_i$ и $a_j$
	равняется $f(a_1)+f(a_2)$.
\end{itemize}
Все альтернативные варианты ранжируются по предпочтительности, и каждому из них
эксперт назначает количественные оценки, как правило, в долях единицы.

Далее эксперт сопоставляет по предпочтительности альтернативный вариант $a_1$ и
сумму остальных альтернативных вариантов. Если он предпочтительнее, то и значение
$f(a_1)$ должно быть больше суммарного значения остальных альтернативных вариантов,
в противном случае – наоборот. Если эти соотношения не выполняются, то оценки должны
быть соответствующим образом скорректированы.

Если $a_1$ менее предпочтителен, чем сумма остальных альтернативных вариантов, то
он сравнивается с суммой остальных альтернативных вариантов, за исключением наихудшего.

Если альтернативный вариант $a_1$ на каком-то шаге оказался предпочтительнее суммы
остальных альтернативных вариантов и для оценок это соотношение подтверждается, то
$a_1$ из дальнейших рассмотрений исключается.

Этот процесс продолжается до тех пор, пока последовательно не будет просмотрены все альтернативные варианты.

При практическом применении в случае достаточно большого числа сравниваемых альтернативных
вариантов в метод могут быть внесены некоторые коррективы, снижающие его трудоёмкость.
Так, например, сразу может определяться сумма наибольшего числа альтернативных вариантов
с отбрасыванием менее предпочтительных вариантов, которая меньше, чем $f(a_1)$, и т. д.

% subsection* Churchman_Akoff_method (end)

\subsection*{метод фон Неймана-Моргенштерна} % (fold)
\label{sub:vonNeuman_Morgnestern}

Способ получения численных оценок альтернатив с помощью, так называемых, лотерей
(вероятностных смесей альтернатив) был предложен фон Нейманом и Моргенштерном.
В его основе лежит предположение, согласно которому эксперт для любой альтернативы
$a_j$, менее предпочтительной, чем $a_{i_0}$, но более предпочтительной, чем $a_{i_1}$,
может указать число $p\in [0,1]$ такое, что альтернатива эквивалентна $a_i$ \textbf{по
проедпочтению} вероятностной смеси альтернатив $\Lcal = p\cdot a_0\oplus (1-p)\cdot a_1$.

Суть лотереи в том, что альтернатива $a_{i_0}$ выбирается с вероятностью $p$ , а
альтернатива $a_{}i_1$ - с вероятностью $(1-p)$ . Очевидно, что если $p$ достаточно
близко к $1$, то альтернатива $a_j$ менее предпочтительна, чем смешанная альтернатива
$\Lcal$ ; если $p$ достаточно близко к $0$, то альтернатива $a_j$ более предпочтительна,
чем смешанная альтернатива $\Lcal$, так как почти гарантирова исход $a_{i_0}\prec a_j$.

Если указанная система предпочтений выполнена, то для каждой из набора основных
альтернатив $(a_i)_{i=1}^n$ определяются числа $(u_i)_{i=1}^n$, характеризующие
численную оценку смешанных альтернатив. Численная оценка смешанной альтернативы
$\Lcal = \bigoplus_{i=1}^n p_i a_i$ равна
\[ U(\Lcal) = \sum_{i=1}^n p_i u_i\,. \]
Смешанная альтернатива $\oplus_{i=1}p_i a_i$ предпочтительней, чем смешанная альтернатива
$\oplus_{i=1}q_i a_i$, если 
\[ \sum_{i=1} p_i u_i \geq \sum_{i=1} q_i u_i\,. \]

Очевидно, что сами числа $u_i$ отражают численную оценку чистых альтернатив
(вырожденные смеси). Действительно альтернатива $a_i$ эквивалентна по предпочтению
лотерее
\[ \bigoplus_{i\neq j} 0\cdot a_j \oplus 1\cdot a_i\,, \]
откуда следует
\[ U(a_i) = 1 \cdot u_i + \sum_{j\neq i} 0 \cdot u_j = u_i \,. \]

Таким образом, устанавливается существование функции полезности $U$ на множестве
всевозможные смесей альтернатив, значение которой характеризует степень
предпочтительности любой смешанной альтернативы. Более предпочтительна та смешанная
альтернатива, для которой значение функции полезности больше.

% subsection* vonNeuman_Morgnestern (end)

\subsection*{Метод векторов предпочтений} % (fold)
\label{sub:preference_vectors}

Метод векторов предпочтений используется для коллективного экспертного ранжирования.
Эксперту предъявляется весь набор вариантов и предлагается для каждого из них указать,
насколько он превосходит другие альтернативные варианты.

Эта информация представляется в виде вектора; его первая компонента -- число
альтернативных вариантов, которые превосходят первый, вторая компонента -- число
альтернативных вариантов, которые превосходят второй и т.д. Если в векторе предпочтений
каждое число встречается только один раз, это значит, что экспертом выполнено строгое
ранжирование вариантов по предпочтениям. В противном случае полученный результат
не является строгим ранжированием и отражает затруднения эксперта при оценке сравнительной
предпочтительности отдельных альтернативных вариантов.

Метод векторов предпочтений отличается сравнительно низкой трудоемкостью и может
использоваться с учетом характера экспертизы. Этот метод может быть использован
в тех случаях, когда у эксперта появляются трудности в использовании других методов
оценки сравнительной предпочтительности альтернативных вариантов.

% subsection* preference_vectors (end)

% \subsection*{Экспертная классификация} % (fold)
% \label{sub:expert_classification}

% Этот метод целесообразно используется когда необходимо определить принадлежность
% оцениваемых альтернативных вариантов к установленным и принятым к использованию
% классам, уровням, сортам и т.д. Он может быть использован тогда, когда конкретные
% классы, к которым должны быть отнесены оцениваемые объекты, заранее не определены.
% Может быть заранее не определено и число классов, на которое производится разбиение
% оцениваемых объектов. Оно может быть установлено после завершения процедуры
% классификации. Если эксперту необходимо отнести каждый из альтернативных вариантов
% к одному из заранее установленных классов, то наиболее распространена процедура
% последовательного предъявления эксперту альтернативных вариантов. В соответствии
% с имеющейся у него информацией об оцениваемом объекте и используемой им оценочной
% системы эксперт определяет класс оцениваемого объекта. После завершения процедуры последовательного предъявления оцениваемых альтернативных вариантов эксперту может
% быть предъявлен результат его оценки в виде распределения вариантов по классам.
% Исходя из общего результата классификации эксперт может внести коррективы в собственные
% оценки. Если проводится коллективная экспертиза, то результаты классификации, указанные
% каждым экспертом, обрабатываются с целью получения результирующей коллективной экспертной
% оценки.

% % subsection* expert_classification (end)

\subsection*{Методы определения результирующей экспертной оценки} % (fold)
\label{sub:Final_methods}
% методы получения интегральной  (результирующей) характеристики проведенной экспертизы (набора экспертных оценок).

После оценки согласованности мнений экспертов приступают к определению групповой
(усредненной) оценки. Еще раз отметим, что поиск такой оценки имеет смысл только
в случае достаточно высокой степени согласованности мнений экспертов в группе.

\noindent \textbf{Метод средних арифметических рангов}\hfill \\
Этот метод сводится к подсчету среднего арифметического значения - подсчитывается
сумма рангов, присвоенных экспертами каждому объекту, и делится на число экспертов.
По средним рангам строится итоговое упорядочение, исходя из принципа - чем меньше
средний ранг, тем выше оценка объекта. Однако, важно отметить, что для порядковой
шкалы неправомерно использовать показатель арифметических средних.

\noindent \textbf{Метод медиан рангов}\hfill \\
Медиана - это значение признака, которое разделяет ранжированный ряд распределения
на две равные части — со значениями признака меньше медианы и со значениями признака
больше медианы. Другими словами, для нахождения медианы, нужно отыскать значение
признака, которое находится в середине упорядоченного ряда признака и которое разделяет
ранжированный ряд распределения на две равные части. Если число членов ряда нечетное,
то медиана определяется значением признака, находящимся в середине ряда. Если ряд
состоит из четного число членов, то медиана определяется как среднее двух центральных
значений. Достоинством расчета среднего значения методом медианы является то, что сумма
абсолютных отклонений рангов от медианы представляет собой минимальную величину
по сравнению с отклонением от любой другой величины.

Итак в рамках данного метода вместо неправомерного поиска средних арифметический
в порядковых шкалах, используются медианы индивидуальных оценок экспертов. Для
этого ответы экспертов располагаются в порядке возрастания рангов по объектам. В
случае равноценности элементов, им присваивается средний ранг. Сумма рангов должна
быть равна сумме порядковых номеров элементов в ранжировке. Усредненное групповое
мнение экспертов по объектам в данном случае формируется из значений медиан по каждому
объекту.

% subsection* Final_methods (end)

% section* expert_grades (end)

\section*{Способы проверки согласованности экспертных оценок} % (fold)
\label{sec:expert_opinion_concordance}
%% Слайд 51-60
% 4. ранговые коэффициенты корреляции, коэффициенты конкордации.

В случае участия в опросе нескольких экспертов расхождения в их оценках неизбежны,
однако величина этого расхождения имеет важное значение. Групповая оценка может
считаться достаточно надежной только при условии хорошей согласованности ответов
отдельных специалистов.

Согласованность мнений экспертов рассчитывается при помощи рангового коэффициента
корреляции и коэффициента конкордации, которые применимы только в тех случаях, когда
результаты экспертного опроса представимы в ранговой шкале.

С помощью рангового коэффициента корреляции устанавливается теснота связи между двумя ранжированными рядами, интерпретируемая как согласованность мнений двух экспертов. В практике анализа согласованности применяется два коэффициента: Спирмена и Кендалла.

Коэффициент конкордации Кенделла может принимать значения в пределах от $-1$ до $1$.
При полной согласованности мнений экспертов коэффициент конкордации равен единице,
при полном разногласии -- $-1$. Наиболее реальным является случай частичной
согласованности мнений экспертов. Коеффициент рассчитывается следующим образом:
\[ \tau_{ij} = \frac{1}{n(n-1)} \sum_{k<l}
\text{sign}\bigl((x_{ik}-x_{il})(x_{jk}-x_{jl})\bigr)\,. \]

Коэффициент ранговой корреляции Спирмэна может изменяться в диапазоне от $1$ до $+1$. При полном совпадении оценок коэффициент равен единице. Равенство коэффициента минус единице наблюдается при наибольшем расхождении в мнениях экспертов. Итак, если 
\[\rho_{ij} = 1-\frac{6}{n(n-1)}\sum_{k=1}^n \bigl( x_{ik} - x_{jk}\bigr)^2\,,\]
где $x_{ik}$ – ранг (важность), присвоенный $i$-му объекту $k$-ым экспертом.

Полученная оценка рангового коэффициента корреляции является случайной величиной и,
следовательно, что означает, что необходима проверка её значимости.

% section* expert_opinion_concordance (end)

\section*{Рассмотренные методы коллективной экспертизы} % (fold)
\label{sec:Collective_expertese}

% 5. Определить, какие типы шкал и методы их обработки используются в рассмотренных на лекциях методах коллективной экспертизы (коллективной многовариантной экспертизы и независимой многовариантной экспертизы).

При выявлении существенно различных точек зрения и классификации экспертов
согласно выявленным типам использовалось анкетирование или интервью, по итогам
формировалась векторные характеристики каждого эксперта. Затем производилась
кластеризация полученных данных в пространстве информативных признаков для выявления
групп экспертов со схожим характеристиками.

На этапе анкетирования применяются либо порядковые либо категоризованные шкалы, так как
они наиболее адекватны для целей выявления мнений и предпочтений опрашиваемых экспертов.


% section* Collective_expertese (end)

\end{document}