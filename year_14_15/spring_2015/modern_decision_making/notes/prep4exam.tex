\documentclass[a4paper]{article}
\usepackage[utf8]{inputenc}

\usepackage{graphicx, url}

\usepackage{amsmath, amsfonts, amssymb, amsthm}
\usepackage{mathptmx}

\newcommand{\Real}{\mathbb{R}}
\newcommand{\Cplx}{\mathbb{C}}
\newcommand{\ex}{\mathbb{E}}
\newcommand{\pr}{\mathbb{P}}
\newcommand{\argmin}{\mathop{\text{argmin}}}
\newcommand{\argmax}{\mathop{\text{argmax}}}
\newcommand{\tr}{\text{tr}}
\newcommand{\one}{\mathbf{1}}
\newcommand{\nil}{\mathbf{0}}
\newcommand{\RSS}{{\text{RSS}}}
\newcommand{\Lcal}{\mathcal{L}}
\newcommand{\Hcal}{\mathcal{H}}
\newcommand{\Fcal}{\mathcal{F}}

\usepackage[english, russian]{babel}
\newcommand{\eng}[1]{\foreignlanguage{english}{#1}}
\newcommand{\rus}[1]{\foreignlanguage{russian}{#1}}

\title{MMDM final exam: theoretical questions}
\author{Nazarov Ivan, \rus{101мНОД(ИССА)}\\the DataScience Collective}
\begin{document}
\selectlanguage{english}
\maketitle
\tableofcontents
\clearpage

\section[OLS]{Question \# 1} % (fold)
\label{sec:question_1}
\textbf{\large \textbf{Q}} : Show that the least square solution to the general
linear fitting problem corresponds to the orthogonal projection of the response
variable onto the plane spanned by the columns of $X$.

\noindent{\large \textbf{A}} .\hfill\\ 
Let $T\in \Real^{n\times 1}$, $\beta\in \Real^{(p+1)\times 1}$ and
$X\in \Real^{n\times (p+1)}$, given by $X = \begin{smallmatrix} \one & \mathcal{X} \end{smallmatrix}$
where $\one$ is a vector of ones $\Real^n$ and is optional. The linear regression
problem attempts find the best linear approximation to $\ex(T|X)$: $y(X) = X\beta$.
LS minimizes the \textbf{R}esidual \textbf{S}um of \textbf{S}squares
\[ \RSS = (T-X\beta)'(T-X\beta) \to \min_\beta \,. \]
The first order conditions of the minimizer in the matrix form are:
\begin{align*}
	\frac{\partial}{\partial \beta}\RSS
	&= \frac{\partial}{\partial \beta} (T'T-\beta'X'T - T'X\beta + \beta'X'X\beta)\\
	&= - 2 T'X + 2 \beta'X'X = 0 \,.
\end{align*}
THe matrix $X'X$ is positive-semidefinite and invertible, provided $X$ is full rank.
Therefore $\frac{\partial^2}{\partial \beta\partial \beta}\RSS = 2 X'X \geq 0$,
whence the Least Squares solution $\hat{\beta} = (X'X)^{-1} X'T$ is indeed
the minimizer of the RSS.

The fitted response vector is 
\[ \hat{T} = X\hat{\beta} = X(X'X)^{-1}X' T = \hat{H} T\,, \]
where $\hat{H} = X(X'X)^{-1}X'$ is a projection matrix onto the linear subspace
spanned by the column vectors of $X$. The best linear approximation to $\ex(T\rvert X)$
is thus the projection of $T$ onto the linear manifold $[X]$ spanned by $X$.

Every projector matrix has the following property: if $v$ is in the spanned linear
subspace, then $\hat{H}v = v$. Indeed, since $v$ is in the span of$X$, there must
exist $\alpha \in \Real^{(p+1)\times 1}$ such that $v = X\alpha$. Therefore
\[ \hat{H} v = X(X'X)^{-1} X'v = X(X'X)^{-1} X'X\alpha = X\alpha = v\,. \]

The LS solution $\hat{T} = \hat{H}T$ with $\hat{e} = T - \hat{T}$ is such that 
$\hat{T}\in [X]$ and $\hat{e} \in [X]^\perp$. Indeed, for any $v\in [X]$, 
\[ \hat{e}'v = (T - \hat{T})'v = T'(I_n - \hat{H})'v = T'( v - \hat{H}v) = 0\,, \]
and $\hat{T} = X \hat{\beta} \in [X]$.

Now the projection theorem in $\Real^n$ states that for any $T\in \Real^n$ there
exist \textbf{unique} $T_*\in [X]$ and $T_\perp \in [X]^\perp$ such that
\[ T = T_* + T_\perp \,. \]
Since our LS solution $(\hat{T}, \hat{e})$satisfies the decomposition and subspace
inclusion requirements, it must coincide with the orthogonal projection. Therefore
the LS solution is the orthogonal projection of the response variable onto the plane
spanned by the columns of $X$.

\clearpage
% section question_1 (end)

\section[LASSO and Ridge]{Question \# 2} % (fold)
\label{sec:question_2}
\textbf{\large \textbf{Q}} : Derive the solutions to the lasso and ridge regression
when the columns of $X$ are orthogonal.

\noindent{\large \textbf{A}} .\hfill\\
The problem of penalized (ridge, LASSO) regression in the matrix form, where $X$
is the $n\times k$ matrix of centred standardized input variables with \textbf{no}
intercept, and $T$ -- the response, is to minimize with respect to $\beta$ :
\[ \RSS^{\text{ridge}} = (t-X\beta)'(t-X\beta) + \lambda \beta'\beta \,, \]
whence the solution is
\[ \hat{\beta}^{\text{ridge}} = \bigl(X'X + \lambda I\bigr)^{-1} X'T \,. \]

Suppose $X$ are orthogonal : $X'X = I_p$. The OLS solution is
\[ \hat{\beta}^{\text{OLS}} = (X'X)^{-1} X'T = X'T \,, \]
and the ridge solution (after simplification) is 
\[
\hat{\beta}^{\text{ridge}} = \frac{1}{1+\lambda} \hat{\beta}^{\text{OLS}}\,,
\]
which shows that the larger the penalty $\lambda$, the smaller the coefficient estimates.

The LASSO problem is the same as ridge regression, but with $L^1$ constraints, to
facilitate automatic feature selection. The solution is
\[
\hat{\beta}^{\text{LASSO}}
= \argmin_\beta (t-X\beta)'(t-X\beta) + \lambda \sum_{j=1}^k \lvert \beta_j\rvert \,.
\]
In the orthogonal $X$ case the LASSO minimization solution is equivalent to (just
expand the brackets and drop constant terms)
\[
\hat{\beta}^{\text{LASSO}}
= \sum_{j=1}^k \beta_j^2 - 2 \beta_j \hat{\beta}^{\text{OLS}}_j + \lambda |\beta_j|\,.
\]
Due to additive separability this in turn is equivalent to $n$ problems:
\[
\hat{\beta}^{\text{LASSO}}_j
= \argmin_{\theta} \theta^2 - 2 \theta \hat{\beta}^{\text{OLS}}_j + \lambda |\theta|\,.
\]
The optimal estimate for the $j$-coefficient is
\[
\hat{\theta}
= \max\bigl\{ \hat{\beta}^{\text{OLS}}_j - \frac{\lambda}{2}, 0 \bigr\}
+ \min\bigl\{ \hat{\beta}^{\text{OLS}}_j + \frac{\lambda}{2}, 0 \bigr\}\,.
\]
Feature selection is due to the zero-value region around $\hat{\beta}^{\text{OLS}}_j$
of width $\lambda$.

\noindent\textbf{FYI} : Actually the derivation of the solution itself is done
by expressing $\beta$ as $\beta_+-\beta_-$ and then solving the constrained convex
minimization problem (with \textbf{KKT})
\[ (\beta_+ - \beta_- - a )^2 + \lambda (\beta_+ + \beta_-) \to \min_{\beta_+,\beta_-}\]
subject to $\beta_+, \beta_-\geq 0$ and $\beta_+ \cdot \beta_-=0$.

\clearpage
% section question_2 (end)

\section[NCS]{Question \# 3} % (fold)
\label{sec:question_3}
\textbf{\large \textbf{Q}} : Show that the natural cubic spline has the minimum
value of $\int_a^b |f''(x)|^2 dx$ among all smooth curves that interpolate the
data, and explain why the solution to the smoothing spline problem is a natural
cubic spline.

\noindent{\large \textbf{A}} .\hfill\\
The goal of fitting a smoothing spline is to find a function $g(x)$ among all
functions with continuous second derivative, that minimizes the following penalized
sum of squares
\[ \sum_{i=1}^n (t_i - g(x_i))^2 + \lambda \int |g''(s)|^2 ds \,, \]
where $(x_i,t_i)_{i=1}^n$ is the data with $x_1<\ldots<x_i<\ldots<x_n$.
This functional basically rewards fitting the data as close as possible with $g$
while penalizing $g$ for irregular behaviour. The \textbf{NCS} is a $C^2[a,b]$-smooth
curve interpolating the data $(\xi_k, g_k)_{k=1}^K$ with a piecewise linear second
derivative, which is zero outside $[\xi_1,\Xi_K]$.

The NCS $g$ has the minimum value of the smoothness integral $\int_a^b |g''(s)|^2 ds$
among all $C^2[a,b]$-smooth curves interpolating the data. Indeed, let $f\in C^2[a,b]$
interpolate the data $(\xi_k, g_k)_{k=1}^K$. Then the function $h = g - f$ has roots
at knots $(\xi_k)_{k=1}^K$. Integrating by parts
\[
\int_a^b h'' g'' ds
= \Bigl. h'(s) g''(s)\Bigr\rvert_a^b - \int_a^b h'(s) g'''(s) ds \,,
\]
and since $g'''$ is zero outside $[\xi_1, \xi_K]$, and piecewise inside, and
$g''$ is zero at the endpoints of $[a,b]$
\[
\ldots
= - \int_{\xi_1}^{\xi_K} h'(s) g'''(s) ds \\
= - \sum_{k=1}^{K-1} \int_{\xi_k}^{\xi_{k+1}} h'(s) g'''(s) ds \,.
\]
Furthermore, since $g'''(x) = \alpha_k$ over each $[\xi_k,\xi_{k+1}]$ and $h(\xi_k)=0$
for all $k$
\[
\ldots
= - \sum_{k=1}^{K-1} \int_{\xi_k}^{\xi_{k+1}} h'(s) \alpha_k ds
= - \sum_{k=1}^{K-1} \alpha_k (h(\xi_{k+1})-h(\xi_k)) = 0\,.
\]
The total variation of the $2$-nd derivative of $f$ is therefore
\begin{align*}
	\int_a^b |f''|^2 ds
	&= \int_a^b \bigl|g'' + h''\bigr|^2 ds
	= \int_a^b |g''|^2 + |h''|^2 ds + 2 \int_a^b h'' g'' ds\\
	&= \int_a^b |g''|^2 + |h''|^2 ds \geq \int_a^b |g''|^2 ds
\end{align*}

Any map $f\in C^2[a,b]$ by a smooth adjustment can be forced to be an interpolating
function of data $(\xi_k, g_k)_{k=1}^K$. Thus it is always possible to reduce the
first part of the objective function. But as has been shown the NCS yields the lowest
penalty integral term, which means that such functions would necessarily produce
lower objective function values.

Formally the NCS is defined as follows :
\[
g(x)
= A_k(x - \xi_k) + B_k + \frac{1}{6}
	\frac{\gamma_{k+1}(x-\xi_k)^3+\gamma_k(\xi_{k+1}-x)^3}{\xi_{k+1}-\xi_k}\,,
\]
for all $x\in (\xi_k,\xi_{k+1}]$ and $k=1, K-1$. The regions beyond the boundaries
by definition must be linear, therefore $g(x) = A_0 (\xi_1-x) + B_0$ for $x\leq \xi_1$
and $g(x) = A_K (x-\xi_1) + B_K$ for $x > \xi_K$.


\clearpage
% section question_3 (end)

\section[Smoothing spline]{Question \# 4} % (fold)
\label{sec:question_4}
\textbf{\large \textbf{Q}} : Derive the solution to the smoothing spline problem,
and show that it can be expressed as
\[ \hat{g} = S_\lambda \mathbf{t} = \sum_{j=1}^n \frac{1}{1+\lambda \eta_{n-j+1}} \langle u_j, \mathbf{t} \rangle u_j \,. \]

\noindent{\large \textbf{A}} .\hfill\\
The smoothing spline problem is to find a smooth function that interpolates the
training data $(x_i,t_i)_{i=1}^n$, with $x_i\neq x_j$, and that minimizes the following
objective :
\[
\sum_{i=1}^n (t_i - g(x_i))^2 + \lambda \int |g''(s)|^2 ds \to \min_{y\in C^2[a,b]}\,.
\]
The optimal solution to this kind of problem turns out to be the NCS. If a vector
of values $g_i$ at $x_i$ is given, then the optimal NCS could be recovered via the
relationship with $2$-nd derivatives $\gamma$ at those points: $Q'g = R'\gamma$,
where $R$ is $(K-2)\times(K-2)$ symmetric and positive-definite matrix, and $Q$
is $K\times(K-2)$. These matrices depend on the unique $x_i$ points.

Therefore the smoothing spline problem reduces to :
\[ (t - g)'(t - g) + \lambda g'K g \to \min_g \,,\]
where $K = Q R^{-1} Q'$. The solution turns out to be given by :
\[ \hat{g} = (I_n + \lambda K)^{-1} \mathbf{t} \,. \]
This vector uniquely identifies the NCS, thereby giving a closed form solution 
the original. Though the solution resembles the linear regression case, the \textbf{
smoother matrix} $S_\lambda = (I_n + \lambda K)^{-1}$, is not a projector.

The matrices $K$ and $S_\lambda$ have the same eigenvectors : if $S_\lambda u = \mu u$,
for some $(\mu, u)$, $u\neq 0$, then
\[
u
= \mu S_\lambda^{-1} u
= \mu (I+\lambda K) u
= \mu u + \mu \lambda K u\,,
\]
whence $u$ is an eigenvector of $K$ corresponding to an eigenvalue $\frac{1-\mu}{\lambda \mu}$.
The converse implication is similar: $K u = \eta u$ implies
\[ ( \lambda K + I ) u = ( \lambda \eta + 1 ) u \,, \]
whence $S_\lambda u = \mu u$ with $\mu = \frac{1}{\lambda \eta + 1}$.

Finally, the matrix $K = Q R^{-1} Q'$ depends on the spacing between the interpolation
knots only, whence its eigenvalues and eigenvectors are determined exclusively by the
data to be interpolated.

In conclusion, if $(\mu_k)_{k=1}^n$ are eigenvalues of $S_\lambda$ in non-increasing
order, then the corresponding eigenvectors $(u_k)_{k=1}^n$ are the eigenvectors
of $K$ with eigenvalues $(\eta_j)_{j=1}^n$, given by 
\[ \eta_{n-j+1} = \frac{1}{1 + \lambda \mu_j} \,. \]

Consider the \textbf{SVD} of $S_\lambda$, which is symmetric and positive semi-definite.
Thus the SVD coincides with the eigenvalue decomposition, where $U = V$: 
\[ S_\lambda = U M U' \,, \]
with $M = \text{diag}(\mu_i)_{i=1}^n$ -- the eigenvalues. The spectral decomposition
of $S_\lambda$ yields :
\[
\hat{g}
= S_\lambda \mathbf{t}
= \sum_{j=1}^n \mu_j u_j u_j' \mathbf{t}
= \sum_{j=1}^n \mu_j u_j \langle u_j, \mathbf{t} \rangle\,.
\]
Since each eigenvector $u_j$ of $S_\lambda$ is independent of the actual value
of $\lambda$, the projection is made onto a fixed subspace, determined by the data
only.
Expressing the eigenvalues of $S_\lambda$ through eigenvalues of $K$ results in :
\[
\hat{g}
= \sum_{j=1}^n \frac{1}{1 + \lambda \eta_{n-j+1}} \langle u_j, \mathbf{t} \rangle u_j \,.
\]
The higher the index of the eigenvector $j$, the smaller the respective eigenvalue
of $S_\lambda$ is, and the more smoothing takes place. Higher $u_j$ correspond to
more and more oscillating functions of $x$ and the smoothing spline penalty ignores
the linear components: $\int_a^b | (p s + q)'' |^2 ds = 0$. Therefore two least
eigenvalues of $K$ ($\eta_n$ and $\eta_{n-1}$) are zero and the matrix $K$ makes
the first $\mu_1$ and $\mu_2$ equal to $1$ :
\[
\hat{g} =
\langle u_1, t \rangle u_1 + \langle u_2, t \rangle u_2 +
+ \sum_{j=3}^n \mu_j \langle u_j, t \rangle u_j\,.
\]

\clearpage
% section question_4 (end)

\section[Laplace]{Question \# 5} % (fold)
\label{sec:question_5}
\textbf{\large \textbf{Q}} : Understand the Laplace approximation.

\noindent{\large \textbf{A}} .\hfill\\
The general idea (in the univariate case) is to use Gaussian distribution to approximate
some un-normalized, but \textbf{unimodal}, probability density
\[ p(x) \propto f(x) \,, \]
with some normalising constant $N = \int f(x) dx$.

Consider the Taylor series expansion of $\log f(x)$ around its mode $x_0$, $f'(x_0) = 0$:
\begin{align*}
	\log f(x)
	&= \log f(x_0) + \frac{f'(x_0)}{f(x_0)} (x-x_0)\\
	&+ \frac{1}{2} \frac{f''(x_0)f(x_0) - \bigl(f'(x_0)\bigr)^2}{\bigl(f(x_0)\bigr)^2} (x-x_0)^2
		+ o\bigl((x-x_0)^2\bigr)\,,
\end{align*}
as $x\to x_0$, whence is some very small neighbourhood of $x_0$
\[
\log f(x) \approx \log f(x_0) + \frac{1}{2} \frac{f''(x_0)}{f(x_0)} (x-x_0)^2\,.
\]
Therefore for 
\[ C = - \frac{f''(x_0)}{f(x_0)}\,, \]
one has 
\[ f(x) \approx f(x_0) \text{exp}\bigl\{ - \frac{1}{2} C (x-x_0)^2 \bigr\} \,. \]
The right hand side is an improperly normalised Gaussian. Thus
\[ f(x) \approx f(x_0) \sqrt{\frac{2\pi}{C}} q(x) \,, \]
for the Gaussian density with mean $x_0$ and variance $C^{-1}$ (precision $C$)
\[
q(x) = \frac{1}{\sqrt{2\pi C^{-1}}} \text{exp}\bigl\{ - \frac{1}{2} C (x-x_0)^2 \bigr\}\,.
\]
Hence $N \approx f(x_0) \sqrt{\frac{2\pi}{C}}$.

In the $d$-dimensional case the Taylor expansion for $x\to x_0$ is
\begin{align*}
\log f(x) &= \log f(x_0) + \nabla f(x_0)' (x-x_0) \\
	&+ \frac{1}{2} \frac{1}{\bigl(f(x_0)\bigr)^2} (x-x_0)'
		\bigl( f(x_0) \nabla^2 f(x_0) - \nabla f(x_0) \nabla f(x_0)' \bigr) (x-x_0) \\
	&+ o(\|x-x_0\|^2) \,,
\end{align*}
for $\nabla f(x_0)\in \Real^{d\times1}$ -- gradient and $\nabla^2 f(x_0)$ -- Hessian
at $x_0$, whence 
\[
f(x) \approx f(x_0) \text{exp}\biggl\{ - \frac{1}{2} (x-x_0)' \Lambda (x-x_0) \biggr\}\,,
\]
for $\Lambda = - f(x_0)^{-1} \nabla^2 f(x_0)$. Therefore
\[
N \approx f(x_0) (2\pi)^{\frac{d}{2}} |\Lambda|^{-\frac{1}{2}} \,.
\]

\clearpage
% section question_5 (end)

\section[Binary Logit]{Question \# 6} % (fold)
\label{sec:question_6}
\textbf{\large \textbf{Q}} : Derive the iterative reweighed least square algorithm for
binary logistic regression.

\noindent{\large \textbf{A}} .\hfill\\

Let $(x_i,t_i)_{i=1}^n \in \Real^{p+1}\times \{0,1\}$ be the training sample for binary
classification. In the two-class case the logistic regression posits that the posterior
class probability is given by
\[ \pr(T=k|X=x) = \sigma(\beta'x) = \sigma( \beta_0 + \beta_1'x ) \,, \]
where $\sigma(t) = \frac{e^t}{1+e^t}$ and $\beta=(\beta_0, \beta_1)'\in \Real^{(p+1)\times 1}$.
Before proceeding, note the useful property of sigmoid function derivative:
\[
\frac{d}{dt} \sigma(t)
= \frac{e^t(1+e^t) - e^{2t} }{(1+e^t)^2}
= \sigma(t)(1-\sigma(t)) \,.
\]

Due to $0$-$1$ coding scheme the log-likelihood of a particular observation $(x,t)$
is
\begin{align*}
l(x,t,\beta)
&= \log p(T=t|X=x)
= \log \sigma(\beta'x)^t (1-\sigma(\beta'x))^{1-t}\\
&= t \log \sigma(\beta'x) + (1-t)\log (1-\sigma(\beta'x))\,,
\end{align*}
whence
\begin{align*}
	\frac{d}{d\beta} l
	&= t \frac{(1-\sigma(\beta'x))\sigma(\beta'x)}{\sigma(\beta'x)} x
		- (1-t) \frac{ (1-\sigma(\beta'x))\sigma(\beta'x) }{(1-\sigma(\beta'x))} x\\
	&= ( t - \sigma(\beta'x) ) x\,.
\end{align*}
The full sample log-likelihood is 
\[ L( \beta ) = \sum_{i=1}^n l(x_i,t_i,\beta) \,, \]
and its gradient (first vector derivative) is
\[
\nabla L
= \sum_{i=1}^n ( t_i - \sigma(\beta'x_i) ) x_i
= X' ( t - \sigma(\beta'X) )
\,, \]
where $X$ is the $n\times (p+1)$ matrix of observations, $t$ is the target vector, and
$\sigma(\beta'X)$ is the $n\times 1$ vector $(\sigma(\beta'x_i))_{i=1}^n$. It appears
to be non-linear in $\beta$. As for the second derivative it is given by
\[
\frac{\partial^2 }{\partial \beta_k \partial \beta_j} L
= \frac{\partial }{\partial \beta_j} \sum_{i=1}^n ( t_i - \sigma(\beta'x_i) ) x_{ik}
= - \sum_{i=1}^n (1-\sigma(\beta'x_i))\sigma(\beta'x_i) x_{ik} x_{ij}\,,
\]
whence
\[ \nabla^2 L = - X'B X\,, \]
for $B = \text{diag}\bigl( (1-\sigma(\beta'x_i))\sigma(\beta'x_i) \bigr)_{i=1}^n$.
The Hessian of $L$ is obviously negative semi-definite, which means that the optimal
$\beta$ maximizes the log-likelihood.

The first order conditions contain $\beta$ inside a non-linear term which makes
analytical expression of the optimal coefficient troublesome if not intractable.
Instead let's employ Newton-Raphson iterative procedure to obtain a sequence of
gradually more accurate estimates of $\beta$.

Recall that the Newton-Raphson method of finding stationary points (roots of its
first derivative) of a twice differentiable function. The log-likelihood $L$ is
a twice differentiable $\Real^{p+1}\to\Real$ function, whence it is legitimate
to approximate it by a second order Taylor quadratic form. Given some point $a$,
$L(\beta) = Q(\beta) + o(\|\beta-a\|^2)$ where
\[ Q(x) = L(a) + \nabla L(a)'(x-a) + \frac{1}{2}(x-a)'\nabla^2 L(a) (x-a)\,. \]
Expanding the brackets yields
\[
Q(x) = L(a) - \nabla L(a)'a + \frac{1}{2}x'\nabla^2 L(a) a
	+ \bigl( \nabla L(a)' - a'\nabla^2 L(a) \bigr)x
	+ \frac{1}{2}x'\nabla^2 L(a) x\,,
\]
and $Q$ is concave, since the Hessian matrix of $L$ at $a$ is negative semi-definite.
The maximum of $Q$ is thus given by
\begin{align*}
	x^*
	&= (\nabla^2 L(a))^{-1} \bigl( \nabla^2 L(a) a - \nabla L(a)\bigl) \\
	&= a - (\nabla^2 L(a))^{-1}\nabla L(a) \,.
\end{align*}
This $x^*$ is guaranteed to yield a value of $L$ closer to maximum.

Therefore the following iterative procedure seems logical: \begin{enumerate}
	\item start with some initial $\beta = \beta^{(0)}$;
	\item update $\beta \to \beta_{\text{new}}$ by a Newton-Raphson step:
	\begin{align*}
		\beta_{\text{new}}
		&= \beta + (X'BX)^{-1} X'( t - \sigma(\beta'X) ) \\
		&= (X'BX)^{-1} X'B \bigl( X \beta - B^{-1}( \sigma(\beta'X) - t ) \bigr) \,,
	\end{align*}
	where the term in brackets is the adjusted response;
	\item repeat step 2 until a convergence criteria is met.
\end{enumerate}
If $Z = X \beta - B^{-1}( \sigma(\beta'X) - t )$, then $\beta_{\text{new}}$ is the
solution produced by weighted least squares with weighing matrix $B$ :
\[ ( Z - X\gamma )'B (Z - X\gamma ) \to \min_\gamma \,, \]
where $B$ must be positive definite. The first order conditions are
\[ - 2 X'BZ + 2 X'BX \gamma = 0 \,. \]

\clearpage
% section question_6 (end)

\section[EM]{Question \# 7} % (fold)
\label{sec:question_7}
\textbf{\large \textbf{Q}} : Derive the EM algorithm.

\noindent{\large \textbf{A}} .\hfill\\
The EM algorithm seeks to maximize the likelihood by means of successive application
of two steps: the \textbf{E}-step and the \textbf{M}-step. Consider a probability
density $p(X|\Theta)$ of observations $X$ given some parameter $\Theta$. For any
probability density $q$ on the space of latent variables $Z$ the following holds:  
\begin{align*}
\log p(X|\Theta)
    &= \int q(Z) \log p(X|\Theta) dZ
     = \mathbb{E}_q \log p(X|\Theta) \\
   %% &= \Bigl[p(X,Z|\Theta) = p(Z|X,\Theta) p(X|\Theta) \Bigr] \\
    &= \mathbb{E}_{Z\sim q} \log \frac{p(X,Z|\Theta)}{p(Z|X\Theta)}
     = \mathbb{E}_{Z\sim q} \log \frac{q(Z)}{p(Z|X,\Theta)}
     + \mathbb{E}_{Z\sim q} \log \frac{p(X,Z|\Theta)}{q(Z)} \\ 
    &= KL\bigl(q\|p(\cdot|X,\Theta)\bigr) + \mathcal{L}\bigl(q, \Theta\bigr)\,,
\end{align*}
since the Bayes theorem posits that $p(X,Z|\Theta) = p(Z|X,\Theta) p(X|\Theta)$.
This equation is the ``\textbf{master identity}''. As the Kullback-Leibler divergence
is \textbf{always} non-negative :
\[ \log p(X|\Theta) \geq \mathcal{L}\bigl(q, \Theta\bigr) \,. \]
By changing $\Theta$ and varying $q$ we can make the lower bound as large as possible.

For some fixed and given $\Theta$, the \textbf{master identity} implies that with
respect to $q$ maximization of $\mathcal{L}$is equivalent to minimization of
$KL\bigl(q\|p(\cdot|X,\Theta)\bigr)$. Since there are no restrictions to $q$, the
optimal minimizer $q^*_\Theta$ is $q^*(Z|\Theta) = p(Z|X,\Theta)$ for all $Z$.

At the optimal distribution $q^*_\Theta$ the \textbf{master identity} becomes
\begin{align*}
\log p(X|\Theta)
&= \mathcal{L}\bigl(q^*_\Theta, \Theta\bigr)
= \mathbb{E}_{Z\sim q^*_\Theta} \log \frac{p(X,Z|\Theta)}{q^*(Z|\Theta)} \\
&= \mathbb{E}_{Z\sim q^*_\Theta} \log p(X,Z|\Theta) - \mathbb{E}_{Z\sim q^*_\Theta} \log q^*(Z|\Theta) \,,
\end{align*}
for any $\Theta$. Thus the problem of log-likelihood maximization reduces to that
of maximizing the sum of expectations on the right-hand side. However this new problem
does not seem to be tractable in general since the optimization parameters $\Theta$
affect both the expected log-likelihood $\log p(X,Z|\Theta)$ under $Z\sim q^*_\Theta$
and the entropy of the optimal distribution of the latent variables $Z$.

Hopefully using an iterative procedure which switches between the computation of $q^*_\Theta$
and the maximization of $\Theta$ might be effective : \begin{description}
	\item[\textbf{E}-step] considering $\Theta_i$ as fixed and given, find
	\[ q^*_{\Theta_i} = \argmin_q \,\, KL\bigl(q\|p(\cdot|X,\Theta_i)\bigr)\,,\]
	and set $q_{i+1} = q^*_{\Theta_i}$ ;
	\item[\textbf{M}-step] considering $q_{i+1}$ as fixed and given, find
	\[ \Theta^*_{i+1} = \argmax_\Theta \,\, \mathbb{E}_{Z\sim q_{i+1}} \log p(X,Z|\Theta)\,,\]
	and put $\Theta_{i+1} = \Theta^*_{i+1}$.
\end{description}

Recall that the \textbf{master identity} is an identity: for all densities $q$ on
$Z$ and for all admissible parameters $\Theta$
\[ \log p(X|\Theta) = KL\bigl(q\|p(\cdot|X,\Theta)\bigr) + \mathcal{L}\bigl(q, \Theta\bigr) \,. \]

The \textbf{EM} algorithm is correct in that its iterations do not decrease the
log-likelihood. Indeed, just after the \textbf{E}-step one has $q_{i+1} = p(Z|X,\Theta_i)$,
whence $KL\bigl(q_{i+1}\|p(\cdot|X,\Theta_i)\bigr) = 0$. Via the \textbf{master identity}
this implies
\[ \log p(X|\Theta_i) = \mathcal{L}(q_{i+1},\Theta_i) \,. \]
After the \textbf{M}-step, since $\Theta_{i+1}$ \emph{improves} $\mathcal{L}(q_{i+1},\Theta)$
compared to its value at $(q_i,\Theta_i)$, one has
\[ \mathcal{L}(q_{i+1},\Theta_i) \leq \mathcal{L}(q_{i+1},\Theta_{i+1}) \,. \]
Therefore the effect of a single complete round of \textbf{EM} on the log-likelihood
itself is:
\[
\log p(X|\Theta_i)
= \mathcal{L}(q_{i+1},\Theta_i)
\leq \mathcal{L}(q_{i+1},\Theta_{i+1})
\leq \mathcal{L}(q_{i+2},\Theta_{i+1})
= \log p(X|\Theta_{i+1}) \,,
\]
where the equality is achieved between the \textbf{E} and the \textbf{M} step within one round.
This implies that \textbf{EM} indeed iteratively improves the log-likelihood.

\clearpage
% section question_7 (end)

\section[Fisher's LDA]{Question \# 8} % (fold)
\label{sec:question_8}
\textbf{\large \textbf{Q}} : Express Fisher's linear discriminant criterion (Rayleigh
quotient), and find the optimal separating hyperplane according to this criterion.

\noindent{\large \textbf{A}} .\hfill\\
Suppose the training sample is $(x_i, t_i)_{i=1}^n \in \Real^p \times\{-1,+1\}$ --
a two-class classification problem. The idea of Fisher's linear discriminant
criterion is to reduce the dimensionality of the data by projecting it onto a line,
chosen in such a way as to facilitate easier binary separation. For $k\in\{-1,+1\}$
let
\[ \bar{m}_k = \frac{1}{|C_k|} \sum_{i\in C_k} x_i \,, \]
where $C_k = \{ i : t_i = k \}$. Further define
\[
\Sigma_k = \frac{1}{|C_k|} \sum_{i\in C_k} ( x_i - \bar{m}_k )( x_i - \bar{m}_k )'\,,
\]
which is the within-class variance matrix. For any $\beta$ let
\[ m_k(\beta) = \beta'\bar{m}_k\,, \] 
and
\[ S_k(\beta) = \frac{1}{|C_k|} \sum_{i\in C_k} ( \beta'x_i - m_k )^2 \,. \]

Fisher's linear discriminant is such a hyperplane, given by the normal vector $\beta$,
that maximizes the Fisher's separation criterion given by
\[ C(\beta) = \frac{\bigl(m_+(\beta)-m_-(\beta)\bigr)^2}{ S_-(\beta) + S_+(\beta) }\,. \]
In fact the separation criterion may be simplified to
\begin{align*}
	C(\beta)
	&= \frac{\bigl(\beta'(\bar{m}_+ - \bar{m}_-)\bigr)^2}{ \beta'( \Sigma_+ + \Sigma_- )\beta }\\
	&= \frac{\beta'(\bar{m}_+ - \bar{m}_-)(\bar{m}_+ - \bar{m}_-)'\beta}{ \beta'( \Sigma_+ + \Sigma_- )\beta }\\
	&= \frac{\beta'S_B\beta}{ \beta'S_W\beta }\,,
\end{align*}
where $S_W = \Sigma_+ + \Sigma_-$ is the within-class variance, and $S_B$ --
the between class variance (scatter) matrix given by
\[ S_B = (\bar{m}_+ - \bar{m}_-)(\bar{m}_+ - \bar{m}_-)'\,. \]

Thus the separation criterion we are trying to maximize is the Rayleigh quotient
\[ \mathcal{R}(\beta) = \frac{\beta'S_B\beta}{\beta'S_W\beta}\,. \]
Since $\mathcal{R}(\beta)$ is homogeneous of degree $0$ in $\beta$ it is possible
to fix the denominator and then maximize the numerator only:
\[ \beta'S_B\beta \to \max_\beta \,,\]
subject to $\beta'S_W\beta = 1$. The optimal solution is given by the first order
conditions (KKT) $S_B\beta - \lambda S_W \beta = 0$ and $\beta'S_W\beta = 1$.
Note that these conditions imply that $\beta'S_B\beta = \lambda$. Thus the problem
further simplifies to finding the largest eigenvalue and the corresponding eigenvector
in the generalized eigenvalue-eigenvector problem $S_B\beta = \lambda S_W\beta$ with
normalisation $\beta'S_W\beta = 1$.

Notice that
\[
S_B \beta_* = (\bar{m}_+ - \bar{m}_-) \bigl( (\bar{m}_+ - \bar{m}_-)'\beta_* \bigr)
\propto \bar{m}_+ - \bar{m}_-\,,
\]
whence the normal to the hyperplane with the best separation criterion is given by
\[ \beta_* \propto S_W^{-1} (\bar{m}_+ - \bar{m}_-) \,. \]
The optimal threshold for the discriminant hyperplane is found by projecting the 
training data into the space spanned by $\beta^*$ and then threshold that best
separates the data is chosen from analysis of the one-dimensional distribution.
With the ``best'' threshold $\beta_0$ the classification rule becomes
\[\text{class}(x) = \begin{cases}
	\text{class}_1, &\text{ if } \beta_*' x + \beta_0 \geq 0\\
	\text{class}_2, &\text{ otherwise }
\end{cases}\,.\]

\clearpage
% section question_8 (end)

\section[CART]{Question \# 9} % (fold)
\label{sec:question_9}
\textbf{\large \textbf{Q}} : Explain and understand the CART procedure.

\noindent{\large \textbf{A}} .\hfill\\
\textbf{C}lassification \textbf{a}nd \textbf{R}egression \textbf{T}rees procedure
is a method for constructing decision trees based on the training data. This method
attempts to find the best approximation to the optimal decision tree, which is hard
to compute efficiently in practice.

Tree based methods permit non-linear regression and classification. Consider training
dataset $(x_i,t_i)_{i=1}^n\in \Real^p \times \Real$. A \emph{regression} tree $T$
models the response as a piecewise function $f:\Real^p \to \Real$ :
\[ f(x) = \sum_{m=1}^M f_m(x) 1_{R_m}(x)\,, \]
where $(R_m)_{m=1}^M$ is a collection of non-overlapping regions covering $\Real^p$, 
and $(f_m(\cdot))_{m=1}^M$ are the regression models fit within each region. Simplest
and most parsimonious regression trees use $f_m(x) = c_m$ and thus model a piecewise
constant regression. If the partition $(R_m)_{m=1}^M$ is known, the $c_m$'s are the
local averages of the response variable within each region $R_m$.

It is very hard to construct a partition that minimizes the training error, that is
why regions are constructed with the help of a greedy procedure, that approximates
the optimal solution via successive binary splits along the most optimal feature
axis. The greedy algorithm approximates the solution to
\[
\sum_{i=1}^n \bigl( t_i - \sum_{m=1}^M c_m 1_{R_m}(x) \bigr)^2
\to \mathop{\text{minimum}}_{(c_m)_{m=1}^M, (R_m)_{m=1}^M}\,.
\]
For a splitting axis (feature) $j=1,\ldots,p$ and a splitting threshold $s$, define
regions
\[ R_1(j,s) = \bigr\{ z \in \Real^p\,\big|\, z_j \leq s \bigr\} \,, \]
and
\[ R_2(j,s) = \bigr\{ z \in \Real^p\,\big|\, z_j > s \bigr\} \,. \]
For observations $D$ (by default $D = \{(x_i,t_i)_{i=1}^n\}$) choose the split
$(j,s)$ that minimizes
\[ 
  \min_{c_1} \sum_{i\,:,x_i\in R_1(i,s)\cap D} \bigl( t_i - c_1 \bigr)^2
+ \min_{c_2} \sum_{i\,:,x_i\in R_2(i,s)\cap D} \bigl( t_i - c_2 \bigr)^2
\to \min_{j,s}\,.
\]
This problem is feasible since for each $j$ there are at most $n$ points to consider
for $s$: it is given by the unique values of $x_{ij}$. The optimal $c_m$ are defined
as the local averages of $D$ within each region $R_m$ :
\[ \hat{c}_m = \frac{1}{\#\{i\,:,x_i\in R_m \cap D\}} \sum_{i\,:,x_i\in R_m \cap D} t_i \,. \]

Once the best split $(j,s)$ has been found, partition the training data into
\[ D_1 = D \cap R_1(j,s)\,\text{ and }\, D_2 = D \cap R_2(j,s)\,, \]
and recursively split on both $D_1$ and $Q_2$. Record the best splits as a binary
tree.

The common approach to constructing decision regression trees is to apply cost-complexity
pruning. The idea is to construct a large tree $T_0$, using the greedy procedure,
and then to prune $T_0$ by merging its terminal nodes (or collapsing the internal
nodes). The pruning is governed by the cost-complexity trade off
\[ C_\alpha(T) = \sum_{m=1}^{M_T} \#\{i\,:,x_i\in R^{(T)}_m\} Q_m(T) + \alpha |T|\,, \]
where $\alpha$ is the trade-off parameter, $|T|$ is the complexity of the tree (the
number of regions in the partition) and $Q_m(T)$ is the \textbf{impurity measure}
of region $m$ of the \textbf{regression} tree $T$:
\[
Q_m(T) = \min_c \frac{1}{\#\{i\,:,x_i\in R^{(T)}_m\}} \sum_{i\,:,x_i\in R^{(T)}_m} ( t_i - c )^2 \,.
\]
The pruning is done by elimination of \textbf{the weakest links}: successively collapse
the node that produces the smallest per-node increase in $C_\alpha(T)$, until left with
a root. The minimizer sub-tree $T_\alpha\subseteq T_0$ of $C_\alpha(T)$ necessarily lies
on the path of one-node collapses.

\emph{Classification} trees $T$ for a $K$-class classification problem are defined
by the regions $(R_m)_{m=1}^M$ and classes $(k_m)_{m=1}^M$ assigned to each region :
\[ f(x) = \sum_{m=1}^M k_m 1_{R_m}(x) \,. \]
Define the proportion of class $k$ observations at node $m$ (falling within region
$R_m$ that is) as
\[ \hat{p}_{mk} = \frac{1}{\#\{i\,:,x_i\in R_m\}} \sum_{i\,:,x_i\in R_m} 1_{t_i = k} \,, \]
and assign the dominating (majority) class to each region $R_m$ 
\[ k(m) = \argmax_{k=1,\ldots,K} \hat{p}_{mk} \,. \]
The misclassification error within each region is thus
\[ \text{mis}_m = 1 - \hat{p}_{m k(m)}\,. \]

In general, the quality of region $m$ of a tree $T$ is measured by an \textbf{impurity
index}, which reflects how poorly:
\begin{itemize}
	\item the regression tree predicts the value of $t$ within $R_m$;
	\item the classification tree distinguishes the classes within $R_m$.
\end{itemize}
For a regression tree $T$ and a region $m$ \textbf{CART} employs the following
the square-loss impurity measure. As for classification tree $T$ there are many
possibilities for impurity indices, among which the most well studied are:
\begin{itemize}
	\item The \textbf{Gini impurity} index, used in \textbf{CART}:
	\[ Q_m(T) = \sum_{k=1}^K \hat{p}_{mk} (1-\hat{p}_{mk}) \,; \]
	\item The \textbf{entropy impurity}, used in \textbf{ID$4.5$}:
	\[ Q_m(T) = - \sum_{k=1}^K \hat{p}_{mk} \log \hat{p}_{mk} \,; \]
	\item Simple misclassification error (not recommended for use):
	\[ Q_m(T) = \sum_{k=1}^K 1-\hat{p}_{mk} \,.\]
\end{itemize}
All these measures can be used for pruning trees, but only the Gini and entropy
impurity indices are recommended for use in growing classification trees.

Due to hierarchical structure trees have high variance, errors at top nodes propagate
to lower ones and small changes in training data can manifest in very different binary
splits.

\subsection{Alternative explanation} % (fold)
\label{sub:alternative_explanation}

Given some training dataset $(x_i,t_i)_{i\in I} \in \Real^p \times T$ constructing
a regression tree amounts to finding a piecewise function
\[ f(x) = \sum_{m=1}^M f_m( x ) 1_{R_m}(x)\,, \]
for $(f_m)_{m=1}^M$ is some parametric family $\Fcal$, such that
\[ \sum_{i\in I} \Lcal\bigl( t_i, f(x_i) \bigr) \to \min_{(f_m)_{m=1}^M}\,. \]

This problem in NP-hard, even in a simplified case with $\Fcal$ being a family of
constant maps $\Real^p\mapsto \Real$. In practice, the optimal $f^*$ is approximated
by the ``forward selection'' procedure which is an iterative algorithm that makes
a series of dichotomous partitions of the feature space.

Let $J\subseteq I$ be a subset of the training data. For a region $R\subseteq \Real^p$
define the observations falling into $R$ as
\[ J(R) = \{ i\in J\,:\,x_i \in R\} \,. \]
The algorithm tries to partition $R$ into $R_1$ and $R_2$ by greedily minimizing
the overall \textbf{split impurity}, defined as
\[ \frac{|J(R_1)|}{|J|} Q(R_1) + \frac{|J(R_2)|}{|J|} Q(R_2) \,, \]
for some given region impurity measure $Q$. The partition is not arbitrary and is
actually the half-spaces created by the separating hyperplane perpendicular to some
dimension of features. Let $a=1,\ldots,p$ denote the axis of the split and $\theta$
-- the splitting threshold. Define the half-spaces as
\begin{align*}
	R_+(a,\theta) &= \bigl\{ z \in\Real^p \big| z_a \in (-\infty,\theta] \bigr\}\,,\\
	R_-(a,\theta) &= \Real^p \setminus R_-(a,\theta)\,.
\end{align*}
For the current subset $J\subseteq I$ of the dataset and region $R\subseteq \Real^p$,
the forward selection determines the optimal split by solving
\[
(a^*,\theta^*) = \argmin_{(a,\theta)}
  \frac{|J(R_-(a,\theta))|}{|J|} Q( R_-(a,\theta) )
+ \frac{|J(R_+(a,\theta))|}{|J|} Q( R_+(a,\theta) ) \,,
\]
and then applies the same greedy splitting for each subset of the dataset in $J_-(a^*,\theta^*)$
and $J_+(a^*,\theta^*)$.

The \textbf{impurity measure} $Q$ of region $R$ gauges how poorly the best local
model in this region does in predicting the target variable. For general regression
trees the measure is defined as the mean loss
\[ Q(R) = \frac{1}{ |J(R)| } \sum_{i\in J(R)} \Lcal\bigl( t_i, f_R(x_i) \bigr) \,, \]
for 
\[ f_R = \argmin_{f\in \Fcal} \sum_{i\in J(R)} \Lcal\bigl( t_i, f(x_i) \bigr) \,, \]
where $\Lcal:T\times T\to \Real$ is the loss function. In CART square loss is used
for constructing a regression tree.

For classification trees there are three most widely used impurity measures each
with its own merits and faults. All of them are based on the empirical class
probabilities conditional on $X\in R$ : for $k = 1,\ldots, K$
\[ p_k(R) = \frac{1}{|J(R)|} \sum_{j\in J(R)} 1_{t_i = k}\]
which estimate $\pr(T=k|X\in R)$. The classification itself is performed via assigning
a class with the highest empirical posterior probability:
\[ k_R = \argmax_{k=1,\ldots, K} p_k(R)\,, \]
which has nothing to do with Baysian classification. The measures are \begin{description}
	\item[Misclassification error] \[ Q(R) = 1 - p_{k_R}(R)\,; \]
	\item[Entropy] \[ Q(R) = - \sum_{k=1}^K p_k(R) \log p_k(R) \,; \]
	\item[Gini index] \[ Q(R) = \sum_{k=1}^K p_k(R) ( 1 - p_k(R)) \,. \]
\end{description}

% subsection alternative_explanation (end)

\clearpage
% section question_9 (end)

\section[AdaBoost]{Question \# 10} % (fold)
\label{sec:question_10}
\textbf{\large \textbf{Q}} : Derive the AdaBoost algorithm.

\noindent{\large \textbf{A}} .\hfill\\
Consider a two-class classification problem: the training data is $(x_i, y_i)_{i=1}^n
\in \Real^p \times \{-1,+1\}$. Boosting is an algorithm to assemble many weak classifiers
into a powerful committee. Computationally it is based on the \textbf{F}orward
\textbf{S}tagewise \textbf{A}daptive \textbf{M}odelling, a general procedure which
constructs an approximation to the global solution of
\[
\sum_{i=1}^n L\bigl( y_i, \sum_{m=1}^M \beta_m b(x_i,\gamma_m) \bigr)
	\to \min_{(\beta_m,\gamma_m)_{m=1}^M}\,,
\]
where $ L:\Real\times \Real\to \Real$ is a loss function, $b(\cdot,\gamma)$ are
\emph{base} functions (classifiers) in the additive expansion with coefficients
$(\beta_m)_{m=1}^M$. FSAM does this approximation via a greedy algorithm that adds
$b(\cdot,\gamma)$ to the expansion one-by-one: \begin{enumerate}
	\item Begin with $f_0(x) = 0$;
	\item for $m=1,\ldots, M$ do: \begin{enumerate}
		\item Given $f_{m-1}$ compute 
		\[
			( \hat{\beta}_m, \gamma_m)
			= \argmin_{\beta,\gamma} \sum_{i=1}^n
				L\bigl( y_i, f_{m-1}(x_i) + \beta b(x_i,\gamma) \bigr)\,;
		\]
		\item Set
		\[ f_m = f_{m-1} + \hat{\beta}_m b(\cdot,\gamma_m)\,, \]
		with previously added terms left \textbf{intact};
	\end{enumerate}
\end{enumerate}
For instance, if base functions are classification trees, then the parameter $\gamma$
stands for the splitting axes, splitting thresholds and class assignments at the
terminal nodes (leaves).

Essentially, AdaBoost is FSAM with exponential loss
\[ L(y,f(x)) = \text{exp}\bigl\{ -y f(x) \bigr\}\,, \]
that seeks an approximation to the population minimizer
\begin{align*}
	f^*(X) &= \argmin_f \ex_{T|X} e^{ - T f(X) } \\
	&= e^{-f(X)} \pr(T=1|X) + e^{f(X)} \pr(T=-1|X)\\
	&= \frac{1}{2} \log\frac{ \pr(T=1|X=x) }{ \pr(T=-1|X=x) } \,.
\end{align*}
The approximation of the global solution to
\[
\sum_{i=1}^n \text{exp}\bigl\{ - y_i \sum_{m=1}^M \beta_m G_m(x_i) \bigr\}
\to \mathop{\text{minimum}}_{ (\beta_m, G_m(\cdot))_{m=1}^M }\,,
\]
where $G_m(x)$ is a weak binary classifier, is given by the additive expansion
\[ f_M(x) = \sum_{m=1}^M \beta_m G_m(x) \,. \]
The final classifier is defined through the discriminant function:
\[
G_M(x)
= \text{sign} \Bigl( \sum_{m=1}^M \beta_m G_m(X) \Bigr) \,,
\]
-- the weighted majority vote, justified by the fact that under the exponential 
loss the population optimal $f(x)$ is the scaled log-odds ratio of class posteriors.\hfill\\

\noindent\textbf{Derivation:}\hfill\\
Consider the FSAM with exponential loss. At each iteration AdaBoost solves the
following problem 
\[
\sum_{i=1}^n w_i^{(m)} \text{exp}\bigl\{ -\beta y_i G(x_i) \bigr\}
\to \min_{\beta,G(\cdot)}\,,
\]
where 
\[ w_i^{(m)} = \text{exp}\bigl\{ -y_i f_{m-1}(x_i) \bigr\}\,. \]
Since $\pm1$ coding scheme is employed one can transform the sum into:
\begin{align*}
	\sum_{i=1}^n w_i^{(m)} \text{exp}\bigl\{ -\beta y_i G(x_i) \bigr\}
	&= \sum_{i=1}^n w_i^{(m)} e^\beta 1_{G(x_i) \neq y_i}
		+ \sum_{i=1}^n w_i^{(m)} e^{-\beta} 1_{G(x_i) = y_i} \\
	&= (e^\beta-e^{-\beta}) \sum_{i=1}^n w_i^{(m)} 1_{G(x_i) \neq y_i}
		+ e^{-\beta} \sum_{i=1}^n w_i^{(m)} \,.
\end{align*}
Since $\beta \geq 0$, the problem of finding the best $G(\cdot)$ reduces to
\[ \sum_{i=1}^n w_i^{(m)} 1_{G(x_i) \neq y_i} \to \min_G\,, \]
i.e. training a weak classifier $G$ on the data weighted by $(w_i^{(m)})_{i=1}^n$.
For instance, for two-class classification trees, the CART algorithm is modified
easily: the Gini impurity index uses
\[
\hat{p}_{mk}
 = \frac{ \sum_{i\,:,x_i\in R_m} w^{(m)}_i 1_{t_i=k} }
 	{ \sum_{i\,:,x_i\in R_m} w^{(m)}_i } \,.
\]

For a given $\hat{G}_m$, which minimizes the weighted misclassification error
\[
\text{err}_m(G)
= \bigl(\sum_{i=1}^n w_i^{(m)}\bigr)^{-1}
	\sum_{i=1}^n w_i^{(m)} 1_{G(x_i) \neq y_i} \,,
\]
the problem of finding the optimal $\beta$ becomes
\[
\hat{\beta}_m = \argmin_\beta
	\bigl( (e^\beta-e^{-\beta}) \text{err}_m(\hat{G}_m) + e^{-\beta} \bigr)
		\sum_{i=1}^n w_i^{(m)} \,.
\]
The first order conditions are
\[ (e^\beta+e^{-\beta}) \text{err}_m(\hat{G}) - e^{-\beta} = 0\,, \]
whence
\[ e^{2\beta} \text{err}_m(\hat{G}) = 1 - \text{err}_m(\hat{G}) \,, \]
and
\[
\hat{\beta}_m = \frac{1}{2} \log \frac{1-\text{err}_m(\hat{G})}{\text{err}_m(\hat{G})}\,.
\]

Given a more or less closed solution to the FSAM step it is interesting to see
how weights $(w_i^{(m)})_{i=1}^n$ change between iterations. After each iteration
\[ f_m = f_{m-1} + \hat{\beta}_m \hat{G}_m(\cdot)\,, \]
whence
\begin{align*}
	w_i^{(m+1)}
	&= \text{exp}\bigl\{ -y_i f_m(x_i) \bigr\} \\
	&= \text{exp}\bigl\{ -y_i f_{m-1}(x_i) \bigr\}
	   \text{exp}\bigl\{ -y_i \hat{\beta}_m \hat{G}_m(x_i) \bigr\}\\
	&= w_i^{(m)}
	   \text{exp}\bigl\{ -y_i \hat{\beta}_m \hat{G}_m(x_i) \bigr\}\,.
\end{align*}
Now, $\pm1$ coding implies that
\[
- y_i \hat{G}_m(x_i)
= 1_{G(x_i) \neq y_i} - 1_{G(x_i) = y_i}
= 2 1_{G(x_i) \neq y_i} - 1\,,
\]
whence
\[ w_i^{(m+1)} = w_i^{(m)} e^{-\hat{\beta}_m} e^{ 2 \hat{\beta}_m 1_{G(x_i) \neq y_i}} \,. \]

To summarize the AdaBoost version of FSAM is
\begin{enumerate}
	\item Initialize the weights to $w_i^{(1)} = \frac{1}{n}$;
	\item For $m=1,\ldots, M$ do: \begin{enumerate}
		\item Fit a binary classifier $\hat{G}_m(x)$ to the weighted training data using
		the weights $w_i^{(m)}$;
		\item Compute the current misclassification error:
		\[
		\text{err}_m
			=\frac{ \sum_{i=1}^n w_i^{(m)} 1_{\hat{G}_m(x_i) \neq y_i}}{\sum_{i=1}^n w_i^{(m)}} \,;
		\]
		\item Calculate 
		\[ \hat{\beta}_m = \frac{1}{2} \log \frac{1-\text{err}_m}{\text{err}_m}\,, \]
		and put $\alpha_m = 2 \hat{\beta}_m$;
		\item Finally update the weights and ensure that they sum to one
		\[ w_i^{(m+1)} \propto w_i^{(m)} e^{ \alpha_m 1_{G_m(x_i)\neq y_i }} \,; \]
	\end{enumerate}
	\item Return the final binary classifier 
		\[ G(x) = \text{sign}\Bigl( \sum_{m=1}^M \beta_m G_m(x) \Bigr) \,. \]
\end{enumerate}

With each iteration, observations difficult to classify correctly receive more
influence, and thus each successive base classifier focuses on observations, which
were handled poorly by previous classifiers in the sequence.

An advantage of the exponential loss is that its sequential minimization leads to
the simple AdaBoost scheme. One drawback, however, is that it penalizes large negative
values much more strongly than cross-entropy.

\clearpage
% section question_10 (end)

\section[Convex optimization]{Question \# 11} % (fold)
\label{sec:question_11}
\textbf{\large \textbf{Q}} : Show that if strong duality holds, then the optimal
solutions to the primal and dual problem must satisfy the KKT conditions (necessary
condition). Show that the KKT conditions are also sufficient when the problem is
convex.

\noindent{\large \textbf{A}} .\hfill\\
Consider a constrained optimization problem in its normal form :
\[ f_0(x) \to \min_x \,, \]
subject to
\[ f_i(x) \leq 0\,\text{ and }\, h_j(x) = 0 \,. \]
The Lagrangian function $\Lcal: D \times \Real_+^m \times \Real^p \to \Real$ is :
\[
\Lcal(x, \lambda, \nu) = f_0(x) + \sum_{i=1}^m \lambda_i f_i(x) + \sum_{j=1}^p \nu_j h_j(x) \,,
\]
where $D \subseteq \Real^n$ is the common domain
\[
D = \bigl( \bigcap_{i=0}^m \text{dom} f_i \bigr) \cap \bigl( \bigcap_{j=1}^p \text{dom} h_j \bigr) \,.
\]
The Lagrange dual function $g : \Real_+^m \times \Real^p \to \Real$ is given by
\[
g(\lambda,\nu) = \inf_x \Lcal(x,\lambda,\nu)\,,
\]
and it is concave in $\lambda$ and $\mu$ ($\inf$ plus affinity of $\Lcal$). It provides
a lower bound on the optimal value $f_0(x^*)$ of the original problem (\textbf{weak duality}):
if $x$ is \textbf{feasible} then
\[ 
g(\lambda,\nu) = \inf_z \Lcal(z,\lambda,\nu)\leq \Lcal(x, \lambda,\nu) \leq f_0(x) \,.
\]
for all $\lambda\geq0$ and $\nu$ (feasibility is important to ensure the right-most
inequality).

The dual problem is finding the ``best'' lower bound $d^* = g(\lambda,\nu)$ for
admissible $\lambda\geq 0$
\[
g(\lambda,\nu) \to \max_{\lambda,\nu}\,,
\]
but there is no guarantee that this lower bound is any good.

\textbf{Strong duality} takes place when the duality gap $f_0(x^*) - d^*\geq 0$ is
strictly $0$, and Slater's Constraint Qualifications for convex optimization ensure
this. The convex problem :
\[ f_0(x) \to \min_x\,\text{st} f_i(x) \leq 0\,\text{and}\, h_j(x) = 0\]
if there exists at least one feasible $x\in \text{int} D$ such that $f_i(x) < 0$ and $h(x) = 0$.

If $p^*>-\infty$ then $d^*>-\infty$ (dual is feasible) and there exist $\lambda^*$
and $\nu^*$ such that $g(\lambda^*, \nu^*) = p^*$.

Suppose that strong duality holds, let $x^*$ be optimal in the primal problem, and
let $(\lambda^*, \nu^*)$ be optimal in the dual. The duality gap is zero, whence
\[
f_0(x^*)
= \inf_z f_0(z) + \sum_{i=1}^m\lambda_i^* f_i(z) + \sum_{j=1}^p\nu_j^* h_i(z)\,.
\]
Since $x^*$ is feasible
\[
f_0(x^*) + \sum_{i=1}^m\lambda_i^* f_i(x^*) + \sum_{j=1}^p\nu_j^* h_i(x^*)
\leq f_0(x^*)\,,
\]
whence for $x^*$, $\lambda^*$ and $\nu^*$
\[
\sum_{i=1}^m \lambda_i^* f_i(x^*) + \sum_{j=1}^p\nu_j^* h_i(x^*) = 0 \,.
\]
Since $ \sum_{j=1}^p\nu_j^* h_i(x^*) = 0$ one has
\[ \sum_{i=1}^m \lambda_i^* f_i(x^*) = 0 \,,\]
which imply the complementary slackness conditions : $\lambda_i^* f_i(x^*) = 0$
for all $i=1,\ldots,m$.

Summary, for any optimization problem if $x^*$ and $(\lambda^*,\nu^*)$ are such that
$f_0(x^*) = g(\lambda^*,\nu^*)$ then $(x^*,\lambda^*,\nu^*)$ satisfy the necessary
\textbf{KKT} conditions.
\textbf{K}arush-\textbf{K}uhn-\textbf{T}ucker conditions:
\begin{itemize}
	\item primal constraints: $f_i(x)\leq 0$ for $i=1,\ldots,m$ and $h_j(x) = 0$ for $j=1,\ldots,p$;
	\item dual constraints: $\lambda_i \geq 0$ to ensure that $g\leq p^*$;
	\item Complementary slackness: $\lambda_i f)i(x) = 0$ for $i=1,\ldots,m$;
	\item Gradient of $\Lcal$ vanishes: $\nabla_x \Lcal(x) = 0$.
	\[\nabla_x \Lcal = \nabla f_0(x) + \sum_{i=1}^m \lambda_i \nabla f_i(x) + \sum_{j=1}^p \nu_j \nabla h_j(x)\]
\end{itemize}

For a convex optimization problem KKT conditions are also sufficient. Indeed, if
a triple $(x^*,\lambda^*,\nu^*)$ satisfies the KKT, then $x^*$ is feasible and
for any feasible $x$
\[
\Lcal(x,\lambda^*,\nu^*)
= f_0(x) + \sum_{i=1}^m\lambda_i^* f_i(x) + \sum_{j=1}^p\nu_j^* h_i(x)
= f_0(x) + \sum_{i=1}^m\lambda_i^* f_i(x)
\]
which is a non-negative weighted $(\lambda_i^*)_{i=1}^m$ sum of convex functions
$f_i$. Thus by the last KKT condition, $x^*$ is the minimizer $\Lcal(x,\lambda^*,\nu^*)$
and $\Lcal(x^*,\lambda^*,\nu^*) \leq g(\lambda^*, \nu^*)$, whence
\[
g(\lambda^*,\nu^*)
= f_0(x^*) + \sum_{i=1}^m \lambda_i^* f_i(x^*) + \sum_{j=1}^p\nu_j^* h_i(x^*)
= f_0(x^*)\,,
\]
due to feasibility and complementary slackness conditions.

\clearpage
% section question_11 (end)

\section[SVM]{Question \# 12} % (fold)
\label{sec:question_12}
\textbf{\large \textbf{Q}} : Derive the maximum margin classifier in the linearly
separable and non-separable case (dual problem, KKT conditions, etc).

\noindent{\large \textbf{A}} .\hfill\\
Consider a standard two-class classification problem on the training data
$(x_i,t_i)_{i=1}^n$ with $t_i\in\{-1,+1\}$. We want to find the best hyperplane
in the sense that the distance to it from the closet data point (\textbf{the margin})
is the largest. Under the $\pm1$ coding the \textbf{SVM} problem is
\[ M \to \max_{\beta, \beta_0}\,, \]
subject to $\|\beta\| = 1$ and
\[t_i(x_i'\beta + \beta_0) \geq M\,,\]
for all $i=1,\ldots,n$. Converting maximization into minimization by $a\mapsto a^{-1}$
and dividing by $M$ the problem equivalently transforms into
\[
\frac{1}{2}\|\beta\|^2 \to \min_{\beta_0,\beta}
\text{ s.t. }
t_i(\beta_0+x_i'\beta)\geq 1\,\forall i=1,\ldots,n\,.
\]
The primal problem's Lagrangian is given by
\[
\Lcal(\beta_0, \beta, \lambda)
= \frac{1}{2} \|\beta\|^2 + \sum_{i=1}^n \lambda_i \bigl( 1 - t_i(x_i'\beta+\beta_0) \bigr)\,,
\]
and the Lagrange dual
\[
g(\lambda) = \inf_{\beta_0,\beta} L(\beta_0, \beta, \lambda)\,.
\]

The objective function is convex and the restrictions are linear, and linear separability
of the data implies that the primal problem is feasible. By Slater's constraint
qualifications \textbf{strong duality} holds. Thus optimal $\beta_0^*$, $\beta^*$
and $\lambda^*$ satisfy the KKT conditions: \begin{itemize}
	\item Primal constraints: $t_i( x_i'\beta^* + \beta^*_0 )\geq1$;
	\item Dual constraints: $\lambda^*_i\geq 0$;
	\item Complementary slackness: $\lambda^*_i \bigl( 1 - t_i(x_i'\beta^*+\beta^*_0) \bigr) = 0$;
	\item The gradient vanishes:
	\[ \frac{\partial \Lcal}{\partial \beta_0} = - \sum_{i=1}^n \lambda_i t_i = 0\,, \]
	and
	\[ \frac{\partial \Lcal}{\partial \beta} = \beta - \sum_{i=1}^n \lambda_i t_i x_i = 0\,. \]
\end{itemize}
Therefore $\beta^*(\lambda) = \sum_{i=1}^n \lambda_i t_i x_i$ and by plugging
$\beta^*(\lambda)$ into $\Lcal(\beta_0, \beta, \lambda)$ the Lagrange dual becomes:
\begin{align*}
	g(\lambda)
	&= \frac{1}{2} \sum_{i=1}^n \sum_{j=1}^n \lambda_j \lambda_i t_j t_i x_i'x_j
		- \sum_{i=1}^n \sum_{j=1}^n \lambda_j \lambda_i t_j t_i x_i' x_j
		+ \sum_{i=1}^n \lambda_i\,,
\end{align*}
Let $H_{ij} = t_j t_i x_i' x_j$, which is the inner product in $\Real^p$ space of
the $i$-th and $j$-th data points. Indeed,
\[ H_{ij} = t_i \langle x_i, x_j \rangle t_j = \langle t_i x_i, t_j x_j \rangle \,.\]
In the matrix form the Lagrange dual is
\[ g(\lambda) = \one' \lambda - \frac{1}{2} \lambda'H\lambda \,, \]
and so the final dual problem becomes
\[ \one' \lambda - \frac{1}{2} \lambda' H \lambda \to \max_\lambda\,,\]
subject to
\[ \lambda \geq 0 \text{ and } \lambda' t = 0\,. \]

For some $i=1,\ldots, n$ consider the complementary slackness condition
\[ \lambda_i \bigl( 1 - t_i ( x_i'\beta^* + \beta^*_0 ) \bigr) = 0 \,. \]
If $\lambda^*_i > 0 $ then $t_i ( x_i'\beta^* + \beta^*_0) = 1$, and such points
necessarily lie on the margin. These are the \textbf{support vectors}. The points
that are far away from the margin, i.e. $t_i ( x_i'\beta^* + \beta^*_0) > 0$ must
have $\lambda^*_i = 0$. The fact that only for a few $i$, $\lambda^*_i>0$ gives
the so called sparse solution.

The support vectors give the optimal value of $\beta^*_0$. Let the set of all support
vectors be
\[ \text{SV} = \{i=1,\ldots,n| \lambda^*_i>0\} \,. \]
\textbf{hopefully it is non-empty}. For all $i\in \text{SV}$, one has
$x_i'\beta^* + \beta^*_0 = t_i $, as $t_i = \pm 1$, whence
\[ 
x_i'\beta^* + \beta^*_0 - t_i 
= \sum_{j=1}^n \lambda^*_j t_j x_i'x_j + \beta^*_0 - t_i \,.
\]
Therefore
\[ \beta^*_0 = t_i - \sum_{j:\lambda^*_i>0} \lambda^*_j t_j x_i'x_j \,. \]
For numerical stability average this expression across all vectors in $\text{SV}$.

\subsection*{non-separable case} % (fold)
\label{sub:non_separable_case}

In order to tackle the non-separable case, permit certain slackness in the classification
for extra flexibility. Let $\xi_i \geq 0$ be the classification slack variable for a data
point $i$. The idea is that each $\xi_i$ can be seen as a penalty for misclassification:
one expects \begin{itemize}
	\item $\xi_i = 0$ for correctly classified points outside of the margin;
	\item $\xi_i\in(0,1]$ for correctly classified points within the margin;
	\item $\xi_i = 1$ for the points on exactly on the separating hyperplane;
	\item $\xi_k > 1$ for data points that are misclassified.
\end{itemize}

As in the separable case, this ``relaxed'' problem of maximizing the margin becomes
\[ \frac{1}{2} \|\beta\|^2 + C \sum_{i=1}^n \xi_i \to \min_{\beta_0,\beta, \xi}\,, \]
subject to classification constraints with some slack :
\[ t_i( \beta_0 + x_i'\beta ) \geq 1 - \xi_i \,, \]
and $\xi_i\geq 0$ for $i=1,\ldots,n$. The larger the $C>0$, the narrower is the optimal
margin. The Lagrangian of this primal problem is given by
\begin{align*}
	\Lcal(\beta_0,\beta,\xi, \lambda,\mu)
	&= \frac{1}{2} \|\beta\|^2 + C \sum_{i=1}^n \xi_i \\
	&+ \sum_{i=1}^n \lambda_i \Bigl( 1 - \xi_i - t_i ( x_i'\beta + \beta_0 ) \Bigr)
	+ \sum_{i=1}^n \mu_i ( - \xi_i )\,.
\end{align*}
The KKT conditions are \begin{itemize}
	\item $t_i \bigl( \beta_0 + x_i'\beta \bigr) \geq 1 - \xi_i$ and $\xi_i \geq 0$;
	\item $\lambda_i,\mu_i \geq 0$;
	\item $\lambda_i \Bigl( 1 - \xi_i - t_i ( x_i'\beta + \beta_0 ) \Bigr) = 0$
		and $\mu_i ( - \xi_i ) = 0$;
	\item \begin{description}
		\item[$\beta_0$ :] $ -\sum_{i=1}^n \lambda_i t_i = 0$;
		\item[$\beta$ :] $ \beta - \sum_{i=1}^n \lambda_i t_i x_i = 0$;
		\item[$\xi$ :] $ C - \lambda_i - \mu_i = 0$.
	\end{description}
\end{itemize}
For $\beta_0$, $\beta$, $\xi$, $\lambda$ and $\mu$ satisfying the KKT,
\[ \beta = \sum_{i=1}^n \lambda_i t_i x_i\,, \]
and the Lagrange dual function (after minor rearrangement) is
\begin{align*}
	g(\lambda,\mu)
	& = \frac{1}{2} \sum_{i=1}^n \sum_{j=1}^n \lambda_i \lambda_j t_i t_j x_i'x_j
		+ \sum_{i=1}^n \xi_i ( C - \lambda_i - \mu_i ) \\
	& - \sum_{i=1}^n \sum_{j=1}^n \lambda_i \lambda_j t_i t_j x_i'x_j
		- \beta_0 \sum_{i=1}^n \lambda_i t_i
		+ \sum_{i=1}^n \lambda_i \,.
\end{align*}
Exploiting the vanishing of the gradient and the dual feasibility conditions, the
function reduces to
\[
g(\lambda,\mu) =
- \frac{1}{2} \sum_{i=1}^n\sum_{j=1}^n \lambda_i\lambda_j t_i t_j x_i'x_j
+ \sum_{i=1}^n \lambda_i \,.
\]
Therefore the dual problem is
\[ \one' \lambda - \frac{1}{2}\lambda'H\lambda \to \max_\lambda \]
subject to $\lambda't = 0$ and $0\leq \lambda\leq C$ -- the box constraint.

Characterization of a solution:
\begin{description}
	\item[$\lambda^*_i = 0$] Then $\mu^*_i = C$, $\xi^*_i = 0$ and $t_i ( \beta^*_0 + x_i'\beta^* ) \geq 1 - 0$;
	\item[$\lambda^*_i > 0$] Then $t_i ( \beta^*_0 + x_i'\beta^* ) = 1 - \xi^*_i$ and $\xi^*_i \geq 0$.
		When $\lambda^*_i < C$, then $\mu^*_i > 0$, whence $\xi^*_i = 0$ and $t_i ( \beta^*_0 + x_i'\beta^* ) = 1$;
	\item[$\lambda^*_i = C$] Then $t_i ( \beta^*_0 + x_i'\beta^* ) = 1 - \xi^*_i$,
	$\mu^*_i=0$ and the point is either inside the margin and correctly classified
	$\xi^*_i \in [0,1]$, or the point is misclassified and $\xi^*_i > 1$.
\end{description}
To get $\beta_0$ compute 
\[ \beta^*_0 = t_i - \sum_{j\in\text{SVE}} \lambda^*_j t_j x_i'x_j \]
where $\text{SVE} = \{i: \lambda^*_i > 0 \text{ and } \xi^*_i = 0 \}$.

% Is there a guarantee that there necessarily exists $i$ with $\xi^* = 0$ and
% $\lambda^*_i>0$? \textbf{Who knows}.

% subsection* non_separable_case (end)

\clearpage
% section question_12 (end)

\section[RKHS]{Question \# 13} % (fold)
\label{sec:question_13}
\textbf{\large \textbf{Q}} : Show that the function K is a kernel if and only if
it is a symmetric positive definite function.

\noindent{\large \textbf{A}} .\hfill\\
A pair $(\Hcal, \langle\cdot,\cdot\rangle)$ is a Hilbert space, if  $\Hcal$ is a
vector space over the field $\Real$, $\langle\cdot,\cdot\rangle$ is an inner product
on $\Hcal$, and $\Hcal$ is complete with respect to the inner product's norm $\|\cdot\|$.

Let $X$ be some set, the nature of which is irrelevant. A function $K:X\times X\to \Real$
is called a \textbf{kernel} if there exists a Hilbert space $(\Hcal, \langle\cdot,\cdot\rangle)$
and a mapping $\Phi : X \to \Hcal$ such that for all $x,y\in X$
\[ K(x,y) = \bigl\langle \Phi(x), \Phi(y) \bigr\rangle \,. \]
The space $\Hcal$ is a new feature space and the map $\Phi$ embeds the original
feature set $X$ in $\Hcal$.

A function $K : X\times X \to \Real$ is symmetric and positive-definite if \begin{itemize}
	\item $K(x,y) = K(y,x)$ for all $x,y\in X$;
	\item for any $n\geq1$ and $(x_i)_{i=1}^n\in X$ the matrix
	\[ K_{xx} = \begin{pmatrix}
			K(x_1,x_1) & K(x_1,x_2) & \cdots & K(x_1,x_1) \\
			K(x_2,x_1) & K(x_2,x_2) & \cdots & K(x_2,x_2) \\
			\vdots & \vdots & \ddots & \vdots \\
			K(x_n,x_1) & K(x_n,x_2) & \cdots & K(x_n,x_n)
		\end{pmatrix} \]
	is positive-definite. In other words
	\[ \sum_{i=1}^n \sum_{j=1}^n \lambda_i K(x_i,x_j)\lambda_j \geq 0\,, \]
	for any $\lambda = (\lambda_i)_{i=1}^n \in \Real$.
\end{itemize}
If $K : X\times X \to \Real$ is symmetric and positive-definite, then
\[ K_{xy} = \begin{pmatrix}
	K(x,x) & K(x,y)\\
	K(y,x) & K(y,y)
\end{pmatrix} \]
must be positive definite. This means that $|K_{xy}|\geq 0$, whence follows an
analogue of Cauchy-Schwartz-Bunyakovsky inequality :
\[ K(x,y)^2 \leq K(x,x) K(y,y)\,. \]

\noindent\textbf{Theorem}\hfill\\
A function $K : X\times X \to \Real$ is a kernel if and only if it is symmetric and
positive-definite.

\noindent\textbf{$\Rightarrow$} If $K$ is a kernel, then there exists a Hilbert space
$(\Hcal, \langle\cdot,\cdot\rangle)$ and a map $\Phi:X\to \Hcal$ such that
\[ K(x,y) = \langle \Phi(x), \Phi(y)\rangle\,. \]
First, symmetry of the inner product implies symmetry of kernel $K$. Next, let 
$n\geq1$, $(x_i)_{i=1}^n\in X$ and $(\lambda_i)_{i=1}^n \in \Real$. Then by bilinearity
of the inner product
\begin{align*}
	\sum_{i=1}^n \sum_{j=1}^n \lambda_i K(x_i,x_j)\lambda_j
	&= \sum_{i=1}^n \sum_{j=1}^n \lambda_i \langle \Phi(x_i), \Phi(x_j)\rangle \lambda_j
	= \langle \sum_{i=1}^n \lambda_i \Phi(x_i), \sum_{j=1}^n \Phi(x_j) \lambda_j\rangle\\
	&= \| \sum_{i=1}^n \lambda_i \Phi(x_i) \lambda_j \|^2
	\geq 0\,,
\end{align*}
and $K$ is positive-definite.

\noindent \textbf{ $\Leftarrow$ : building the space}\hfill\\
Consider a symmetric and positive-definite function $K : X \times X \to \Real$.
Define
\[
\Hcal_0 = \bigl\{ \sum_{i=1}^n \alpha_i K(\cdot, x_i)\,\big|\,
n\geq 1,\, (x_i)_{i=1}^n\in X\, \text{ and }\, (\alpha_i)_{i=1}^n\in \Real \bigr\}\,.
\]
Being a set of linear combinations this set is a linear space over $\Real$.

Let $f,g\in \Hcal_0$, with representations $f = \sum_{i=1}^n \alpha_i K(\cdot, x_i)$ and
$g = \sum_{j=1}^m \beta_j K(\cdot, y_j)$. Define
\[ \langle f, g \rangle_0 = \sum_{i=1}^n \sum_{j=1}^m \alpha_i \beta_j K(x_i, y_j)\,. \]
For another representation of $f$, given by $f = \sum_{l=1}^q \xi_l K(\cdot, z_l)$:
\begin{align*}
	\sum_{l=1}^q \sum_{j=1}^m \xi_i \beta_j K(z_l, y_j)
	&= \sum_{j=1}^m \beta_j \sum_{l=1}^q \xi_i K(y_j, z_l) 
	 = \sum_{j=1}^m \beta_j f(y_j) \\
	&= \sum_{j=1}^m \beta_j \sum_{i=1}^n \alpha_i K(y_j, x_i) 
	 = \sum_{i=1}^n \sum_{j=1}^m \alpha_i \beta_j K(x_i, y_j) \,,
\end{align*}
which means the inner product $\langle \cdot,\cdot \rangle_0$ is well-defined.
Let $\|f\|_0 = \sqrt{\langle f, f\rangle_0}$ for any $f\in \Hcal_0$.

For any $x,y\in X$ the functions $K(\cdot,x), K(\cdot,y)$ are in $\Hcal_0$, whence
by construction of $\langle\cdot,\cdot\rangle_0$
\[ \langle K(\cdot,x), K(\cdot,y) \rangle_0 = K(x,y) \,.\]
This means that with respect to this ``inner product'', ``kernel'' $K$ reproduces itself.
As a corollary 
\[ \| K(\cdot,x) \|_0 = \sqrt{ K(x,x) } \,.\]

The symmetry of $K$ implies symmetry of $\langle \cdot,\cdot \rangle_0$, and
bilinearity follows by construction. For any $f\in \Hcal_0$
\[ \langle f, f \rangle_0 = \sum_{i=1}^n \sum_{j=1}^n \alpha_i \alpha_j K(x_i, x_j) \geq 0\,, \]
since $K$ being positive-definite. Furthermore for any function $f\in \Hcal_0$
and $x\in X$ one has
\begin{align*}
	f(x) &= \sum_{i=1}^n \alpha_i K(x, y_i)
	 = \sum_{i=1}^n \alpha_i \langle K(\cdot,x), K(\cdot,y_i) \rangle_0 \\
	&= \bigl\langle \sum_{i=1}^n \alpha_i K(\cdot,y_i), K(\cdot,x) \bigr\rangle_0 
	 = \langle f, K(\cdot,x) \rangle_0 \,,
\end{align*}
which is the so-called \textbf{reproducing} property of $K$ with respect to
$\langle \cdot,\cdot \rangle_0$. A particularly interesting corollary is that
for any $f\in \Hcal_0$
\[
\langle f, f \rangle_0
= \bigl\langle f, \sum_{i=1}^p \alpha_i K(\cdot, x_i) \bigr\rangle_0
= \sum_{i=1}^p \alpha_i \langle f, K(\cdot, x_i) \rangle_0
= \sum_{i=1}^p \alpha_i f(x_i)\,,
\]
which actually follows from the construction of the inner product. Furthermore,
if $f\in \Hcal_0$ is such that $\langle f, f\rangle_0 = 0$, then for any $x\in X$
\[
|f(x)|
= \bigl| \langle f, K(\cdot,x) \rangle_0 \bigr|
\leq \sqrt{ \langle f,f\rangle_0 } \|K(\cdot,x)\|_0
= 0 \,,
\]
whence $f(\cdot) = \nil$, where $\nil\in \Hcal_0$ is the zero function. The converse
implication, that for $f=\nil$ $\langle f,f\rangle_0 = 0$ is trivial.

By this point the space $(\Hcal_0,\langle\cdot,\cdot\rangle_0)$ is just an inner
product space, but with a very important and fruitful property of reproducibility
of the ``kernel'' $K$.

\noindent\textbf{Lemma \# 1 : } \hfill \\
Reproducibility implies that for any $f,g\in\Hcal_0$ and $x\in X$
\[
| f(x) - g(x) |
\leq |\langle f-g,K(\cdot,x)\rangle_0|
\leq \|f-g\|_0 \sqrt{ K(x,x) }\,,
\]
which means that the evaluation functional $\delta_x:\Hcal_0\to\Real$ given by
\[ \delta_x f = f(x) \,,\]
is bounded. Since it is linear by definition, it must be also continuous in an inner
product space $(\Hcal_0,\langle\cdot,\cdot\rangle_0)$.

\noindent \textbf{ $\Leftarrow$ : Cauchy sequences in $\Hcal_0$}\hfill\\
A sequence $(f_n)_{n\geq1} \in \Hcal_0$ is Cauchy if for all $\epsilon>0$ there
is $N\geq1$ such that
\[ \|f_n - f_m\|_0 < \epsilon \,, \]
for all $n,m\geq N$. Main properties include: \begin{itemize}
	\item If $(f_n)_{n\geq1}\in \Hcal_0$ is Cauchy, then it is bounded in $\Hcal_0$.
		Indeed, by definition there is $N\geq 1$ such that $ \|f_n - f_N\|_0 < 1 $ for
		all $n\geq N$, whence $\|f_n\|_0 \leq \|f_N\|_0 + 1$ by the triangle inequality.
		Letting $M_N = \max_{i=1\ldots,N-1}\|f_i\|$ and setting $M = \max\{M_N, \|F_N\|+1\}$
		yields an upper bound on $\|f_n\|_0$ for all $n\geq1$.
	\item Two Cauchy sequences $(f_n)_{n\geq1},(g_n)_{n\geq1}\in \Hcal_0$ are ``equivalent'',
		$f_n \sim g_n$ if for any $\epsilon>0$ there is $N\geq 1$ such that
		\[ \| f_n - g_n \|_0 < \epsilon\,, \]
		for all $n\geq N$. This relation, is symmetric, reflexive and transitive,
		which can be shown via a standard $2\epsilon$ argument.
	\item Any stationary sequence $(g)_{n\geq1}\in \Hcal_0$ is Cauchy. Any Cauchy
		sequence $(g_n)_{n\geq1}\in \Hcal_0$ with $g_n\sim g$, must converge
		pointwise to $g$.
\end{itemize}

\noindent\textbf{Lemma \# 2} : if $(f_n)_{n\geq1}\in \Hcal_0$ is a Cauchy sequence,
then $\|f_n\|_0 \to 0$ if and only if $f_n\to\nil$ pointwise.

\noindent\textbf{Proof: $\Rightarrow$} \hfill\\
Indeed, if $\|f_n\|_0 \to 0$, then for any $x\in X$
\[ |f_n(x)| \leq \|f_n\|_0 \sqrt{ K(x,x) }\,, \]
whence $f_n$ converges pointwise to $\nil$ in $\Real$.

\noindent\textbf{Proof: $\Leftarrow$} \hfill\\
Conversely, let $(f_n)_{n\geq1}\in \Hcal_0$ be Cauchy. Then \begin{itemize}
	\item there is $M>0$ such that $\|f_n\|_0\leq M$ for all $n\geq 1$;
	\item for any $\epsilon>0$ there exists $N_0\geq1$ with
		\[ \|f_n - f_{N_0} \|_0 < \frac{\epsilon}{2M}\,, \]
		for all $n\geq N_0$.
\end{itemize}
Now $f_{N_0} = \sum_{j=1}^p \beta_j K(\cdot,y_j)$ and by construction
\[ \langle f_{N_0}, f_n \rangle_0 = \sum_{j=1}^p \beta_j f_n(y_j)\,. \]
Since $f_n \to \nil$ pointwise for all $j=1,\ldots,p$ there exists $N_j\geq 1$
such that and $n\geq N_j$
\[ |f_n(y_j)| \leq \frac{\epsilon}{2 p ( |\beta_j| + 1 ) }\,. \]
Therefore for all $n \geq N = \max_{j=0,1,\ldots,p}N_j$
\begin{align*}
	\|f_n\|_0^2
	& \leq |\langle f_n-f_{N_0},f_n \rangle_0| + |\langle f_{N_0}, f_n \rangle_0| \\
	& \leq \|f_n-f_{N_0}\|_0 \|f_n\|_0 + \sum_{j=1}^p |\beta_j| |f_n(y_j)| \\
	& < \frac{\epsilon}{2M} M + \sum_{j=1}^p |\beta_j| \frac{\epsilon}{2 p ( |\beta_j| + 1 ) }\\
	& < \frac{\epsilon}{2} + p\frac{\epsilon}{2p}\,.
\end{align*}

\noindent \textbf{ $\Leftarrow$ : Completing $\Hcal_0$}\hfill\\
Let $(f_n)_{n\geq1}\in \Hcal_0$ be a Cauchy sequence. For every $x\in X$ lemma \#1
implies that $(f_n(x))_{n\geq1}\in\Real$ is Cauchy, whence there exists $f(x)\in \Real$
by completeness of $\Real$ such that $f_n\to f$ pointwise.

Completing the $\Hcal_0$ is performed by populating $\Hcal$ with limiting points
of all Cauchy sequences in $\Hcal_0$. Since stationary sequences are Cauchy, this
means that $\Hcal_0\subseteq \Hcal$. Furthermore, this $\Hcal$ a linear space due to
linearity of pointwise convergence.

Next, extend the inner product $\langle\cdot,\cdot\rangle_0$ to $\Hcal$ in the following
way: any $f,g\in \Hcal$ are limits of some Cauchy sequences $(f_n)_{n\geq1}$ and
$(g_n)_{n\geq1}$ in $\Hcal_0$, thus put
\[
\langle f, g \rangle = \lim_{n\to \infty} \langle f_n, g_n \rangle_0\,.
\]
This is indeed an extension, since any $f, g \in \Hcal_0$ are pointwise limits of
stationary sequences of themselves : $f_n=f$ and $g_n=g$ in $\Hcal_0$, -- whence
\[
\langle f, g\rangle
= \lim_{n\to \infty} \langle f_n, g_n \rangle_0
= \langle f, g \rangle_0\,.
\]
For any $f\in \Hcal$ put $\|f\| = \sqrt{ \langle f,f\rangle}$.

\noindent\textbf{Lemma \# 3} : This extension is well defined.

\noindent\textbf{Proof: existence} \hfill\\
Let $(f_n)_{n\geq1},(g_n)_{n\geq1}\in \Hcal_0$ be Cauchy. Since every Cauchy sequence
is bounded there necessarily exists $M\geq 0$ such that $\|f_n\|_0,\|g_n\|_0 \leq M$
for all $n\geq1$. Now, by the triangle and kernel Cauchy-Schwartz-Bunyakovsky inequalities
\begin{align*}
\bigl| \langle f_n, g_n\rangle_0 - \langle f_m, g_m\rangle_0 \bigr|
&\leq \|f_n\|_0\|f_n-f_m\|_0+\|g_m\|_0\|g_n-g_m\|_0\\
&\leq M ( \|f_n-f_m\|_0+\|g_n-g_m\|_0)\,.
\end{align*}
The fact that both sequences are Cauchy, implies that there is $N\geq1$ such that
for all $n, m\geq N$ one has
\[ \|f_n-f_m\|_0,\|g_n-g_m\|_0 < \frac{\epsilon}{2(M+1)}\,. \]
Hence $(\langle f_n, g_n\rangle_0)_{n\geq1}$ is Cauchy in $\Real$, and thus indeed
converges.

\noindent\textbf{Proof: independence} \hfill \\
Let $f,h\in \Hcal$. Then there exist $(f_n)_{n\geq1},(g_n)_{n\geq1}\in \Hcal_0$
such that $f_n\to f$ and $g_n \to g$ pointwise. If $(f_n')_{n\geq1}$ and $(g_n')_{n\geq1}$
are another Cauchy sequences in $\Hcal_0$ with $f_n' \to f$ and $g_n' \to g$ pointwise,
then by lemma \#2 $\|f_n-f_n'\|_0\to 0$ and $\|g_n-g_n'\|_0\to 0$. Since Cauchy sequences
in $\Hcal_0$ are bounded, there is $M\geq0$ such that $\|f_n'\|_0,\|g_n\|_0\leq M$
for all $n\geq1$. Therefore
\begin{align*}
\bigl| \langle f_n, g_n\rangle_0 - \langle f_n', g_n'\rangle_0 \bigr|
&\leq \|f_n\|_0\|f_n-f_n'\|_0+\|g_n'\|_0\|g_n-g_n'\|_0\\
&\leq M ( \|f_n-f_n'\|_0+\|g_n-g_n'\|_0)\,,
\end{align*}
whence
\[
\lim_{n\to \infty} \langle f_n, g_n \rangle_0
= \lim_{n\to \infty} \langle f_n', g_n' \rangle_0\,.
\]

\noindent\textbf{Lemma \# 4} : This extension is an inner product.

\noindent\textbf{Proof} \hfill \\
It is symmetric and non-negative by being constructed from $\langle\cdot,\cdot\rangle_0$.
If $f\in\Hcal$ then there is $(f_n)_{n\geq1}\in \Hcal_0$ with $f_n\to f$ pointwise.
If $\langle f,f\rangle = 0$, then $\|f_n\|_0\to 0$ and $f_n \to \nil$ pointwise by lemma \# 2,
whence $f = \nil$. Conversely, $\nil\in\Hcal_0$ implies that
\[ \langle \nil,\nil \rangle = \langle \nil,\nil \rangle_0 = 0 \,. \]
This extension is bilinear due to the fact that any linear combination of Cauchy sequences is
a Cauchy sequence and the pointwise limit of a linear combination is a linear
combination of limits. Thus
\begin{align*}
	\langle \alpha f + g, h \rangle
	&= \lim_{n\to\infty}\langle \alpha f_n + g_n, h_n \rangle_0\\
	&= \alpha \lim_{n\to\infty}\langle f_n, h_n \rangle_0
		+ \lim_{n\to\infty}\langle g_n, h_n \rangle_0\\
	&= \alpha \langle f, h \rangle + \langle g, h \rangle\,.
\end{align*}

\noindent\textbf{Lemma \# 5} : This extension preserves reproducibility of $K$.

\noindent\textbf{Proof} \hfill \\
If $f\in \Hcal$, then it is a pointwise limit of some Cauchy sequence $(f_n)_{n\geq1}$
in $\Hcal_0$. Then for any $x\in X$, $K(\cdot, x)\in \Hcal_0 \subseteq \Hcal$ and
\[
\langle f, K(\cdot, x)\rangle
= \lim_{n\to\infty} \langle f_n, K(\cdot, x)\rangle_0
= \lim_{n\to\infty} f_n(x)
= f(x)\,.
\]

\noindent\textbf{Lemma \# 6} : If $(f_n)_{n\geq1}$ is Cauchy in $\Hcal_0$ and $f_n\to f$
pointwise for some $f:X\to \Real$, then $f_n\to f$ in $\Hcal$, i.e. $\|f_n-f\|\to 0$.

\noindent\textbf{Proof} \hfill \\
Indeed, let $\epsilon>0$. Since $\|f_n-f_m\|_0 \to \|f_n-f\|$ by construction as
$m\to \infty$, there is $N_1\geq1$ such that
\[ \|f_n-f\| < \|f_n-f_m\|_0 + \frac{\epsilon}{2}\,, \]
for all $n\geq N_1$. Furthermore, since $(f_n)_{n\geq1}$ is Cauchy in $\Hcal_0$,
there is $N_2\geq1$ with $\|f_n-f_m\|_0 < \frac{\epsilon}{2}$ for all $n,m\geq N_2$.
Therefore for any $n,m\geq N_1\wedge N_2$
\[ \|f_n-f\| < \|f_n-f_m\|_0 + \frac{\epsilon}{2} < \epsilon \,, \]
whence $\|f_n-f\| \to 0$.

\noindent\textbf{Lemma \# 7} : The set $\Hcal_0$ is dense in $(\Hcal,\langle\cdot,\cdot\rangle)$.

\noindent\textbf{Proof} \hfill \\
If $f\in\Hcal$, then by definition it is a limiting point of some Cauchy sequence
$(f_n)_{n\geq1}\in\Hcal_0$. Then $\|f_n-f\|\to 0$ in $\Hcal$ by lemma \#6, whence
for any $\epsilon>0$ there is $N\geq1$ such that $\|f_N-f\|<\epsilon$. Thus $\Hcal_0$
is dense in $\Hcal$ with respect to the norm $\|\cdot\|$.

\noindent\textbf{Lemma \# 8} : The space $(\Hcal,\langle\cdot,\cdot\rangle)$ is complete.

\noindent\textbf{Proof} \hfill \\
If $(f_n)_{n\geq1}\in\Hcal$ is Cauchy, then by lemma \#7 there exists $(g_n)_{n\geq1}\in \Hcal_0$
such that for all $n\geq1$
\[ \|f_n - g_n\| < \frac{1}{2^n} \,, \]
which means that $\| f_n-g_n\|\to 0$.
Now for any $\epsilon>0$ there exists $N\geq1$ such that for all $n,m\geq N$
\[
\| g_m - g_n \|
\leq \| g_m - f_m \| + \| f_m - f_n \| + \| f_n - g_n \|
< \frac{\epsilon}{3} + \frac{\epsilon}{3} + \frac{\epsilon}{3} \,,
\]
whence $(g_n)_{n\geq1}\in \Hcal_0$ is Cauchy. Therefore by lemma \# 1 there exists
a function $g : X\to \Real$ such that $g_n\to g$ pointwise. Thus $g\in \Hcal$ and
\[ \|f_n - g\| \leq \|f_n - g_n\| + \|g_n - g\|\,, \]
implying that $\|f_n - g\| \to 0$. Finally, by the reproducing property of $K$
in $\Hcal$ (lemma \# 5) for any $x\in X$
\[ |f_n(x) - g(x) | \leq \sqrt{K(x,x)} \|f_n - g\|\,, \]
whence $f_n\to g$ pointwise.

Thus the pair $(\Hcal,\langle\cdot,\cdot\rangle)$ is a Hilbert space with \begin{itemize}
	\item $f(x) = \langle f, K(\cdot, x)\rangle$ for all $x\in X$ and all $f\in\Hcal$;
	\item $\Hcal$ is a completion of the space of finite linear combinations of
	partially evaluated kernels $K(\cdot,x)$ for any $x\in X$:
	\[ \bigl\{ \sum_{i=1}^n \alpha_i K(\cdot,x_i)\,:\,n\geq1,\,(\alpha_i)_{i=1}^n\in \Real,\,(x_i)_{i=1}^n\in X \bigr\} \]
\end{itemize}

\noindent \textbf{ $\Leftarrow$ : Feature maps}\hfill\\
The reproducing property of $K$ in $(\Hcal,\langle\cdot,\cdot\rangle)$ hints at
the possible feature maps $\Phi:X\to \Hcal$ such that for all $x,y\in X$
\[ K(x,y) = \langle \Phi(y), \Phi(x)\rangle\,. \]
Indeed, by self-reproducing property of $K$ with respect to the inner product
$\Phi(x) = K(\cdot,x)$ do the trick. Therefore $K$ is indeed a kernel.

\clearpage
% section question_13 (end)

\section[Representer]{Question \# 14} % (fold)
\label{sec:question_14}
\textbf{\large \textbf{Q}} : Prove the representer theorem. Application to kernel
ridge regression.

\noindent{\large \textbf{A}} .\hfill\\
Let $X$ be some non-empty set and $\Hcal$ be a Hilbert space of maps $f:X\to\Real$.
Then $\Hcal$ is \textbf{R}eproducing \textbf{K}ernel \textbf{H}ilbert \textbf{S}pace
with the inner product $\langle\cdot, \cdot\rangle$ and a reproducing kernel $K: X\times X\to \Real$
if \begin{itemize}
	\item for all $x\in X$, $K(\cdot,x) \in \Hcal$ ;
	\item for all $x\in X$ and $f\in \Hcal$, $f(x) = \langle f, K(\cdot, x) \rangle$;
\end{itemize}
Then $K$ is a kernel with feature maps $K(\cdot,x)$ and the associated Hilbert space
$\Hcal$. By the reproducing property the $K$ is unique.

\noindent\textbf{Theorem} : \hfill\\
Let $X$ be a non empty set, $K$ be a kernel on $X\times X$ with the corresponding
RKHS $\Hcal$. Given a training sample $(x_i, t_i)_{i=1}^n \in X \times \Real$, a
strictly increasing function $\Omega:\Real_+\to\Real$ and an arbitrary loss function
$L:( X\times \Real^2)^n \to \Real$, the solution $f^*$ to the problem
\[
L\Bigl( \bigl(x_i,t_i,f(x_i)\bigr)_{i=1}^n \Bigr) + \Omega(\|f\|) \to \min_{f\in \Hcal}\,,
\]
admits the following representation :
\[ f^* = \sum_{i=1}^n \alpha_i K( \cdot, x_i ) \,. \]

\noindent\textbf{Proof} \hfill \\
Consider the subspace of $\Hcal$ spanned by $(K(\cdot, x_i ))_{i=1}^n$ :
\[
\Hcal_* = \bigl\{ \sum_{i=1}^n \gamma_i K(\cdot, x_i ) \,\:\,(\gamma_j)_{j=1}^n \bigr\}\,.
\]
For any $f\in \Hcal$ the projection theorem in Hilbert spaces establishes that there
exists a unique pair $f_\perp \in\Hcal_*^\perp$ and $f_* \in \Hcal_*$ such that
\[ f = f_\perp + f_* \,. \]
For any $i=1,\ldots,n$ reproducing property of the kernel $K$ implies that
\[
f(x_i)
= \langle f_\perp + f_*, K(\cdot, x_i)\rangle
= \langle f_*, K(\cdot, x_i)\rangle\,,
\]
because $K(\cdot, x_i)\in \Hcal_*$. Therefore
\[
f(x_i)
= \bigl\langle \sum_{j=1}^n \gamma_j K(\cdot, x_j ), K(\cdot, x_i)\bigr\rangle 
= \sum_{j=1}^n \gamma_j \langle K(\cdot, x_j ), K(\cdot, x_i)\rangle 
= \sum_{j=1}^n \gamma_j K(x_i, x_j)\,.
\]
Hence, the whole cost function is independent of the orthogonal component of $f$.
Next, by orthogonality and strict monotonicity
\[
\Omega\bigl( \|f\| \bigr)
= \Omega\bigl( \sqrt{\|f_*\|^2 + \|f_\perp\|^2} \bigr)
\geq \Omega\bigl( \|f_*\| \bigr)\,,
\]
which means that ignoring the orthogonal component of $f$ reduces the second term.
Therefore for any $f\in \Hcal$ its projection $f_*$ onto $\Hcal_*$ is such that
\[
L\Bigl( \bigl(x_i,t_i,f(x_i)\bigr)_{i=1}^n \Bigr) + \Omega(\|f\|)
\geq L\Bigl( \bigl(x_i,t_i,f_*(x_i)\bigr)_{i=1}^n \Bigr) + \Omega(\|f_*\|)\,,
\]
whence the solution to the optimization problem cannot have any other form but
\[ f^* = \sum_{i=1}^n \gamma_i K(\cdot, x_i)\,. \]

\noindent\textbf{Kernel ridge regression} : \hfill\\
Consider the problem
\[ \sum_{i=1} (t_i - y(x_i))^2 + \lambda \|y\|^2 \to \min_{y\in \Hcal}\,,\]
for some RKHS $\Hcal$ with kernel $K$. The representer theorem implies that the
solution must have representation
\[ y(x) = \sum_{i=1}^n \alpha_i K(x, x_i)\,. \]
In the matrix notation $y(x) = K_x' \alpha$ for $K_x = (K(x,x_i))_{i=1}^n \in \Real^{n\times 1}$
and $\alpha = (\alpha_i)_{i=1}^n \in \Real^{n\times 1}$. The cost function
becomes	
\[
\sum_{i=1}^n (t_i - y(x_i))^2 = (t-K\alpha)'(t-K\alpha)\,,
\]
for the Gram matrix
\[
K = \begin{pmatrix} K_{x_1}'\\ K_{x_2}'\\ \vdots\\ K_{x_n}'\\ \end{pmatrix}
 = \begin{pmatrix}
		K(x_1,x_1) & K(x_1,x_2) & \cdots & K(x_1,x_1) \\
		K(x_2,x_1) & K(x_2,x_2) & \cdots & K(x_2,x_2) \\
		\vdots & \vdots & \ddots & \vdots \\
		K(x_n,x_1) & K(x_n,x_2) & \cdots & K(x_n,x_n)
	\end{pmatrix}\,.
\]
Due to reproducibility the penalty term reduces to
\begin{align*}
	\lambda \|y\|^2
	&= \lambda \sum_{i=1}^n \sum_{j=1}^n \alpha_i \alpha_j \langle K(\cdot,x_i), K(\cdot,x_j)\rangle\\
	&= \lambda \sum_{i=1}^n \sum_{j=1}^n \alpha_i K(x_i,x_j) \alpha_j
	= \lambda \alpha' K \alpha\,.
\end{align*}
And therefore the problem is equivalent to
\[ (t-K\alpha)'(t-K\alpha) + \lambda \alpha' K \alpha \to \min_\alpha\,,\]
with the first order conditions
\[ 2 K'(t-K\alpha) + 2 \lambda K \alpha = 0 \,, \]
which gives the optimal $\alpha^* = (K+\lambda I)^{-1} t$, since $K$ is invertible.
The prediction of $t$ at a new $x$ is given by
\[ \hat{t}(x) = K_x'\alpha^* = K_x' (K + \lambda I)^{-1} t \,, \]
for $K_x = (K(x,x_i))_{i=1}^n$.

% section question_14 (end)

\end{document}

