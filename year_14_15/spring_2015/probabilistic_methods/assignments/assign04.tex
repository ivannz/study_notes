\documentclass[a4paper]{article}
\usepackage[utf8]{inputenc}
%\usepackage{fullpage}
\usepackage{graphicx, url}

\usepackage{amsmath, amsfonts}
\usepackage{mathtools}

\newcommand{\Real}{\mathbb{R}}
\newcommand{\Cplx}{\mathbb{C}}

\newcommand{\Ccal}{\mathcal{C}}
\newcommand{\Ncal}{\mathcal{N}}
\newcommand{\Dcal}{\mathcal{D}}
\newcommand{\Ical}{\mathcal{I}}

\newcommand{\ex}{{\mathbb{E}}}
\newcommand{\pr}{{\mathbb{P}}}
\newcommand{\var}{{\text{var}}}

\newcommand{\defn}{\overset{\Delta}{=}}

\newcommand{\re}{\operatorname{Re}\nolimits}
\newcommand{\im}{\operatorname{Im}\nolimits}

\usepackage[english, russian]{babel}
\newcommand{\eng}[1]{\foreignlanguage{english}{#1}}
\newcommand{\rus}[1]{\foreignlanguage{russian}{#1}}

\title{\rus{Домашняя работа по курсу \\ ``Теоретико-вероятностные методы в статистике''}}
\author{\rus{Назаров Иван,} \rus{101мНОД(ИССА)}}

\begin{document}
\selectlanguage{russian}
\maketitle

\section{Задание \# 2} % (fold)
\label{sec:task_2}

Найдём расстояние Кульбака-Лейблера между плотностями $p_1$ и $p_2$, где
\[
p_i(x) = (2\pi)^{-\tfrac{1}{2}} (\sigma_i^2)^{-\tfrac{1}{2}}
\text{exp}\bigl\{ -\frac{1}{2\sigma_i^2} (x-\mu_i)^2 \bigr\}\,.
\]
Рассчитаем сначала
\begin{align*}
	\log\frac{p_1(X)}{p_2(X)}
	&= -\frac{1}{2} \log 2\pi - \frac{1}{2} \log \sigma_1^2 - \frac{1}{2\sigma_1^2} (x-\mu_1)^2 \\
	&-\bigl( -\frac{1}{2} \log 2\pi - \frac{1}{2} \log \sigma_2^2 - \frac{1}{2\sigma_2^2} (x-\mu_2)^2 \bigr)\\
	&= - \frac{1}{2} \Bigl( \log \frac{\sigma_1^2}{\sigma_2^2}
		+ \frac{1}{2\sigma_1^2} (x-\mu_1)^2
		- \frac{1}{2\sigma_2^2} (x-\mu_2)^2 \Bigr )\,.
\end{align*}
Итак
\[ \text{KL}(p_1\|p_2) = \ex_{x\sim p_1} \log\frac{p_1(x)}{p_2(x)}\,, \]
откуда
\begin{align*}
	\text{KL}(p_1\|p_2)
	&= - \frac{1}{2} \Bigl( \log \frac{\sigma_1^2}{\sigma_2^2} 
 		+ \frac{1}{\sigma_1^2} \ex_{x\sim p_1} (x-\mu_1)^2
		- \frac{1}{\sigma_2^2} \ex_{x\sim p_1} (x-\mu_2)^2
	  \Bigr )\\
	&= - \frac{1}{2} \Bigl( \log \frac{\sigma_1^2}{\sigma_2^2} 
 		+ \frac{\sigma_1^2}{\sigma_1^2}
		- \frac{1}{\sigma_2^2} \ex_{x\sim p_1} \bigl[ (x-\mu_1)^2 + 2(x-\mu_1)(\mu_1-\mu_2) + (\mu_1-\mu_2)^2 \bigr]
	  \Bigr )\\
	&= - \frac{1}{2} \Bigl( 1 + \log \frac{\sigma_1^2}{\sigma_2^2} 
 		- \frac{\sigma_1^2}{\sigma_2^2} - \frac{(\mu_1-\mu_2)^2}{\sigma_2^2} \Bigr )\,,
\end{align*}
и аналогично:
\[
\text{KL}(p_2\|p_1)
= - \frac{1}{2} \Bigl( 1 + \log \frac{\sigma_2^2}{\sigma_1^2} 
- \frac{\sigma_2^2}{\sigma_1^2} - \frac{(\mu_2-\mu_1)^2}{\sigma_1^2} \Bigr ) \,.
\]

% section task_2 (end)

\section{Задание \# 3} % (fold)
\label{sec:task_3}

Пусть на отрезке $[0,1]$ имеется процесс $\{Y(t)\}$ заданный
\[ Y(t) = \theta S(t) + \xi(t)\,, \]
где $\theta\in\{-1,+1\}$ с равными вероятностями, $S(t)$ известная функция и
$\xi(t)\sim \text{WN}$.

Пусть в моменты времени $(t_k)_{k=1}^n$ наблюдается выборка значений $y_k = Y(t_k)$
(вектор $y$) с соответствующими $s_k = S(t_k)$ (вектор $s$), причём $\|s\| > 0$.
Правдоподобие наблюдений равно:
\[ p(Y^n;\theta) = (2\pi)^{-\frac{n}{2}} \text{exp}\bigl\{ -\frac{1}{2} \sum_{k=1}^n (y_k - \theta s_k)^2\bigr\}\,. \]

Проверяем гипотезы $H_0:\theta\in\Theta_0 = \{-1\}$ против $H_1:\theta\in \Theta_1 = \{+1\}$.
Тогда
\[
\int_{\Theta_0} p(Y^n;\theta) \pi(\theta) d\theta
= \frac{1}{2} (2\pi)^{-\frac{n}{2}} \text{exp}\bigl\{ -\frac{1}{2} \sum_{k=1}^n (y_k + s_k)^2 \bigr\} \,,
\]
и
\[
\int_{\Theta_1} p(Y^n;\theta) \pi(\theta) d\theta
= \frac{1}{2} (2\pi)^{-\frac{n}{2}} \text{exp}\bigl\{ -\frac{1}{2} \sum_{k=1}^n (y_k - s_k)^2 \bigr\} \,.
\]

Оптимальный Байесовский тест задаётся функцией
\[
\phi(Y^n)
= \mathbf{1}\biggl\{ \int_{\Theta_1} p(Y^n;\theta) \pi(\theta) d\theta
\geq \int_{\Theta_0} p(Y^n;\theta) \pi(\theta) d\theta \biggr\} \,,
\]
что в нашем случае выражается в
\[
\phi(Y^n)
= \mathbf{1} \bigl\{ \sum_{k=1}^n (y_k + s_k)^2 \geq \sum_{k=1}^n (y_k - s_k)^2 \bigr\} \,,
\]
которое вырождается в 
\[
\phi(Y^n)
= \mathbf{1} \bigl\{ \sum_{k=1}^n y_k s_k \geq 0 \bigr\} \,.
\]
По сути определяется проекция вектора наблюдений на вектор известных значений.

Рассмотрим ошибки первого и второго рода этого теста: \begin{align*}
	\alpha &= \ex_\theta \phi(Y^n)\, \text{ \rus{при} } \theta\in\Theta_0\\
	\beta &= 1-\ex_\theta \phi(Y^n)\, \text{ \rus{при} } \theta\in\Theta_1\,.
\end{align*}
В нашем случае
\begin{align*}
	\ex_\theta \phi(Y^n)
	&= \pr\bigl( \sum_{k=1}^n Y_k s_k \geq 0 \bigr) \\
	&= \pr\bigl( \theta\sum_{k=1}^n s_k^2 + \sum_{k=1}^n\xi_k s_k \geq 0 \bigr) \\
	&= \pr\bigl( \sum_{k=1}^n \xi_k s_k \geq -\theta\sum_{k=1}^n s_k^2 +  \bigr)\,,
\end{align*}
где $\xi_k = \xi(t_k)$ шум в моменты $t_k$ (вектор $\xi$). В силу того, что $\|s\|>0$
и вектор $\xi$ Гауссовкский, справедливо, что $s'\xi \sim \mathcal{N}(0,s's)$,
откуда вытекает, что
\[ \ex_\theta \phi(Y^n) = \pr\bigl( s'\xi \geq -\theta s's ) = 1 - \Phi( -\theta \sqrt{s's} ) \,, \]
для любых $\theta\in \{-1,+1\}$ где $\Phi(x)$ функция распределения стандартной
нормальной случайной величины. Таким образом
\begin{align*}
	\alpha &= 1 - \Phi\bigl( \sqrt{ \sum_{k=1}^n s_k^2 } \bigr) \\
	\beta &= \Phi\bigl( -\sqrt{ \sum_{k=1}^n s_k^2 } \bigr) \,,
\end{align*}
откуда $\alpha = \beta$, то есть ошибки первого и второго рода совпадают.

% section task_3 (end)

\section{Задание \# 4} % (fold)
\label{sec:task_4}

Рассмотрим наблюдения случайного процесса в дискретном времени 
\[ Y_k = \beta_1 + X_k \beta_1 + \sigma \xi_k\,,\,\, k=1,\ldots, n\,, \]
где $\beta_1$ и $\beta_2$ -- неизвестные параметры, $(X_k)_{k=1}^n$ -- известные числа
и $\xi_k\sim \mathcal{N}(0,1)$ НОРСЛ.

Правдоподобие наблюдений:
\[
\mathcal{L}(Y^n, X^n;\beta_1, \beta_2, \sigma^2)
= -\frac{n}{2}\log 2\pi - \frac{n}{2} \log \sigma^2 - \frac{1}{2 \sigma^2} \sum_{k=1}^n \bigl(Y_k - \beta_1 - \beta_2 X_k\bigr)^2\,,
\]
выпуклая функция от $\beta$ и достигает максимума значениях, удовлетворяющим
условиям первого порядка:
\begin{align*}
	\beta_1:\,& \frac{1}{2\sigma^2} \sum_{k=1}^n \bigl(Y_k - \beta_1 - \beta_2 X_k\bigr) = 0\\
	\beta_2:\,& \frac{1}{2\sigma^2} \sum_{k=1}^n \bigl(Y_k - \beta_1 - \beta_2 X_k\bigr)X_k  = 0\,,
\end{align*}
откуда следует
\[ \beta_1 = \bar{Y} - \beta_2 \bar{X} \,, \]
и
\[ \overline{YX} - \beta_2 \overline{X^2} = \beta_1 \bar{X} \,, \]
где $\bar{X} = \tfrac{1}{n} \sum_{k=1}^n X_k$ (аналогично для $\overline{XY}$ и $\overline{X^2}$).
Подставив полученное выражение для $\beta_1$ во второе условие получаем:
\[
\overline{YX} - (\bar{Y} - \bar{X} \beta_2) \bar{X} - \beta_2 \overline{X^2} = 0 \,,
\]
откуда \[
\overline{YX} - \bar{Y}\bar{X} = \beta_2 \overline{X^2} - \beta_2 \bar{X}^2\,.
\]
Таким образом
\begin{align*}
	\hat{\beta}_2 &= \frac{\overline{YX} - \bar{Y}\bar{X}}{ \overline{X^2} - \bar{X}^2 } \\
	\hat{\beta}_1 &= \bar{Y} - \hat{\beta}_2 \bar{X}\,.
\end{align*}
 
% section task_4 (end)

\section{Задание \# 5} % (fold)
\label{sec:task_5}

Рассмотрим стационарный авторрегрессионный процесс первого порядка в дискретном времени:
\[ Y_k = \alpha Y_{k-1} + \xi_k\,,\]
где $|\alpha| < 1$ и $\xi_k \sim\text{WN}$.

Во-первых, для любого целого $m\geq 0$ справедливо :
\[ Y_k = \alpha^m Y_{k-m} + \sum_{i=0}^{m-1} \alpha^i \xi_{k-i}\,. \]
Во-вторых, в силу стационарности $\ex Y_k = \ex Y_s$ для любых $k,s$. Поскольку
\[
\ex Y_k
= \alpha \ex Y_{k-1} + \ex \xi_k
= \alpha \ex Y_{k-1}
= \alpha \ex Y_k\,,\]
то получаем $\ex Y_k = 0$ для любого $k\geq1$ так как $\alpha < 1$.
В-третьих, опять-таки в силу стационарности выполняется, что :
\[
\ex Y_k^2
= \alpha^2 \ex Y_{k-1}^2 + 2\ex Y_{k-1} \xi_k + \ex \xi_k^2
= \alpha^2 \ex Y_{k-1}^2 + 2\ex Y_{k-1} \ex \xi_k + 1
= \alpha^2 \ex Y_k^2 + 1 \,, \]
откуда $\ex Y_k^2 = (1-\alpha^2)^{-1}$.

Рассчитаем автоковариационную функцию процесса для $m\geq 0$ :
\begin{align*}
	\gamma_m &= \ex Y_k Y_{k-m}
	 = \alpha^m \ex Y_{k-m}^2 + \sum_{i=0}^{m-1} \alpha^i \ex Y_{k-m} \xi_{k-i} \\
	&=\biggl[ Y_{k-m} \text{\rus{ и }} (\xi_{k-i})_{i=0}^{m-1} \text{\rus{ независимы }} \biggr] \\
	&= \alpha^m \ex Y_{k-m}^2 + \sum_{i=0}^{m-1} \alpha^i \ex Y_{k-m} \ex \xi_{k-i} \\
	&= \frac{\alpha^m}{1-\alpha^2} \,.
\end{align*}
в силу стационарности для $m\leq 0$ выполнено
\[ \ex Y_k Y_{k-m} = \ex Y_k Y_{k-(-m)}  = \gamma_m \,. \]

Спектральная плотность стационарного процесса определяется выражением
\[
f_Y(\lambda)
= \sum_{m=-\infty}^{+\infty} e^{2\pi i\, \lambda m} \ex Y_0 Y_m\,,
\]
где $|\lambda| \leq \frac{1}{2}$. В силу симметричности $\gamma_m$ справедливо
\[
f_Y(\lambda) = \gamma_0 + \sum_{m\geq 1} \bigl( e^{2\pi i \lambda m} + e^{2\pi i \lambda -m} \bigr) \gamma_m \,.
\]
Подставив значение $\gamma_m$ получаем
\[
f_Y(\lambda) = \frac{1}{1-\alpha^2} + \frac{1}{1-\alpha^2} \biggl(
		\sum_{m\geq 1} \bigl( \alpha e^{2\pi i \lambda} \bigr)^m
		+ \sum_{m\geq 1} \bigl( \alpha e^{ -2\pi i \lambda} \bigr)^m
	\biggr)\,.
\]

Заметим, что для любого $\lambda$ комплексное число $z = \alpha e^{\pm 2\pi i \lambda}$
по модулю строго меньше $1$:
\[ \bigl\lvert \alpha e^{\pm 2\pi i \lambda}\bigr\rvert = |\alpha| < 1\,. \]
Поэтому ряды в выражении $f_Y(\lambda)$ сходятся. Следовательно
\begin{align*}
	f_Y(\lambda)
	&= \frac{1}{1-\alpha^2} \biggl( 1
		+ \frac{\alpha e^{2\pi i \lambda}}{1-\alpha e^{2\pi i \lambda}}
		+ \frac{\alpha e^{-2\pi i \lambda}}{1-\alpha e^{-2\pi i \lambda}}
		\biggr)\\
	&= \frac{1}{1-\alpha^2} \frac{
		1 - \alpha e^{-2\pi i \lambda} - \alpha e^{2\pi i \lambda } + \alpha^2
		+ \alpha e^{-2\pi i \lambda} - \alpha^2 + \alpha e^{2\pi i \lambda} - \alpha^2
	}{1 - \alpha e^{-2\pi i \lambda} - \alpha e^{2\pi i \lambda } + \alpha^2} \\
	&= \frac{1}{1-\alpha^2} \frac{ 1 - \alpha^2}{1 +
		\alpha^2 - \alpha e^{-2\pi i \lambda} - \alpha e^{2\pi i \lambda } } \,,
\end{align*}
откуда по формуле Эйлера:
\begin{align*}
	f_Y(\lambda)
	&= \biggl( 1 + \alpha^2 - \alpha ( e^{-2\pi i \lambda} + e^{2\pi i \lambda } ) \biggr)^{-1}\\
	&= \biggl( 1 + \alpha^2 - 2\alpha \cos 2\pi \lambda \biggr)^{-1} \,.
\end{align*}

% section task_5 (end)

\section{Задание \# 6} % (fold)
\label{sec:task_6}
Рассмотри наблюдения заданные
\[ Y_k  = \theta + \sigma \xi_k\,, \]
где $\theta$ -- неизвестный случайный параметр с плотностью
\[\pi(\theta) = \frac{1}{2\lambda} \text{exp}\bigl\{ -\frac{|\theta|}{\lambda} \bigr\} \,,\]
и $\xi_k\sim \mathcal{N}(0,1)$.
Найдём оценку, максимизирующую апостериорную плотность $\theta$
\[ p(\theta|Y^n) \propto p(Y^n|\theta) \pi(\theta) \,,\]
где константа пропорциональности не зависит от $\theta$.

Правдоподобие наблюдений задаётся следующим выражением:
\begin{align*}
	p(Y^n|\theta)
	&= (2\pi \sigma^2)^{-\frac{n}{2}} \text{exp}\bigl\{ - \frac{1}{2\sigma^2} \sum_{k=1}^n (Y_k - \theta)^2 \bigr\} \\
	&= (2\pi \sigma^2)^{-\frac{n}{2}}
		\text{exp}\bigl\{ - \frac{1}{2\sigma^2} \sum_{k=1}^n (Y_k - \bar{Y}_n)^2 \bigr\}
		\text{exp}\bigl\{ - \frac{n}{2\sigma^2}( \bar{Y}_n - \theta)^2 \bigr\} \,,
\end{align*}
поскольку $\sum_{k=1}^n (Y_k - \bar{Y}_n)(\bar{Y}-\theta) = 0$, где 
\[\bar{Y}_n = \frac{1}{n} \sum_{k=1}^n Y_k \,.\]
Легко увидеть, что правдоподобие зависит от параметра $\theta$ только через близость
к выборочному среднему $\bar{Y}_n$. Таким образом оптимальное $\theta$ можно найти
из решения следующей эквивалентной задачи :
\[
\text{exp}\bigl\{ - \frac{n}{2\sigma^2}( \bar{Y}_n - \theta)^2 \bigr\}
\text{exp}\bigl\{ - \frac{|\theta|}{\lambda} \bigr\} \to \max_\theta \,,
\]
которая в свою очередь сводится к
\[
(\bar{Y}_n - \theta)^2 + A |\theta| \to \min_\theta \,,
\]
где $A = \frac{2\sigma^2}{n\lambda} > 0$.

Эту задачу можно переформулировать в виде условной задачи 
\begin{align*}
	(\bar{Y}_n - \theta_+ + \theta_-)^2 + A (\theta_+ + \theta_- ) \to \min_{\theta_-,\theta_+}\\
	\text{s.t. }\, \theta_-,\theta_+ \geq 0\,,\, \theta_- \theta_+ = 0\,.
\end{align*}
Лагранжиан:
\[
\mathcal{L} (\theta_-, \theta_+,\lambda_-,\lambda_+,\mu) =
	(\bar{Y}_n - \theta_+ + \theta_-)^2 + A (\theta_+ + \theta_- )
	- \lambda_- \theta_- - \lambda_+ \theta_+ - \mu \theta_- \theta_+ \,,
\]
Условия первого порядка на $\theta_-$ и $\theta_+$ выражены:
\begin{align*}
	\theta_+:\quad & - 2(\bar{Y}_n - \theta_+ + \theta_- ) + A - \lambda_+ - \mu \theta_- = 0\\
	\theta_-:\quad&   2(\bar{Y}_n - \theta_+ + \theta_- ) + A - \lambda_- - \mu \theta_+ = 0\,.
\end{align*}
Домножим условия на $\theta_+$ и $\theta_-$ соответственно. Условие допустимости
$\theta_- \theta_+ = 0$ и условия дополняющей нежёсткости $\lambda_-\theta_- =0$
и $\lambda_+\theta_+ = 0$ упрощают условия первого порядка:
\begin{align*}
	\theta_+:\, & - 2(\bar{Y}_n - \theta_+ + \theta_- - \frac{A}{2} ) \theta_+ = 0\\
	\theta_-:\, &   2(\bar{Y}_n - \theta_+ + \theta_- + \frac{A}{2} ) \theta_- = 0\,.
\end{align*}

Набор $\theta_- = \theta_+ = 0$ удовлетворяет как условиям первого порядка,
так и допустимости.

Рассмотрим решение $\theta_+ = 0$ и $\theta_- > 0$. Тогда из второго условия
вытекает:
\[ \theta_- = - \bar{Y}_n - \frac{A}{2}\,, \]
что допустимо в случае если $\bar{Y}_n < -\tfrac{A}{2}$.

Для решения $\theta_- = 0$ и $\theta_+ > 0$ из первого условия получаем:
\[ \theta_+ = \bar{Y}_n - \frac{A}{2} \,, \]
допустимое когда $\bar{Y}_n > \tfrac{A}{2}$.

Условия $\bar{Y}_n < -\tfrac{A}{2}$ и $\bar{Y}_n > \tfrac{A}{2}$не могут выполняться
одновременно, откуда вытекает, что :
\begin{itemize}
	\item Пара $\theta_- = 0$ и $\theta_+ > 0$ является решением, тогда и только тогда,
	когда $\bar{Y}_n > \tfrac{A}{2}$;
	\item Пара $\theta_+ = 0$ и $\theta_- > 0$ -- если и только если $\bar{Y}_n > \tfrac{A}{2}$;
	\item В случае $\bar{Y}_n\in [-\tfrac{A}{2}, +\tfrac{A}{2}]$ допустимых решений кроме
	$\theta_- = \theta_+ = 0$ нет.
\end{itemize}
Обобщая сделанные замечания получаем следующее решение для $\theta = \theta_+-\theta_-$ :
\[
\hat{\theta}
= \max\bigl\{ \bar{Y}_n - \frac{A}{2}, 0 \bigr\}
+ \min\bigl\{ \bar{Y}_n + \frac{A}{2}, 0 \bigr\}
\]

% section task_6 (end)

\section{Задание \# 7} % (fold)
\label{sec:task_7}

Неравенство Чернова состоит в следующем. Пусть $X$ некоторая вещественная случайная
величина. Тогда для любого $z\in \Real$ выполняется : 
\[ \pr(X\geq z ) \leq \inf_{\lambda > 0} e^{-\lambda z} \ex e^{\lambda X} \,.\]

Рассмотрим набор $(Y_k)_{k=1}^n \sim\text{Exp}(1)$ независимых и одинаково распределённых
случайных величин. Заметим, что случайная величина $S_n$ определённая как:
\[ S_n = \sum_{k=1}^n Y_k \,,\]
имеет Гамма распределение с параметрами $n$ и $\beta = 1$ с плотностью
\[ f(x) = \frac{\beta^n}{\Gamma(n)} x^{n-1} e^{-\beta x}\,, x\geq 0\,. \]

В силу независимости и одинаковой распределённости
\[
\ex\bigl(e^{\lambda S_n}\bigr)
= \ex\bigl(\prod_{k=1}^n e^{\lambda Y_k}\bigr)
= \prod_{k=1}^n \ex\bigl(e^{\lambda Y_k}\bigr)
= \ex\bigl(e^{\lambda Y_1}\bigr)^n \,.
\]
Теперь,
\[
\ex e^{\lambda Y_1}
= \int_0^\infty e^{-(1-\lambda)x} dx
= \begin{cases}
	+\infty, &\text{ \rus{если} } \lambda \geq 1\\
	(1-\lambda)^{-1},&\text{ \rus{иначе}}
\end{cases}\,.
\]
откуда для $\lambda < 1$
\[ \ex\bigl(e^{\lambda S_n}\bigr) = (1-\lambda)^{-n} \,. \]

Теперь согласно неравенству Чернова для любого $x\geq 0$ (нет смысла рассматривать
отрицательные $x$) :
\[ \pr(S_n \geq x )\leq \inf_{\lambda \in (0,1)} e^{-\lambda x} (1-\lambda)^{-n} \,.\]

Найдём хорошее $\lambda$ из эквивалентной выпуклой задачи:
\[ - \lambda x - n \log(1-\lambda) \to \inf_{\lambda \in (0,1)} \,. \]
Минимизирующее $\lambda$ задаётся решением уравнения :
\[ - x + \frac{n}{1-\lambda} = 0\,, \text{--} \]
то есть выражением $\lambda = 1-\frac{n}{x}$ при условии $\lambda > 0$.

Таким образом при $n < x$ верхняя граница равна
\[ \pr(S_n \geq x )\leq e^{n - x} \biggr(\frac{n}{x}\biggr)^{-n}\,. \]
При $n\geq x$ положительность производной выражения под знаком $\inf$ для всех 
$\lambda \in (0,1)$ означает что оно строго возрастает, откуда следует, что $\inf$ принимает
значение равное пределу при $\lambda\to 0+$. Таким образом
\[ \pr(S_n \geq x )\leq 1\,, \]
в этом случае.

% section task_7 (end)

% [ВШЭ ТВМММ] Зачётная работа Назаров Иван.pdf

\end{document}

