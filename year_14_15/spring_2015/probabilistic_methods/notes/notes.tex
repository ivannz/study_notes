\documentclass[a4paper]{article}
\usepackage[utf8]{inputenc}

\usepackage{graphicx, url}

\usepackage{amsmath, amsfonts, amssymb, amsthm}
\usepackage{xfrac, mathptmx}

\newcommand{\obj}[1]{{\left\{ #1 \right \}}}
\newcommand{\clo}[1]{{\left [ #1 \right ]}}
\newcommand{\clop}[1]{{\left [ #1 \right )}}
\newcommand{\ploc}[1]{{\left ( #1 \right ]}}

\newcommand{\brac}[1]{{\left ( #1 \right )}}
\newcommand{\induc}[1]{{\left . #1 \right \vert}}
\newcommand{\abs}[1]{{\left | #1 \right |}}
\newcommand{\nrm}[1]{{\left\| #1 \right \|}}
\newcommand{\brkt}[1]{{\left\langle #1 \right\rangle}}
\newcommand{\floor}[1]{{\left\lfloor #1 \right\rfloor}}

\newcommand{\Real}{\mathbb{R}}
\newcommand{\Cplx}{\mathbb{C}}

\newcommand{\Fcal}{\mathcal{F}}
\newcommand{\borel}{\mathcal{B}}

\newcommand{\defn}{\mathop{\overset{\Delta}{=}}\nolimits}

%% \usepackage[english, russian]{babel}
%% \newcommand{\eng}[1]{\foreignlanguage{english}{#1}}
%% \newcommand{\rus}[1]{\foreignlanguage{russian}{#1}}

\title{Probabilistic methods in modelling}
\author{Nazarov Ivan, \rus{101мНОД(ИССА)}\\the DataScience Collective}
\begin{document}
%% \selectlanguage{english}
\maketitle

Theory of Stochastic processes
Brownian motion
Martingales
Poisson point processes

Generalizationo of the central limit theorem:
-- normal distribution
-- stable processes
-- Fisher-Tipett-Gendenko thorem for maxima

Based on the law of large numbers:
-- Extreme value theory

Additional chapters of probability

Elementary probability: Kolmogorov axioms, Measure theoretic approach.
Bernoulli law or large numbers.
Stochastic processes require measure theoretic foundations for their definition on some abstract space.

%% \selectlanguage{russian}
\section{Lecutre \#1} % (fold)
\label{sec:lecutre_1}


Suppose $\brac{X_k}_{k=1}^n$ is a finite collection of random variables which are independent and identically distributed.

What is the asymptotics of some function of $\brac{X_k}_{k=1}^n$. The probability distribution 

Consider a measure space $(\Omega, \Fcal, P)$ and an RV $X(\omega)$ from $\Omega$ to $(\mathcal{X}, \Sigma)$ on it.
This induces a measure in $\mathcal{X}$ -- the image space. see image~1.
\[\mathbb{P}_X \defn X_\# \mathbb{P} = \mathbb{P}\brac{X\in A} = \int 1_{X^{-1}(A)}d\mathbb{P}\]

For limiting theorems the image space is sufficient.

Abstract spaces are needed in stochastic processes (and with filtration).

Convergence in distribution is needed in limiting theorems.

Consider $X\in \R$.
How is the distribution of $X$ defined?
It is defined using the semi-ring of half-closed intervals $\ploc{a,b}$.
Or, the most popular method -- Cumulative Distribution function.

An RV is not a measurable map, but a measure representing its probability.
A map $F:\R\to \clo{0,+\infty}$ is a distribution function if \begin{enumerate}
	\item $F$ is non decreasing;
	\item $F(-\infty) = 0$ and $F(+\infty) = 1$.
	\item $F$ is right-continuous and bounded;
\end{enumerate}
in short $F$ must be c\'adl\'ag. The crucial thing is $F(\R)=1$.

Modern statistical physics was founded on Gibbs's idea.
A physical system might be in different states, thus Gibbs and Bolzmann postulated that each state has some likelihood.

What is the likelihood of picking an even number in $\mathbb{Z}$? Suppose there is some probability measure on $\mathbb{Z}$ given by $\brac{p_n}_{n\in \mathbb{Z}}$ 
\[\sum_{n\text{ div } 2} p_n\]

Let $A\subseteq \mathbb{Z}$ then the density of $A$ is \[\rho(A) \defn \lim_{n\to+\infty, m\to -\infty} \frac{\abs{\obj{\induc{k\in \mathbb{Z}}\,n\leq k\leq m }}}{n+m+1}\]
However $\rho(\cdot)$ fails to be countably sub-additive.

Classification of distribution functions on $\R$. \begin{description}
	\item[Atomic:] \hfill \\
		there is a countable and measurable subset $A\in \borel(\R)$ and a collection $\brac{p_k}_{k\in A}$ such that
			\[F(x) = \sum_{k\in A} p_k 1_{\obj{k}}(x)\]
	\item[Singular:] \hfill \\
		all the rest. 
	\item[Absolutely continuous:] \hfill \\
		there exists a lebesgue-measurable map $p:\R\to\R$ with $F(x)=\int p(x) dx$
\end{description}

Lebesgue theorem: every distribution function can be represented as a weighted sum of three basic types of distributions.


Classification of local behaviour of distributions in $\R$.

Consider some $x_0\in \R$ and consider the difference $F(x_0+\Delta)-F(x_0-\Delta)$ for $\Delta > 0$.
The function $F$ has the singularity order $\alpha$ if \[F(x_0+\Delta)-F(x_0-\Delta) = \Omega(\Delta^\alpha)\]

\begin{description}
	\item[Atomic:] $\alpha = 0$;
	\item[Continuous:] $\alpha = 1$;
\end{description}

If $F(x)=\min\obj{\sqrt{x}, 1} 1_{\clop{0,+\infty}}(x)$ then $0$ has singularity order $\frac{1}{2}$.

Cantor distribution has uncountably many points with $\alpha = 0$, and uncountably many points with $\alpha = \frac{\log 2}{\log 3}$.
A good idea is to represent any $x\in \clo{0,1}$ in base-3 representation. The ``interesting'' points have either $0$ or $2$ in their base-3 form.



Problem \#1
	Given $U\sim \mathcal{U}\clo{0,1}$ and $Y=f(U)$. Find $f$ such that the density $p_Y$ in given by
	\[\lambda e^{-\lambda y} 1_{\clop{0,+\infty}}\]
	If $F(x)$ is absolutely continuous and $X\sim F$, then $F(X)\sim \mathcal{U}\clo{0,1}$. (Skorokhod)

Problem \#2
	Suppose $(X,Y)$ is uniformly distributed inside the unit circle. What is the distribution of $(x',y')$ with the radius given by $r = \sqrt{-\log \pho}$.
	Using the Jacobian transformation, the distribution of the transformed pair is jointly gaussian.



% section lecutre_1 (end)

\end{document}
