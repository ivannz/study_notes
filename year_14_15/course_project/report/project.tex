\documentclass[a4paper]{article}
\usepackage[utf8]{inputenc}
% \usepackage{fullpage}

\usepackage{graphicx, url}

\usepackage{amsmath, amsfonts, xfrac}
\usepackage{mathtools}

\newcommand{\obj}[1]{{\left\{ #1 \right \}}}
\newcommand{\clo}[1]{{\left [ #1 \right ]}}
\newcommand{\clop}[1]{{\left [ #1 \right )}}
\newcommand{\ploc}[1]{{\left ( #1 \right ]}}

\newcommand{\brac}[1]{{\left ( #1 \right )}}
\newcommand{\crab}[1]{{\left ] #1 \right [}}
\newcommand{\induc}[1]{{\left . #1 \right \vert}}
\newcommand{\abs}[1]{{\left | #1 \right |}}
\newcommand{\nrm}[1]{{\left\| #1 \right \|}}
\newcommand{\brkt}[1]{{\left\langle #1 \right\rangle}}

\newcommand{\floor}[1]{{\left\lfloor #1 \right\rfloor}}

\newcommand{\Rbar}{{\bar{\mathbb{R}}}}
\newcommand{\Real}{\mathbb{R}}
\newcommand{\Zinf}{\clo{ 0, +\infty }}
\newcommand{\Cplx}{\mathbb{C}}
\newcommand{\Tcal}{\mathcal{T}}
\newcommand{\Dcal}{\mathcal{D}}
\newcommand{\Hcal}{\mathcal{H}}
\newcommand{\Ccal}{\mathcal{C}}
\newcommand{\Scal}{\mathcal{S}}
\newcommand{\Ncal}{\mathcal{N}}
\newcommand{\Ecal}{\mathcal{E}}
\newcommand{\Fcal}{\mathcal{F}}
\newcommand{\borel}[1]{\mathcal{B}\brac{#1}}
\newcommand{\Ex}[1]{\mathbb{E}\brac{#1}}
\newcommand{\Var}[1]{\text{Var}\brac{#1}}

\newcommand{\pwr}[1]{\mathcal{P}\brac{#1}}
\newcommand{\Dyns}[1]{\mathfrak{D}\brac{#1}}
\newcommand{\Ring}[1]{\mathcal{R}\brac{#1}}
\newcommand{\Supp}[1]{\operatorname{supp}\nolimits\brac{#1}}

\newcommand{\defn}{\mathop{\overset{\Delta}{=}}\nolimits}
\newcommand{\lpto}{\mathop{\overset{L^p}{\to}}\nolimits}

\newcommand{\re}{\operatorname{Re}\nolimits}
\newcommand{\im}{\operatorname{Im}\nolimits}

\usepackage[english, russian]{babel}
\newcommand{\eng}[1]{\foreignlanguage{english}{#1}}
\newcommand{\rus}[1]{\foreignlanguage{russian}{#1}}
\selectlanguage{english}

\title{Studying Self-similar Processes Using the Croosing Tree}
\author{Nazarov Ivan, \rus{101мНОД(ИССА)}}

\begin{document}
\pagenumbering{gobble}
%% Russian title page
\selectlanguage{russian}
\begin{titlepage}
    \selectlanguage{russian}
    \thispagestyle{empty}
    \vbox to \textheight {
        \renewcommand{\baselinestretch}{1}\selectfont
        \begin{center}
            \textbf{%\Large
            Правительство Российской Федерации\\[0.5cm]
            Федеральное государственное автономное образовательное учреждение 
            высшего профессионального образования\\[0.5cm]
            Национальный Исследовательский Университет\\[0.5cm]
            <<Высшая Школа Экономики>>}\\[1.5cm]

            \textbf{%\Large
            Факультет компьютерных наук\\[0.5cm]}
        \end{center}
        \textbf{%\Large
            Магистерская программа Науки о Данных\\[0.5cm]
            Кафедра кафедра кафедра\\[1.0cm]}

            \begin{center}
            {\large \bfseries КУРСОВАЯ РАБОТА}\\[1.0cm]
            \end{center}
            {\large \bfseries На тему ``Исследование самоподобных процессов с помощью дерева пересечений''}\\[0.5cm]
        %\end{center}

        \vspace{2.0cm}

        \begin{flushright}
            Студент группы \# 101мНОД\\
            Назаров Иван Николаевич\\[0.5cm]
            Руководитель ВКР\\
            Декруэ Жофри Жерар\\[3cm]
        \end{flushright}

        \vspace{2.0cm}

        \vfill
        \begin{center}
            Москва, 2015\\[3.0cm]
            % \includegraphics{hsecmyk}\\[1cm]
        \end{center}
    }
\end{titlepage}
\clearpage

%% English title page
\selectlanguage{english}
\begin{titlepage}
	\selectlanguage{english}
	\thispagestyle{empty}
	\vbox to \textheight{
		\renewcommand{\baselinestretch}{1}\selectfont
		\begin{center}
			\textsc{\LARGE
			National Research University\\[0.5cm]
			Higher School of Economics}\\[1.5cm]

			\textsc{\Large
			Master’s programme in Data Science}\\[0.5cm]

			\rule{\linewidth}{0.5mm}\\[1.0cm]

			{\huge \bfseries Course Project}\\[0.5cm]
			{\large \bfseries on}\\[0.5cm]
			{\huge \bfseries ``Studying Self-similar Processes Using the Crossing Tree''}\\[0.5cm]
		\end{center}

		\vspace{2.0cm}

		\begin{flushright}
			\large Ivan \textsc{Nazarov}\\[0.5cm]
			\rus{101мНОД(ИССА)}\\[3cm]
		\end{flushright}
		
		\vspace{2.0cm}

		\vfill
		\begin{center}
			Moscow\\
			2015\\[3cm]
			% \includegraphics{hsecmyk}\\[1cm]
		\end{center}
	}
\end{titlepage}

\clearpage

%% Draft title page
\selectlanguage{english}
\maketitle
\begin{abstract}
Time-series data presenting scale invariance do not posses a well-defined time scale. Instead, their dynamics are understood when studied across a whole range of scales. Examples of data with empirical scale-invariance include network traffic, financial time-series, and other natural phenomena in physics and biology. The crossing-tree is a recent tool to analyze this kind of signals. It provides an ad-hoc representation of the data which is adapted to its dynamics, and thus represents an alternative to wavelet decompositions. In this project, the student will first learn about scale invariance, and how the crossing-tree has been used previously as a tool to analyze self-similar signals. The next step is to analyze self-similar processes with stationary increments (known as H-SSSI processes) using the crossing tree. It is expected that for this class of processes, the crossing tree presents common features which need to be extracted.
\end{abstract}
\tableofcontents
\clearpage
\pagenumbering{arabic}

%% The project itself
\selectlanguage{english}

\section{Introduction} % (fold)
\label{sec:introduction}
Crossing tree

% section introduction (end)

\section{Literature review} % (fold)
\label{sec:literature_review}
Reviewed papers \cite{jones2004}, \cite{jonesshen2005} and \cite{decrouez2013}.

% section literature_review (end)

%% End of the report: lists of object and references
% \clearpage \listoffigures
% \clearpage \listoftables

\clearpage
\selectlanguage{english}
\bibliographystyle{amsplain}
\bibliography{literature}

%% Supplementary material
% \chapter*{Appendix} % (fold)
\label{cha:appendix}

\section*{Details on the practical procedure} % (fold)
\label{sec:details_on_the_practical_procedure}

In the appendix a brief description of the parctical crossing tree costruction
procedure is given.

Recall that the tree is constructed in two passes: initial detection passages over
levels of $\delta \mathbb{Z}$ grid, which also discrds within-band movements, and
the pruning phase, where re-crossings are eliminated.

Before the first pass through the data the time series $x_j$ is shifted and scaled to
series $z_i = \frac{1}{ \delta }\bigl(x_j - x_0\bigr)$, so that the constructed 
is rooted at $t=0$, since the process sets off from the origin.

The first pass sweeps through consecutive increments of the series given by pairs
$(t_i, z_i)$ and $(t_{i+1}, z_{i+1})$ for $i=1, \ldots, N-1$. For each increment
its direction is determined by the sign of the difference $\Delta_i=z_{i+1}-z_i$,
and depending on it, the range of grid levels passed is computed. The range of an
increment $i$ a subset of integers $R_i = [a_i,b_i]\cup\mathbb{Z}$, where the
boundaries $a_i$ and $b_i$ are determined using the following logic
\begin{description}
    \item[Upcrossing] if $\Delta_i > 0$ then $a_i = \lceil z_i \rceil$
    and $b_i = \lfloor z_{i+1}\rfloor$;
    \item[Downcrossing] for $\Delta_i < 0$ the range given by $b_i = \lceil z_{i+1} \rceil$
    and $a_i = \lfloor z_i\rfloor$.
\end{description}
In an rare case of sideways movement, $R_i = \emptyset$. Also note that $[a,b] = \emptyset$
whenever $b<a$. During the first pass,

Movements of the normalised process $(z_i)$ that wiggle strictly within a band
between levels of the grid $\delta \mathbb{Z}$ and do not do not pass or touch
through a level have empty range, and are discarded.

The second pass checks if $b_{j-} = a_j$ for any crossing $j\in J$, where
$J = \{j : R_j\neq \emptyset\}$ and the index $j-$ for any $j\in J$ is defined as
$j- = \max\{i\in J : i < j\}$, or $-\infty$ if $j\in J$ is the very first apparent
crossing of the grid $\delta \mathbb{Z}$, and $a_{j-}$ is taken to be $-\infty$.

If the first level passed by $j\in J$ coincides with the last levels crossed by $j-$
then the very first crossing event in the $j$-th increment is a re-crossing evnet,
and thus should be eliminated. This is done by adjusting the $a_i$ in the direction
of the increment $j$. The ranges are undated accordingly and empty ones are discarded.

These passes compute the levels the sample path apparently crossed, and the crossing
times are estimated using linear interpolation between times $t_i$ and $t_{i+1}$ with
weights determined by passed level. The formula the crossing time of level $l\in R_i$
during the $i$-th increment is
\[
\tau_{il} = t_i + \bigl(t_{i+1} - t_i\bigr) \frac{l - z_i}{z_{i+1} - z_i}
\]

The crossing times of a coarser resolution grid are computed based on the 
data provided the next finer resolution in the same way the second pass functions:
since the resolution is halved between successive tree levels, the crossing times
estimated at resolution $\delta 2^{n+1}$ are by construction a subset of crossing
times of a grid $\delta 2^n$. Inded, if an increment $j$ passed a line $m \delta 2^{n+1}$
then this very same passed a line $2m \delta 2^n$ of the finer grid and the interpolation
coefficient used to compute the crossing time is unchanged
\begin{align*}
	\frac{m - \frac{x_j - x_0}{\delta 2^{n+1}}}{ \frac{x_{j+1} - x_j}{\delta 2^{n+1}} }
	&= \frac{2m - \frac{x_j - x_0}{\delta 2^n}}{ \frac{x_{j+1} - x_j}{\delta 2^n} } \\
\end{align*}
Thus, since the incrementis the same, the corrins time reamins the same.

% section* details_on_the_practical_procedure (end)

% chapter* appendix (end)

\end{document}
