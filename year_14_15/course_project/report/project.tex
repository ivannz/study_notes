\documentclass[a4paper,14pt]{extreport}
\usepackage{geometry}

\usepackage[utf8]{inputenc}
\usepackage[russian, english]{babel}

\usepackage{fullpage}
\linespread{1.5}

\usepackage{graphicx, url}
\usepackage{amsmath, amsfonts, xfrac}
\usepackage{mathtools}

\usepackage{tikz}
\usetikzlibrary{positioning}

% \usepackage{natbib}

\newcommand{\Real}{\mathbb{R}}
\newcommand{\Cplx}{\mathbb{C}}

\newcommand{\pr}{\mathbb{P}}
\newcommand{\ex}{\mathbb{E}}
\newcommand{\var}{\text{var}}
\newcommand{\Pcal}{\mathcal{P}}
\newcommand{\Dcal}{\mathcal{D}}
\newcommand{\Ncal}{\mathcal{N}}
\newcommand{\Lcal}{\mathcal{L}}

\newcommand{\defn}{\mathop{\overset{\Delta}{=}}\nolimits}
\newcommand{\lpto}{\mathop{\overset{L^p}{\to}}\nolimits}

\newcommand{\re}{\operatorname{Re}\nolimits}
\newcommand{\im}{\operatorname{Im}\nolimits}

\newcommand{\eng}[1]{\foreignlanguage{english}{#1}}

\newcommand{\rus}[1]{\foreignlanguage{russian}{#1}}
% \selectlanguage{english}

\title{Studying Self-similar Processes Using the Crossing Tree}
\author{Nazarov Ivan, \rus{101мНОД(ИССА)}\
	{\small Supervised by: Decrouez Geoffrey Gerard}}

\begin{document}
\pagenumbering{gobble}
%% Russian title page
\selectlanguage{russian}
\begin{titlepage}
    \selectlanguage{russian}
    \thispagestyle{empty}
    \vbox to \textheight {
        \renewcommand{\baselinestretch}{1}\selectfont
        \begin{center}
            \textbf{%\Large
            Правительство Российской Федерации\\[0.5cm]
            Федеральное государственное автономное образовательное учреждение 
            высшего профессионального образования\\[0.5cm]
            Национальный Исследовательский Университет\\[0.5cm]
            <<Высшая Школа Экономики>>}\\[1.5cm]

            \textbf{%\Large
            Факультет компьютерных наук\\[0.5cm]}
        \end{center}
        \textbf{%\Large
            Магистерская программа Науки о Данных\\[0.5cm]
            Кафедра кафедра кафедра\\[1.0cm]}

            \begin{center}
            {\large \bfseries КУРСОВАЯ РАБОТА}\\[1.0cm]
            \end{center}
            {\large \bfseries На тему ``Исследование самоподобных процессов с помощью дерева пересечений''}\\[0.5cm]
        %\end{center}

        \vspace{2.0cm}

        \begin{flushright}
            Студент группы \# 101мНОД\\
            Назаров Иван Николаевич\\[0.5cm]
            Руководитель ВКР\\
            Декруэ Жофри Жерар\\[3cm]
        \end{flushright}

        \vspace{2.0cm}

        \vfill
        \begin{center}
            Москва, 2015\\[3.0cm]
            % \includegraphics{hsecmyk}\\[1cm]
        \end{center}
    }
\end{titlepage}
\clearpage

%% English title page
\selectlanguage{english}
\begin{titlepage}
	\selectlanguage{english}
	\thispagestyle{empty}
	\vbox to \textheight{
		\renewcommand{\baselinestretch}{1}\selectfont
		\begin{center}
			\textsc{\LARGE
			National Research University\\[0.5cm]
			Higher School of Economics}\\[1.5cm]

			\textsc{\Large
			Master’s programme in Data Science}\\[0.5cm]

			\rule{\linewidth}{0.5mm}\\[1.0cm]

			{\huge \bfseries Course Project}\\[0.5cm]
			{\large \bfseries on}\\[0.5cm]
			{\huge \bfseries ``Studying Self-similar Processes Using the Crossing Tree''}\\[0.5cm]
		\end{center}

		\vspace{2.0cm}

		\begin{flushright}
			\large Ivan \textsc{Nazarov}\\[0.5cm]
			\rus{101мНОД(ИССА)}\\[3cm]
		\end{flushright}
		
		\vspace{2.0cm}

		\vfill
		\begin{center}
			Moscow\\
			2015\\[3cm]
			% \includegraphics{hsecmyk}\\[1cm]
		\end{center}
	}
\end{titlepage}

\clearpage

%% Draft title page
\selectlanguage{english}
\maketitle
\begin{abstract}
\textbf{Shamelessly copied from Geoffrey's project ptich}.\hfill \\
Time-series data presenting scale invariance do not posses a well-defined time scale.
Instead, their dynamics are understood when studied across a whole range of scales.
Examples of data with empirical scale-invariance include network traffic, financial
time-series, and other natural phenomena in physics and biology. The crossing-tree
is a recent tool to analyze this kind of signals. It provides an ad-hoc representation
of the data which is adapted to its dynamics, and thus represents an alternative to
wavelet decompositions. In this project, the student will first learn about scale
invariance, and how the crossing-tree has been used previously as a tool to analyze
self-similar signals. The next step is to analyze self-similar processes with stationary
increments (known as H-SSSI processes) using the crossing tree. It is expected that
for this class of processes, the crossing tree presents common features which need
to be extracted.
\end{abstract}

\selectlanguage{english}
\tableofcontents
\clearpage
\pagenumbering{arabic}

%% The project itself
\selectlanguage{english}

\chapter*{Introduction} % (fold)
\label{cha:introduction}

% chapter* introduction (end)

%% Include the contents of the second chapter
\clearpage
 \chapter{the Crossing Tree} % (fold)
\label{cha:the_crossing_tree}

The crossing tree as a tool for analysing self-similarity was introduced by Jones
and Shen in \cite{jones2004}, where it was successfully used in detecting structural
changes in time-series data, which manifested in a change in the Hurst exponent.
In this section I describe the theoretic construct of the crossing tree and how it
is built on sample data.

\section{Crossing Times} % (fold)
\label{sec:crossing_times}

Consider a path of real-valued continuous process $\{X(t)\}$. At the heart of the
crossing tree lie crossing times of a uniformly spaced grid on $\Real$ given by
$X(0) + \delta 2^n \mathbb{Z}$ for some $n\geq 0$ and base grid spacing $\delta > 0$.
\footnotemark Each $n\geq 0$ represents the ``resolution'' through which sample
paths of $\{X(t)\}$ are studied.
\footnotetext{The addition $x_0+A$, for $A$ in an affine space such as $\Real$
represents shifting of every element of the set $A$ by a common value $x_0$.}

In particular, the crossing times $T_k^n$ of process $\{X(t)\}$ at resolution $n\geq 0$
are defined as follows: let $T_0^n = 0$ and for any $k\geq 0$ put
\[
T_{k+1}^n
= \inf \Bigl\{ t \geq T^n_k : \bigr | X(t) - X(T^n_k) \bigr | \geq \delta 2^n \Bigr\}
\]
In other words $T_{k+1}^n$, if it is finite, is the first passage time of a path of
$X(t)$ across a level of $\delta 2^n \mathbb{Z}$ that is different from the line of
the same grid crossed previously. Fixing the zero-th crossing at $0$ automatically
aligns the grid with the process, so in effect, without loss of generality one may
consider processes $X_t$, which start at the origin $X(0)=0$.

By definition, the crossing times of some fixed level $n\geq 0$ are non-decreasing:
$T_k^n \leq T_{k+1}^n$ for all $k\geq 0$. However, since the paths of the process are
almost surely continuous, it is possible to show that $T_k^n < T_{k+1}^n$. Indeed,
by right-continuity of $X(t)$ (a fixed path of $\{X(t)\}$) for $\delta 2^{n-1}$
there is $\eta>0$ with $|X(s)-X(T_k^n)| < \delta 2^{n-1}$ for all $s\in [T_k^n, T_k^n + \eta)$.
However, by definition of $T_{k+1}^n$ there in the $\eta$ vicinity of $T_{k+1}^n$
there exists $\hat{s}$ such that $|X(\hat{s})-X(T_k^n)|\geq \delta 2^n$. Thus, if
$T_k^n = T_{k+1}^n$ then one has $\delta 2^n \leq \delta 2^{n-1}$ -- a contradiction.

Another important property of crossing times is that if $T_{k+1}^n < +\infty$, then
$|X(T_{k+1}^n)-X(T_k^n)| = \delta 2^n$. Indeed, the definition of $T_{k+1}^n$ implies
that there exits a sequence $s_j\downarrow T_{k+1}^n$ with $|X(T_{k+1}^n)-X(T_k^n)|\geq \delta 2^n$
. Since the path $X(t)$ is continuous at $T_{k+1}^n$ it must be true that
\[
|X(T_{k+1}^n)-X(T_k^n)| = \lim_{j\to \infty} |X(s_j)-X(T_k^n)| \geq \delta 2^n
\]
Now, since the function $t\to |X(t)-X(T_k^n)| - \delta 2^n$ is continuous at $T_{k+1}^n$,
$|X(T_{k+1}^n)-X(T_k^n)| > \delta 2^n$ would entail the existence of $s\in [T_k^n, T_{k+1}^n)$
such that $|X(s)-X(T_k^n)| - \delta 2^n > 0$. This would contradict the fact that
$T_{k+1}^n$ is a lower bound of the set of all $s\geq T_k^n$ with $|X(s)-X(T_k^n)| \geq \delta 2^n$.

So far the crossing times of an almost surely continuous process $\{X(t)\}$, have
the following properties (almost surely): \begin{enumerate}
	\item $T_k^n < T_{k+1}^n$ for all $k\geq 0$;
	\item $|X(T_{k+1}^n)-X(T_k^n)| = \delta 2^n$ for all $k\geq0$ with $T_{k+1}^n<+\infty$.
\end{enumerate}

In order to see the natural tree structure to the crossings, one has to investigate
the properties of crossing times at different resolutions. For the following suppose,
for some integers $k,m\geq 0$ it holds that $T_k^n\in [T_m^{n+1}, T_{m+1}^{n+1})$ and
$X(T_m^{n+1}) = X(T_k^n)$.

By definition of $T_{k+1}^n$ for all $s\in[T_k^n, T_{k+1}^n)$ it must be true that
\[ |X(s) - X(T_k^n)| < \delta 2^n \]
fr otherwise $T_{k+1}^n$ would not have been the least lower bound. Since the process
takes the same value at both times $T_m^{n+1}$ and $T_k^n$, it must be true that
$|X(s) - X(T_m^{n+1})| < \delta 2^{n+1}$ for such $s$, whence $s\leq T_{m+1}^{n+1}$.
Therefore $T_{k+1}^n\leq T_{m+1}^{n+1}$.

However, if one further assumes that $T_{m+1}^{n+1} < +\infty$, i.e. that the $n+1$-st
crossing takes place, then $T_{k+1}^n \neq T_{m+1}^{n+1}$. Indeed, \begin{itemize}
	\item $|X(T_{k+1}^n)-X(T_k^n)| = \delta 2^n$;
	\item $|X(T_{m+1}^{n+1})-X(T_m^{n+1})| = \delta 2^{n+1}$.
\end{itemize}
together with $X(T_m^{n+1}) = X(T_k^n)$ entail a contradiction $\delta 2^n = \delta 2^{n+1}$.

The assumption that a crossing of a level of a coarser grid $\delta 2^{n+1} \mathbb{Z}$
took place, permits further refinement of crossing times' behaviour: under the assumptions
above $T_{k+2}^n\leq T_{m+1}^{n+1}$ holds. Indeed, suppose otherwise. Then by definition
of $T_{k+2}^n$ and because $T_{k+1}^n \leq T_{m+1}^{n+1}$, it must be true that
\[ |X(T_{m+1}^{n+1}) - X(T_{k+1}^n)| < \delta 2^n \]
However $|X(T_{m+1}^{n+1}) - X(T_m^{n+1})| = \delta 2^{n+1}$ by the established
properties of crossing times on the grid of $n+1$ resolution. Then by the triangle
inequality and the assumption that $X(T_m^{n+1}) = X(T_k^n)$ one gets
\begin{align*}
    |X(T_{m+1}^{n+1}) - X(T_m^{n+1})| 
    & \leq |X(T_{m+1}^{n+1}) - X(T_{k+1}^n)| + |X(T_{k+1}^n) - X(T_m^{n+1})| \\
    & = |X(T_{m+1}^{n+1}) - X(T_{k+1}^n)| + |X(T_{k+1}^n) - X(T_k^n)|
\end{align*}
whence the following contradiction emerges
\[
\delta 2^{n+1}
\leq |X(T_{m+1}^{n+1}) - X(T_{k+1}^n)| + \delta 2^n
< \delta 2^{n+1}
\]
Therefore, under the stated conditions $T_{k+2}^n\leq T_{m+1}^{n+1}$.

% section crossing_times (end)

\section{a Tree Structure} % (fold)
\label{sec:a_tree_structure}

Let's define the \emph{$n$-th level crossing} as the slice of the path of the process
$\{X(t)\}$ over the time half-interval $[T_k^n,T_{k+1}^n)$ for some integer $k\geq 0$,
during which it makes a move of $\pm \delta 2^n$. The uncovered relationship between
crossing times of consecutive levels imply that if $T_k^n \in [T_m^{n+1}, T_{m+1}^{n+1})$
with $X(T_m^{n+1}) = X(T_k^n)$ and $T_{m+1}^{n+1} < +\infty$ then
\[
\bigl[T_k^n,T_{k+1}^n\bigr) \cup \bigl[T_{k+1}^n,T_{k+2}^n\bigr)
\subseteq \bigl[T_m^{n+1},T_{m+1}^{n+1}\bigr)
\]
The first condition, $X(T_m^{n+1}) = X(T_k^n)$ aligns the different resolution grid
with one another. The last requirement that $[T_m^{n+1}, T_{m+1}^{n+1})$ be a finite
interval is natural, since it actually means that the $n+1$-st level crossing actually
occurred.

These observations actually state that within the $n+1$-st level crossing there musts
be exactly an even number of crossing of the $n$-th level (subcrossings). Indeed,
if $T_k^n = T_m^{n+1}$ and $T_{m+1}^{n+1}<+\infty$, then there are two possibilities
for the crossing time $T_{k+2}^n$: \begin{enumerate}
	\item the process crossed a new level of $n+1$-st grid $\delta 2^{n+1}$ in which
	case $T_{m+1}^{n+1}\leq T_{k+2}^n$. Hence there are exactly \emph{two} subcrossings.
	\item the process moved back to the level $X(T_k^n)$, which did not incur a crossing
	of $n+1$-st grid, since $X(T_{k+2}^n)=X(T_m^{n+1})$. Yet this fluctuation was
	registered by the grid of resolution $n$. In this case, the properties of crossing]
	times entail \[ T_{k+2}^n < T_{k+3}^n < T_{k+4}^n \leq T_{m+1}^{n+1} \]
	Since $T_{m+1}{n+1}$ is finite, recurent application of this argument implies that
	there must be exactly an even number of $n$-th level subcrossings in an $n+1$-st
	level crossing.
\end{enumerate}
This described behaviour is depicted in figure~\ref{fig:xing_tree_structure}.

%% Draw a simple illustration of the movements within one crossing
\begin{figure}[htb]\begin{center}
    \tikzset{bullet/.style={circle,draw, fill=black,minimum size=5pt,inner sep=0pt},}
    \begin{tikzpicture}[grow=right, sloped]
%% Crossings
        \node [bullet] (o) {};
        \node [bullet] (u) [above right = 3em and 9em of o.west, anchor = west] {};
        \node [bullet] (d) [below right = 3em and 9em of o.west, anchor = west] {};
        \node [bullet] (uu) [above right = 3em and 9em of u.west, anchor = west] {};
        \node [bullet] (ud) [above right = 3em and 9em of d.west, anchor = west] {};
        \node [bullet] (dd) [below right = 3em and 9em of d.west, anchor = west] {};
%% The anchor and labels
        \node [draw=none] [left = 1em of o.west, anchor = east] {$X(T_k^n)$};
        \node [draw=none] [right = 1em of ud.east, anchor = west] {$X(T_k^n)=X(T_{k+2}^n)$};
        \node [draw=none] [right = 1em of uu.east, anchor = west] {$X(T_k^n)+\delta 2^{n+1}$};
        \node [draw=none] [right = 1em of dd.east, anchor = west] {$X(T_k^n)-\delta 2^{n+1}$};
%% Crossing times nodes at the bottom
        \node [draw=none] (t2) [below =of dd.south, anchor = west] {$T_{k+2}^n$};
        \node [draw=none] (t1) [left = 9em of t2.west, anchor = west] {$T_{k+1}^n$};
        \node [draw=none] (t0) [left = 9em of t1.west, anchor = west] {$T_k^n$};
%% Dumy nodes on the top level
        \node [fill=none,draw=none] (dummy2) [above = 1em of uu.north, anchor = north] {};
        \node [fill=none,draw=none] (dummy1) [left = 9em of dummy2.west, anchor = west] {};
        \node [fill=none,draw=none] (dummy0) [left = 9em of dummy1.west, anchor = west] {};
%% Paths to levels
        \path[thick,draw]
            (o) edge node[above] {$+\delta 2^n$} (u)
            (o) edge node[below] {$-\delta 2^n$} (d)
            (u) edge node[above] {$+\delta 2^n$} (uu)
            (u) edge node[below] {$-\delta 2^n$} (ud)
            (d) edge node[above] {$+\delta 2^n$} (ud)
            (d) edge node[below] {$-\delta 2^n$} (dd);
%% Crossing times
        \draw[black, dashed] (dummy2.west -| t2.west) -- (t2.west -| t2.west) {};
        \draw[black, dashed] (dummy1.west -| t1.west) -- (t1.west -| t1.west) {};
        \draw[black, dashed] (dummy0.west -| t0.west) -- (t0.west -| t0.west) {};
    \end{tikzpicture}
%% A very informative caption
    \caption{The structure of possible movements during the crossing $\bigl[T_k^n,T_{k+2}^n\bigr)$.}
\label{fig:xing_tree_structure}
\end{center}\end{figure}

Thus the properties of crossing times of a continuous process permit a natural tree
structure to crossings. If tree levels are enumerated from bottom up then \begin{itemize}
	\item a node at the $n+1$-st tree level represents a crossing of the $\delta 2^{n+1} \mathbb{Z}$;
	\item $n$-th level children (offspring) of an $n+1$-st level node represent the
	multiple crossings of a finer gird $\delta 2^n \mathbb{Z}$ that took place during
	the larger crossing. 
\end{itemize}

The above exposition considered the crossing tree for successively coarser resolutions
$\delta 2^n \mathbb{Z}$ for $n\geq0$. This has been done purely for the following reason:
in practice, on real sample path of some supposedly continuous process it is never
possible no meaningfully go beyond some basic finest resolution given by $\delta>0$.
This has to do with the fact that processes are sampled at finite frequency or over
regularly spaced but finitely many point in time. Nevertheless theoretical crossing
tree can be extended to crossings of $\delta 2^n$ grid for all integer $n\in\mathbb{Z}$.

% section a_tree_structure (end)

\section{Crossing tree on times series data} % (fold)
\label{sec:crossing_tree_on_times_series_data}

In this section I briefly describe how the crossing tree is constructed for a sample
path of some continuous process, given by time series data $(t_i, x_i)_{i=0}^N$.

In practice the crossing trees are built iteratively from the finest resolution
specified by the base grid spacing $\delta > 0$. The crossing times of the leaves
of the tree are computed for a given fixed $\delta>0$ in two passes: the firs pass
computes passage times on the passed $\delta \mathbb{Z}$ grid levels, and the second
pass adjusts the passage times and passed levels to eliminate re-crossings of the
same level. The assumption that the time series data came from a continuous process
is the main reason, that enables a sequential procedure for estimation of the crossing
times of grid $\delta\mathbb{Z}$: $X(T_{k+1})^n = X(T_k^n)+\pm\delta2^n$ if $T_{k+1}^n<\infty$.
One is encouraged to refer to appendix for details of the procedure.

The choice of $\delta$ is extremely important as it affects the estimates of the
crossing times and levels in the following way, reminiscent of the bias-variance
trade-off:
\begin{itemize}
    \item the grid with too low a value of $\delta$ would make the resolution so fine
    as to regard the studied process as a continuous piecewise linear function. Each
    increment would be very likely to produce long consecutive unidirectional streaks
    of crossed levels, which would overestimate the number of crossings with only 2
    subcrossings by introducing excess number of artificial crossings events.
    \item Too large $\delta$ leads to a tree with fewer crossings and offspring.
\end{itemize}


% section crossing_tree_on_times_series_data (end)

% chapter the_crossing_tree (end)


%% Include the contents of the third chapter
\clearpage
\chapter{$H$-SSSI processes} % (fold)
\label{cha:h_sssi_processes}

This section will cover the basic definitions and properties of self-similar processes
studied in this paper. In general we will consider a process $\bigl\{X(t)\bigr\}$ to be
a continuous-time real-valued stochastic processes defined for all $t\in [0,+\infty)$.
The contents of this section are quite general and are based on the following papers and
text books: \cite{bulinskii2005teoriya2516755}, \cite{Bai20141710}, \cite{Chronopoulou:1114288}
\cite{embrechts2000introduction} and \cite{embrechtsselfsimilar} to name a few.

\section{Definition} % (fold)
\label{sec:definition}

Before proceeding with the definitions and properties, it is necessary to clarify, what
is meant by stochastic equivalence of random processes. Processes $\bigl\{X(t)\bigr\}$ and
$\bigl\{Y(t)\bigr\}$ are equivalent in finite-dimensional distributions, or symbolically
$\{X(t)\} \overset{\Dcal}{=} \{Y(t)\}$, if for all $n\geq1$ and all $(t_k)_{k=1}^n\in [0,+\infty)$
with $t_k<t_{k+1}$ it is true that
\[ \bigl(X(t_k)\bigr)_{k=1}^n \overset{\Dcal}{\sim} \bigl(Y(t_k)\bigr)_{k=1}^n \]
where $A\overset{\Dcal}{\sim} B$ denotes equality of distribution of random variables
$A$ and $B$.

A process $\bigl\{X(t)\bigr\}_{t\geq 0}$ is called \textbf{self-similar}, of \textbf{SS} for
short, if for any $a>0$ there exists $b>0$ such that
\[ \bigl\{X(at)\bigr\} \overset{\Dcal}{=} \bigl\{b X(t)\bigr\} \]
their finite-dimensional distributions coincide.

A process $\bigl\{X(t)\bigr\}$ is said to be stochastically continuous at $t\geq0$ if
$\lim_{h\to 0} \pr\bigl\{ |X(t+h)-X(t)| \geq \epsilon \bigr\} = 0$ for arbitrary
$\epsilon > 0$.

It was shown by Lamperti in 1962 that whenever a self-similar process $\bigl\{X(t)\bigr\}$
has non-degenerate point distributions (non-trivial), and is stochastically continuous
at $t=0$, there necessarily exists a constant $H\geq 0$ such that any $a>0$ the constant
$b$ in the definition of self-similarity is given by $b=a^H$, i.e
\[ \bigl\{X(at)\bigr\} \overset{\Dcal}{=} \bigl\{a^H X(t)\bigr\} \]
Uniqueness of $H$ follows from the fact that $b_1 X\overset{\Dcal}{\sim} b_2 X$ implies
$b_1=b_2$ for any non-degenerate random variable $X$.

This theorem gives the definition of an $H$-\textbf{S}elf-\textbf{S}imilarity: a process
$\bigl\{X(t)\bigr\}$ is $H$-SS with Hurst exponent $H$ if for all $a>0$ it holds
\[ \bigl\{X(t)\bigr\} \overset{\Dcal}{=} \bigl\{a^{-H} X(at)\bigr\} \]

%% bulinskii2005teoriya2516755 p.105
The processes, to which the crossing tree is applied in this paper, also have stationary
increments. A stochastic processes $\bigl\{X(t)\bigr\}$ is said to have \textbf{S}tationary
\textbf{I}ncrements, or SI, if all finite-dimensional distributions of the process
$\bigl\{X(t+s) - X(t)\bigr\}_{s\geq0}$ are independent of $t$, which equivalently means
that
\[ \bigl\{X(t+s)-X(t)\bigr\} \overset{\Dcal}{=} \bigl\{X(s)-X(0)\bigr\} \]

Particularly nice corollary to the definition of an $H$-self similarity process is that
if $\bigl\{X(t)\bigr\}$ is $H$-SS then $X(t)\overset{\Dcal}{\sim}t^H X(1)$. In turn,
this allows, for example to show that all zero-mean stochastic processes with stationary
increments share a similar auto-correlation pattern for all $t\neq s$
\[ \ex X(s) X(t) = \frac{1}{2}\Bigl( t^{2H} + s^{2H} - |t-s|^{2H}\Bigr) \ex|X(1)|^2 \]
Indeed, the identity $2 a b = a^2 + b^2 - (a-b)^2$ implies that\begin{align*}
	\ex X(s) X(t)
	&= \frac{1}{2}\biggl( \ex X(s)^2 + \ex X(t)^2 - \ex\bigl( X(s) - X(t) \bigr)^2 \biggr) \\
	&= \frac{1}{2}\biggl( \ex X(s)^2 + \ex X(t)^2 - \ex X(|s-t|)^2 \biggr) \\
	&= \frac{1}{2}\biggl( |s|^{2H} \ex X(1)^2 + |t|^{2H} \ex X(1)^2 - |s-t|^{2H} \ex X(1)^2 \biggr)
\end{align*}
where the second and the third lines follow from stationarity of increments and the
mentioned corollary respectively.

%% bulinskii2005teoriya2516755 p.47
Lastly, recall that a process $\bigl\{X(t)\bigr\}$ is said to have \textbf{I}ndependent
\textbf{I}ncrements if for any integer $n\geq1$ and every $(t_k)_{k=0}^n\in[0,\infty)$
with $0=t_0$ and $t_k < t_{k+1}$ the random variables $\bigl(X(t_k) - X(t_{k-1})\bigr){k=1}^n$
and $X(0)$ are jointly independent.

% section definition (end)

\section{Fractional Brownian Motion} % (fold)
\label{sec:fractional_brownian_motion}

It was mentionend in section~\ref{sec:the_crossing_tree_of_brownian_motion} theat Brownina
Motion is an example of a self-similar process. In fact it is an $\sfrac{1}{2}$-SSSI process.
Indeed, let $a>0$ and consider a process $V(t) = \frac{1}{\sqrt{a}} B(at)$. Obviously $V(0) = 0$ almost
surely, since $V(0) = \sfrac{1}{\sqrt{a}} B(a 0) = 0$. Furthermore this process inherits
stationary increments from $B(t)$ : \begin{align*}
	\{ V(t+s) - V(t) \} &= \biggl\{ \frac{1}{\sqrt{a}}( B(at+as) - B(at) ) \biggr\}\\
	&\overset{\Dcal}{=} \biggl\{ \frac{1}{\sqrt{a}}( B(as) ) \biggr\} \\
	&= \{ V(s) \}
\end{align*}
and independence of increments as well -- just pick $(t_k)_{k=1}^n$ with $t_k<t_{k+1}$ and
observe that $\bigl(V(t_{k+1}) - V(t_k)\bigr)_{k=1}^n$ are mutually independent since
$\bigl(B(s_{k+1}) - B(s_k)\bigr)_{k=1}^n$ for $s_k = a t_k$ are. Path continuity follows
from continuity of sampel paths of $B(t)$ and the fact that $t\to a t$ is a continuous map. 
Finally $\mathcal{N}(0, at) \sim \sqrt{a} \mathcal{N}(0,t)$ implies that $V(t)$ is Brownian
Motion as well.

Another example of an $H$-SSSI process, which is frequently used as a reference for
self-similarity and scale invariance studies is the \textbf{f}ractional \textbf{B}rownian
\textbf{M}otion, -- a generalisation of the $\tfrac{1}{2}$-SS Brownian Motion to a general
$H$-SS Gaussian process with $H\in (0, 1)$.

Formally, fractional Brownian Motion introduced in \cite{doi:10.1137/1010093} is a
mean zero Gaussian process $\bigl\{X(t)\bigr\}$, (its every finite-dimensional joint
distributions are multivariate normal), with the covariance structure of a general
zero-mean $H$-SSSI stochastic process:
\[ \ex X(s) X(t) = \frac{1}{2} \bigl(t^{2H} + s^{2H} - |t-s|^{2H}\bigr) \]
%% bulinskii2005teoriya2516755 p.51

Theorem 1.3.3 on p.~6 of \cite{embrechtsselfsimilar} establishes that a fractional
Browninan Motion $\{B_h(t)\}_{t\geq 0}$ is an $H$-SSSI process, for the follwing integral
representation up to a mutliplicative constant
\[
\int_{-\infty}^0 (t-s)^{H-\tfrac{1}{2}} - (-s)^{H-\tfrac{1}{2}} dB(s)
+ \int_0^t |t-s|^{H-\tfrac{1}{2}} dB(s)
\]
If $H = 1$ then the fractional broader motion degenreates to $B_1(t) = tB(1)$ almost
surely. The class of all Fractional Brownian motions conides with the class of all
Gaussian self-similar processes with stationary incerments, see \cite{embrechtsselfsimilar}. 

In fact, fractional Brownian motion is an example of a broader class of self-similar
processes known as the \textbf{Hermite} processes. They exhibit non-Gaussianity and have
strongly dependent incerments.

% section fractional_brownian_motion (end)

\section{Hermite processes} % (fold)
\label{sec:hermite_processes}

Hermite processes inherit their name from the stochastic integral kernel used
in their definition, and are an extremenly important example of processes finite-
dimensional joint distributions of which depart significantly Gaussian. 

A probabilistic Hermite polynomial of order $k\geq0$ is defined as 
\[ H_k(x) = (-1)^k e^{-\frac{x^2}{2}} \frac{d^k}{dx^k} e^{-\frac{x^2}{2}} \]
and is a solution to the following differential equation
\[
 \frac{d}{dx}\biggl( e^{-\frac{x^2}{2}} \frac{d}{dx} f\biggr) + \lambda e^{-\frac{x^2}{2}} f = 0
\]
These polynomials constitute an orthogonal basis of the Hilbert space $\Lcal^2(\Real, \mu)$
with measure $\mu$ being the Lebesgue integral $\int e^{-\frac{x^2}{2}} dx$ and the inner
product given by
\[
\langle f, g\rangle = \int_\Real f g d\mu = \int_\Real f g e^{-\frac{x^2}{2}} dx
\]

The Hermite process of order $m$ with self-similarity parameter $H\in(\tfrac{1}{2},1)$,
denoted by $\bigl\{Z_H^p(t)\bigr\}_{t\geq 0}$, is defined via s mutlivaraive stochastic
integral
\[
Z_H^p(t) = \underset{\Real^m}{\int \cdots \int} \Biggl(
\int^t_0 \prod_{k=1}^m (u-x_k)_+^{-\frac{1}{2}-\frac{1-H}{m}} du\Biggr) dB(x_1) \ldots dB(x_q)
\]
over independent realizations of Gaussain white moise $dB(x_j)$.
The Hermite process of order $1$ is fractional Brownian Motion, whereas higher order
Hermite processes correspond to non-Gaussian $H$-SSSI, \cite{Bai20141710}.

% section hermite_processes (end)

\section{Weierstrass function} % (fold)
\label{sec:weierstrass_function}

Yet another example of non-Gaussian $H$-SSSI process is the random Weierstrass function
defined as the limiting random function of the sum of randomly phase shifted and scaled
cosines
\[
f(t)
= \sum_{k\in \mathbb{Z}} \lambda_0^{-kH}\bigl\{ \cos(\phi_k)
- \cos(2\pi \lambda_0^k t + \phi_k) \bigr\} 
\]
for $(\phi_k)_{k\in\mathbb{Z}}\sim \mathcal{U}(0,2\pi)$ iid -- the random phase shift of
the $k$-th layer harmonics. The parameter $\lambda_0$ govern the base scale of the
process and is knows as ``the fundamental harmonic''. The function $f_h(t)$ has
continuous paths, yet is nowhere differentiable, even in the deterministic case
when all phase shifts are identically zero. The Weierstrass function is known to be
$H$-self similar (see~\cite{decrouez2013estimation}).

% section weierstrass_function (end)

% chapter h_sssi_processes (end)

%% End of the report: lists of object and references
% \clearpage \listoffigures
% \clearpage \listoftables

\clearpage

\selectlanguage{english}

% \bibliographystyle{amsplain}
\bibliographystyle{ugost2008ls}
\bibliography{literature}

%% Supplementary material
\selectlanguage{english}
\chapter*{Appendix} % (fold)
\label{cha:appendix}

\section*{Details on the practical procedure} % (fold)
\label{sec:details_on_the_practical_procedure}

In the appendix a brief description of the parctical crossing tree costruction
procedure is given.

Recall that the tree is constructed in two passes: initial detection passages over
levels of $\delta \mathbb{Z}$ grid, which also discrds within-band movements, and
the pruning phase, where re-crossings are eliminated.

Before the first pass through the data the time series $x_j$ is shifted and scaled to
series $z_i = \frac{1}{ \delta }\bigl(x_j - x_0\bigr)$, so that the constructed 
is rooted at $t=0$, since the process sets off from the origin.

The first pass sweeps through consecutive increments of the series given by pairs
$(t_i, z_i)$ and $(t_{i+1}, z_{i+1})$ for $i=1, \ldots, N-1$. For each increment
its direction is determined by the sign of the difference $\Delta_i=z_{i+1}-z_i$,
and depending on it, the range of grid levels passed is computed. The range of an
increment $i$ a subset of integers $R_i = [a_i,b_i]\cup\mathbb{Z}$, where the
boundaries $a_i$ and $b_i$ are determined using the following logic
\begin{description}
    \item[Upcrossing] if $\Delta_i > 0$ then $a_i = \lceil z_i \rceil$
    and $b_i = \lfloor z_{i+1}\rfloor$;
    \item[Downcrossing] for $\Delta_i < 0$ the range given by $b_i = \lceil z_{i+1} \rceil$
    and $a_i = \lfloor z_i\rfloor$.
\end{description}
In an rare case of sideways movement, $R_i = \emptyset$. Also note that $[a,b] = \emptyset$
whenever $b<a$. During the first pass,

Movements of the normalised process $(z_i)$ that wiggle strictly within a band
between levels of the grid $\delta \mathbb{Z}$ and do not do not pass or touch
through a level have empty range, and are discarded.

The second pass checks if $b_{j-} = a_j$ for any crossing $j\in J$, where
$J = \{j : R_j\neq \emptyset\}$ and the index $j-$ for any $j\in J$ is defined as
$j- = \max\{i\in J : i < j\}$, or $-\infty$ if $j\in J$ is the very first apparent
crossing of the grid $\delta \mathbb{Z}$, and $a_{j-}$ is taken to be $-\infty$.

If the first level passed by $j\in J$ coincides with the last levels crossed by $j-$
then the very first crossing event in the $j$-th increment is a re-crossing evnet,
and thus should be eliminated. This is done by adjusting the $a_i$ in the direction
of the increment $j$. The ranges are undated accordingly and empty ones are discarded.

These passes compute the levels the sample path apparently crossed, and the crossing
times are estimated using linear interpolation between times $t_i$ and $t_{i+1}$ with
weights determined by passed level. The formula the crossing time of level $l\in R_i$
during the $i$-th increment is
\[
\tau_{il} = t_i + \bigl(t_{i+1} - t_i\bigr) \frac{l - z_i}{z_{i+1} - z_i}
\]

The crossing times of a coarser resolution grid are computed based on the 
data provided the next finer resolution in the same way the second pass functions:
since the resolution is halved between successive tree levels, the crossing times
estimated at resolution $\delta 2^{n+1}$ are by construction a subset of crossing
times of a grid $\delta 2^n$. Inded, if an increment $j$ passed a line $m \delta 2^{n+1}$
then this very same passed a line $2m \delta 2^n$ of the finer grid and the interpolation
coefficient used to compute the crossing time is unchanged
\begin{align*}
	\frac{m - \frac{x_j - x_0}{\delta 2^{n+1}}}{ \frac{x_{j+1} - x_j}{\delta 2^{n+1}} }
	&= \frac{2m - \frac{x_j - x_0}{\delta 2^n}}{ \frac{x_{j+1} - x_j}{\delta 2^n} } \\
\end{align*}
Thus, since the incrementis the same, the corrins time reamins the same.

% section* details_on_the_practical_procedure (end)

% chapter* appendix (end)


\end{document}
