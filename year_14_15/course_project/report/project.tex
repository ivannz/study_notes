\documentclass[a4paper]{report}

\usepackage[utf8]{inputenc}
\usepackage[russian, english]{babel}

% \usepackage{fullpage}
% \linespread{1.5}

\usepackage{graphicx, url}
\usepackage{amsmath, amsfonts, xfrac}
\usepackage{mathtools}

% \usepackage{natbib}

\newcommand{\Real}{\mathbb{R}}
\newcommand{\Cplx}{\mathbb{C}}

\newcommand{\pr}{\mathbb{P}}
\newcommand{\ex}{\mathbb{E}}
\newcommand{\var}{\text{var}}
\newcommand{\Pcal}{\mathcal{P}}

\newcommand{\defn}{\mathop{\overset{\Delta}{=}}\nolimits}
\newcommand{\lpto}{\mathop{\overset{L^p}{\to}}\nolimits}

\newcommand{\re}{\operatorname{Re}\nolimits}
\newcommand{\im}{\operatorname{Im}\nolimits}

\newcommand{\eng}[1]{\foreignlanguage{english}{#1}}

\newcommand{\rus}[1]{\foreignlanguage{russian}{#1}}
% \selectlanguage{english}

\title{Studying Self-similar Processes Using the Croosing Tree}
\author{Nazarov Ivan, \rus{101мНОД(ИССА)}}

\begin{document}
\pagenumbering{gobble}
%% Russian title page
\selectlanguage{russian}

\begin{titlepage}
    \selectlanguage{russian}
    \thispagestyle{empty}
    \vbox to \textheight {
        \renewcommand{\baselinestretch}{1}\selectfont
        \begin{center}
            \textsc{\LARGE
            Национальный Исследовательский Университет\\[0.5cm]
            Высшая Школа Экономики}\\[1.5cm]

            \textsc{\Large
            Магистерская программа Науки о Данных}\\[0.5cm]

            \rule{\linewidth}{0.5mm}\\[1.0cm]

            {\huge \bfseries Курсовая работа}\\[0.5cm]
            {\large \bfseries на тему}\\[0.5cm]
            {\huge \bfseries ``Исследование самоподобных процессов с помощью дерева пересечений''}\\[0.5cm]
        \end{center}

        \vspace{2.0cm}

        \begin{flushright}
            \large Иван \textsc{Назаров}\\[0.5cm]
            \rus{101мНОД(ИССА)}\\[3cm]
        \end{flushright}

        \vspace{2.0cm}

        \vfill
        \begin{center}
            Москва\\
            2015\\[3cm]
            % \includegraphics{hsecmyk}\\[1cm]
        \end{center}
    }
\end{titlepage}

\clearpage

%% English title page
\selectlanguage{english}
\begin{titlepage}
    \selectlanguage{english}
    \thispagestyle{empty}
    \vbox to \textheight {
        \renewcommand{\baselinestretch}{1}\selectfont
        \begin{center}
            \textbf{%\Large
            Government of Russian Federation\\[0.5cm]
            Federal State Autonomous Educational Institution of Higher Professional Education\\[0.5cm]
            National Research University\\[0.5cm]
            ``Higher School of Economics''}\\[1.5cm]

            \textbf{%\Large
            Faculty of Computer Science\\[0.5cm]}
        \end{center}
        \textbf{%\Large
            Master's programme in Data Science\\[0.5cm]
            Department of Complex Systems Modelling Technologies\\[1.0cm]}
            
            \begin{center}
            {\large \bfseries COURSE PROJECT}\\[1.0cm]
            \end{center}
            {\large \bfseries On ``Studying Self-similar Processes Using the Crossing Tree''}\\[0.5cm]
        %\end{center}

        \vspace{2.0cm}

        \begin{flushright}
        	Student of group 101m \\
            Ivan Nazarov\\[0.5cm]
            Scientific Advisor\\
            Geoffrey G. Decrouez\\[3cm]
        \end{flushright}

        \vspace{2.0cm}

        \vfill
        \begin{center}
            Moscow, 2015\\[3.0cm]
            % \includegraphics{hsecmyk}\\[1cm]
        \end{center}
    }
\end{titlepage}

\clearpage

%% Draft title page
\selectlanguage{english}
\maketitle
\begin{abstract}
\textbf{Shamelessly copied from Geoffrey's project ptich}.\hfill \\
Time-series data presenting scale invariance do not posses a well-defined time scale.
Instead, their dynamics are understood when studied across a whole range of scales.
Examples of data with empirical scale-invariance include network traffic, financial
time-series, and other natural phenomena in physics and biology. The crossing-tree
is a recent tool to analyze this kind of signals. It provides an ad-hoc representation
of the data which is adapted to its dynamics, and thus represents an alternative to
wavelet decompositions. In this project, the student will first learn about scale
invariance, and how the crossing-tree has been used previously as a tool to analyze
self-similar signals. The next step is to analyze self-similar processes with stationary
increments (known as H-SSSI processes) using the crossing tree. It is expected that
for this class of processes, the crossing tree presents common features which need
to be extracted.
\end{abstract}

\selectlanguage{english}
\tableofcontents
\clearpage
\pagenumbering{arabic}

%% The project itself
\selectlanguage{english}

\chapter*{Introduction} % (fold)
\label{cha:introduction}

% chapter introduction (end)

\chapter{Crossing tree formalism} % (fold)
\label{cha:crossing_tree_formalism}

\section{Stochastic self-similarity} % (fold)
\label{sec:stochastic_self_similarity}

\subsection{General definition} % (fold)
\label{sub:general_definition}

% subsection general_definition (end)

\subsection{Hurst self-similar stochastic processes} % (fold)
\label{sub:hurst_self_similar_stochastic_processes}

% subsection hurst_self_similar_stochastic_processes (end)

% section stochastic_self_similarity (end)

\section{the Crossing tree} % (fold)
\label{sec:the_crossing_tree}

\subsection{Definition} % (fold)
\label{sub:definition}
Cite the orginal papers \cite{jones2004} and \cite{jonesshen2005}.

% subsection definition (end)

\subsection{Practical construction} % (fold)
\label{sub:practical_construction}

% subsection practical_construction (end)

% section the_crossing_tree (end)

\section{Fractional Brownian Motion} % (fold)
\label{sec:fractional_brownian_motion}

\subsection{Definiton} % (fold)
\label{sub:definiton}
Brief history, relation to H-SSSI processes.

Mention that BM is fBM for $H=0.5$.

\subsubsection{Sample paths} % (fold)
\label{ssub:sample_paths}

Reference the generation circulant embedding generation algorithm with complexity.

% subsubsection sample_paths (end)

% subsection definiton (end)

% section fractional_brownian_motion (end)

\section{Brownian Motion} % (fold)
\label{sec:brownian_motion}

\section{Literature review} % (fold)
\label{sec:literature_review}
Reviewed papers \cite{jones2004}, \cite{jonesshen2005} and \cite{decrouez2013}.

% section literature_review (end)

\subsection{Properties of the crossing tree} % (fold)
\label{sub:properties_of_the_crossing_tree}

% subsection properties_of_the_crossing_tree (end)

\subsection{Simulation study} % (fold)
\label{sub:simulation_study_bm}

% subsection simulation_study_bm (end)

% section brownian_motion (end)

\section{Conjecture for FBM with $H\in (\sfrac{1}{2},1)$} % (fold)
\label{sec:conjecture_for_fbm}

\subsection{Statement} % (fold)
\label{sub:statement}

% subsection statement (end)

\subsection{Simulation study} % (fold)
\label{sub:simulation_study_fbm}

See the paper by Owen Jones and Shen (2005) for the outline.

Each MonteCarlo realisation generates a discretized sample path $(t_i, X_i)_{i=0}^N$ of
a particular continuous stochastc process $X(t)$, where $X_i = X(t_i)$ and $(t_i)_{i=0}^N\in [0,1]$
is a uniformly spaced mesh with
\[0 = t_0 \geq \ldots \geq t_i < t_{i+1} \geq \ldots \geq t_N = 1\]

Main plots: let $\Delta X_i = X_i - X_{i-1}$ for $i=1,\ldots, N$.
\begin{itemize}
	\item for $\delta = \text{std}\bigl(\Delta X_i \bigr)$, where $\text{std}(Y^n)$ is
	the square root of the unbiased sample estimator of varaince of $Y$:
	\[ \text{std}(Y^n) = \sqrt{ \frac{1}{n-1} \sum_{i=1}^n \bigl( Y_i - \bar{Y}_n \bigr)^2 }\]
	and $\bar{Y}_n$ is tha sample mean: $\bar{Y}_n = \frac{1}{n}\sum_{i=1}^n Y_i$. 

	\item for $\delta = \text{iqr}\bigl(\Delta X_i \bigr)$, where $\text{iqr}(Y^n)$ is
	the \textbf{i}nter\textbf{q}uartile \textbf{r}ange of the sample $Y^n = (Y_i)_{i=1}^n$
	defiend as the difference of the $75\%$ and the $25\%$ sample quartiles of the sample:
	\[\text{iqr} = \hat{F}^{-1}_n\Bigl(\frac{3}{4}\Bigr) - \hat{F}^{-1}_n\Bigl(\frac{1}{4}\Bigr)\]
	where $\hat{F}^{-1}_n(p)$ is the generalized inverse of the empirical CDF of the sample $Y^n$
	given by
	\[\hat{F}_n(y) = \frac{1}{n} \sum_{i=1}^n 1_{(-\infty,y]}(Y_i)\]
	Ratinale for the IQR is that it is robust (this is poor!).
	\item for $\delta = \hat{F}^{-1}_n\Bigl(\frac{3}{4}\Bigr)$ -- the $75\%$ empirical qauntile of
	the process of increments $(\Delta_i)_{i=1}^N$.
\end{itemize}

Each group of plots should have: \begin{itemize}
	\item a plot of subcrossing distribution averaged across the MC realisations with
	the theoretical values;
	\item add histograms of $\pr(Z_n = 2k)$ (separately for $k=1,2,\ldots$) with
	superimposed theoretical $\theta = 2^{1-H^{-1}}$ and averaged values;
	$Z\sim\text{Geom}$ with porbability given by
	\[\pr(Z = 2k) = (1-\theta)^{k-1} \theta\]
	for $k\geq 1$.
	\item a table of excursion distribution averaged across the MC realisations;
	\item add histograms of $\pr(+-|++)$ and $\pr(+-|--)$ with superimposed theoretical
	and averaged values. Let $\mu = \ex Z$, which is $\frac{2}{\theta}$. Then the
	hypothesized probability of an up-down excursion $\delta 2^n$ in an upcrossing
	of resolution $\delta 2^{n+1}$ is $\frac{1}{\sqrt{\mu}}$.
\end{itemize}

% subsection simulation_study_fbm (end)

\subsection{Discussion} % (fold)
\label{sub:discussion}

% subsection discussion (end)

% section conjecture_for_fbm (end)

%% End of the report: lists of object and references
% \clearpage \listoffigures
% \clearpage \listoftables

\clearpage

\selectlanguage{english}

% \bibliographystyle{amsplain}
\bibliographystyle{ugost2008ls}
\bibliography{literature}

%% Supplementary material
% \selectlanguage{english}
% \chapter*{Appendix} % (fold)
\label{cha:appendix}

\section*{Details on the practical procedure} % (fold)
\label{sec:details_on_the_practical_procedure}

In the appendix a brief description of the parctical crossing tree costruction
procedure is given.

Recall that the tree is constructed in two passes: initial detection passages over
levels of $\delta \mathbb{Z}$ grid, which also discrds within-band movements, and
the pruning phase, where re-crossings are eliminated.

Before the first pass through the data the time series $x_j$ is shifted and scaled to
series $z_i = \frac{1}{ \delta }\bigl(x_j - x_0\bigr)$, so that the constructed 
is rooted at $t=0$, since the process sets off from the origin.

The first pass sweeps through consecutive increments of the series given by pairs
$(t_i, z_i)$ and $(t_{i+1}, z_{i+1})$ for $i=1, \ldots, N-1$. For each increment
its direction is determined by the sign of the difference $\Delta_i=z_{i+1}-z_i$,
and depending on it, the range of grid levels passed is computed. The range of an
increment $i$ a subset of integers $R_i = [a_i,b_i]\cup\mathbb{Z}$, where the
boundaries $a_i$ and $b_i$ are determined using the following logic
\begin{description}
    \item[Upcrossing] if $\Delta_i > 0$ then $a_i = \lceil z_i \rceil$
    and $b_i = \lfloor z_{i+1}\rfloor$;
    \item[Downcrossing] for $\Delta_i < 0$ the range given by $b_i = \lceil z_{i+1} \rceil$
    and $a_i = \lfloor z_i\rfloor$.
\end{description}
In an rare case of sideways movement, $R_i = \emptyset$. Also note that $[a,b] = \emptyset$
whenever $b<a$. During the first pass,

Movements of the normalised process $(z_i)$ that wiggle strictly within a band
between levels of the grid $\delta \mathbb{Z}$ and do not do not pass or touch
through a level have empty range, and are discarded.

The second pass checks if $b_{j-} = a_j$ for any crossing $j\in J$, where
$J = \{j : R_j\neq \emptyset\}$ and the index $j-$ for any $j\in J$ is defined as
$j- = \max\{i\in J : i < j\}$, or $-\infty$ if $j\in J$ is the very first apparent
crossing of the grid $\delta \mathbb{Z}$, and $a_{j-}$ is taken to be $-\infty$.

If the first level passed by $j\in J$ coincides with the last levels crossed by $j-$
then the very first crossing event in the $j$-th increment is a re-crossing evnet,
and thus should be eliminated. This is done by adjusting the $a_i$ in the direction
of the increment $j$. The ranges are undated accordingly and empty ones are discarded.

These passes compute the levels the sample path apparently crossed, and the crossing
times are estimated using linear interpolation between times $t_i$ and $t_{i+1}$ with
weights determined by passed level. The formula the crossing time of level $l\in R_i$
during the $i$-th increment is
\[
\tau_{il} = t_i + \bigl(t_{i+1} - t_i\bigr) \frac{l - z_i}{z_{i+1} - z_i}
\]

The crossing times of a coarser resolution grid are computed based on the 
data provided the next finer resolution in the same way the second pass functions:
since the resolution is halved between successive tree levels, the crossing times
estimated at resolution $\delta 2^{n+1}$ are by construction a subset of crossing
times of a grid $\delta 2^n$. Inded, if an increment $j$ passed a line $m \delta 2^{n+1}$
then this very same passed a line $2m \delta 2^n$ of the finer grid and the interpolation
coefficient used to compute the crossing time is unchanged
\begin{align*}
	\frac{m - \frac{x_j - x_0}{\delta 2^{n+1}}}{ \frac{x_{j+1} - x_j}{\delta 2^{n+1}} }
	&= \frac{2m - \frac{x_j - x_0}{\delta 2^n}}{ \frac{x_{j+1} - x_j}{\delta 2^n} } \\
\end{align*}
Thus, since the incrementis the same, the corrins time reamins the same.

% section* details_on_the_practical_procedure (end)

% chapter* appendix (end)


\end{document}
