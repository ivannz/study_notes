\chapter{the Crossing tree and $H$-SSSI processes} % (fold)
\label{cha:the_crossing_tree_and_h_sssi_processes}

Inspired by the simplicity of the statistical properties of the crossing tree for
Brownian Moitons Decrouez in~\cite{DecrouezConjecture} formulated the following
conjecture: whenever $\{X(t)\}$ is a continuous $H$-SSSI process, then $X(0)= 0$
almost surely and the process is uniquely identified by the statistical properties
of the crossing tree, namely the crossing durations, the excursion and the offspring
distributions in the following way: \begin{itemize}
	\item the properly scaled crossing durations $\tfrac{1}{\delta^2 4^n} W_k^n$
    are identically distributed with mean 1 and finite variance;
    \item $Z_k^n\sim 2+2\cdot\text{Geom}(2^{1-H^{-1}})$ are identically distributed,
    but possibly correlated;
    \item the $V_k^n \sim \text{Bern}(\frac{1}{\sqrt{\mu}})$ for $\mu = \ex Z_k^n$.
\end{itemize}

In order to gather empirical evidence either supporting or refuting this claim,
extensive numerical study was in order. The Monte-Carlo simulation was performed
on each of the processes mentioned in the previous chapter (~\ref{cha:chapter_3}).
The software part of the experiment was implemented in Python in conjunction with
Numpy and FFTW, a standalone open source library dedicated to computing Fast Fourier
Transforms.\footnote{The source code of the developed toolkit for crossing tree
construction and analysis is publicly available online at
\url{https://github.com/ivannz/study_notes/tree/master/year_14_15/course_project/code/project}}

% chapter the_crossing_tree_and_h_sssi_processes (end)

\section{Generation of fBM, Hermite and Weierstrass processes} % (fold)
\label{sec:generation_of_fbm_hermite_and_weierstrass_processes}

The basic building block of the FBM and the Hermite processes is Gaussian noise
with particular correlation structure, as evidence by their stochastic integral representations.

Generation of the fractional Brownian motion for this study is based on the Circulant
Embedding method of Dietrich and Newsam for generating fractional Gaussian noise,
(\cite{WRCR:WRCR6232}). In short, the method utilizes the structure of the correlation
matrix of fGN to embed it into a larger circulant Toeplitz matrix, suitable for
imposing the necessary covariance structure upon independent standard normal random
variables via standard forward Fast Fourier Transform. For details the reader is
encouraged to refer to the original paper~\cite{WRCR:WRCR6232} and a more recent
one~\cite{Perrin:1058211}.

As for the Hermite processes, synthesis of their sample path was based on the fundamental
theorem of Lamperti (\cite{lamperti}) which states that $H$-sssi process is the only 
limiting law of normalized partial sum of a stationary random sequence. Formally,
if $(X_i)_{i\geq1}$ is stationary and for some regularly varying function $a_n\to \infty$
and the following limiting exists
\[ \frac{1}{a_n} \sum_{i=1}^{[nt]} X_i \overset{\Dcal}{\rightarrow} y(t) \]
with convergence is in all finite-dimensional distributions, then the limiting
process $(Y_t)_{t\geq0}$ is $H$-SSSI for some $H>0$.

For instance, if $(X_i)_{i\geq 1}$ is an independent and identically distributed,
sequence then the limit of normed partial sums is the Brownian Motion, which is
$\frac{1}{2}$-SSSI. In case if the stationary sequence $X_i$ is \textbf{l}ong-
\textbf{r}ange \textbf{d}ependent, i.e. with slowly decaying autocorrelation,
the limit $Y(t)$ is often $H$-SSSI with $H>\frac{1}{2}$.

Sample paths of the Weierstrass process were, simulated on a uniformly spaced grid
$(t_k)_{i=0}^N\in [0,1]$ with $0 = t_0< t_1 < \ldots < t_{N-1} < t_N = 1$. The spacing
was chosen so that the trigonometric functions are adequately sampled over the unit interval
accrrding to suggestions given in~\cite{decrouez2013estimation}.

% section generation_of_fbm_hermite_and_weierstrass_processes (end)

\section{Simulation results} % (fold)
\label{sec:results}

In order to study the empirical support, or lack thereof, of the hypothesised properties
of the crossing tree as well as see how accurate in practice the crossing tree methodology
confirms the theoretical result in \cite{ECP1673}, an extensive Monte Carlo experiment
was preformed.

The crossing trees were constructed for a base scale dependent on the particular
realisation of the process, but in such a way as to enable meaningful comparisons
of the crossing tree data between different Monte-Carlo replications and between
different classes of $H$-SSSI processes. For a particular sample path $(x_j)_{j=0}^N$
of the process $\{X(t)\}$ the base scale was set to \[
\delta = \text{med}\bigl( |\Delta x_j| \bigr)
\]
where $\Delta x_j = x_j-x_{j-1}$ and $\text{med}(\cdot)$ returns the median of
the sample. The rationale behind the median of absolute increments of the sample
path was to strike a balance between the biasedness of the crossing tree parameters,
resulting from linear interpolation of the crossing times and the inaccuracy of
the due to too coarse a resolution. Other choices for the base scale were considered,
such as the standard deviation and the \textbf{i}nter\textbf{q}uartile \textbf{r}ange
measure of statistical dispersion of the increments. Both did not produce any significantly
different results from the median, except only that they tended to produce trees with
fewer levels and did not study the process at enough resolutions.

It seems natural to begin with the study of the fractional Brownian Motion. To this
end 1000 Monte-Carlo simulations of sample paths of the FBM process were simulated.
The process was confined to the unit interval and discretized to have $2^{21}$ points.

Before proceeding to evaluation of statistical properties of the crossing trees,
it is necessary to determine the range of tree levels for which self-similarity
is apparent. 
\begin{figure}[htb]\begin{center}
    \includegraphics[width=6in]{images/fbm-fig_05_med_1000-21}
    \caption{The plot of the estimated of the Hurst exponent $H$ based on the offspring
    data of a single level of $1000$ Monte-Carlo simulations of fBm.}
\label{fig:fbm_hurst_crossing_tree}
\end{center}\end{figure}

Figure~\ref{fig:fbm_hurst_crossing_tree} suggests that scale-invariance properties
for the studied fractional processes and the chosen method of computation of the base
scale manifest their effects at across levels from 7 to 8. In fact the $\chi^2$ test
described in section~\ref{sec:analysis_using_the_crossing_tree} seems to have the lowest
empirical rejection rate exactly at these levels (see table~\ref{tbl:chi_sq_test_for_fbm_only}).
\begin{table}[h]\begin{center}
	\begin{tabular}{l||c|c|c|c|c|c|}
	Process 		&  $6-7$ & $7-8$ & $8-9$ &  $6-8$ &  $7-9$ &  $6-9$ \\ \hline\hline
	FBM-$0.50$ 		& $10.6$ & $\mathbf{8.2}$ & $9.1$ & $13.6$ & $12.2$ & $16.2$ \\ \hline 
	FBM-$0.60$ 		& $10.9$ & $\mathbf{9.1}$ & $9.4$ & $14.2$ & $13.1$ & $16.4$ \\ \hline 
	FBM-$0.70$ 		&  $9.4$ & $\mathbf{7.0}$ & $7.2$ & $12.6$ & $11.2$ & $15.8$ \\ \hline 
	FBM-$0.80$ 		& $10.3$ & $\mathbf{6.1}$ & $6.7$ & $12.8$ & $10.7$ & $16.9$ \\ \hline 
	FBM-$0.90$ 		& $11.1$ & $\mathbf{7.8}$ & $7.9$ & $18.0$ & $12.8$ & $27.7$ \\ \hline 
 	\end{tabular}
	\caption{The table of empirical rejection rate at significance level of $\alpha = 5\%$
	of the $\chi^2$ test for self-similarity between levels of the crossing tree. }
\label{tbl:chi_sq_test_for_fbm_only}
\end{center}\end{table}
Since the self-similarity of the simulated fractional Brownian Motion paths seems
to become apparent at levels 7 and 8 on average, is seems logical to pool the crossing
tree data in order to obtain more reliable estimates.

\begin{figure}[htb]\begin{center}
    \includegraphics[width=6in]{images/fig_01_med_FBM_1000-21}
    \caption{The log-plot of the offspring distributions estimated on 1000 sample discrete paths
    of the fBm process of length $2^{21}+1$ and Hurst exponents in the range from $0.5$ to $0.9$.}
\label{fig:fbm_offspring_distribution}
\end{center}\end{figure}

Restricting the attention to the Brownian Motion case (fBm with $H=\tfrac{1}{2}$)
one can easily see that the numerical evidence is well aligned with the theoretical
result of Jones and Rolls (see~\ref{fig:fbm_offspring_distribution}). Similarly, the
theoretical probability of an up-down crossing conditional on an upcrossing is matched
very closely by the numerical evidence (see~\ref{fig:fbm_offspring_up_down}). The
results are similar in the down-up excursions' case (see~\ref{fig:fbm_offspring_down_up}).

\begin{figure}[htb]\begin{center}
    \includegraphics[width=6in]{images/fbm-fig_03_up-down_med_1000-21}
    \caption{The estimated conditional probability of an up-down excursion given upward
    orientation of the parent crossing.}
\label{fig:fbm_offspring_up_down}
\end{center}\end{figure}

\begin{figure}[htb]\begin{center}
    \includegraphics[width=6in]{images/fbm-fig_03_down-up_med_1000-21}
    \caption{The box-plot of the estimates of the probability of an down-up excursion
    conditional on the orientation of the parent crossing being downward.}
\label{fig:fbm_offspring_down_up}
\end{center}\end{figure}

As for the conjectured distribution, the offspring distribution in the performed
experiments seems to suggest that the levels of the tree, at least for the fractional
Brownian motion, tend to be populated by crossings with more than predicted number
of subcrossings.

Now let's turn to other $H$-SSSI processes: Hermite and Weierstrass. This time $10^4$
random replications of sampele paths of size $2^{17}$ were generated for each processes.
This size limitation was dictated by deteriorating numerical accuracy of the procedure,
responsible for generating Hermite processes. Thus in order to render results comparable,
sample paths of FBM and Weierstrass processes were limited to $2^{17}$ points as well. 

Figure~\ref{fig:all_hurst_crossing_tree} suggests that the processes studied share common
statistical properties of the crossing tree.
\begin{figure}[htb]\begin{center}
    \includegraphics[width=6in]{images/fig_05_med_10000-17_7-8}
    \caption{Estimates of the Hurst exponent $H$ based on a single level of the crossing tree for
    the all stidued $H$-SSSI processes.}
\label{fig:all_hurst_crossing_tree}
\end{center}\end{figure}

The following table~\ref{tbl:chi_sq_test_for_all_01} summarizes the empirical rejection
rates of the $\chi^2$ test for self similarity across the level indicated in the header
of the table.
\begin{table}[h]\begin{center}
	\begin{tabular}{l||c|c|c|c|c|c|}
	Process 		& $6-7$ &  $7-8$ &  $8-9$ &  $6-8$ &  $7-9$ &  $6-9$ \\ \hline\hline
	  FBM-$0.50$	& $9.7$ & $15.7$ &     -- & $13.8$ &  $\mathbf{3.6}$ &  $5.4$ \\ \hline
	  FBM-$0.60$	& $8.8$ &  $9.3$ &  $\mathbf{5.6}$ & $14.2$ & $14.3$ & $16.5$ \\ \hline
	  FBM-$0.70$	& $7.6$ &  $\mathbf{6.6}$ & $10.5$ & $13.4$ & $15.7$ & $17.8$ \\ \hline
	  FBM-$0.80$	& $6.5$ &  $4.8$ &  $\mathbf{4.5}$ & $11.5$ & $13.1$ & $16.5$ \\ \hline
	  FBM-$0.90$	& $\mathbf{4.5}$ &  $5.1$ &  $6.9$ & $11.7$ & $13.6$ & $19.8$ \\ \hline\hline

	  WEI-$0.50$	& $9.6$ & $12.4$ &     -- & $13.9$ &  $\mathbf{2.2}$ &  $5.1$ \\ \hline
	  WEI-$0.60$	& $\mathbf{8.2}$ &  $9.6$ & $11.1$ & $13.2$ & $14.8$ & $15.5$ \\ \hline
	  WEI-$0.70$	& $7.0$ &  $\mathbf{5.9}$ &  $\mathbf{5.9}$ & $12.9$ & $14.6$ & $15.8$ \\ \hline
	  WEI-$0.80$	& $6.2$ &  $5.5$ &  $\mathbf{4.9}$ & $13.1$ & $13.2$ & $17.0$ \\ \hline
	  WEI-$0.90$	& $5.0$ &  $\mathbf{1.9}$ &  $8.7$ & $12.2$ &  $9.7$ & $21.0$ \\ \hline\hline

	HRM-2-$0.60$ 	& $9.6$ &  $9.2$ &  $\mathbf{7.1}$ & $15.2$ & $15.9$ & $19.6$ \\ \hline
	HRM-2-$0.70$ 	& $8.5$ &  $\mathbf{7.4}$ & $11.3$ & $13.9$ & $15.1$ & $18.2$ \\ \hline
	HRM-2-$0.80$ 	& $6.5$ &  $\mathbf{5.4}$ &  $5.6$ & $12.5$ & $12.3$ & $17.8$ \\ \hline
	HRM-2-$0.90$ 	& $6.4$ &  $\mathbf{4.9}$ &  $0.0$ & $12.7$ & $14.5$ & $18.7$ \\ \hline\hline

	HRM-3-$0.60$ 	& $9.6$ &  $\mathbf{9.2}$ & $13.7$ & $15.6$ & $17.9$ & $20.3$ \\ \hline
	HRM-3-$0.70$ 	& $\mathbf{7.9}$ &  $8.1$ &  $\mathbf{7.9}$ & $14.5$ & $16.4$ & $19.2$ \\ \hline
	HRM-3-$0.80$ 	& $5.5$ &  $\mathbf{4.3}$ &  $5.4$ & $11.8$ & $13.1$ & $16.7$ \\ \hline
	HRM-3-$0.90$ 	& $\mathbf{5.6}$ &  $7.7$ & $12.5$ & $13.4$ & $14.1$ & $19.8$ \\ \hline\hline

	HRM-4-$0.60$ 	& $9.7$ &  $\mathbf{9.3}$ & $12.3$ & $16.5$ & $17.4$ & $21.4$ \\ \hline
	HRM-4-$0.70$ 	& $8.1$ &  $7.2$ &  $\mathbf{6.0}$ & $15.4$ & $16.8$ & $20.4$ \\ \hline
	HRM-4-$0.80$ 	& $5.7$ &  $7.4$ &  $\mathbf{1.8}$ & $11.7$ & $13.7$ & $16.8$ \\ \hline
	HRM-4-$0.90$ 	& $5.1$ &  $\mathbf{2.9}$ &  $8.3$ &  $9.8$ & $12.9$ & $19.0$ \\ \hline\hline

 	\end{tabular}
	\caption{The table of empirical rejection rate at significance level of $\alpha = 5\%$
	of the $\chi^2$ test for self-similarity between levels of the crossing tree. }
\label{tbl:chi_sq_test_for_all_01}
\end{center}\end{table}
There does not seem to be a common range of levels, at the corresponding resolutions of which
all processes exhibit scale-invariance. This might be attributed to the fact that the path
of the generated processes were insufficiently long to adequately populate the higher
levels of the crossing tree. Nevertheless, it seems reasonable to expect a certain degree
of self-similarity over level 7 to 8, which with due caution, could yield empirical evidence
regarding the conjecture.

Indeed, the offspring distribution plot (\ref{fig:all_xing_probs}) seems to suggest that
even though the theoretical distribution seems to underestimate the real probability of
a crossing of a particular size (in the number of subcrossings), there is still evidence
for similarity of the properties of the crossing tree across different $H$-SSSI processes.
\begin{figure}[htb]\begin{center}
    \includegraphics[width=6in]{images/fig_02_med_10000-17_7-8}
    \caption{Empirical probabilities of the crossing sizes against the hypothesized
    theoretical probabilities for $10^4$ sample paths of length $2^{17}$.}
\label{fig:all_xing_probs}
\end{center}\end{figure}

The tables \ref{tbl:empirical_probs_01} anf \ref{tbl:empirical_probs_02} compare the 
conjectured probabilities against the empirical ones. Though there is no strong evidence
for the conjecture, one cannot say that it should be discarded. Indeed, looking at
the box-plots (\ref{fig:all_offspring_down_up} and \ref{fig:all_offspring_up_down})
of the conditional distribution of excursions, it is possible to see that even
though there is significant margin of error there the probabilities seem to agree
quite well with the hypothesised distribution.
\begin{table}[h]\begin{center}
	\begin{tabular}{l||l|l|l|l|}
					$H=0.6$ & $Z_k = 2$ & $Z_k = 4$ & $Z_k = 6$ & $Z_k = 8$ \\ \hline\hline
	\multirow{2}{*}{FBM} 	& $0.630$ & $0.233$ & $0.086$ & $0.032$ \\ \cline{2-5}
							& $0.644\pm0.033$ & $0.224\pm0.027$ & $0.083\pm0.017$ & $0.031\pm0.011$ \\ \hline\hline
	\multirow{2}{*}{WEI} 	& $0.630$ & $0.233$ & $0.086$ & $0.032$ \\ \cline{2-5}
							& $0.642\pm0.027$ & $0.226\pm0.026$ & $0.084\pm0.017$ & $0.031\pm0.010$ \\ \hline\hline
	\multirow{2}{*}{HRM-2} 	& $0.630$ & $0.233$ & $0.086$ & $0.032$ \\ \cline{2-5}
							& $0.663\pm0.034$ & $0.215\pm0.026$ & $0.078\pm0.015$ & $0.028\pm0.009$ \\ \hline\hline
	\multirow{2}{*}{HRM-3} 	& $0.630$ & $0.233$ & $0.086$ & $0.032$ \\ \cline{2-5}
							& $0.666\pm0.038$ & $0.214\pm0.026$ & $0.076\pm0.015$ & $0.027\pm0.009$ \\ \hline\hline
	\multirow{2}{*}{HRM-4} 	& $0.630$ & $0.233$ & $0.086$ & $0.032$ \\ \cline{2-5}
							& $0.667\pm0.039$ & $0.215\pm0.025$ & $0.076\pm0.016$ & $0.027\pm0.009$ \\ \hline\hline
	\end{tabular}
	\caption{The table of empirical probabilities of the first four values of the number
	of subcrossings in a parent crossing for $H$-SSSI processes with $H=0.6$. Levels from
	6 to 8 were pooled to get the estimates.}
\label{tbl:empirical_probs_01}
\end{center}\end{table}

\begin{table}[h]\begin{center}
	\begin{tabular}{l||l|l|l|l|}
					$H=0.8$ & $Z_k = 2$ & $Z_k = 4$ & $Z_k = 6$ & $Z_k = 8$ \\ \hline\hline
	\multirow{2}{*}{FBM} 	& $0.841$ & $0.134$ & $0.021$ & $0.003$ \\ \cline{2-5}
 							& $0.855\pm0.023$ & $0.112\pm0.017$ & $0.026\pm0.006$ & $0.006\pm0.002$ \\ \hline\hline
	\multirow{2}{*}{HRM-2} 	& $0.841$ & $0.134$ & $0.021$ & $0.003$ \\ \cline{2-5}
 							& $0.848\pm0.027$ & $0.118\pm0.022$ & $0.026\pm0.006$ & $0.005\pm0.002$ \\ \hline\hline
	\multirow{2}{*}{HRM-3} 	& $0.841$ & $0.134$ & $0.021$ & $0.003$ \\ \cline{2-5}
 							& $0.846\pm0.024$ & $0.121\pm0.019$ & $0.026\pm0.006$ & $0.005\pm0.002$ \\ \hline\hline
	\multirow{2}{*}{HRM-4} 	& $0.841$ & $0.134$ & $0.021$ & $0.003$ \\ \cline{2-5}
 							& $0.844\pm0.021$ & $0.122\pm0.016$ & $0.026\pm0.006$ & $0.005\pm0.002$ \\ \hline\hline
	\multirow{2}{*}{WEI} 	& $0.841$ & $0.134$ & $0.021$ & $0.003$ \\ \cline{2-5}
 							& $0.854\pm0.019$ & $0.113\pm0.015$ & $0.026\pm0.005$ & $0.006\pm0.002$ \\ \hline\hline
	\end{tabular}
	\caption{The table of empirical probabilities of the first four values of the number
	of subcrossings in a parent crossing for $H$-SSSI processes with $H=0.8$. Levels from
	6 to 8 were pooled to get the estimates.}
\label{tbl:empirical_probs_02}
\end{center}\end{table}

\begin{figure}[htb]\begin{center}
    \includegraphics[width=6in]{images/fig_03_up-down_med_10000-17_7-8}
    \caption{The empirical estimates of the probability of an up-down excursion conditional on
    the upward orientation of the parent crossing.}
\label{fig:all_offspring_up_down}
\end{center}\end{figure}

\begin{figure}[htb]\begin{center}
    \includegraphics[width=6in]{images/fig_03_down-up_med_10000-17_7-8}
    \caption{The same figure as \ref{fig:all_offspring_up_down} but for down-up excursions
    conditional on the orientation of the parent crossing being downward.}
\label{fig:all_offspring_down_up}
\end{center}\end{figure}

% section results (end)

