\chapter{the Crossing Tree} % (fold)
\label{cha:the_crossing_tree}

The crossing tree as a tool for analysing self-similarity was introduced by Jones
and Shen in \cite{jones2004}, where it was successfully used in detecting structural
changes in time-series data, which manifested in a change in the Hurst exponent.
In this section I describe the theoretic construct of the crossing tree and how it
is built on sample data.

\section{Crossing Times} % (fold)
\label{sec:crossing_times}

Consider a path of real-valued continuous process $\{X(t)\}$. At the heart of the
crossing tree lie crossing times of a uniformly spaced grid on $\Real$ given by
$X(0) + \delta 2^n \mathbb{Z}$ for some $n\geq 0$ and base grid spacing $\delta > 0$.
\footnotemark Each $n\geq 0$ represents the ``resolution'' through which sample
paths of $\{X(t)\}$ are studied.
\footnotetext{The addition $x_0+A$, for $A$ in an affine space such as $\Real$
represents shifting of every element of the set $A$ by a common value $x_0$.}

In particular, the crossing times $T_k^n$ of process $\{X(t)\}$ at resolution $n\geq 0$
are defined as follows: let $T_0^n = 0$ and for any $k\geq 0$ put
\[
T_{k+1}^n
= \inf \Bigl\{ t \geq T^n_k : \bigr | X(t) - X(T^n_k) \bigr | \geq \delta 2^n \Bigr\}
\]
In other words $T_{k+1}^n$, if it is finite, is the first passage time of a path of
$X(t)$ across a level of $\delta 2^n \mathbb{Z}$ that is different from the line of
the same grid crossed previously. Fixing the zero-th crossing at $0$ automatically
aligns the grid with the process, so in effect, without loss of generality one may
consider processes $X_t$, which start at the origin $X(0)=0$.

By definition, the crossing times of some fixed level $n\geq 0$ are non-decreasing:
$T_k^n \leq T_{k+1}^n$ for all $k\geq 0$. However, since the paths of the process are
almost surely continuous, it is possible to show that $T_k^n < T_{k+1}^n$. Indeed,
by right-continuity of $X(t)$ (a fixed path of $\{X(t)\}$) for $\delta 2^{n-1}$
there is $\eta>0$ with $|X(s)-X(T_k^n)| < \delta 2^{n-1}$ for all $s\in [T_k^n, T_k^n + \eta)$.
However, by definition of $T_{k+1}^n$ there in the $\eta$ vicinity of $T_{k+1}^n$
there exists $\hat{s}$ such that $|X(\hat{s})-X(T_k^n)|\geq \delta 2^n$. Thus, if
$T_k^n = T_{k+1}^n$ then one has $\delta 2^n \leq \delta 2^{n-1}$ -- a contradiction.

Another important property of crossing times is that if $T_{k+1}^n < +\infty$, then
$|X(T_{k+1}^n)-X(T_k^n)| = \delta 2^n$. Indeed, the definition of $T_{k+1}^n$ implies
that there exits a sequence $s_j\downarrow T_{k+1}^n$ with $|X(T_{k+1}^n)-X(T_k^n)|\geq \delta 2^n$
. Since the path $X(t)$ is continuous at $T_{k+1}^n$ it must be true that
\[
|X(T_{k+1}^n)-X(T_k^n)| = \lim_{j\to \infty} |X(s_j)-X(T_k^n)| \geq \delta 2^n
\]
Now, since the function $t\to |X(t)-X(T_k^n)| - \delta 2^n$ is continuous at $T_{k+1}^n$,
$|X(T_{k+1}^n)-X(T_k^n)| > \delta 2^n$ would entail the existence of $s\in [T_k^n, T_{k+1}^n)$
such that $|X(s)-X(T_k^n)| - \delta 2^n > 0$. This would contradict the fact that
$T_{k+1}^n$ is a lower bound of the set of all $s\geq T_k^n$ with $|X(s)-X(T_k^n)| \geq \delta 2^n$.

So far the crossing times of an almost surely continuous process $\{X(t)\}$, have
the following properties (almost surely): \begin{enumerate}
	\item $T_k^n < T_{k+1}^n$ for all $k\geq 0$;
	\item $|X(T_{k+1}^n)-X(T_k^n)| = \delta 2^n$ for all $k\geq0$ with $T_{k+1}^n<+\infty$.
\end{enumerate}

In order to see the natural tree structure to the crossings, one has to investigate
the properties of crossing times at different resolutions. For the following suppose,
for some integers $k,m\geq 0$ it holds that $T_k^n\in [T_m^{n+1}, T_{m+1}^{n+1})$ and
$X(T_m^{n+1}) = X(T_k^n)$.

By definition of $T_{k+1}^n$ for all $s\in[T_k^n, T_{k+1}^n)$ it must be true that
\[ |X(s) - X(T_k^n)| < \delta 2^n \]
fr otherwise $T_{k+1}^n$ would not have been the least lower bound. Since the process
takes the same value at both times $T_m^{n+1}$ and $T_k^n$, it must be true that
$|X(s) - X(T_m^{n+1})| < \delta 2^{n+1}$ for such $s$, whence $s\leq T_{m+1}^{n+1}$.
Therefore $T_{k+1}^n\leq T_{m+1}^{n+1}$.

However, if one further assumes that $T_{m+1}^{n+1} < +\infty$, i.e. that the $n+1$-st
crossing takes place, then $T_{k+1}^n \neq T_{m+1}^{n+1}$. Indeed, \begin{itemize}
	\item $|X(T_{k+1}^n)-X(T_k^n)| = \delta 2^n$;
	\item $|X(T_{m+1}^{n+1})-X(T_m^{n+1})| = \delta 2^{n+1}$.
\end{itemize}
together with $X(T_m^{n+1}) = X(T_k^n)$ entail a contradiction $\delta 2^n = \delta 2^{n+1}$.

The assumption that a crossing of a level of a coarser grid $\delta 2^{n+1} \mathbb{Z}$
took place, permits further refinement of crossing times' behaviour: under the assumptions
above $T_{k+2}^n\leq T_{m+1}^{n+1}$ holds. Indeed, suppose otherwise. Then by definition
of $T_{k+2}^n$ and because $T_{k+1}^n \leq T_{m+1}^{n+1}$, it must be true that
\[ |X(T_{m+1}^{n+1}) - X(T_{k+1}^n)| < \delta 2^n \]
However $|X(T_{m+1}^{n+1}) - X(T_m^{n+1})| = \delta 2^{n+1}$ by the established
properties of crossing times on the grid of $n+1$ resolution. Then by the triangle
inequality and the assumption that $X(T_m^{n+1}) = X(T_k^n)$ one gets
\begin{align*}
    |X(T_{m+1}^{n+1}) - X(T_m^{n+1})| 
    & \leq |X(T_{m+1}^{n+1}) - X(T_{k+1}^n)| + |X(T_{k+1}^n) - X(T_m^{n+1})| \\
    & = |X(T_{m+1}^{n+1}) - X(T_{k+1}^n)| + |X(T_{k+1}^n) - X(T_k^n)|
\end{align*}
whence the following contradiction emerges
\[
\delta 2^{n+1}
\leq |X(T_{m+1}^{n+1}) - X(T_{k+1}^n)| + \delta 2^n
< \delta 2^{n+1}
\]
Therefore, under the stated conditions $T_{k+2}^n\leq T_{m+1}^{n+1}$.

% section crossing_times (end)

\section{a Tree Structure} % (fold)
\label{sec:a_tree_structure}

Let's define the \emph{$n$-th level crossing} as the slice of the path of the process
$\{X(t)\}$ over the time half-interval $[T_k^n,T_{k+1}^n)$ for some integer $k\geq 0$,
during which it makes a move of $\pm \delta 2^n$. The uncovered relationship between
crossing times of consecutive levels imply that if $T_k^n \in [T_m^{n+1}, T_{m+1}^{n+1})$
with $X(T_m^{n+1}) = X(T_k^n)$ and $T_{m+1}^{n+1} < +\infty$ then
\[
\bigl[T_k^n,T_{k+1}^n\bigr) \cup \bigl[T_{k+1}^n,T_{k+2}^n\bigr)
\subseteq \bigl[T_m^{n+1},T_{m+1}^{n+1}\bigr)
\]
The first condition, $X(T_m^{n+1}) = X(T_k^n)$ aligns the different resolution grid
with one another. The last requirement that $[T_m^{n+1}, T_{m+1}^{n+1})$ be a finite
interval is natural, since it actually means that the $n+1$-st level crossing actually
occurred.

These observations actually state that within the $n+1$-st level crossing there musts
be exactly an even number of crossing of the $n$-th level (subcrossings). Indeed,
if $T_k^n = T_m^{n+1}$ and $T_{m+1}^{n+1}<+\infty$, then there are two possibilities
for the crossing time $T_{k+2}^n$: \begin{enumerate}
	\item the process crossed a new level of $n+1$-st grid $\delta 2^{n+1}$ in which
	case $T_{m+1}^{n+1}\leq T_{k+2}^n$. Hence there are exactly \emph{two} subcrossings.
	\item the process moved back to the level $X(T_k^n)$, which did not incur a crossing
	of $n+1$-st grid, since $X(T_{k+2}^n)=X(T_m^{n+1})$. Yet this fluctuation was
	registered by the grid of resolution $n$. In this case, the properties of crossing]
	times entail \[ T_{k+2}^n < T_{k+3}^n < T_{k+4}^n \leq T_{m+1}^{n+1} \]
	Since $T_{m+1}{n+1}$ is finite, recurent application of this argument implies that
	there must be exactly an even number of $n$-th level subcrossings in an $n+1$-st
	level crossing.
\end{enumerate}
This described behaviour is depicted in figure~\ref{fig:xing_tree_structure}.

%% Draw a simple illustration of the movements within one crossing
\begin{figure}[htb]\begin{center}
    \tikzset{bullet/.style={circle,draw, fill=black,minimum size=5pt,inner sep=0pt},}
    \begin{tikzpicture}[grow=right, sloped]
%% Crossings
        \node [bullet] (o) {};
        \node [bullet] (u) [above right = 3em and 9em of o.west, anchor = west] {};
        \node [bullet] (d) [below right = 3em and 9em of o.west, anchor = west] {};
        \node [bullet] (uu) [above right = 3em and 9em of u.west, anchor = west] {};
        \node [bullet] (ud) [above right = 3em and 9em of d.west, anchor = west] {};
        \node [bullet] (dd) [below right = 3em and 9em of d.west, anchor = west] {};
%% The anchor and labels
        \node [draw=none] [left = 1em of o.west, anchor = east] {$X(T_k^n)$};
        \node [draw=none] [right = 1em of ud.east, anchor = west] {$X(T_k^n)=X(T_{k+2}^n)$};
        \node [draw=none] [right = 1em of uu.east, anchor = west] {$X(T_k^n)+\delta 2^{n+1}$};
        \node [draw=none] [right = 1em of dd.east, anchor = west] {$X(T_k^n)-\delta 2^{n+1}$};
%% Crossing times nodes at the bottom
        \node [draw=none] (t2) [below =of dd.south, anchor = west] {$T_{k+2}^n$};
        \node [draw=none] (t1) [left = 9em of t2.west, anchor = west] {$T_{k+1}^n$};
        \node [draw=none] (t0) [left = 9em of t1.west, anchor = west] {$T_k^n$};
%% Dumy nodes on the top level
        \node [fill=none,draw=none] (dummy2) [above = 1em of uu.north, anchor = north] {};
        \node [fill=none,draw=none] (dummy1) [left = 9em of dummy2.west, anchor = west] {};
        \node [fill=none,draw=none] (dummy0) [left = 9em of dummy1.west, anchor = west] {};
%% Paths to levels
        \path[thick,draw]
            (o) edge node[above] {$+\delta 2^n$} (u)
            (o) edge node[below] {$-\delta 2^n$} (d)
            (u) edge node[above] {$+\delta 2^n$} (uu)
            (u) edge node[below] {$-\delta 2^n$} (ud)
            (d) edge node[above] {$+\delta 2^n$} (ud)
            (d) edge node[below] {$-\delta 2^n$} (dd);
%% Crossing times
        \draw[black, dashed] (dummy2.west -| t2.west) -- (t2.west -| t2.west) {};
        \draw[black, dashed] (dummy1.west -| t1.west) -- (t1.west -| t1.west) {};
        \draw[black, dashed] (dummy0.west -| t0.west) -- (t0.west -| t0.west) {};
    \end{tikzpicture}
%% A very informative caption
    \caption{The structure of possible movements during the crossing $\bigl[T_k^n,T_{k+2}^n\bigr)$.}
\label{fig:xing_tree_structure}
\end{center}\end{figure}

Thus the properties of crossing times of a continuous process permit a natural tree
structure to crossings. If tree levels are enumerated from bottom up then \begin{itemize}
	\item a node at the $n+1$-st tree level represents a crossing of the $\delta 2^{n+1} \mathbb{Z}$;
	\item $n$-th level children (offspring) of an $n+1$-st level node represent the
	multiple crossings of a finer gird $\delta 2^n \mathbb{Z}$ that took place during
	the larger crossing. 
\end{itemize}

The above exposition considered the crossing tree for successively coarser resolutions
$\delta 2^n \mathbb{Z}$ for $n\geq0$. This has been done purely for the following reason:
in practice, on real sample path of some supposedly continuous process it is never
possible no meaningfully go beyond some basic finest resolution given by $\delta>0$.
This has to do with the fact that processes are sampled at finite frequency or over
regularly spaced but finitely many point in time. Nevertheless theoretical crossing
tree can be extended to crossings of $\delta 2^n$ grid for all integer $n\in\mathbb{Z}$.

% section a_tree_structure (end)

\section{Crossing tree on times series data} % (fold)
\label{sec:crossing_tree_on_times_series_data}

In this section I briefly describe how the crossing tree is constructed for a sample
path of some continuous process, given by time series data $(t_i, x_i)_{i=0}^N$.

In practice the crossing trees are built iteratively from the finest resolution
specified by the base grid spacing $\delta > 0$. The crossing times of the leaves
of the tree are computed for a given fixed $\delta>0$ in two passes: the firs pass
computes passage times on the passed $\delta \mathbb{Z}$ grid levels, and the second
pass adjusts the passage times and passed levels to eliminate re-crossings of the
same level. The assumption that the time series data came from a continuous process
is the main reason, that enables a sequential procedure for estimation of the crossing
times of grid $\delta\mathbb{Z}$: $X(T_{k+1})^n = X(T_k^n)+\pm\delta2^n$ if $T_{k+1}^n<\infty$.
One is encouraged to refer to appendix for details of the procedure.

The choice of $\delta$ is extremely important as it affects the estimates of the
crossing times and levels in the following way, reminiscent of the bias-variance
trade-off:
\begin{itemize}
    \item the grid with too low a value of $\delta$ would make the resolution so fine
    as to regard the studied process as a continuous piecewise linear function. Each
    increment would be very likely to produce long consecutive unidirectional streaks
    of crossed levels, which would overestimate the number of crossings with only 2
    subcrossings by introducing excess number of artificial crossings events.
    \item Too large $\delta$ leads to a tree with fewer crossings and offspring.
\end{itemize}


% section crossing_tree_on_times_series_data (end)

% chapter the_crossing_tree (end)
