\chapter*{Appendix} % (fold)
\label{cha:appendix}

\section*{Details on the practical procedure} % (fold)
\label{sec:details_on_the_practical_procedure}

In the appendix a brief description of the parctical crossing tree costruction
procedure is given.

Recall that the tree is constructed in two passes: initial detection passages over
levels of $\delta \mathbb{Z}$ grid, which also discrds within-band movements, and
the pruning phase, where re-crossings are eliminated.

Before the first pass through the data the time series $x_j$ is shifted and scaled to
series $z_i = \frac{1}{ \delta }\bigl(x_j - x_0\bigr)$, so that the constructed 
is rooted at $t=0$, since the process sets off from the origin.

The first pass sweeps through consecutive increments of the series given by pairs
$(t_i, z_i)$ and $(t_{i+1}, z_{i+1})$ for $i=1, \ldots, N-1$. For each increment
its direction is determined by the sign of the difference $\Delta_i=z_{i+1}-z_i$,
and depending on it, the range of grid levels passed is computed. The range of an
increment $i$ a subset of integers $R_i = [a_i,b_i]\cup\mathbb{Z}$, where the
boundaries $a_i$ and $b_i$ are determined using the following logic
\begin{description}
    \item[Upcrossing] if $\Delta_i > 0$ then $a_i = \lceil z_i \rceil$
    and $b_i = \lfloor z_{i+1}\rfloor$;
    \item[Downcrossing] for $\Delta_i < 0$ the range given by $b_i = \lceil z_{i+1} \rceil$
    and $a_i = \lfloor z_i\rfloor$.
\end{description}
In an rare case of sideways movement, $R_i = \emptyset$. Also note that $[a,b] = \emptyset$
whenever $b<a$. During the first pass,

Movements of the normalised process $(z_i)$ that wiggle strictly within a band
between levels of the grid $\delta \mathbb{Z}$ and do not do not pass or touch
through a level have empty range, and are discarded.

The second pass checks if $b_{j-} = a_j$ for any crossing $j\in J$, where
$J = \{j : R_j\neq \emptyset\}$ and the index $j-$ for any $j\in J$ is defined as
$j- = \max\{i\in J : i < j\}$, or $-\infty$ if $j\in J$ is the very first apparent
crossing of the grid $\delta \mathbb{Z}$, and $a_{j-}$ is taken to be $-\infty$.

If the first level passed by $j\in J$ coincides with the last levels crossed by $j-$
then the very first crossing event in the $j$-th increment is a re-crossing evnet,
and thus should be eliminated. This is done by adjusting the $a_i$ in the direction
of the increment $j$. The ranges are undated accordingly and empty ones are discarded.

These passes compute the levels the sample path apparently crossed, and the crossing
times are estimated using linear interpolation between times $t_i$ and $t_{i+1}$ with
weights determined by passed level. The formula the crossing time of level $l\in R_i$
during the $i$-th increment is
\[
\tau_{il} = t_i + \bigl(t_{i+1} - t_i\bigr) \frac{l - z_i}{z_{i+1} - z_i}
\]

The crossing times of a coarser resolution grid are computed based on the 
data provided the next finer resolution in the same way the second pass functions:
since the resolution is halved between successive tree levels, the crossing times
estimated at resolution $\delta 2^{n+1}$ are by construction a subset of crossing
times of a grid $\delta 2^n$. Inded, if an increment $j$ passed a line $m \delta 2^{n+1}$
then this very same passed a line $2m \delta 2^n$ of the finer grid and the interpolation
coefficient used to compute the crossing time is unchanged
\begin{align*}
	\frac{m - \frac{x_j - x_0}{\delta 2^{n+1}}}{ \frac{x_{j+1} - x_j}{\delta 2^{n+1}} }
	&= \frac{2m - \frac{x_j - x_0}{\delta 2^n}}{ \frac{x_{j+1} - x_j}{\delta 2^n} } \\
\end{align*}
Thus, since the incrementis the same, the corrins time reamins the same.

% section* details_on_the_practical_procedure (end)

% chapter* appendix (end)
