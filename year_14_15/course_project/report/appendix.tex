\chapter*{Appendix} % (fold)
\label{cha:appendix}

\section*{Details on the practical procedure} % (fold)
\label{sec:details_on_the_practical_procedure}

We briefly describe of the procedure for constructing the crossing tree on time series
data $(t_i, x_i)_{i=0}^N$.

Recall that the tree is constructed in two passes: initial detection passages over
levels of $\delta \mathbb{Z}$ grid, which also discards within-band movements, and
the pruning phase, where re-crossing events are eliminated.

Before the first pass through the data, the time series $(x_i)$ is shifted and scaled
to series $z_i = \frac{1}{ \delta }\bigl(x_i - x_0\bigr)$, so that the constructed
tree is rooted at the crossing time $t=0$, since the process sets off from the origin.

The first pass sweeps through consecutive increments of the series given by pairs
$(t_i, z_i)$ and $(t_{i+1}, z_{i+1})$ for $i=1, \ldots, N-1$. For each increment,
its direction is determined by the sign of the difference $\Delta_i=z_{i+1}-z_i$,
and, depending on it, the range of grid levels traversed by it is computed. The
range of the $i$-th increment a subset of integers $R_i = [a_i,b_i]\cup\mathbb{Z}$,
where the boundaries $a_i$ and $b_i$ are determined using the following logic:
\begin{description}
    \item[Upcrossing] if $\Delta_i > 0$ then $a_i = \lceil z_i \rceil$
    and $b_i = \lfloor z_{i+1}\rfloor$;
    \item[Downcrossing] for $\Delta_i < 0$ the range given by $b_i = \lceil z_{i+1} \rceil$
    and $a_i = \lfloor z_i\rfloor$.
\end{description}
In a rare case of sideways movement $\Delta_i = 0$, the range is set to be $\emptyset$.
Also note that $[a,b] = \emptyset$ whenever $b<a$.

Movements of the normalised process $(z_i)$ that wiggle strictly within a band between
levels of the grid $\delta \mathbb{Z}$ and do not do not pass through or touch a level
of the grid have empty range, and are discarded.

This sequential elimination of between grid lines movements crucially depends on
the presumption that the process is continuous and linear interpolation between
the endpoints of each increment is valid.

The second pass checks if $b_{j-} = a_j$ for any crossing $j\in J$, where
$J = \{j : R_j\neq \emptyset\}$, and the index $j-$ for any $j\in J$ is defined as
$j- = \max\{i\in J : i < j\}$, or $-\infty$ if $j\in J$ is the very first apparent
crossing of the grid $\delta \mathbb{Z}$, and $a_{j-}$ is taken to be $-\infty$.

If the first level passed by $j\in J$ coincides with the very last level crossed
by $j-$ then the very first crossing event in the $j$-th increment is a re-crossing
event, and thus should be eliminated. This is done by adjusting the $a_i$ in the
direction of the $j$-th increment. The ranges are updated accordingly and empty
ones are discarded.

These passes compute the levels the sample path of the process appears to have
crossed, and the crossing times are estimated by linearly interpolating between
times $t_i$ and $t_{i+1}$ (endpoints of the $i$-th increment) with weights determined
by passed levels of the grid. The formula of the crossing time of some level $l\in R_i$
during the $i$-th increment is
\[ \tau_{il} = t_i + \bigl(t_{i+1} - t_i\bigr) \frac{l - z_i}{z_{i+1} - z_i} \,. \]

The crossing tree is constructed from the finest resolution up by gradually making
the resolution coarser. The crossing times at some grid are computed based on the 
times and levels provided by the least finest resolution, which is still finer than
the this one, in the same way the second pass functions: since the resolution is
halved between successive tree levels, the crossing times estimated at resolution
$\delta 2^{n+1}$ are by construction a subset of crossing times of a finer grid
$\delta 2^n$. Indeed, if an increment $j$ passed a line $m \delta 2^{n+1}$ then
this very same passed a line $2m \delta 2^n$ of the finer grid and the interpolation
coefficient used to compute the crossing time is unchanged:
\[
\biggl( m - \frac{x_j - x_0}{\delta 2^{n+1}} \biggr) \frac{x_{j+1} - x_j}{\delta 2^{n+1}}
= \biggl( 2m - \frac{x_j - x_0}{\delta 2^n} \biggr) \frac{x_{j+1} - x_j}{\delta 2^n} \,.
\]
Thus, since the same increment crossed the grid line, its time remains unchanged.

% section* details_on_the_practical_procedure (end)

% chapter* appendix (end)
