\chapter{$H$-sssi processes} % (fold)
\label{cha:h_sssi_processes}

This section covers the basic definitions and properties of self-similar processes
studied in this project. We consider a process $\bigl\{X(t)\bigr\}$ to be a continuous
-time real-valued stochastic process defined for all $t\in [0,+\infty)$. The contents
of this section are quite general and are based on the following papers and textbooks:
\cite{bulinskii2005teoriya2516755}, \cite{Bai20141710}, \cite{Chronopoulou:1114288}
\cite{embrechts2000introduction} and \cite{embrechtsselfsimilar} to name but a few.

\section{Definition} % (fold)
\label{sec:definition}

Before proceeding with the definitions and properties, we clarify what is meant by
stochastic equivalence of random processes. Processes $\bigl\{X(t)\bigr\}$ and
$\bigl\{Y(t)\bigr\}$ are equivalent in finite-dimensional distributions, or symbolically
$\{X(t)\} \overset{\Dcal}{=} \{Y(t)\}$, if for all $n\geq1$ and all $(t_k)_{k=1}^n\in [0,+\infty)$
with $t_k<t_{k+1}$,
\[ \bigl(X(t_k)\bigr)_{k=1}^n \overset{\Dcal}{\sim} \bigl(Y(t_k)\bigr)_{k=1}^n \text{ holds}\,,\]
where $A\overset{\Dcal}{\sim} B$ denotes equality of distribution of random variables
$A$ and $B$.

A process $\bigl\{X(t)\bigr\}_{t\geq 0}$ is called \textbf{self-similar}, or \textbf{ss}
for short, if for any $a>0$ there exists $b>0$ such that
\[ \bigl\{X(at)\bigr\} \overset{\Dcal}{=} \bigl\{b X(t)\bigr\} \,,\]
that is, their finite-dimensional distributions coincide.

A process $\bigl\{X(t)\bigr\}$ is said to be stochastically continuous at $t\geq0$ if
$\lim_{h\to 0} \pr\bigl\{ |X(t+h)-X(t)| \geq \epsilon \bigr\} = 0$ for arbitrary
$\epsilon > 0$.

It was shown by Lamperti in 1962 that whenever a self-similar process $\bigl\{X(t)\bigr\}$
has non-degenerate point distributions (non-trivial), and is stochastically continuous
at $t=0$, there (necessarily) exists a constant $H\geq 0$ such that any $a>0$ the constant
$b$ in the definition of self-similarity is given by $b=a^H$, i.e
\[ \bigl\{X(at)\bigr\} \overset{\Dcal}{=} \bigl\{a^H X(t)\bigr\} \,. \]
Uniqueness of $H$ follows from the fact that $b_1 X\overset{\Dcal}{\sim} b_2 X$ implies
$b_1=b_2$ for any non-degenerate random variable $X$.

This theorem gives the definition of $H$-\textbf{s}elf-\textbf{s}imilarity: \label{def:hsssi}
a process $\bigl\{X(t)\bigr\}$ is $H$-ss with Hurst exponent $H$ if for all $a>0$,
\[ \bigl\{X(t)\bigr\} \overset{\Dcal}{=} \bigl\{a^{-H} X(at)\bigr\} \]

%% bulinskii2005teoriya2516755 p.105
A stochastic processes $\bigl\{X(t)\bigr\}$ is said to have \textbf{S}tationary
\textbf{I}ncrements, or \textbf{si}, if all finite-dimensional distributions of
the process $\bigl\{X(t+s) - X(t)\bigr\}_{s\geq0}$ are independent of $t$, which
equivalently means that
\[ \bigl\{X(t+s)-X(t)\bigr\} \overset{\Dcal}{=} \bigl\{X(s)-X(0)\bigr\} \,. \]

A particularly nice corollary to the definition of an $H$-self-similar process is that
if $\bigl\{X(t)\bigr\}$ is $H$-ss then $X(t)\overset{\Dcal}{\sim}t^H X(1)$. In turn,
this allows, for example to show that all zero-mean stochastic processes with stationary
increments share similar auto-correlation pattern for all $t\neq s$
\[ \ex X(s) X(t) = \frac{1}{2}\Bigl( t^{2H} + s^{2H} - |t-s|^{2H}\Bigr) \ex|X(1)|^2 \,. \]
Indeed, the identity $2 a b = a^2 + b^2 - (a-b)^2$ implies that\begin{align*}
	\ex X(s) X(t)
	&= \frac{1}{2}\biggl( \ex X(s)^2 + \ex X(t)^2 - \ex\bigl( X(s) - X(t) \bigr)^2 \biggr) \\
	&= \frac{1}{2}\biggl( \ex X(s)^2 + \ex X(t)^2 - \ex X(|s-t|)^2 \biggr) \\
	&= \frac{1}{2}\biggl( |s|^{2H} \ex X(1)^2 + |t|^{2H} \ex X(1)^2 - |s-t|^{2H} \ex X(1)^2 \biggr) \,,
\end{align*}
where the second and the third lines follow from the stationarity of the increments
and the mentioned corollary, respectively.

%% bulinskii2005teoriya2516755 p.47
Lastly, recall that a process $\bigl\{X(t)\bigr\}$ is said to have \textbf{i}ndependent
\textbf{i}ncrements if for any integer $n\geq1$ and every $(t_k)_{k=0}^n\in[0,\infty)$
with $0=t_0$ and $t_k < t_{k+1}$ the random variables $\bigl(X(t_k) - X(t_{k-1})\bigr){k=1}^n$
and $X(0)$ are jointly independent.

To summarize a process $\bigl\{X(t)\bigr\}$ is $H$-sssi, if it is $H$-self-similar
with Hurst exponent $H$ and has stationary increments.

% section definition (end)

\section{Fractional Brownian motion} % (fold)
\label{sec:fractional_brownian_motion}

It was mentionend in section~\ref{sec:the_crossing_tree_of_brownian_motion} that Brownian
motion is an example of a self-similar process. In fact it is an $\sfrac{1}{2}$-sssi process.
Indeed, let $a>0$ and consider the process $V(t) = \frac{1}{\sqrt{a}} B(at)$. Obviously
$V(0) = 0$ almost surely, since $V(0) = \sfrac{1}{\sqrt{a}} B(a 0) = 0$. Furthermore
this process inherits stationary increments from $B(t)$ : \begin{align*}
	\{ V(t+s) - V(t) \} &= \biggl\{ \frac{1}{\sqrt{a}}( B(at+as) - B(at) ) \biggr\}\\
	&\overset{\Dcal}{=} \biggl\{ \frac{1}{\sqrt{a}}( B(as) ) \biggr\} \\
	&= \{ V(s) \} \,,
\end{align*}
and independence of increments as well: pick $(t_k)_{k=1}^n$ with $t_k<t_{k+1}$ and
observe that $\bigl(V(t_{k+1}) - V(t_k)\bigr)_{k=1}^n$ are mutually independent since
$\bigl(B(s_{k+1}) - B(s_k)\bigr)_{k=1}^n$ for $s_k = a t_k$ are. Path continuity follows
from continuity of sample paths of $B(t)$ and the fact that $t\to a t$ is a continuous
map.  Finally $\mathcal{N}(0, at) \sim \sqrt{a} \mathcal{N}(0,t)$ implies that $V(t)$
is Brownian motion as well.

Another example of an $H$-sssi process, which is frequently used as a reference for
self-similarity and scale invariance studies is the \textbf{f}ractional \textbf{B}rownian
\textbf{m}otion, or \textbf{fBm}. It is a generalisation of the $\tfrac{1}{2}$-ss
Brownian motion to a general $H$-ss Gaussian process with $H\in (0, 1)$.

Formally, fractional Brownian motion introduced in \cite{doi:10.1137/1010093} is a
zero-mean Gaussian process $\bigl\{X(t)\bigr\}$, (its every finite-dimensional joint
distributions are multivariate normal), with the covariance structure of a general
zero-mean $H$-sssi stochastic process:
\[ \ex X(s) X(t) = \frac{1}{2} \bigl(t^{2H} + s^{2H} - |t-s|^{2H}\bigr) \,. \]
%% bulinskii2005teoriya2516755 p.51

Theorem 1.3.3 on p.~6 of \cite{embrechtsselfsimilar} establishes that a fractional
Brownian motion $\{B_h(t)\}_{t\geq 0}$ is an $H$-sssi process, for the following
integral representation up to a multiplicative constant
\[
\int_{-\infty}^0 (t-s)^{H-\tfrac{1}{2}} - (-s)^{H-\tfrac{1}{2}} dB(s)
+ \int_0^t |t-s|^{H-\tfrac{1}{2}} dB(s) \,.
\]
If $H = 1$ then the fractional Brownian motion degenerates to $B_1(t) = tB(1)$ (almost
surely). The class of all fractional Brownian motions coincides with the class of all
Gaussian self-similar processes with stationary increments, see \cite{embrechtsselfsimilar}. 

In fact, fractional Brownian motion is an example of a broader class of self-similar
processes known as the \textbf{Hermite} processes. They exhibit non-Gaussianity and have
strongly dependent increments.

% section fractional_brownian_motion (end)

\section{Hermite processes} % (fold)
\label{sec:hermite_processes}

Hermite processes inherit their name from the stochastic integral kernel used in
their definition, and are an extremely important example of processes, which have 
non-Gaussian finite-dimensional joint distributions.

A probabilistic Hermite polynomial of order $k\geq0$ is defined as 
\[ H_k(x) = (-1)^k e^{-\frac{x^2}{2}} \frac{d^k}{dx^k} e^{-\frac{x^2}{2}} \,,\]
and is a solution to the following differential equation
\[
 \frac{d}{dx}\biggl( e^{-\frac{x^2}{2}} \frac{d}{dx} f\biggr) + \lambda e^{-\frac{x^2}{2}} f = 0 \,.
\]
These polynomials constitute an orthogonal basis of the Hilbert space $\Lcal^2(\Real, \mu)$
with measure $\mu$ being the Lebesgue integral $\int e^{-\frac{x^2}{2}} dx$ and the inner
product given by
\[
\langle f, g\rangle = \int_\Real f g d\mu = \int_\Real f g e^{-\frac{x^2}{2}} dx \,.
\]

The Hermite process of order $m$ with self-similarity parameter $H\in(\tfrac{1}{2},1)$,
denoted by $\bigl\{Z_H^p(t)\bigr\}_{t\geq 0}$, is defined via s multiple stochastic
integral
\[
Z_H^p(t) = \underset{\Real^m}{\int \cdots \int} \Biggl(
\int^t_0 \prod_{k=1}^m (u-x_k)_+^{-\frac{1}{2}-\frac{1-H}{m}} du\Biggr) dB(x_1) \ldots dB(x_q)
\]
over independent realizations of Gaussain white noise $dB(x_j)$. The Hermite process
of order $1$ is fractional Brownian motion, whereas higher order Hermite processes
correspond to non-Gaussian $H$-sssi, \cite{Bai20141710}.

% section hermite_processes (end)

\section{Weierstrass function} % (fold)
\label{sec:weierstrass_function}

Yet another example of non-Gaussian self-similar process is the random Weierstrass
function defined as the limiting random function of the sum of randomly phase shifted
and scaled cosines
\[
f_H(t)
= \sum_{k\in \mathbb{Z}} \lambda_0^{-kH}\bigl\{ \cos(\phi_k)
- \cos(2\pi \lambda_0^k t + \phi_k) \bigr\} \,,
\]
for $(\phi_k)_{k\in\mathbb{Z}}\sim \mathcal{U}(0,2\pi)$ iid -- the random phase
shift of the $k$-th layer harmonics. The parameter $\lambda_0$ governs the base
scale of the process and is knows as ``the fundamental harmonic'' (see~\cite{decrouez2013estimation}).
The function $f_H(t)$ has continuous paths, yet is nowhere differentiable, even
in the deterministic case when all phase shifts are identically zero. The Weierstrass
function is not an $H$-sssi process, but it exhibits discrete self-similarity, which
means that there exists $a_0>0$ such that self-similarity holds for values of $a$
on a discrete grid $a_0\mathbb{N}$, that is for all $t\geq 0$
\[ f_H(at) = a^H f(t)\,. \]

The $H$ in the definition is in fact not the Hurst index, but a so called H\"older
exponent, which means that for some $C\in[0,+\infty)$ it is true that for all $s,t\geq0$
\[ \bigl\lvert f_H(s) - f_H(t) \bigr\rvert \leq C |s-t|^H \,. \]
Functions, or random processes, for which there exists $H\geq0$ such that this
condition is satisfied are known as \textbf{H\"older continuous}. This H\"older
continuity implies scale-invariance of the Weierstrass function. Indeed, for any
$a\in a_0\mathbb{N}$ the condition implies that
\[ a^{-H}\bigl\lvert f_H(as) - f_H(at) \bigr\rvert \leq C |s-t|^H\,, \]
for all $s,t\geq 0$. Furthermore, observe that by definition $f_H(0)=0$ and for
all $t\geq0$:
\begin{align*}
	\bigl\lvert f_H(at) - a^H f_H(t) \bigr\rvert
	&\leq \bigl\lvert f_H(at) - f_H(t) \bigr\rvert
	+ \bigl\lvert a^H f_H(t) - f_H(t) \bigr\rvert \\
	&\leq C |a - 1|^H |t|^H + |a^H - 1|\bigl\lvert f_H(t)\bigr\rvert\\
	&\leq C \bigl( |a - 1|^H  + |a^H - 1| \bigr) |t|^H \,,
\end{align*}
which makes it obvious that the Weierstrass function has self-similarity.

% section weierstrass_function (end)

% chapter h_sssi_processes (end)