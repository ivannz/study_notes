\chapter{$H$-SSSI processes} % (fold)
\label{cha:h_sssi_processes}

This section will cover the basic definitions and properties of self-similar processes
studied in this paper. In general we will consider a process $\bigl\{X(t)\bigr\}$ to be
a continuous-time real-valued stochastic processes defined for all $t\in [0,+\infty)$.
The contents of this section are quite general and are based on the following papers and
text books: \cite{bulinskii2005teoriya2516755}, \cite{Bai20141710}, \cite{Chronopoulou:1114288}
\cite{embrechts2000introduction} and \cite{embrechtsselfsimilar} to name a few.

\section{Definition} % (fold)
\label{sec:definition}

Before proceeding with the definitions and properties, it is necessary to clarify, what
is meant by stochastic equivalence of random processes. Processes $\bigl\{X(t)\bigr\}$ and
$\bigl\{Y(t)\bigr\}$ are equivalent in finite-dimensional distributions, or symbolically
$\{X(t)\} \overset{\Dcal}{=} \{Y(t)\}$, if for all $n\geq1$ and all $(t_k)_{k=1}^n\in [0,+\infty)$
with $t_k<t_{k+1}$ it is true that
\[ \bigl(X(t_k)\bigr)_{k=1}^n \overset{\Dcal}{\sim} \bigl(Y(t_k)\bigr)_{k=1}^n \]
where $A\overset{\Dcal}{\sim} B$ denotes equality of distribution of random variables
$A$ and $B$.

A process $\bigl\{X(t)\bigr\}_{t\geq 0}$ is called \textbf{self-similar}, of \textbf{SS} for
short, if for any $a>0$ there exists $b>0$ such that
\[ \bigl\{X(at)\bigr\} \overset{\Dcal}{=} \bigl\{b X(t)\bigr\} \]
their finite-dimensional distributions coincide.

A process $\bigl\{X(t)\bigr\}$ is said to be stochastically continuous at $t\geq0$ if
$\lim_{h\to 0} \pr\bigl\{ |X(t+h)-X(t)| \geq \epsilon \bigr\} = 0$ for arbitrary
$\epsilon > 0$.

It was shown by Lamperti in 1962 that whenever a self-similar process $\bigl\{X(t)\bigr\}$
has non-degenerate point distributions (non-trivial), and is stochastically continuous
at $t=0$, there necessarily exists a constant $H\geq 0$ such that any $a>0$ the constant
$b$ in the definition of self-similarity is given by $b=a^H$, i.e
\[ \bigl\{X(at)\bigr\} \overset{\Dcal}{=} \bigl\{a^H X(t)\bigr\} \]
Uniqueness of $H$ follows from the fact that $b_1 X\overset{\Dcal}{\sim} b_2 X$ implies
$b_1=b_2$ for any non-degenerate random variable $X$.

This theorem gives the definition of an $H$-\textbf{S}elf-\textbf{S}imilarity: a process
$\bigl\{X(t)\bigr\}$ is $H$-SS with Hurst exponent $H$ if for all $a>0$ it holds
\[ \bigl\{X(t)\bigr\} \overset{\Dcal}{=} \bigl\{a^{-H} X(at)\bigr\} \]

%% bulinskii2005teoriya2516755 p.105
The processes, to which the crossing tree is applied in this paper, also have stationary
increments. A stochastic processes $\bigl\{X(t)\bigr\}$ is said to have \textbf{S}tationary
\textbf{I}ncrements, or SI, if all finite-dimensional distributions of the process
$\bigl\{X(t+s) - X(t)\bigr\}_{s\geq0}$ are independent of $t$, which equivalently means
that
\[ \bigl\{X(t+s)-X(t)\bigr\} \overset{\Dcal}{=} \bigl\{X(s)-X(0)\bigr\} \]

Particularly nice corollary to the definition of an $H$-self similarity process is that
if $\bigl\{X(t)\bigr\}$ is $H$-SS then $X(t)\overset{\Dcal}{\sim}t^H X(1)$. In turn,
this allows, for example to show that all zero-mean stochastic processes with stationary
increments share a similar auto-correlation pattern for all $t\neq s$
\[ \ex X(s) X(t) = \frac{1}{2}\Bigl( t^{2H} + s^{2H} - |t-s|^{2H}\Bigr) \ex|X(1)|^2 \]
Indeed, the identity $2 a b = a^2 + b^2 - (a-b)^2$ implies that\begin{align*}
	\ex X(s) X(t)
	&= \frac{1}{2}\biggl( \ex X(s)^2 + \ex X(t)^2 - \ex\bigl( X(s) - X(t) \bigr)^2 \biggr) \\
	&= \frac{1}{2}\biggl( \ex X(s)^2 + \ex X(t)^2 - \ex X(|s-t|)^2 \biggr) \\
	&= \frac{1}{2}\biggl( |s|^{2H} \ex X(1)^2 + |t|^{2H} \ex X(1)^2 - |s-t|^{2H} \ex X(1)^2 \biggr)
\end{align*}
where the second and the third lines follow from stationarity of increments and the
mentioned corollary respectively.

%% bulinskii2005teoriya2516755 p.47
Lastly, recall that a process $\bigl\{X(t)\bigr\}$ is said to have \textbf{I}ndependent
\textbf{I}ncrements if for any integer $n\geq1$ and every $(t_k)_{k=0}^n\in[0,\infty)$
with $0=t_0$ and $t_k < t_{k+1}$ the random variables $\bigl(X(t_k) - X(t_{k-1})\bigr){k=1}^n$
and $X(0)$ are jointly independent.

% section definition (end)

\section{Fractional Brownian Motion} % (fold)
\label{sec:fractional_brownian_motion}

It was mentionend in section~\ref{sec:the_crossing_tree_of_brownian_motion} theat Brownina
Motion is an example of a self-similar process. In fact it is an $\sfrac{1}{2}$-SSSI process.
Indeed, let $a>0$ and consider a process $V(t) = \frac{1}{\sqrt{a}} B(at)$. Obviously $V(0) = 0$ almost
surely, since $V(0) = \sfrac{1}{\sqrt{a}} B(a 0) = 0$. Furthermore this process inherits
stationary increments from $B(t)$ : \begin{align*}
	\{ V(t+s) - V(t) \} &= \biggl\{ \frac{1}{\sqrt{a}}( B(at+as) - B(at) ) \biggr\}\\
	&\overset{\Dcal}{=} \biggl\{ \frac{1}{\sqrt{a}}( B(as) ) \biggr\} \\
	&= \{ V(s) \}
\end{align*}
and independence of increments as well -- just pick $(t_k)_{k=1}^n$ with $t_k<t_{k+1}$ and
observe that $\bigl(V(t_{k+1}) - V(t_k)\bigr)_{k=1}^n$ are mutually independent since
$\bigl(B(s_{k+1}) - B(s_k)\bigr)_{k=1}^n$ for $s_k = a t_k$ are. Path continuity follows
from continuity of sampel paths of $B(t)$ and the fact that $t\to a t$ is a continuous map. 
Finally $\mathcal{N}(0, at) \sim \sqrt{a} \mathcal{N}(0,t)$ implies that $V(t)$ is Brownian
Motion as well.

Another example of an $H$-SSSI process, which is frequently used as a reference for
self-similarity and scale invariance studies is the \textbf{f}ractional \textbf{B}rownian
\textbf{M}otion, -- a generalisation of the $\tfrac{1}{2}$-SS Brownian Motion to a general
$H$-SS Gaussian process with $H\in (0, 1)$.

Formally, fractional Brownian Motion introduced in \cite{doi:10.1137/1010093} is a
mean zero Gaussian process $\bigl\{X(t)\bigr\}$, (its every finite-dimensional joint
distributions are multivariate normal), with the covariance structure of a general
zero-mean $H$-SSSI stochastic process:
\[ \ex X(s) X(t) = \frac{1}{2} \bigl(t^{2H} + s^{2H} - |t-s|^{2H}\bigr) \]
%% bulinskii2005teoriya2516755 p.51

Theorem 1.3.3 on p.~6 of \cite{embrechtsselfsimilar} establishes that a fractional
Browninan Motion $\{B_h(t)\}_{t\geq 0}$ is an $H$-SSSI process, for the follwing integral
representation up to a mutliplicative constant
\[
\int_{-\infty}^0 (t-s)^{H-\tfrac{1}{2}} - (-s)^{H-\tfrac{1}{2}} dB(s)
+ \int_0^t |t-s|^{H-\tfrac{1}{2}} dB(s)
\]
If $H = 1$ then the fractional broader motion degenreates to $B_1(t) = tB(1)$ almost
surely. The class of all Fractional Brownian motions conides with the class of all
Gaussian self-similar processes with stationary incerments, see \cite{embrechtsselfsimilar}. 

In fact, fractional Brownian motion is an example of a broader class of self-similar
processes known as the \textbf{Hermite} processes. They exhibit non-Gaussianity and have
strongly dependent incerments.

% section fractional_brownian_motion (end)

\section{Hermite processes} % (fold)
\label{sec:hermite_processes}

Hermite processes inherit their name from the stochastic integral kernel used
in their definition, and are an extremenly important example of processes finite-
dimensional joint distributions of which depart significantly Gaussian. 

A probabilistic Hermite polynomial of order $k\geq0$ is defined as 
\[ H_k(x) = (-1)^k e^{-\frac{x^2}{2}} \frac{d^k}{dx^k} e^{-\frac{x^2}{2}} \]
and is a solution to the following differential equation
\[
 \frac{d}{dx}\biggl( e^{-\frac{x^2}{2}} \frac{d}{dx} f\biggr) + \lambda e^{-\frac{x^2}{2}} f = 0
\]
These polynomials constitute an orthogonal basis of the Hilbert space $\Lcal^2(\Real, \mu)$
with measure $\mu$ being the Lebesgue integral $\int e^{-\frac{x^2}{2}} dx$ and the inner
product given by
\[
\langle f, g\rangle = \int_\Real f g d\mu = \int_\Real f g e^{-\frac{x^2}{2}} dx
\]

The Hermite process of order $m$ with self-similarity parameter $H\in(\tfrac{1}{2},1)$,
denoted by $\bigl\{Z_H^p(t)\bigr\}_{t\geq 0}$, is defined via s mutlivaraive stochastic
integral
\[
Z_H^p(t) = \underset{\Real^m}{\int \cdots \int} \Biggl(
\int^t_0 \prod_{k=1}^m (u-x_k)_+^{-\frac{1}{2}-\frac{1-H}{m}} du\Biggr) dB(x_1) \ldots dB(x_q)
\]
over independent realizations of Gaussain white moise $dB(x_j)$.
The Hermite process of order $1$ is fractional Brownian Motion, whereas higher order
Hermite processes correspond to non-Gaussian $H$-SSSI, \cite{Bai20141710}.

% section hermite_processes (end)

\section{Weierstrass function} % (fold)
\label{sec:weierstrass_function}

Yet another example of non-Gaussian $H$-SSSI process is the random Weierstrass function
defined as the limiting random function of the sum of randomly phase shifted and scaled
cosines
\[
f(t)
= \sum_{k\in \mathbb{Z}} \lambda_0^{-kH}\bigl\{ \cos(\phi_k)
- \cos(2\pi \lambda_0^k t + \phi_k) \bigr\} 
\]
for $(\phi_k)_{k\in\mathbb{Z}}\sim \mathcal{U}(0,2\pi)$ iid -- the random phase shift of
the $k$-th layer harmonics. The parameter $\lambda_0$ govern the base scale of the
process and is knows as ``the fundamental harmonic''. The function $f_h(t)$ has
continuous paths, yet is nowhere differentiable, even in the deterministic case
when all phase shifts are identically zero. The Weierstrass function is known to be
$H$-self similar (see~\cite{decrouez2013estimation}).

% section weierstrass_function (end)

% chapter h_sssi_processes (end)