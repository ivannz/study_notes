Dmitriy Petrovich Alexeevskiy

\Omega = S^2 \subset \Real^2
What activates the retina? 
INPUT function I:S^2 \to\Real^+
 

Visualization:
Function $\to$ contours: $\obj{ \induc{ \omega\in \Omega} F\brac{\omega} = C }$
Vision by contours (gradient)

Generalized functional on functional spaces $\phi:\Fcal\to \Cplx$ where \Fcal\subseteq $\Cplx^\Omega$

Mean intensity within the receptive field according to receptive profile on a neuron (convolution integral)

INPUT is highly irregular: too many level sets (contours generalize level sets)

retina regularization -> neuronal regularization (90\% feedback from the visual cortex (?) ) (external kinked corpus (?)) : local contour information

Laplacian of Gaussian -- isotropic filter (does not favour any direction) \ref{D. Marr; E. Hildreth (29 February 1980). "Theory of Edge Detection". Proceedings of the Royal Society of London. Series B, Biological Sciences (The Royal Society) 207 (1167)}
integral of Gaussian -- ``smoothed'' value of a function
laplacian: \sum_{k=1}^N \frac{\partial }{\partial x_k}

Gradient is more important than the intensity in the INPUT

(differential geometry)
Consider the differential one-form df and a one-distribution of Ker(df)

two visual systems: $\mathbb{M}$ -- moving objects, $\mathbb{P}$ immobile objects; humans do not perceive polarization.

Focus on the $\mathbb{P}$ visual system.

the infinitesimal object of a curve is its tangent
k-jet of $\phi\brac{t}$ : Taylor expansion up to the k-th power.


1-jet -- direction. Space of 1-jets -- the contact space: M^3 \defn \bigcup_z proj T_zS^2 -- direction manifold of the retina, R\subseteq \Real^2.

\brac{x,y}\in S^2, \brac{\brac{x,y}, \theta}\in M^3 -- \theta direction angle

spectral indicatrisse (?)

the geometry of colour

\brac{x\brac{t},y\brac{t},\tehta\brac{t}} \in M^3 
2-sphere \to spiral

....

ds^2 - g_{ij} dx^i dx^j
\Delta = g^{ij} \partial_{x^i} \partial_{x^j}
\partial^2_t u = \Delta u

Time reversion in heat-diffusion equation solution -- sharpens the image


