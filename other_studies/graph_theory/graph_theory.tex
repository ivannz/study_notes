\documentclass[a4paper]{article}
\usepackage[utf8]{inputenc}

\usepackage{graphicx, url}

\usepackage{amsmath, amsfonts, xfrac}
\usepackage{mathtools}

\newcommand{\obj}[1]{{\left\{ #1 \right \}}}
\newcommand{\clo}[1]{{\left [ #1 \right ]}}
\newcommand{\clop}[1]{{\left [ #1 \right )}}
\newcommand{\ploc}[1]{{\left ( #1 \right ]}}

\newcommand{\brac}[1]{{\left ( #1 \right )}}
\newcommand{\induc}[1]{{\left . #1 \right \vert}}
\newcommand{\abs}[1]{{\left | #1 \right |}}
\newcommand{\nrm}[1]{{\left\| #1 \right \|}}
\newcommand{\brkt}[1]{{\left\langle #1 \right\rangle}}
\newcommand{\floor}[1]{{\left\lfloor #1 \right\rfloor}}
\newcommand{\ceil}[1]{{\left\lceil #1 \right\rceil}}

\newcommand{\Real}{\mathbb{R}}
\newcommand{\Cplx}{\mathbb{C}}
\newcommand{\Pwr}{\mathcal{P}}

\newcommand{\Ctx}{\mathbb{K}}
\newcommand{\Pat}{\mathbb{P}}
\newcommand{\supp}{\text{supp}}
\newcommand{\conf}{\text{conf}}


\newcommand{\defn}{\mathop{\overset{\Delta}{=}}\nolimits}

\usepackage[english, russian]{babel}
\newcommand{\eng}[1]{\foreignlanguage{english}{#1}}
\newcommand{\rus}[1]{\foreignlanguage{russian}{#1}}

\begin{filecontents}{references.bib}
@book{diestel2006,
  title={Graph Theory},
  author={Diestel, R.},
  isbn={9783540261834},
  lccn={99057468},
  series={Electronic library of mathematics},
  url={http://books.google.ru/books?id=aR2TMYQr2CMC},
  year={2006},
  publisher={Springer}
}
\end{filecontents}


\title{Summary of Diestel's graph theory}
\author{Nazarov Ivan, \rus{101мНОД(ИССА)}\\the DataScience Collective}

\begin{document}
\maketitle

\begin{abstract}
	A summary of a great book on graph theory.
\end{abstract}

\tableofcontents
\clearpage

Summary of book~\cite{diestel2006}.

\section{Chapter 1} % (fold)
\label{sec:chapter_1}

\subsection{Introduction to graphs} % (fold)
\label{sub:introduction_to_graphs}

A graph is an ordered pair $G=(V,E)$ with $E\subseteq V\times V$.
The number of vertices in $G$, $\abs{V}$, is known as the \textbf{order} of $G$ and is also denoted by $\abs{G}$.
At the same time $\abs{E}$ the number of edges in $G$ is the \textbf{size} of $G$, denoted by $\nrm{G}$.

If the orientation of a graph is irrelevant and looped edges are not prohibited,
then by a slight abuse of notation the product $V\times V$ can be treated as the set of all two-element subsets of $V$.

A vertex $v\in V$ is \textbf{incident} to an edge $e\in E$, written as $v\in e$, if $e=(u,v)$ or $(v,u)$ for some $u\in V$.
Vertices $x,y\in V$ are \textbf{adjacent}, or neighbours in $G$, if $(x,y)$ or $(y,x)\in E$. Otherwise, vertices $x$ and $y$ are \textbf{independent}.
Edges $e,f\in V$, $e\neq f$, are \textbf{adjacent} if there exists $v\in V$ such that $v\in e,f$.
A graph $(V,E)$ is \textbf{complete} if every $e_1, e_2\in E$ are adjacent.

A collection of nodes $S\subseteq V$ or edges $S\subseteq E$ is \textbf{stable}(\rus{внутренне устойчиво}), if every $x,y\in S$ are non-adjacent in $G$.

A graph $G_1 = (V_1,E_1)$ is isomorphic to $G_2=(V_2,E_2)$ if there exists an bijection $\phi:V_1\to V_2$ such that $(x,y)\in E_1$ if and only if $(\phi(x),\phi(y))\in E_2$.
Isomorphic graphs are identified.

A graph invariant is a map $\psi:\Gamma\to\Omega$ is called a graph invariant if it takes the same value in $\Omega$ for every isomorphic graph.

A graph $G'$ is a subgraph of $G$ if $V'\subseteq V$ and $E'\subseteq E$.
If $G'\subseteq G$ and $(u,v)\in E'$ for any $(u,v)\in E$ with $u,v\in V'$, then $G'$ is an induced subgraph of $G$.
A subgraph $G'$ of $G$ is \emph{spanning} if $V'=V$.

Minimality or maximality of any graph with some property refers to subgraph ordering.

Consider a graph (undirected) $G=(V,E)$ with no more that one edge per a pair of vertices.
For any $v\in V$ the set of neighbours of $v$ is defined as
\[N_G(v) \defn \obj{\induc{w\in V}\, (w,v)\in E}\]

For any $v\in V$ the \textbf{degree} of $v$, $d(v)$, is defined as
\[d(v) \defn \abs{\obj{\induc{ e\in E }\, v\in e }}\]
The minimum degree is $\delta(G)\defn \min_{v\in V} d(v)$ and the maximum -- $\Delta(G) \defn \max_{v\in V} d(v)$.
The \textbf{average} degree of $G$ is defined as \[d(G) \defn \frac{1}{\abs{V}} \sum_{v\in V} d(v)\] and obviously $\delta(G)\leq d(G)\leq \Delta(G)$.

The value $d(G)$, the average degree, measures globally what the vertex degree quantifies locally.
The ratio $\epsilon(G) \defn \sfrac{\abs{E}}{\abs{V}}$ is the average number of edges per vertex.
By definition there must be the following relation: $\epsilon(G) = \frac{1}{2} d(G)$.

The sum of all vertex degrees in a graph is equal to twice the number of edges.
Indeed, each vertex degree $d(v)$ accounts for all edges $(v,u)$ with $u\in N_G(v)$,
but then the degree of each vertex $w\in N_G(v)$ also counts every edge $(w,v)\in E$.
Therefore each edge is counted twice in the following sum
\[\sum_{v\in V} d(v)\]

The number of edges of odd degree is even in any graph.
Indeed, the set of all vertices can be split into two subsets:
nodes with odd degree $V'$ and all others -- with even degree.
Therefore \[\sum_{v\in V}d(v) = \sum_{v\in V'}d(v) + \sum_{v\notin V'} d(v) = 2 \abs{E}\]
implies that $\sum_{v\in V'} d(v)$, a sum of odd numbers, must be an even number, whence $\abs{V'}$ has to be even.





% subsection introduction_to_graphs (end)


% section chapter_1 (end)

%% diestel2006, p.~25
A \textbf{tree} is a connected acyclic graph, whereas dropping the
connectedness requirement yields a \textbf{forest}.

The following are equivalent:
\begin{itemize}
	\item A graph $T$ is a tree;
	\item The graph $T$ is minimally connected: dropping any edge from $T$ produces a disconnected graph, though $T$ is connected;
	\item Every pair of vertices in $T$ is joined by unique path within $T$;
	\item Tree $T$ is maximally acyclic: $T$ is acyclic, but linking any pair of non-adjacent nodes by an edge generates a cycle.
\end{itemize}

For any $u,v\in T$ the unique path joining them is denoted by $xTy$.

The connectedness minimality of a tree actually implies that every
connected graph contains a \textbf{spanning tree} -- a tree $T$ such
that it contains all vertices of $G$.
Indeed removing edges one-by-one from $G$ would yield a graph $G'$
such that removing an extra edge yields a disconnected graph.
This means that $G'$ is a tree.

The vertices of a tree can be enumerated in such a way
$\brac{v_i}_{i=1}^n$ that for every $k\geq 1$ the node $v_k$ has
a unique neighbour among $\brac{v_i}_{i=1}^{k-1}$.

A connected graph with $n$ vertices is a tree if and only if it has
$n-1$ edges.

A tree $T$ is a \textbf{rooted} tree if there is a distinguished
root vertex $r$.
Since every pair of vertices in a tree are connected by a unique
path, it is possible to define a partial order relation on the nodes
of $T$ as follows: $x\leq y$ whenever $x\in xTr$.
This is a so called \textbf{tree-order}.
\begin{description}
	\item[Reflexivity] \hfill \\
		For any $x\in T$ it is obviously true that $x$ lies on a
		path from $r$ to $x$ in $T$;
	\item[Asymmetry] \hfill \\
		If $x,y\in T$ are such that $x\leq y$ and $y\leq x$, then
		$x\in yTr$, meaning that $xTr\subseteq yTr$, and $y\in xTr$, 
		which in turn implies that $yTr\subseteq xTr$;
	\item[Transitivity] \hfill \\
		If $x\leq y$ and $y\leq z$ in $T$, then $y\in zTr$ and
		$x\in yTr$. Since path are unique, $yTr\subseteq zTr$ and
		$x\in yTr$ implies that $xTr$. Therefore it must be true
		that $x\in zTr$.
\end{description}

A rooted tree $T$ which is a subgraph of some $G=(V,E)$ is normal if 
the ends of every edge $(u,v)$ of $G$ are comparable in the tree-
order of $T$ whenever $u,v\in T$.

The \textbf{down-closure} of a node $y$ is defined as
\[\ceil{y} \defn \obj{\induc{w\in T}\, w\leq y}\]
For any $u,v\in \ceil{y}$ it is true that $u,v\in yTr$. Since $u$
and $v$ lie on the same path it path uniqueness implies that either
$u\leq v$ or $v\leq u$. Therefore the down-closure is a complete
poset -- a \textbf{chain}.





\bibliographystyle{plain}
\bibliography{references}

\end{document}
