\section{Dilworth's theorem} % (fold)
\label{sec:dilworth_s_theorem}

Consider a set of shapes to be cut optimally and efficiently from a sheet
of some material. Arrange the shapes according to the geomatric containment
partial order induced by ``subset'' relation. Every chain (linearly ordered
subset of a poset) in this containment Hasse diagramme is producible from a
single shape specimen.

The question is how many chains cover this ordered set?

\subsection{Definitions} % (fold)
\label{sub:definitions}
Let $(J,\geq)$ be a poset with partial order $\geq$. Any $i,j \in J$ are
comparable if either $(i,j)\in \geq$ or $j\geq i$, and incomparable otherwise.

A subset of pairwise comparable elements is a \textbf{chain}, similarly a set
of pairwise incomparable elements is an \textbf{anti-chain}.

\textbf{Width} is the size of the largest anti-chain and the \textbf{height} or
the \textbf{length} is the size of the largest chain minus 1 (the number of edges).

% subsection definitions (end)

\noindent the Theorem\hfill\\
A poset $(J,R)$ has width $n$ if and only if it can be covered by $n$ chains,
but not by fewer.

An obvious corollary is that if a poset can be covered by $n$ chains, then its
width cannot be less than $n$.

$\Rightarrow$ Width $n$ needs $n$ chains, since any two elements in the anti-chain
are incomparable. Therefore thay cannot be in any chain.

$\Leftarrow$

% section dilworth_s_theorem (end)

Applications: marriage theorem, max flow min cut over transportation graphs
(with edge capacity). Cut through the flow network, and define the capacity
of the cut as the sum of capacities of edges over the cut.

Another application: take a poset of pairs $(\mathbb{N}\times \mathbb{N}, \geq)$
with the order corresponding to component-wise partial order. This poset contains
no infinite anti-chains, but it contains an anti-chain of any finite order (proof
by induction). Thus it cannot it be covered by finitely many chains.


\section{Ganter's Colloquium} % (fold)
\label{sec:ganters_colloquium}

Is Formal Concept analysis useful?

Is it maths? Even mathemticians do not know what mathematics is, but they
have a clear feeling of what it might be.

Lattice theory, as a study of a paritcular algebraic structure, originated in
the 20-th century. Dedekind called it a ``dualgruppe''. Willie based the
research on FCA in philosophical foundations.

The main direction on any Hasse diagram of a lattice is eaither upward or
downward. The lattice permits the reconstruction if the original data, whereas
the dimensionality reduced data is not a fithful representation.

\rus{м14НД ИССА}

\noindent Definition \hfill \\
A formal context is a triple $(G,M,I)$ with $I\subseteq G\times M$ -- a relation
representing ``an object $g$ has an attribute $m$'' for any $(g,m)\in I$.

In a lattice diagram the nodes are the formal concepts and the edges represent
the ``subconcept'' relation.

\noindent ...\hfill\\

For any $A\subseteq G$ and $B\subseteq M$ put
\begin{align*}
	A' &= \{ m\in M\,:\, \forall g\in A\, (g,m)\in I \}\\
	B' &= \{ g\in G\,:\, \forall m\in B\, (g,m)\in I \}\,.
\end{align*}
The mapping $X\mapsto X''$ is a closure operator:\begin{itemize}
	\item idempotent: $(X'')'' = X''$;
	\item monotonic: $X\subseteq Y$ implies $X''\subseteq Y''$;
	\item extensivity: $X\subseteq X''$.
\end{itemize}

Nifty application: construct a lattice of formal concepts over McDonald's
allergen table.

Another application: see Revenko, ICFCA 2014.

Concept lattices not necessarily need to be computed in order to perform
FCA. For instance, Iceberg lattice (algrothim to compute it is Titanic, which
basically ``searches for an iceberg'' in the lattice -- not an effective one).

% section ganters_colloquium (end)

%% Not in the speech:
In pattern structures $\bigl(G, (D, \sqcap), \delta\bigr)$ the Galois connection
is
\[ A' = \sqcap\limits_{ g\in A } \delta(g) \,, \]
and
\[ d' = \{ g\in G\,:\, d \sqsubseteq \delta(g) \} \,, \]
with the connection property:
\[ d \sqsubseteq A' \Leftrightarrow A \subseteq d' \,,\]
since $a \sqsubset b$ if and only if $a = a\sqcap b$ for any $a,b\in D$.
%% End

