\documentclass[12pt]{article}         % класс документа - статья. Также report, book и др.
%\usepackage[14pt]{extsizes}
\usepackage{amstext,amsmath,amssymb}            % пакеты для формул
\usepackage{mathtext}           % позволяет использовать русские буквы в формулах
\usepackage{geometry}           % пакет для задания полей страницы командой \geometry
%\geometry{left=3cm,right=1.5cm,top=2cm,bottom=2cm}
\usepackage[cp1251]{inputenc}  % кодировка текста
\usepackage[T2A]{fontenc}       %пакет Т2А необходим для правильного отображения кириллицы и переноса слов
%\inputencoding{cp1251}          % тоже кодировка...
\usepackage[english,russian]{babel}     % языковой пакет - последний язык главный
\usepackage[unicode]{hyperref}  %создаёт гиперссылки на список литературы в pdf-файле
\usepackage{bm}                 % boldmath - пакет для жирного шрифта
\ifx\pdfoutput\undefined
\usepackage[dvips]{graphicx}
\else
\usepackage[pdftex]{graphicx}
\usepackage{epstopdf}
\fi   % пакет для включения рисунков в форматах png,pdf,jpg,mps,tif
\graphicspath{{fig/}}           % папка с рисунками

\usepackage{amsfonts}           % греческие символы и, возможно, что-то ещё
\usepackage{indentfirst}        % одинаковый отступ для первого параграфа и всего остального
\usepackage{cite}               % команда /cite{1,2,7,9} даёт ссылки
\usepackage{multicol}
\usepackage{array}              % нужен для создания таблиц
%\usepackage{pscyr}              % шрифты
\usepackage{url}
\def\tmdefault{ftm}
\input glyphtounicode.tex       % для поиска и копирования
\pdfgentounicode=1

\linespread{1.3}                % полтора интервала. Если 1.6, то два интервала
\pagestyle{plain}               % номерует страницы
\usepackage[font=small,format=plain,labelfont=bf,up,textfont=it,up]{caption}
%\numberwithin{figure}{section}  % для сквозной нумерации
%\numberwithin{table}{section}
%\numberwithin{equation}{section}

\setlength{\parskip}{1ex} %--skip lines between paragraphs
\setlength{\parindent}{0pt} %--don't indent paragraphs

\textheight=24cm % высота текста
\textwidth=16cm % ширина текста
\oddsidemargin=0pt % отступ от левого края
\topmargin=-1.5cm % отступ от верхнего края
\parindent=12pt % абзацный отступ
%\parskip=2pt % интервал между абзацами
\tolerance=2000 % терпимость к "жидким" строкам
\flushbottom % выравнивание высоты страниц


\usepackage{listings}
\usepackage{color}
\definecolor{Purple}{rgb}{0.44,0.00,0.94}
\definecolor{MyDarkGreen}{rgb}{0.0,0.4,0.0}
\definecolor{Blue}{rgb}{0.00,0.00,1.00}
\lstloadlanguages{Matlab}%
\lstset{language=Matlab,                        % Use MATLAB
        frame=single,                           % Single frame around code
        basicstyle=\small\ttfamily,             % Use small true type font
        keywordstyle=[1]\color{Blue}\bf,        % MATLAB functions bold and blue
        keywordstyle=[2]\color{Purple},         % MATLAB function arguments purple
        keywordstyle=[3]\color{Blue}\underbar,  % User functions underlined and blue
        identifierstyle=,                       % Nothing special about identifiers
                                                % Comments small dark green courier
        commentstyle=\color{MyDarkGreen}\small,
        stringstyle=\color{Purple},             % Strings are purple
        showstringspaces=false,                 % Don't put marks in string spaces
        tabsize=5,                              % 5 spaces per tab
        breaklines=true,
        %%% Put standard MATLAB functions not included in the default
        %%% language here
        morekeywords={imread,double,size,reshape,image},
        %
        %%% Put MATLAB function parameters here
        morekeywords=[2]{on, off, interp},
        %
        %%% Put user defined functions here
        morekeywords=[3]{FindESS, homework_example},
        %
        morecomment=[l][\color{Blue}]{...},     % Line continuation (...) like blue comment
        numbers=left,                           % Line numbers on left
        firstnumber=1,                          % Line numbers start with line 1
        numberstyle=\tiny\color{Blue},          % Line numbers are blue
        stepnumber=5                            % Line numbers go in steps of 5
        }


\begin{document}
\begin{center}
%\textbf{\large <<Машинное обучение>>}
\textbf{\large <<<Machine Learning>>}

%\textit{1 курс магистратуры, Компьютерная лингвистика}
\textit{Computer lingusistics, masters 1 year}

\end{center}
\begin{center}
% {\large Групповой или индивидуальный проект}
{\large Projects to be completed either indvidually or in groups}
\end{center}
\hrule
% Авторы:  Д.И.~Игнатов
Authors: Ignatov, D.


% Срок сдачи итогового отчета: 25.03.2016 (выбор задачи до 11.03.2016)
The project is to be chosen by 2016-03-11, final report is due 2016-03-25

\hrule


% \section*{Постановка задачи}
\section*{Problem statement}
% Под групповым проектом подразумевается коллективное выполнение задания, связанного с применением методов разработки данных и машинного обучения. Перед тем как приступить к выполнению проекта необходимо:
Group project entails solving a machine learning and data mining problem or challenge in collaboration with other students.
Before engaging with the project, it is necessary to:
\begin{enumerate}

  % \item Сформировать группы не более трех человек. Альтернативно можно работать над индивидуальным проектом.
  \item Form groups of at least one (individual project) and at most three students (group projects);
  % \item Найти (выбрать) набор данных для анализа.
  \item Pick a dataset to be analysed in the project;
  % \item Сформулирвать постановку задачи, описать данные и составить план решения.
  \item Provide concise and clear statement of research goals, a brief summary of the dataset
  and outline a roadmap for a likely solution.

\end{enumerate}
% Необходимо проинформировать проверяющих о выборе задачи до дедлайна (11.03.2016), а затем можно перейти к выполнению проекта. Подходящие наборы данных, например, можно найти на сайтах:
It is important to submit the project proposal for approval by a the faculty staff 
member before the first deadline (2016-03-11). Only after the project has been approved
it advisable to start working on it. Below is a short list of web repositories with
freely available dataset (the list is not exhaustive and the selection is not limited
to it):

 % \


 \begin{tabular}{|l|}

   \hline
\href{http://archive.ics.uci.edu/ml/}{UC Irvine Machine Learning Repository}\\
\url{http://www.kaggle.com/competitions}{}\\
\url{http://www.openml.org/}{}\\
\url{http://www-stat.stanford.edu/~tibs/ElemStatLearn/}{}\\
\url{http://lib.stat.cmu.edu/datasets}{}\\
\url{http://www.statsci.org/datasets.html}{}\\
\url{http://www.amstat.org/publications/jse/jse_data_archive.htm}{}\\
\url{http://www.physionet.org/physiobank/database}{}\\
\url{http://biostat.mc.vanderbilt.edu/twiki/bin/view/Main/DataSets}{}.\\
   \hline

 \end{tabular}

 \

% Приветствуется работа с текстовыми (лингвистическими) данными, например, \url{http://universaldependencies.org/}, а также \href{http://trec.nist.gov/}{TREC}, \href{http://www.clef-initiative.eu/dataset/test-collection}{}, \url{http://pan.webis.de/data.html} и т.п.
Projects concerning textual data analysis and natural language processing are encouraged.
To this end one may have a look at these web sites (but not limited to):
\url{http://universaldependencies.org/}, \href{http://trec.nist.gov/}{TREC},
\href{http://www.clef-initiative.eu/dataset/test-collection}{},
and \url{http://pan.webis.de/data.html} et c..


 \

% Примерное содержание отчета по проекту следующее:
The following structure of the report is suggested:
\begin{enumerate}
  % \item Формулировка задачи
  \item Problem statement;
  % \item Описание данных
  \item Dataset summary;
  % \item Обоснование выбора методов
  \item Methodology (justify the selected ML/DM approach);
  % \item Постановка / результаты экспериментов
  \item Experminet setup and results;
  % \item Сравнение методов
  \item Discussion (Comparisons, interpretations et c.);
  % \item Выводы
  \item Conslusion.
\end{enumerate}

\

% \textbf{Q.:} \textit{Есть ли ограничения снизу/сверху по размеру данных, текста отчета и набору применяемых алгоритмов машинного обучения?}
\textbf{Q.:} \textit{Are there any restriction to the size of the dataset, the report size, and/or the set of applied machine learning algorithms?}

% \textbf{A.:} \textit{Ограничения снизу есть. Данные не должны быть слишком маленькими, например, размером не менее 50 объектов $\times$ 10 признаков. Текст должен быть похож на подробный и понятный преподавателю или сокурснику рассказ о том, что Вы сделали, с таблицами, графиками, скриншотами и прочими вспомогательными иллюстрациями. В принципе, чем больше методов применено, тем лучше. Сравнение и интерпретация результатов обязательны. Если Вы, например, решили применить кластеризацию, то необходимо сравнить результаты работы несколько методов. Применение своего оригинального метода весьма приветствуется. Необходимо продемонстрировать всю цепочку работы с данными, включающую в себя их сбор, предобработку (шкалирование, удаление выбросов, отбор или извлечение признаков и т.п.), применение методов, сравнение, анализ ошибок и интерпретацию результатов.}
\textbf{A.:} \textit{Yes, there are. The dataset must not be too small, at least 50
objects with at least 10 features. The report must be relevant to the proposal, coherent
and sufficiently detailed, so that any reader, be it the teacher or a student, could be
able to understand the work you did, the research steps undertaken and the goals achieved.
This entails reasonable and relevant usage of tables, plots, or other visualization tools.
In general the more data analysis approached are used the better. Comparisons and
interpretations are crucial. For instance, when using clustering it is important to
compare the results of different methods, and analyse the effects of parameter tuning
within the same algorithm (at least empirically). Implementation of new ideas, methods
or algorithms are strongly encouraged.
It is necessary to demonstrate the complete data analysis pipeline: data collection,
data preprocessing (scaling, outlier elimination, feature selection, et c.), method
application, comparison, analysis and interpretation.}

The defense of the project will take place during the exam week. The highest grade is 10.

The following packages are suggested for the use in the project:
\begin{itemize}
  \item Orange \url{http://orange.biolab.si/};
  \item Weka \url{www.cs.waikato.ac.nz/ml/weka};
  \item Scikit-leran \url{http://scikit-learn.org/stable/}
  \item Matlab, or R.
\end{itemize}
% (см. иллюстрацию базовых возможностей в туториале).
For the comparison of the basic capabilities please refer to the illustration in the tutorial.
% Приветствуется работа с применением частых множеств признаков, см., например, пакет SPMF \url{http://www.philippe-fournier-viger.com/spmf/}.
The use of frequent itemset mining, for example SPMF package (\url{http://www.philippe-fournier-viger.com/spmf/})
is encouraged, but not mandatory.

% Письма с выбором задачи и финальным отчетом необходимо прислать на почту: \textit{anya\_potapenko@mail.ru}, \textit{dmitrii.ignatov@gmail.com} и \textit{akutuzov72@gmail.com}. Тема письма должна иметь вид \textbf{[CL-ML2016-Project] Фамилия Имя}.
Emails with the project proposals and the final version of the reports are to be sent to these emails:
``to'' \textit{dmitrii.ignatov@gmail.com}, with a ``cc'' to both \textit{anya\_potapenko@mail.ru} and
\textit{akutuzov72@gmail.com}.
The title of the letter must have the following format (no quotes)\hfill\\
\textbf{[CL-ML2016-Project] ``LAST NAME'' ``FIRST NAME''}.



\end{document}
