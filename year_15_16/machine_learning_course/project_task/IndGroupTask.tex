\documentclass[12pt]{article}         % класс документа - статья. Также report, book и др.
%\usepackage[14pt]{extsizes}
\usepackage{amstext,amsmath,amssymb}            % пакеты для формул
\usepackage{mathtext}           % позволяет использовать русские буквы в формулах
\usepackage{geometry}           % пакет для задания полей страницы командой \geometry
%\geometry{left=3cm,right=1.5cm,top=2cm,bottom=2cm}
\usepackage[cp1251]{inputenc}  % кодировка текста
\usepackage[T2A]{fontenc}       %пакет Т2А необходим для правильного отображения кириллицы и переноса слов
%\inputencoding{cp1251}          % тоже кодировка...
\usepackage[english,russian]{babel}     % языковой пакет - последний язык главный
\usepackage[unicode]{hyperref}  %создаёт гиперссылки на список литературы в pdf-файле
\usepackage{bm}                 % boldmath - пакет для жирного шрифта
\ifx\pdfoutput\undefined
\usepackage[dvips]{graphicx}
\else
\usepackage[pdftex]{graphicx}
\usepackage{epstopdf}
\fi   % пакет для включения рисунков в форматах png,pdf,jpg,mps,tif
\graphicspath{{fig/}}           % папка с рисунками

\usepackage{amsfonts}           % греческие символы и, возможно, что-то ещё
\usepackage{indentfirst}        % одинаковый отступ для первого параграфа и всего остального
\usepackage{cite}               % команда /cite{1,2,7,9} даёт ссылки
\usepackage{multicol}
\usepackage{array}              % нужен для создания таблиц
%\usepackage{pscyr}              % шрифты
\usepackage{url}
\def\tmdefault{ftm}
\input glyphtounicode.tex       % для поиска и копирования
\pdfgentounicode=1

\linespread{1.3}                % полтора интервала. Если 1.6, то два интервала
\pagestyle{plain}               % номерует страницы
\usepackage[font=small,format=plain,labelfont=bf,up,textfont=it,up]{caption}
%\numberwithin{figure}{section}  % для сквозной нумерации
%\numberwithin{table}{section}
%\numberwithin{equation}{section}

\setlength{\parskip}{1ex} %--skip lines between paragraphs
\setlength{\parindent}{0pt} %--don't indent paragraphs

\textheight=24cm % высота текста
\textwidth=16cm % ширина текста
\oddsidemargin=0pt % отступ от левого края
\topmargin=-1.5cm % отступ от верхнего края
\parindent=12pt % абзацный отступ
%\parskip=2pt % интервал между абзацами
\tolerance=2000 % терпимость к "жидким" строкам
\flushbottom % выравнивание высоты страниц


\usepackage{listings}
\usepackage{color}
\definecolor{Purple}{rgb}{0.44,0.00,0.94}
\definecolor{MyDarkGreen}{rgb}{0.0,0.4,0.0}
\definecolor{Blue}{rgb}{0.00,0.00,1.00}
\lstloadlanguages{Matlab}%
\lstset{language=Matlab,                        % Use MATLAB
        frame=single,                           % Single frame around code
        basicstyle=\small\ttfamily,             % Use small true type font
        keywordstyle=[1]\color{Blue}\bf,        % MATLAB functions bold and blue
        keywordstyle=[2]\color{Purple},         % MATLAB function arguments purple
        keywordstyle=[3]\color{Blue}\underbar,  % User functions underlined and blue
        identifierstyle=,                       % Nothing special about identifiers
                                                % Comments small dark green courier
        commentstyle=\color{MyDarkGreen}\small,
        stringstyle=\color{Purple},             % Strings are purple
        showstringspaces=false,                 % Don't put marks in string spaces
        tabsize=5,                              % 5 spaces per tab
        breaklines=true,
        %%% Put standard MATLAB functions not included in the default
        %%% language here
        morekeywords={imread,double,size,reshape,image},
        %
        %%% Put MATLAB function parameters here
        morekeywords=[2]{on, off, interp},
        %
        %%% Put user defined functions here
        morekeywords=[3]{FindESS, homework_example},
        %
        morecomment=[l][\color{Blue}]{...},     % Line continuation (...) like blue comment
        numbers=left,                           % Line numbers on left
        firstnumber=1,                          % Line numbers start with line 1
        numberstyle=\tiny\color{Blue},          % Line numbers are blue
        stepnumber=5                            % Line numbers go in steps of 5
        }


\begin{document}
\begin{center}
\textbf{\large <<Машинное обучение>>}

\textit{1 курс магистратуры, Компьютерная лингвистика}

\end{center}
\begin{center}
{\large Групповой или индивидуальный проект}
\end{center}
\hrule
Авторы:  Д.И.~Игнатов


Срок сдачи итогового отчета: 25.03.2016 (выбор задачи до 11.03.2016)

\hrule


\section*{Постановка задачи}
Под групповым проектом подразумевается коллективное выполнение задания, связанного с применением методов разработки данных и машинного обучения. Перед тем как приступить к выполнению проекта необходимо:

\begin{enumerate}

  \item Сформировать группы не более трех человек. Альтернативно можно работать над индивидуальным проектом.
  \item Найти (выбрать) набор данных для анализа.
  \item Сформулирвать постановку задачи, описать данные и составить план решения.

\end{enumerate}
Необходимо проинформировать проверяющих о выборе задачи до дедлайна (11.03.2016), а затем можно перейти к выполнению проекта. Подходящие наборы данных, например, можно найти на сайтах:

 \


 \begin{tabular}{|l|}

   \hline
\href{http://archive.ics.uci.edu/ml/}{UC Irvine Machine Learning Repository}\\
\url{http://www.kaggle.com/competitions}{}\\
\url{http://www.openml.org/}{}\\
\url{http://www-stat.stanford.edu/~tibs/ElemStatLearn/}{}\\
\url{http://lib.stat.cmu.edu/datasets}{}\\
\url{http://www.statsci.org/datasets.html}{}\\
\url{http://www.amstat.org/publications/jse/jse_data_archive.htm}{}\\
\url{http://www.physionet.org/physiobank/database}{}\\
\url{http://biostat.mc.vanderbilt.edu/twiki/bin/view/Main/DataSets}{}.\\
   \hline

 \end{tabular}

 \

Приветствуется работа с текстовыми (лингвистическими) данными, например, \url{http://universaldependencies.org/}, а также \href{http://trec.nist.gov/}{TREC}, \href{http://www.clef-initiative.eu/dataset/test-collection}{}, \url{http://pan.webis.de/data.html} и т.п.


 \

Примерное содержание отчета по проекту следующее:
\begin{enumerate}
  \item Формулировка задачи
  \item Описание данных
  \item Обоснование выбора методов
  \item Постановка / результаты экспериментов
  \item Сравнение методов
  \item Выводы
\end{enumerate}

\

\textbf{Q.:} \textit{Есть ли ограничения снизу/сверху по размеру данных, текста отчета и набору применяемых алгоритмов машинного обучения?}

\textbf{A.:} \textit{Ограничения снизу есть. Данные не должны быть слишком маленькими, например, размером не менее 50 объектов $\times$ 10 признаков. Текст должен быть похож на подробный и понятный преподавателю или сокурснику рассказ о том, что Вы сделали, с таблицами, графиками, скриншотами и прочими вспомогательными иллюстрациями. В принципе, чем больше методов применено, тем лучше. Сравнение и интерпретация результатов обязательны. Если Вы, например, решили применить кластеризацию, то необходимо сравнить результаты работы несколько методов. Применение своего оригинального метода весьма приветствуется. Необходимо продемонстрировать всю цепочку работы с данными, включающую в себя их сбор, предобработку (шкалирование, удаление выбросов, отбор или извлечение признаков и т.п.), применение методов, сравнение, анализ ошибок и интерпретацию результатов.}

Защита проекта будет проходить во время зачёта. Максимальный балл 10.

В качестве программных продуктов для решения задачи рекомендуются программные продукты Orange \url{http://orange.biolab.si/}, Weka \url{www.cs.waikato.ac.nz/ml/weka}, Scikit-leran \url{http://scikit-learn.org/stable/}, Matlab или R (см. иллюстрацию базовых возможностей в туториале).


%Приветствуется работа с применением частых множеств признаков, см., например, пакет SPMF \url{http://www.philippe-fournier-viger.com/spmf/}.


Письма с выбором задачи и финальным отчетом необходимо прислать на почту: \textit{anya\_potapenko@mail.ru}, \textit{dmitrii.ignatov@gmail.com} и \textit{akutuzov72@gmail.com}. Тема письма должна иметь вид \textbf{[CL-ML2016-Project] Фамилия Имя}.


\end{document}
