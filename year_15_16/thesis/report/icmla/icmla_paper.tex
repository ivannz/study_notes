\documentclass[conference]{IEEEtran}
% \documentclass[a4paper]{article}
% \documentclass[a4paper,14pt]{extarticle}

%% >
% \usepackage{extsizes}
\linespread{1.3}
\usepackage[T2A]{fontenc}
%% <

\usepackage[utf8]{inputenc}

%% >
\usepackage{indentfirst}
%% <

% \usepackage{geometry}
% \geometry{top=2cm} % отступ сверху
% \geometry{bottom=2cm} % отступ снизу
% \geometry{left=3.5cm}
% \geometry{right=2cm} % отступ справа

% \usepackage{fullpage}
\usepackage[square,numbers,sort&compress]{natbib}

\usepackage[mathcal]{euscript}
\usepackage{booktabs}

\usepackage{graphicx, url}

\usepackage{amsmath, amsfonts, amssymb, amsthm}
\usepackage{mathptmx}
\usepackage{algorithm2e}
\usepackage{subcaption}
\usepackage{caption}
\usepackage{float}

\newcommand{\ex}{\mathop{\mathbb{E}}\nolimits}
\newcommand{\pr}{\mathop{\mathbb{P}}\nolimits}
\newcommand{\Bcal}{\mathcal{B}}
\newcommand{\Hcal}{\mathcal{H}}
\newcommand{\Ncal}{\mathcal{N}}
\newcommand{\Kcal}{\mathcal{K}}
\newcommand{\BigO}{\mathcal{O}}
\newcommand{\Xcal}{\mathcal{X}}
\newcommand{\Ycal}{\mathcal{Y}}
\newcommand{\Zcal}{\mathcal{Z}}
\newcommand{\Real}{\mathbb{R}}
\newcommand{\nil}{\mathbf{0}}
\newcommand{\one}{\mathbf{1}}
\newcommand{\Lcal}{\mathcal{L}}
\newcommand{\Cplx}{\mathbb{C}}
\newcommand{\diag}{\mathop{\text{diag}}\nolimits}

\usepackage[english, russian]{babel}
\newcommand{\eng}[1]{\foreignlanguage{english}{#1}}
\newcommand{\rus}[1]{\foreignlanguage{russian}{#1}}

\def\bbljan{Jan.}
\def\bblfeb{Feb.}
\def\bblmar{Mar.}
\def\bblapr{Apr.}
\def\bblmay{May}
\def\bbljun{Jun.}
\def\bbljul{Jul.}
\def\bblaug{Aug.}
\def\bblsep{Sep.}
\def\bbloct{Oct.}
\def\bblnov{Nov.}
\def\bbldec{Dec.}

\title{Conformalized Kernel Ridge Regression}
\author{Nazarov, I. N., Burnaev, E. V.}

\begin{document}
\selectlanguage{english}
\maketitle

\section{Introduction} % (fold)
\label{sec:introduction}

In many applied situations, like anomaly detection in telemetry of some equipment,
online filtering and monitoring of potentially interesting events, or power grid
load balancing, it is necessary not only to make optimal predictions with respect
to some loss, but also to be able to quantify the degree of confidence in the obtained
forecasts. At the same time it is necessary to take into consideration exogenous
variables, that in certain ways affect the object of study.

Predictive modelling techniques are concerned with studying pattern of dependence
between the target variable $y$ and the input variable $x$ based on a finite train
sample of possibly noisy observations $(X, y)$ from $\Omega\times \Ycal \subseteq
\Xcal\times\Real$. These methods postulate that there exits a stable unobserved
relationship $y_x = \ex(y|X=x) = f(x)$: the observations of $y$ at any given
$x\in \Omega$ are corrupted by additive noise:
\begin{equation} \label{eq:signal_model}
  y_x = f(x) + \xi_x \,,
\end{equation}
with $\ex \xi_x = 0$ for all $x\in \Omega$. Regression methods allow one to construct
an estimate $\hat{f}$ of the true relation $f$, which can be used to produce point
predictions at any $x$.

These methods, however, do not readily provide the user with any sort of prediction
confidence measure, and in order to acquire such measure additional probabilistic
assumptions regarding the observation model (eq.~\ref{eq:signal_model}) are required.
With their help, it is usually quite straightforward to get a posterior probability
measure for unobserved data given the training data using standard Bayesian derivation
methods.

Consider Kernel Ridge Regression -- a model that combines ridge regression with the
kernel trick. It learns a function $\hat{f}$ which, given training sample $X = (x_i)_{i=1}^n$,
$X\in \Xcal^{n\times 1}$ and $y \in \Real^{n\times 1}$, solves
\begin{equation*}
  \|y - f\|^2 + \lambda \|f\|^2 \to \min_{f \in \Hcal} \,,
\end{equation*}
with $f \in \Real^{n\times 1}$ being a column-vector of $f$ evaluated at every input
in $X$. Here $\Hcal$ is the canonical Reproducing Kernel Hilbert Space associated with
a Mercer-type kernel $\Kcal:\Xcal\times \Xcal\mapsto \Real$. The Representer theorem,
\cite{scholkopf2002}, states that the solution $\hat{f}:\Xcal\mapsto\Real$ is of the
form $f(x) = k_x' \beta$, for $k_x : x\mapsto \Phi(x)$ and $\Phi = (\phi(x_i))_{i=1}^n \in \Hcal^{n\times 1}$.
If $e_j$ is the $j$-th unit vector in $\Real^{n\times 1}$, and $K_{XX}$ is the Gram
matrix of $\Kcal$ over $(x_i)_{i=1}^n$, then $f(x_j) = k_{x_j}' \beta = e_j' K_{XX} \beta$
and  $\| f \|^2 = \beta' K_{XX} \beta$. Thus the kernel ridge regression problem is
equivalent to this finite-dimensional convex minimization problem:
\begin{equation*}
  \|y - K_{XX} \beta \|^2 + \lambda \beta' K_{XX} \beta
    \to \min_{\beta\in \Real^{n\times 1}} \,,
\end{equation*}
which yields the optimal weight vector $\hat{\beta}$ and prediction at $x^*\in \Xcal$
given by
\begin{equation} \label{eq:krr_approx}
  \hat{\beta} = (\lambda I_n + K_{XX})^{-1} y \, \text{ and }\,
  \hat{y}(x^*) = k_{x^*}'\hat{\beta} \,,
\end{equation}
respectively.

In this basic form KRR does not give any kind of estimate of confidence in the prediction
it makes, however if it is viewed from the Gaussian process perspective, \cite{rasmussen2006},
which is essentially a Bayesian version of KRR, then predictive confidence intervals
are readily available.

A Gaussian Process $(y_x)_{x\in \Xcal}$, with mean $m : \Xcal \mapsto \Real$ and
covariance kernel  $\Kcal : \Xcal \times \Xcal \mapsto \Real$ is a random process
such that for any $n\geq1$ and any $X = (x_i)_{i=1}^n \in \Xcal$ the $n\times 1$
vector $y_X = (y(x_i))_{i=1}^n$ is Gaussian:
\begin{equation*}
  y_X \sim \Ncal_n(m_X, K_{XX}) \,,
\end{equation*}
where $m_X = (m(x_i))_{i=1}^n$. The conditional distribution of targets in a test sample
$y_{X^*} = (y_{x^*_j})_{j=1}^l$, with respect to the train sample $y_X = (y_{x_i})_{i=1}^n$
is given by
\begin{equation} \label{eq:cond_distr}
  y_{X^*}\vert_{y_X}
    \sim \Ncal_l\bigl(
      m_{X^*} + K_{X^*X} Q_X (y_X - m_X),
      \Sigma_K(X^*)
    \bigr)
    \,,
\end{equation}
where $\Sigma_K(X^*) = K_{X^*X^*} - K_{X^*X} Q_X K_{XX^*}$, $Q_X = \bigl(K_{XX}\bigr)^{-1}$,
and $K_{XX^*} = (\Kcal(x_i, x^*_j))\in \Real^{n\times l}$.

Bayesian version of KRR posits a prior on the underlying functions $f\sim GP(0, \sigma^2 \Kcal)$
and independent Gaussian white noise $\xi_x\sim \Ncal(0, \sigma^2 \lambda)$ in model~\ref{eq:signal_model},
for $\sigma^2 > 0$. In this setting, eq.~\ref{eq:cond_distr} implies that the distribution
of a yet unobserved target $y_{x^*}$ at $x^*\in \Xcal$, conditional on the train data $(X, y_X)$,
is
\begin{equation} \label{eq:gp_cond_dist}
{y_{x^*}}_{|y_X}
  \sim \Ncal\bigl(\hat{y}_{y_X}(x^*), \sigma^2 \sigma_K^2(x^*)\bigr) \,,
\end{equation}
with $\hat{y}_{y_X}(x^*) = K_X(x^*)' Q_X y_X$, and
\begin{equation*}
  \sigma_K^2(x^*)
    = \lambda + K(x^*, x^*) - K_X(x^*)' Q_X K_X(x^*) \,,
\end{equation*}
where $Q_X = \bigl(\lambda I_n + K_{XX}\bigr)^{-1}$, $K_{XX} = (K(x_i,x_j))_{ij}$,
and $K_X = (K(x_i, \cdot))_{i=1}^n: \Xcal \mapsto \Real^{n\times1}$. Thus, the $1 - \alpha$
confidence interval is thus given by
\begin{equation} \label{eq:gp_conf_int}
\Gamma^\alpha_{y_X}(x^*)
  = \hat{y}_{y_X}(x^*)
  + \sigma \sqrt{\sigma_K^2(x^*)}
  \times [z_{\frac{\alpha}{2}}, z_{1-\frac{\alpha}{2}}]
  \,,
\end{equation}
where $z_\gamma$ is the $\gamma$ quantile of $\Ncal(0, 1)$. Additionally, this version
naturally permits estimation of parameters of the underlying kernel $\Kcal$ through
maximization of the joint likelihood of the train data $(X, y_X)$:
\begin{align} \label{eq:bkrr_likelihood}
  \Lcal
    &= -\frac{n}{2} \log 2\pi
    - \frac{n}{2}\log \sigma^2
    - \frac{1}{2}\log \lvert R_X \rvert \nonumber\\
    &- \frac{1}{2\sigma^2} y' R_X^{-1} y
    \,,
\end{align}
where $R_X = \lambda I_n + K_{XX}$, and $K_{XX}$ depends on the hyper-parameters of
$\Kcal$ (shape, precision et c.).

% The distributional assumptions used to derive this posterior distribution are too
% restrictive: the true relationship $f(x)$ is drawn from a Gaussian Process, and the
% additive noise $\xi_x$ is Gaussian (though many uncontrollable factors are usually
% quite faithfully modelled by a Gaussian distribution according to the CLT).

It is desirable to have distribution-free method that measures confidence of predictions
of a machine learning algorithm. One such method is ``Conformal prediction'' -- an
approach developed in \cite{vovk2005}, which under standard independence assumptions
yields a set in the space of targets, that contains yet unobserved data with a pre-specified
probability. In this study, we provide empirical evidence supporting the claim that
when model assumptions do hold, the conformal confidence sets, constructed over the
Kernel Ridge Regression with isotropic Gaussian kernel do not perform worse than
the prediction confidence intervals of a Bayesian version of the KRR. The report is
structured as follows: in section~\ref{sec:conformal_prediction} a concise overview
of what conformal prediction is and what is required to construct such kind of confidence
predictor is given. Section~\ref{sec:conformalized_krr} describes the particular steps
needed to build a conformal predictor atop the kernel ridge regression. The main
empirical study is reported in section~\ref{sec:numerical_study}, where we study the
properties of the predictor in a batch learning setting for a KRR with specific kernel
(isotropic Gaussian). We conclude this study with discussion and prospects of future
research on this topic (sec.~\ref{sec:conclusion_and_further_work}).

% section introduction (end)

\section{Conformal prediction} % (fold)
\label{sec:conformal_prediction}

Conformal prediction is a distribution-free technique designed to yield a statistically
valid confidence sets for predictions made by machine learning algorithms. The key
advantage of the method is that it offers coverage probability guarantees under
standard IID assumptions, even in cases when assumptions of the underlying prediction
algorithm fail to be satisfied. The method was introduced in \cite{vovk2005} for online
supervised and unsupervised learning.

Let $\Zcal$ denote the object-target space $\Xcal \times \Ycal$. At the core of a
conformal predictor is a measurable map $A: \Zcal^*\times \Zcal \mapsto \Real$, a
Non-Conformity Measure (NCM), which quantifies how much different $z_{n+1} \in \Zcal$
is relative to a sample $Z_{:n} = (z_i)_{i=1}^n\in\Zcal$. A conformal predictor over
$A$ is a procedure, which for every sample $Z_{:n}$, a test object $x_{n+1} \in \Xcal$,
and a confidence level $\alpha \in (0,1)$, gives a confidence set $\Gamma_{Z_{:n}}^\alpha(x^*)$
for the target value $y_{n+1}$:
\begin{equation} \label{eq:conf_pred_set}
  \Gamma_{Z_{:n}}^\alpha(x_{n+1})
    = \bigl\{ y\in \Ycal \,:\, p_{Z_{:n}}(\tilde{z}^y_{n+1}) \geq \alpha \bigr\} \,,
\end{equation}
where $\tilde{z}^y_{n+1} = (x_{n+1}, y)$ a synthetic test observation with target
label $y$. The function $p:\Zcal^*\times (\Xcal\times \Ycal)\mapsto [0,1]$ measures
the likelihood of $\tilde{z}$ based on its non-conformity with $Z_{:n}$, and is
given by
\begin{equation} \label{eq:conf_p_value}
  p_{Z_{:n}}(\tilde{z})
    = (n+1)^{-1} \bigl\lvert\{ i \,:\,
      \eta_i^{\tilde{z}} \geq \eta_{n+1}^{\tilde{z}} \}\bigr\rvert \,,
\end{equation}
where $i=1,\ldots, n+1$, and $\eta_i^{\tilde{z}} = A(S^{\tilde{z}}_{-i}, S^{\tilde{z}}_i)$
-- the non-conformity of the $i$-th observation with respect to the augmented sample
$S^{\tilde{z}} = (Z_{:n}, {\tilde{z}}^y_{n+1}) \in \Zcal^{n+1}$. For any $i$, $S^{\tilde{z}}_i$
is the $i$-th element of the sample, and $S^{\tilde{z}}_{-i}$ is the sample with the $i$-th
observation omitted.

% Conformal prediction can also be constructed atop ``conformity measures'', as opposed
% to ``non-conformity measures'', which in this case requires that the ``$\geq$'' sign
% be switched to ``$\leq$'' in eq.~\ref{eq:conf_p_value} if a conformity measure is used.

For every possible value $z$ of an object $Z_{n+1}$ the conformal procedure tests
$H_0: Z_{n+1} = z$, and then inverts the test to get a confidence region. The hypothesis
tests are designed to have a fixed empirical type-I error rate $\alpha$ based on
the observed sample $Z_{:n}$ and hypothesized $z$. In classification, the class label
hypotheses for new $x_{n+1}$ are tested, whereas in regression the test are conducted
for target levels $y_{n+1}$.

In \cite{vovk2005}, chapter 2, is has been shown, that for sequences of iid examples
$(z_n)_{n \geq1} \sim P$, the coverage probability of the prediction set $\Gamma^\alpha$,
\ref{eq:conf_pred_set}, is at least $1-\alpha$ and successive errors are independent
in online learning and prediction setting. The outlined procedure guarantees unconditional
validity: for any $\alpha \in (0,1)$
\begin{equation} \label{eq:conservative_coverage}
  \pr_{Z_{:n}\sim P} \bigl(
    y_n \notin \Gamma^\alpha_{Z_{:(n-1)}}(x_n)
  \bigr) \leq \alpha \,,
\end{equation} 
where $(x_n, y_n) = z_n$.
Intuitively, the event $y_n \notin \Gamma^\alpha_{Z_{:(n-1)}}(x_n)$ is equivalent
to $\eta_n = A(Z_{-n}, Z_n)$ being among the largest $\lfloor n\alpha\rfloor$ values
of $\eta_i = A(Z_{-i}, Z_i)$, which is equal to $\frac{\lfloor n\alpha\rfloor}{n}$,
due to exchangeability of $Z_{:n}$ (for a rigorous proof see \cite{vovk2005},
chapter 8).

The choice of \textbf{NCM} affects the size of the confidence sets and the computational
burden of the conformal procedure. In the general case computing eq.~\ref{eq:conf_pred_set}
requires exhaustive search through the target space $\Ycal$, which is infeasible in general
regression setting. However, for specific non-conformity measures it is possible to
come up with efficient procedures for computing the confidence region as demonstrated
in \cite{vovk2005} and sec.~\ref{sec:conformalized_krr} of this work.

\subsection{Confromalized ridge regression} % (fold)
\label{sub:conformalized_ridge_regression}

Bayesian Ridge Regression, a Bayesian KRR with linear kernel, basically assumes
that $y = x'\beta + \epsilon_x$ for any $x\in \Xcal = \Real^{d\times 1}$ and
$\Ycal = \Real$, and
\begin{enumerate}
  \item $\beta\sim\Ncal_d(0, \sigma^2 \lambda^{-1} I_p)$;
  \item $\epsilon_x\sim \Ncal(0, \sigma^2)$ and iid for all $x\in \Real^{d\times 1}$;
  \item $\beta$ and $\epsilon_x$ are independent.
\end{enumerate}
The properties of ``Conformalized Ridge Regression'' (CRR) are studied in \cite{burnaevV14}
for a conformity measure
\begin{align} \label{eq:crr_ncm}
  \eta_i &= A(Z_{-i}, Z_i) \nonumber \\
    &= \bigl\lvert\{j\,:\, \hat{r}_j \geq \hat{r}_i \} \bigr\rvert \wedge
       \bigl\lvert\{j\,:\, \hat{r}_j \leq \hat{r}_i \} \bigr\rvert \,,
\end{align}
where $\hat{r} = \hat{r}_Z$ are the ridge regression residuals given by
\begin{equation} \label{eq:rr_insample_resid}
  \hat{r}_Z = (I_n - X (\lambda I_p + X'X)^{-1} X') y \,,
\end{equation}
and $X\in \Real^{n\times d}$ is the design matrix.

The main result of \cite{burnaevV14} states that under relaxed BRR assumptions if
a sequence $(x_n)_{n\geq1}\in\Xcal$ is i.i.d. with an non-singular second moment
matrix $\ex x_1x_1' \succeq 0$, then for all sufficiently large $n$ the CRR
produces confidence regions that lose little efficiency compared to Bayesian
confidence interval when Gaussianity is valid. In particular, the upper endpoint
of the prediction interval deviates asymptotically as $\mathcal{O}_p\bigl(n^{-\frac{1}{2}}\bigr)$.

In the paper it was also shown that for any $\alpha \in (0,1)$ the confidence
region $\Gamma^\alpha$ produced by CRR procedure for the conformity measure in
eq.~\ref{eq:crr_ncm} is equivalent to the intersection of confidence sets yielded
by conformal procedures with non-conformity measures given by $\eta_i = \hat{r}_i$
and $\eta_i = -\hat{r}_i$ at significance levels $\frac{\alpha}{2}$. Individually,
these NCMs define a \textbf{upper} and \textbf{lower} CRR sets respectively, and
together constitute a ``two-sided'' conformal procedure.

In the next section the construction of conformal prediction procedures will be
discussed. Two methods in particular: in sec.~\ref{sub:ridge_regression_confidence_machine}
we consider the original ridge regression approach from \cite{vovk2005}, and in
sec.~\ref{sub:kernel_crr} a ``two-sided'' version, \cite{burnaevV14}, which is
more amenable to theoretical analysis.

% subsection conformalized_ridge_regression (end)

% section conformal_prediction (end)

\section{Conformalized kernel ridge regression} % (fold)
\label{sec:conformalized_krr}

In this section we describe the construction of confidence regions of the conformal
procedure eq.~\ref{eq:conf_pred_set} for the case of the non-conformity measures
based on kernel ridge regression.

\subsection{Ridge Regression Confidence Machine} % (fold)
\label{sub:ridge_regression_confidence_machine}

We consider two versions of the NCM for the Ridge Regression Confidence Machine
(RRCM),proposed in \cite{vovk2005}, chapter 2, for the regression setting: an in-sample
and a \textbf{l}eave-\textbf{o}ne-\textbf{o}ut version. For this purpose consider
a sample $(X, y) = (x_i, y_{x_i})_{i=1}^n$, and for any $i=1\ldots, n$ put $X =
(X_{-i}, x_i)$, and $y = (y_{-i}, y_i)$. Let $K_{-i}$ be the Gram matrix of
$\Kcal$ over $X_{-i}$.

The ``in-sample'' NCM $A_{\text{in}}$, which measures the absolute value of the
regression residual, is given by:
\begin{align}
  A_{\text{in}}\bigl((X_{-i}, y_{-i}), (x_i, y_i)\bigr)
    &= |e_i' \hat{r}_{\text{in}}(X, y)| \nonumber\\
    &= |y_i - \hat{y}_{|(X, y)}(x_i)| \label{eq:ins_ncm}
    \,.
\end{align}
where $\hat{y}_{|(X, y)}(x_i)$ is the prediction over the complete dataset $(X, y)$.
The ``loo'' NCM is defined similarly, but uses deleted residuals:
\begin{align*}
  A_{\text{loo}}\bigl((X_{-i}, y_{-i}), (x_i, y_i)\bigr)
    &= |e_i' \hat{r}_{\text{loo}}(X, y)| \\
    &= |y_i - \hat{y}_{|(X_{-i}, y_{-i})}(x_i)|
    \,.
\end{align*}
For any $i$ the residuals are related by
\begin{equation}
  e_i' \hat{r}_{\text{in}}(X, y)
    = \lambda m_i^{-1} e_i' \hat{r}_{\text{loo}}(X, y)
    \,,
\end{equation}
where $\lambda m_i^{-1} = \lambda e_i'Q_X e_i$ is the KRR ``leverage'' score of
the $i$-th observation, and
\begin{equation} \label{eq:krr_leverage}
  m_i = \lambda + \Kcal(x_i, x_i) - k_{-i}(x_i)' Q_{-i} k_{-i}(x_i) \,,
\end{equation}
with $k_{-i}(x_i)$ -- the $n-1\times 1$ vector of $(\Kcal(x_j, x_i))_{i\neq j}$,
and $Q_{-i} = (K_{-i} + \lambda I_{n-1})^{-1}$.

For the NCM $A$ the $1 - \alpha$ conformal confidence interval for the $n$-th
observation is given by
\begin{equation} \label{eq:conf_ci}
  \Gamma_{X_{-n}, y_{-n}}^\alpha(x_n)
    = \bigl\{ z\in \Real \,:\, p_n\bigl((X, \tilde{y}_n^z)\bigr) \geq \alpha \bigr\}
    \,,
\end{equation}
where $\tilde{y}_i^z = (y_{-i}, z)$ -- the augmented target sample $y$ with the
$i$-th value replaced by $z$. The ``conformal likelihood'' of the $i$-th observation
in some sample $(X, y)$ is given by
\begin{equation*}
  p_j\bigl((X, y)\bigr)
    = n^{-1} \bigl\lvert \bigl\{
        j = 1,\ldots, n \, : \,
        \eta_j \geq \eta_i
    \bigr\} \bigr\rvert
    \,,
\end{equation*}
for $\eta_i = A\bigl((X_{-i}, y_{-i}), (x_i, y_i)\bigr)$.

Efficient construction of the confidence set for the NCM (\ref{eq:ins_ncm}) for
in-sample and deleted residuals relies on linear dependence on the target of the
$n$-th observation:
\begin{equation} \label{eq:krr_in_resid}
  \hat{r}_i^z
    = e_i' \hat{r}_{\text{in}}(X, \tilde{y}_n^z)
    = \lambda c_i + \lambda b_i z
    \,,
\end{equation}
with $c_i = e_i' C_{-n}\bigl((X, y), x_n\bigr)$ and
\begin{align*}
  C_{-n}\bigl((X, y), x_n\bigr)
    &= \begin{pmatrix} Q_{-n} y_{-n} \\ 0 \end{pmatrix} \\
    &- B_{-n}(x_n) K_{-n}(x_n)' Q_{-n} y_{-n}
    \,,
\end{align*}
where the vector $B_{-n}(x_n)\in\Real^{n\times 1}$ is given by
\begin{equation} \label{eq:krr_in_resid_B}
  B_{-n}(x_n)
    = \begin{pmatrix} - Q_{-n} K_{-n}(x_n) \\ 1 \end{pmatrix} m_n^{-1}
    \,.
\end{equation}

Since absolute values of the residuals are compared, it is possible to consistently
manipulate the signs of each entry in $C$ and $B$ to ensure that $e_i'B\geq 0$ for
all $i$. The regions $S_i = \{z\in\Real\,:\, |\hat{r}_i^z| \geq |\hat{r}_n^z|\}$, for
$i=1,\ldots, n$, are either closed intervals, complements of open intervals,
one-side closed half-rays in $\Real$, depending on the values of $C$ and $B$. In
particular, with $p_i$ and $q_i$ denoting the values $-\frac{c_i+c_n}{b_i+b_n}$ and
$\frac{c_i-c_n}{b_n-b_i}$, respectively (whenever each is defined), each region
$S_i$ has one of the representations:
\begin{enumerate}
%% a picture of shifted and scaled x->|x| helps in derivation of this.
  \item $b_i=b_n=0$: $S_i = \Real$ if $|c_i| \geq |c_n|$, or $S_i = \emptyset$
  otherwise;
  \item $b_n = b_i > 0$: $S_i$ is either $(-\infty, p_i]$ if $c_i < c_n$, $[p_i, +\infty)$ if
  $c_i > c_n$, or $\Real$ otherwise;
  \item $b_n > b_i \geq 0$: $S_i$ is either $[p_i, q_i]$ if $c_i b_n \geq c_n b_i$,
  or $[q_i, p_i]$ otherwise;
  \item $b_i > b_n \geq 0$: $S_i$ is $\Real\setminus (q_i, p_i)$ when $c_i b_n \geq c_n b_i$,
  or $\Real\setminus (p_i, q_i)$ otherwise.
\end{enumerate}
Let $P$ and $Q$ be the sets of all well-defined $p_i$ and $q_i$ respectively, and
let $(g_j)_{j=0}^{J+1}$ enumerate distinct values of $\{\pm\infty\} \cup P \cup Q$,
so that $g_j < g_{j+1}$ for all $j$. Then the confidence region is a closed subset
of $\Real$ constructed from sets $G^m_j = [g_j, g_{j+m}]\cap \Real$ for $m=0, 1$:
\begin{equation} \label{eq:rrcm_conf_ci}
  \Gamma_{X_{-n}, y_{-n}}^\alpha(x_n)
    = \bigcup_{m\in\{0,1\}} \bigcup_{j\,:\, N^m_j \geq n \alpha} G^m_j
    \,,
\end{equation}
where $N^m_j = |\{i \,:\, G^m_j \subseteq S_i\}|$ is the coverage frequency of $G^m_j$.
In general, the resulting confidence set might contain isolated singletons $G^0_j$.

This set, can be constructed efficiently in $\BigO(n \log{} n)$ time with $\BigO(n)$
memory footprint. Indeed, it is necessary to sort at most $J\leq 2n$ distinct endpoints
of $G_j$, then locate the values $p_i$ and $q_i$ associated with each region $S_i$
($\BigO(n \log{} n)$). Then, since the building blocks $G^m_j$ of $\Gamma^\alpha$
are either singletons ($m=0$), or intervals made up from adjacent singletons ($m=1$),
coverage numbers $N^m_j$ can be computed in at most $\BigO(n)$ time.

% subsection ridge_regression_confidence_machine (end)

\subsection{Kernel two-sided confidence predictor} % (fold)
\label{sub:kernel_crr}
Another possibility is to use the two-sided procedure proposed in \cite{burnaevV14}
and mentioned in sec.~\ref{sub:conformalized_ridge_regression}. Recall that, in order
to construct a CRR confidence region for significance level $\alpha$ (eq.~\ref{eq:crr_ncm}),
it is necessary to intersect confidence sets yielded by conformal procedures with NCM
scores given by $\hat{r}_i^z$ (upper) and $-\hat{r}_i^z$ (lower) at significance level
$\frac{\alpha}{2}$. Confidence regions based on CRR, much like RRCM, can use any kind
of residual: leave-one-out, or in-sample. In this section we give a description of an
efficient algorithm for constructing this confidence region.

For the upper CRR consider the regions $U_i = \{z\in\Real\,:\, \hat{r}_i^z \geq \hat{r}_n^z\}$,
$i=1,\ldots, n$ are either empty, full $\Real$ or one-side closed half-rays. Since
$\hat{r}_i^z = \lambda c_i + \lambda b_i z$, $U_i$ takes one of the following forms:
\begin{enumerate}
  \item $b_i=b_n$: $U_i = \Real$ if $c_i\geq c_n$, and $\emptyset$ otherwise;
  \item $b_i\neq b_n$: $U_i = [q_i, +\infty)$ if $b_i>b_n$, or
  $U_i = (-\infty, q_i]$ otherwise;
\end{enumerate}
with $q_i = \frac{c_i-c_n}{b_n-b_i}$. The forms of regions $L_i$ for the lower CRR
are computed similarly, but with the signs of $c_i$ and $b_i$ flipped for each $i=1, \ldots, n$.

Both upper and lower confidence regions are built similarly to the kernel RRCM region
eq.~\ref{eq:rrcm_conf_ci} in sec.~\ref{sub:ridge_regression_confidence_machine}. The
final the Kernel CRR confidence set is:
\begin{equation} \label{eq:crr_conf_ci}
  \Gamma_{X_{-n}, y_{-n}}^\alpha(x_n)
    = \Gamma_{X_{-n}, y_{-n}}^{\alpha,\text{u}}(x_n)
    \cap \Gamma_{X_{-n}, y_{-n}}^{\alpha,\text{l}}(x_n)
    \,.
\end{equation}
This intersection can be computed efficiently in $\BigO(n \log{} n)$, since the regions
are built form sets anchored at a finite set $Q$ with at most $n+2$ values. Therefore,
the CRR confidence set for a fixed significance level $\alpha$ has $\BigO(n\log{} n)$
complexity.

% subsection kernel_crr (end)

% section conformalized_krr (end)

\section{Numerical study} % (fold)
\label{sec:numerical_study}

Validity of conformal predictors in the online learning setting has been shown in
\cite{vovk2005}, chapter 2, however, no result of this kind is known in the batch
learning setting. Our experiments aim to evaluate the empirical performance of the
conformal prediction in this setting: with dedicated train and test datasets. In
this section we conduct a set of experiments to examine the validity of the regions,
produced by the conformal Kernel Ridge Regression and compare its efficiency to
the Bayesian confidence intervals.

We primarily test the conformalized KRR predictions in cases when the input space
$\Xcal$ is a compact set in $\Real^{d\times 1}$ for either $d=1$ or $d=2$. The rationale
for this is that since conformal procedure is oblivious to the structure of the
input dataset, there is no reason for its validity to deteriorate with increasing
dimensionality of $\Xcal$. Indeed, the conformal confidence region eq.~\ref{eq:conf_pred_set}
the input data from $\Xcal$ in the training and the test sets are fed into the NCM
$A$, which, in general, can be an arbitrary computable function, and never ``leak''
into the procedure itself. The dimensionality of the input data, however, may impact
the efficiency (the width) of the resulting confidence region. And our experiments,
to a certain extent, assess their effect.

We consider the isotropic Gaussian kernel for both the Conformal Kernel ridge regression
and the Gaussian Process Regression:
\begin{equation} \label{eq:gauss_kenrel}
  K(x,x')
  = \mathop{\text{exp}}\bigl\{-\theta \|x - x'\|^2\bigr\}
  \,,
\end{equation}
where $\theta>0$ is the precision parameter. This kernel is widely used in practice,
because it has very nice properties. In particular, if $\Xcal$ is compact its canonical
RKHS is universal: $\Hcal_K$ is dense in the set $C_0(\Xcal)$ of continuous bounded
maps with respect to the uniform norm $\|\cdot\|_\infty$, see.~\cite{steinwart2002}.

% Another example of a universal kernel is the Laplacian kernel 
% \begin{equation*}
%   K(x,x')
%   = \mathop{\text{exp}}\bigl\{-\theta \|x - x'\|_1\bigr\}
%   \,,
% \end{equation*}
% where $\|\cdot\|_1$ is the $L_1$ norm on $\Xcal$. In contrast to the Gaussian, the Laplacian
% kernel 
% Its spectral density is the Cauchy
% density, as opposed to the Gaussian density in the case of Gaussian kernel. The

\subsection{Setup} % (fold)
\label{sub:setup}

The off-line validity and efficiency is studied in two settings: the fully Gaussian
case, and the non-Gaussian cases. In the first case, we study the validity and
efficiency of GPR and conformal confidence regions on a sample path of a Gaussian
process on $\Xcal$. We focus consider $4$ alternatives, depending on whether the
parameter $\theta$ or $\lambda$ of the KRR are equal, or not, to their counterparts
in the kernel of the synthesized Gaussian process.

In the second setting, we concentrate on deliberately non-Gaussian test functions,
and examine the effects of moderate Gaussian additive noise on the performance of
both Bayesian and Conformal confidence regions. In this setting, the assumptions
of the confidence regions of Gaussian Process Regression are valid to a certain
extent: a deterministic function is contaminated by moderate Gaussian noise. We
explicitly examine two cases: whether the regularization in the kernel ridge regression
is equal to the theoretical level of the noise in the data, or not.

Below is a list of working conjectures we aim to find empirical evidence for: \begin{itemize}
  \item conformal confidence regions for different NCMs and residuals are at least
  asymptotically equivalent tin terms of coverage and efficiency;
  \item the Bayesian confidence regions perform gradually worse the further the data
  is from the GPR assumptions;
  \item in the fully Gaussian case both Bayesian and conformal confidence intervals
  possess the asymptotic validity guarantees, and the conformal procedure yields
  asymptotically efficient (close to the Bayesian) regions;
  \item the asymptotic validity (at least conservative) of the conformal confidence
  set is expected to hold in all cases.
\end{itemize}

We study the effects of the kernel parameters and the choice of the conformal procedure
on the validity and efficiency of the confidence regions by varying the following
hyper-parameters: \begin{enumerate}
  \item the magnitude of the Gaussian noise-to-signal ratio in the data: negligible
  $\gamma = 10^{-6}$ and moderate $\gamma = 10^{-1}$;
  \item the size $n$ of the train sample: small ($n \in \{ 25, 50, 100\}$),
  moderate, and large ($n \in \{200 k\,:\, k=1, \ldots, 8 \}$);
  \item the NCM $A$ for the confidence region: either the RRCM, or the CRR;
  \item the type of residual $\hat{r}$: either $\hat{r}_{\text{in}}$, or $\hat{r}_{\text{loo}}$;
  \item the noise-to-signal (regularization) ratio $\lambda$ in the kernel ridge
  regression is chosen from two regimes: high $\lambda = 10^{-1}$ (smoothing), and
  low $\lambda=10^{-6}$ (interpolation);
  \item the precision of the Gaussian kernel $\theta$ is picked from
  $\{10, 10^2, 10^3\}$, since the input domain is the unit cube, for which
  $\|x-x'\| \leq \sqrt{d}$ for all $x,x'\in[0,1]^d$;
  \item the fixed $\theta$ or its MLE estimate ($\theta$ that minimizes \ref{eq:bkrr_likelihood});
\end{enumerate}
For a given test function $f:\Xcal \mapsto \Real$ on some compact domain $\Xcal$
each experiment for $n, \theta, \lambda, A$ and $\hat{r}$ consists of the following
steps:
\begin{enumerate}
  \item The test inputs, $X^*$, are given by a grid with constant spacing in $\Xcal$;
  \item Train inputs, $X$, are sampled from a uniform distribution over $\Xcal$;
  \item For all $x\in X_{\text{pool}} = X \cup X^*$, draw target values $y_x = f(x)$
  from the data generation process;
  \item let $\tilde{T} = (x, y_x)_{x\in X}$;
  \item perform $L$ independent replications of the following steps: for $l=1,\ldots, L$
  \begin{enumerate}
    \item draw an independent random sample of size $n$ without replacement from $\tilde{T}$;
    \item fit a Gaussian Process regression with $\beta = 0$ and $\Kcal$ given by
    eq.~\ref{eq:krig_kernel} for the Gaussian kernel $K$ with the specified precision
    $\theta$ with fixed $\lambda > 0$;
    \item for each $x^* \in X^*$ construct, the Bayesian KRR region, $\Bcal_l^\alpha(x^*)$,
    and the conformal confidence regions, $\Gamma_l^\alpha(x^*; A, \hat{r})$ using
    the specified NCM $A$, the residuals $\hat{r}$, and the kernel $\Kcal$ (with
    estimated $\sigma^2$);
    \item estimate the coverage rate and the width of the convex hull of the region
    over the test sample $X^*$: $p_l(R) = |X^*|^{-1}\sum_{x\in X^*} 1_{y_x\in R_x}$,
    and $w_l(R) = \inf\{b-a\,:\,R \subseteq [a, b]\}$, where $R$ is a confidence
    region;
  \end{enumerate}
\end{enumerate}
For the $1$-D case the domain $\Xcal$ is $[0,1]$ and the test input set is given by
$\{k N^{-1}\,:\,k=0,\ldots, N+1\}$ for $N=1000$. In $2$-D case we consider functions
defined on $\Xcal=[-1,1]^2$, and use $X^* = \{k N^{-1}\,:\,k=0,\ldots, N+1\}^2$
for $N=50$ as the test set.

To measure the quality of fit of the kernel ridge regression for each KRR set of
hyper-parameters, we compute the ratio of the root mean squared error to the standard
derivation on the test set (``RMSE-std'' on the plot the results section). This
reflects how well the KRR predictions fares against the constant prediction.

%% What is the estimate of \sigma^2

% subsection setup (end)

\subsection{Results: $1$-d} % (fold)
\label{sub:results_1_d}

We begin with the examination of the fully-Gaussian setup with $\Xcal = [0, 1]$.
To illustrate the constructed confidence regions, we generated a sample path of
the $1$-d Gaussian process with $y_x\sim GP(0, \gamma \delta_{x,x'} + K_{x,x'})$
for the Gaussian kernel (eq.~\ref{eq:gauss_kenrel}) on a regular grid $G=\{500^{-1} k \,:\, k=0,
\ldots, 500\}$, and then use its sub-grid of size $51$ with stepping $0.9\cdot 20^{-1}$
in $[0.05, 0.95]$ for training the Kernel Ridge regression and constructing the
confidence regions. The training sample was deliberately chosen to highlight the
blowup of the size of the confidence bands near the boundaries of the train sample.
Figure~\ref{fig:gauss_1d_prof_gpr} illustrates one realization of $y_x$ and the
constructed Bayesian confidence regions (eq.~\ref{eq:gp_conf_int}). Here the abbreviation
``GPR-p'' stands for a confidence region for the true value $f_x$ in eq.~\ref{eq:signal_model},
and ``GPR-f'' -- for the observed value. Recall that the non-parametric vanilla
conformal procedures yield predictions for the observed values $y_x$ only. That
is why in the following we are going to focus on the ``GPR-f'' confidence intervals,
and the ``GPR-p'' band is presented for illustrative purposes only.

On the provided illustration (fig.~\ref{fig:gauss_1d_prof_gpr}) the confidence bands
on are too wide in the negligible noise case, whereas the bands in seem to have better
coverage rate due to higher noise. Figure~\ref{fig:gauss_1d_prof_conf} shows sample
conformal confidence bands in the same setting. Near the endpoints of the $[0,1]$,
the confidence regions dramatically increase their size, reflecting increasing uncertainty.

In can be argued theoretically, that confidence regions for any observation $x_n$
sufficiently far away from the bulk of the training dataset have constant size,
determined only by the train sample (fig.~\ref{fig:limit_1d_ci_size}). Indeed, as
$\|x_n\|^2\to \infty$ ($n$-fixed) the vector $B_{-n}$ in (eq.~\ref{eq:krr_in_resid_B})
approaches the $n$-th unit vector $e_n$, since for the Gaussian kernel the vector
$\|K_{-n}(x_n)\|^2\to 0$. Since the kernel is bounded, the value $m_n$ (eq.~\ref{eq:krr_leverage})
is a bounded function of $x_n$, which, in turn, implies that eventually all RRCM
regions $S_i$ assume the form of closed intervals $[-|q_i|, |q_i|]$, where
$q_i = m_n (e_i'Q_{-n}y_{-n}) + o(\|x_n\|^2)$, $i \neq n$. The same is true for
the CRR procedure. Therefore, the conformal procedure essentially reverts to a
constant-size confidence region, determined by the  $n^{-1}\lfloor n(1-\alpha)\rfloor$-th
order statistic of $(|q_i|)_{i=1}^n$. Analogous effects can be observed for the
Gaussian Process confidence interval (eq.~\ref{eq:gp_conf_int}).
% Is this true for any stationary kernel on $\Real^{d\times 1})$? Bochner's theorem, maybe?
\begin{figure}%[t, width=0.5\textwidth]
  \centering
  \begin{subfigure}[b]{0.5\linewidth}
    \includegraphics[width=0.9\linewidth]{../images/output_pdf/profile/gaussian/0.1_0.1/50/profile_gaussian_0,1_0,1_100_5p-GPR_50.pdf}
  \end{subfigure}%~
  \begin{subfigure}[b]{0.5\linewidth}
    \includegraphics[width=0.9\linewidth]{../images/output_pdf/profile/gaussian/1e-06_0.1/50/profile_gaussian_1e-06_0,1_100_5p-GPR_50.pdf}
  \end{subfigure}%
  \caption{A sample path of a $1$-d Gaussian Process on $\Xcal$ ($y_x$,  $\hat{y}_x$),
  and the forecast and prediction confidence bands (GPR-f and GPR-p, respectively).
  \textit{Left:} a sample path with $\gamma=10^{-1}$; \textit{right:} $\gamma=10^{-6}$.}
  \label{fig:gauss_1d_prof_gpr}
\end{figure}

\begin{figure}%[b, width=0.5\textwidth]
  \centering
  \begin{subfigure}[b]{0.5\linewidth}
    \includegraphics[width=0.9\linewidth]{../images/output_pdf/profile/gaussian/0.1_0.1/50/profile_gaussian_0,1_0,1_100_5p-RRCM_50.pdf}
  \end{subfigure}%~
  \begin{subfigure}[b]{0.5\linewidth}
    \includegraphics[width=0.9\linewidth]{../images/output_pdf/profile/gaussian/1e-06_0.1/50/profile_gaussian_1e-06_0,1_100_5p-RRCM_50.pdf}
  \end{subfigure}\\
  \begin{subfigure}[b]{0.5\linewidth}
    \includegraphics[width=0.9\linewidth]{../images/output_pdf/profile/gaussian/0.1_1e-06/200/profile_gaussian_0,1_1e-06_100_5p-RRCM_200.pdf}
  \end{subfigure}%~
  \begin{subfigure}[b]{0.5\linewidth}
    \includegraphics[width=0.9\linewidth]{../images/output_pdf/profile/gaussian/1e-06_1e-06/200/profile_gaussian_1e-06_1e-06_100_5p-RRCM_200.pdf}
  \end{subfigure}%
  \caption{Sample RRCM confidence bands (eq.~\ref{eq:rrcm_conf_ci}).}
  \label{fig:gauss_1d_prof_conf}
\end{figure}

\begin{figure}%[b, width=0.5\textwidth]
  \centering
  \begin{subfigure}[b]{0.5\linewidth}
    \includegraphics[width=0.9\linewidth]{../images/output_pdf/profile/gaussiantest/1e-06_0.1/200/profile_gaussian_1e-06_0,1_100_1p-GPR_200.pdf}
  \end{subfigure}%
  \begin{subfigure}[b]{0.5\linewidth}
    \includegraphics[width=0.9\linewidth]{../images/output_pdf/profile/gaussiantest/1e-06_0.1/200/profile_gaussian_1e-06_0,1_100_1p-RRCM_200.pdf}
  \end{subfigure}\\
  \begin{subfigure}[b]{0.5\linewidth}
    \includegraphics[width=0.9\linewidth]{../images/output_pdf/profile/gaussiantest/0.1_0.1/200/profile_gaussian_0,1_0,1_100_1p-GPR_200.pdf}
  \end{subfigure}%
  \begin{subfigure}[b]{0.5\linewidth}
    \includegraphics[width=0.9\linewidth]{../images/output_pdf/profile/gaussiantest/0.1_0.1/200/profile_gaussian_0,1_0,1_100_1p-RRCM_200.pdf}
  \end{subfigure}
  \caption{Limiting behaviour of GPR (\textit{left}) and RRCM (\textit{right})
  confidence regions for a sample path of a Gaussian process with negligible noise
  (\textit{top}, $\gamma=10^{-6}$) and high noise-to-signal level (\textit{bottom},
  $\gamma=10^{-1}$).}
  \label{fig:limit_1d_ci_size}
\end{figure}

Now, let's consider the experimental setting with negligible noise in the observed
processes, disregarding for a while the fact that in the noiseless case the necessity
of confidence measures of predictions seems questionable.

The first important observation is that conformal procedures (eq.\ref{eq:conf_pred_set})
can allow for some uncertainty by construction. Indeed, for a sample $(X, y)$, the
NCMs based on KRR (eq.~\ref{eq:krr_in_resid} and eq.~\ref{eq:loo_resid}) and continuous
functions of $y_n$. Therefore the vector of non-conformity scores $(\eta^z_i)_{i=1}^n$
also depends continuously on $y_n$, which mean that for small perturbations of $y_n$
the relative ordering of $\eta^z_i$ remains the same. Hence, in the case of perfectly
noiseless observations, the confidence regions have positive width, thereby having
necessarily higher coverage rate. The GPR based confidence regions are subject to
the same effect in the noiseless case but for another reason: the predictive variance
includes the noise-to-signal ratio.

This conservativeness is indeed confirmed by the experimental results (see fig.~
\ref{fig:gaussian_1d_low_noise}). Within each column the confidence regions were
computed under exactly identical conditions: for each hyper-parameter setting (excluding
$\gamma$) we draw $L$ random samples from some fixed realization of a Gaussian process
(with kernel precision $\theta_0$), compute the ML estimate of $\theta$, if necessary,
construct the confidence intervals, and then average the coverage rate for each test
input, to average out the effects of random choice of the input data.

Judging by the coverage rate, conformal confidence regions are much more narrower
than the Bayesian intervals. This suggests, that the procedure (eq.~\ref{eq:conf_pred_set})
adapts to the noise level not through the regularization $\lambda$, but through
the sample distribution on the non-conformity scores (fig.~\ref{fig:gaussian_1d_low_noise_c2}
and \ref{fig:gaussian_1d_low_noise_c2}). Indeed, fig.~\ref{fig:gaussian_1d_low_noise_width}
clearly show that GPR intervals tend to be wider.

With lower kernel precision parameter ($\theta$) the KRR estimate of the underlying
$f$ becomes less variable, thus enabling better coverage by the GPR and conformal
regions (fig.~\ref{fig:gaussian_1d_low_noise_arb} columns \subref{fig:gaussian_1d_low_noise_arb_c1}
and \subref{fig:gaussian_1d_low_noise_arb_c2}).

It is noteworthy, the conformal procedure yields correct coverage despite relatively
poor fit (the last row in fig.~\ref{fig:gaussian_1d_low_noise_arb}). This suggests,
a certain flexibility of a KRR-based conformal prediction: it is able to maintain
the required coverage rate regardless of the quality of prediction within the NCM.
The region sizes at fig.~\ref{fig:gaussian_1d_low_noise_width_arb_c2} implies that
the GPR regions kept being symmetric around a less accurate prediction, which hints
at the reason why their coverage rate suffered. At the same time the coverage rate
of the conformal regions, symmetry of which is not required, demonstrated no significant
deviations from specified confidence levels. It should be noted, however, that all
intervals in col.~\subref{fig:gaussian_1d_low_noise_width_arb_c2} are wider than
those of other columns in fig.~\ref{fig:gaussian_1d_low_noise_width_arb}.

\begin{figure}%[b, width=0.5\textwidth]
  \centering
  \begin{subfigure}[b]{0.25\linewidth}
    \includegraphics[width=0.95\linewidth]{../images/output_pdf/exp_1d/gaussian/1e-06_1e-06/coverage/GPR-f/coverage_gaussian_1e-06_1e-06_100_GPR-f.pdf}
  \end{subfigure}%
  \begin{subfigure}[b]{0.25\linewidth}
    \includegraphics[width=0.95\linewidth]{../images/output_pdf/exp_1d/gaussian/1e-06_0.1/coverage/GPR-f/coverage_gaussian_1e-06_0,1_100_GPR-f.pdf}
  \end{subfigure}%
  \begin{subfigure}[b]{0.25\linewidth}
    \includegraphics[width=0.95\linewidth]{../images/output_pdf/exp_1d/gaussian/1e-06_1e-06/coverage/GPR-f/coverage_gaussian_1e-06_1e-06_auto_GPR-f.pdf}
  \end{subfigure}%
  \begin{subfigure}[b]{0.25\linewidth}
    \includegraphics[width=0.95\linewidth]{../images/output_pdf/exp_1d/gaussian/1e-06_0.1/coverage/GPR-f/coverage_gaussian_1e-06_0,1_auto_GPR-f.pdf}
  \end{subfigure}\\
  \begin{subfigure}[b]{0.25\linewidth}
    \includegraphics[width=0.95\linewidth]{../images/output_pdf/exp_1d/gaussian/1e-06_1e-06/coverage/RRCM/coverage_gaussian_1e-06_1e-06_100_RRCM.pdf}
  \end{subfigure}%
  \begin{subfigure}[b]{0.25\linewidth}
    \includegraphics[width=0.95\linewidth]{../images/output_pdf/exp_1d/gaussian/1e-06_0.1/coverage/RRCM/coverage_gaussian_1e-06_0,1_100_RRCM.pdf}
  \end{subfigure}%
  \begin{subfigure}[b]{0.25\linewidth}
    \includegraphics[width=0.95\linewidth]{../images/output_pdf/exp_1d/gaussian/1e-06_1e-06/coverage/RRCM/coverage_gaussian_1e-06_1e-06_auto_RRCM.pdf}
  \end{subfigure}%
  \begin{subfigure}[b]{0.25\linewidth}
    \includegraphics[width=0.95\linewidth]{../images/output_pdf/exp_1d/gaussian/1e-06_0.1/coverage/RRCM/coverage_gaussian_1e-06_0,1_auto_RRCM.pdf}
  \end{subfigure}\\
  \begin{subfigure}[b]{0.25\linewidth}
    \includegraphics[width=0.95\linewidth]{../images/output_pdf/exp_1d/gaussian/1e-06_1e-06/coverage/RRCM-loo/coverage_gaussian_1e-06_1e-06_100_RRCM-loo.pdf}
  \end{subfigure}%
  \begin{subfigure}[b]{0.25\linewidth}
    \includegraphics[width=0.95\linewidth]{../images/output_pdf/exp_1d/gaussian/1e-06_0.1/coverage/RRCM-loo/coverage_gaussian_1e-06_0,1_100_RRCM-loo.pdf}
  \end{subfigure}%
  \begin{subfigure}[b]{0.25\linewidth}
    \includegraphics[width=0.95\linewidth]{../images/output_pdf/exp_1d/gaussian/1e-06_1e-06/coverage/RRCM-loo/coverage_gaussian_1e-06_1e-06_auto_RRCM-loo.pdf}
  \end{subfigure}%
  \begin{subfigure}[b]{0.25\linewidth}
    \includegraphics[width=0.95\linewidth]{../images/output_pdf/exp_1d/gaussian/1e-06_0.1/coverage/RRCM-loo/coverage_gaussian_1e-06_0,1_auto_RRCM-loo.pdf}
  \end{subfigure}\\
  \begin{subfigure}[b]{0.25\linewidth}
    \includegraphics[width=0.95\linewidth]{../images/output_pdf/exp_1d/gaussian/1e-06_1e-06/coverage/CRR/coverage_gaussian_1e-06_1e-06_100_CRR.pdf}
  \end{subfigure}%
  \begin{subfigure}[b]{0.25\linewidth}
    \includegraphics[width=0.95\linewidth]{../images/output_pdf/exp_1d/gaussian/1e-06_0.1/coverage/CRR/coverage_gaussian_1e-06_0,1_100_CRR.pdf}
  \end{subfigure}%
  \begin{subfigure}[b]{0.25\linewidth}
    \includegraphics[width=0.95\linewidth]{../images/output_pdf/exp_1d/gaussian/1e-06_1e-06/coverage/CRR/coverage_gaussian_1e-06_1e-06_auto_CRR.pdf}
  \end{subfigure}%
  \begin{subfigure}[b]{0.25\linewidth}
    \includegraphics[width=0.95\linewidth]{../images/output_pdf/exp_1d/gaussian/1e-06_0.1/coverage/CRR/coverage_gaussian_1e-06_0,1_auto_CRR.pdf}
  \end{subfigure}\\
  \begin{subfigure}[b]{0.25\linewidth}
    \includegraphics[width=0.95\linewidth]{../images/output_pdf/exp_1d/gaussian/1e-06_1e-06/coverage/CRR-loo/coverage_gaussian_1e-06_1e-06_100_CRR-loo.pdf}
  \end{subfigure}%
  \begin{subfigure}[b]{0.25\linewidth}
    \includegraphics[width=0.95\linewidth]{../images/output_pdf/exp_1d/gaussian/1e-06_0.1/coverage/CRR-loo/coverage_gaussian_1e-06_0,1_100_CRR-loo.pdf}
  \end{subfigure}%
  \begin{subfigure}[b]{0.25\linewidth}
    \includegraphics[width=0.95\linewidth]{../images/output_pdf/exp_1d/gaussian/1e-06_1e-06/coverage/CRR-loo/coverage_gaussian_1e-06_1e-06_auto_CRR-loo.pdf}
  \end{subfigure}%
  \begin{subfigure}[b]{0.25\linewidth}
    \includegraphics[width=0.95\linewidth]{../images/output_pdf/exp_1d/gaussian/1e-06_0.1/coverage/CRR-loo/coverage_gaussian_1e-06_0,1_auto_CRR-loo.pdf}
  \end{subfigure}\\
  \begin{subfigure}[b]{0.25\linewidth}
    \includegraphics[width=0.95\linewidth]{../images/output_pdf/exp_1d/gaussian/1e-06_1e-06/gaussian_1e-06_1e-06_100_ratio.pdf}
    \caption{} \label{fig:gaussian_1d_low_noise_c1}
  \end{subfigure}%
  \begin{subfigure}[b]{0.25\linewidth}
    \includegraphics[width=0.95\linewidth]{../images/output_pdf/exp_1d/gaussian/1e-06_0.1/gaussian_1e-06_0,1_100_ratio.pdf}
    \caption{} \label{fig:gaussian_1d_low_noise_c2}
  \end{subfigure}%
  \begin{subfigure}[b]{0.25\linewidth}
    \includegraphics[width=0.95\linewidth]{../images/output_pdf/exp_1d/gaussian/1e-06_1e-06/gaussian_1e-06_1e-06_auto_ratio.pdf}
    \caption{} \label{fig:gaussian_1d_low_noise_c3}
  \end{subfigure}%
  \begin{subfigure}[b]{0.25\linewidth}
    \includegraphics[width=0.95\linewidth]{../images/output_pdf/exp_1d/gaussian/1e-06_0.1/gaussian_1e-06_0,1_auto_ratio.pdf}
    \caption{} \label{fig:gaussian_1d_low_noise_c4}
  \end{subfigure}%
  \caption{Coverage rate dynamics in the fully Gaussian low-noise case $\gamma=10^{-6}$
  for different $(\theta, \lambda)$:
  \subref{fig:gaussian_1d_low_noise_c1}~--~$(\theta_0, 10^{-6})$,
  \subref{fig:gaussian_1d_low_noise_c2}~--~$(\theta_0, 10^{-1})$,
  \subref{fig:gaussian_1d_low_noise_c3}~--~$(\hat{\theta}_\text{ML}, 10^{-6})$,
  \subref{fig:gaussian_1d_low_noise_c4}~--~$(\hat{\theta}_\text{ML}, 10^{-1})$.
  Rows from \textit{top} to \textit{bottom}: ``GPR-f'', ``RRCM'', ``RRCM-loo'',
  ``CRR'', ``CRR-loo'', and the last depicting ``RMSE-std'' ratio.}
  \label{fig:gaussian_1d_low_noise}
\end{figure}

\begin{figure}%[b, width=0.5\textwidth]
  \centering
  \begin{subfigure}[b]{0.25\linewidth}
    \includegraphics[width=0.95\linewidth]{../images/output_pdf/exp_1d/gaussian/1e-06_1e-06/width/GPR-f/width_gaussian_1e-06_1e-06_100_GPR-f.pdf}
  \end{subfigure}%
  \begin{subfigure}[b]{0.25\linewidth}
    \includegraphics[width=0.95\linewidth]{../images/output_pdf/exp_1d/gaussian/1e-06_0.1/width/GPR-f/width_gaussian_1e-06_0,1_100_GPR-f.pdf}
  \end{subfigure}%
  \begin{subfigure}[b]{0.25\linewidth}
    \includegraphics[width=0.95\linewidth]{../images/output_pdf/exp_1d/gaussian/1e-06_1e-06/width/GPR-f/width_gaussian_1e-06_1e-06_auto_GPR-f.pdf}
  \end{subfigure}%
  \begin{subfigure}[b]{0.25\linewidth}
    \includegraphics[width=0.95\linewidth]{../images/output_pdf/exp_1d/gaussian/1e-06_0.1/width/GPR-f/width_gaussian_1e-06_0,1_auto_GPR-f.pdf}
  \end{subfigure}\\
  \begin{subfigure}[b]{0.25\linewidth}
    \includegraphics[width=0.95\linewidth]{../images/output_pdf/exp_1d/gaussian/1e-06_1e-06/width/RRCM/width_gaussian_1e-06_1e-06_100_RRCM.pdf}
  \end{subfigure}%
  \begin{subfigure}[b]{0.25\linewidth}
    \includegraphics[width=0.95\linewidth]{../images/output_pdf/exp_1d/gaussian/1e-06_0.1/width/RRCM/width_gaussian_1e-06_0,1_100_RRCM.pdf}
  \end{subfigure}%
  \begin{subfigure}[b]{0.25\linewidth}
    \includegraphics[width=0.95\linewidth]{../images/output_pdf/exp_1d/gaussian/1e-06_1e-06/width/RRCM/width_gaussian_1e-06_1e-06_auto_RRCM.pdf}
  \end{subfigure}%
  \begin{subfigure}[b]{0.25\linewidth}
    \includegraphics[width=0.95\linewidth]{../images/output_pdf/exp_1d/gaussian/1e-06_0.1/width/RRCM/width_gaussian_1e-06_0,1_auto_RRCM.pdf}
  \end{subfigure}\\
  \begin{subfigure}[b]{0.25\linewidth}
    \includegraphics[width=0.95\linewidth]{../images/output_pdf/exp_1d/gaussian/1e-06_1e-06/width/RRCM-loo/width_gaussian_1e-06_1e-06_100_RRCM-loo.pdf}
  \end{subfigure}%
  \begin{subfigure}[b]{0.25\linewidth}
    \includegraphics[width=0.95\linewidth]{../images/output_pdf/exp_1d/gaussian/1e-06_0.1/width/RRCM-loo/width_gaussian_1e-06_0,1_100_RRCM-loo.pdf}
  \end{subfigure}%
  \begin{subfigure}[b]{0.25\linewidth}
    \includegraphics[width=0.95\linewidth]{../images/output_pdf/exp_1d/gaussian/1e-06_1e-06/width/RRCM-loo/width_gaussian_1e-06_1e-06_auto_RRCM-loo.pdf}
  \end{subfigure}%
  \begin{subfigure}[b]{0.25\linewidth}
    \includegraphics[width=0.95\linewidth]{../images/output_pdf/exp_1d/gaussian/1e-06_0.1/width/RRCM-loo/width_gaussian_1e-06_0,1_auto_RRCM-loo.pdf}
  \end{subfigure}\\
  \begin{subfigure}[b]{0.25\linewidth}
    \includegraphics[width=0.95\linewidth]{../images/output_pdf/exp_1d/gaussian/1e-06_1e-06/width/CRR/width_gaussian_1e-06_1e-06_100_CRR.pdf}
  \end{subfigure}%
  \begin{subfigure}[b]{0.25\linewidth}
    \includegraphics[width=0.95\linewidth]{../images/output_pdf/exp_1d/gaussian/1e-06_0.1/width/CRR/width_gaussian_1e-06_0,1_100_CRR.pdf}
  \end{subfigure}%
  \begin{subfigure}[b]{0.25\linewidth}
    \includegraphics[width=0.95\linewidth]{../images/output_pdf/exp_1d/gaussian/1e-06_1e-06/width/CRR/width_gaussian_1e-06_1e-06_auto_CRR.pdf}
  \end{subfigure}%
  \begin{subfigure}[b]{0.25\linewidth}
    \includegraphics[width=0.95\linewidth]{../images/output_pdf/exp_1d/gaussian/1e-06_0.1/width/CRR/width_gaussian_1e-06_0,1_auto_CRR.pdf}
  \end{subfigure}\\
  \begin{subfigure}[b]{0.25\linewidth}
    \includegraphics[width=0.95\linewidth]{../images/output_pdf/exp_1d/gaussian/1e-06_1e-06/width/CRR-loo/width_gaussian_1e-06_1e-06_100_CRR-loo.pdf}
    \caption{} \label{fig:gaussian_1d_low_noise_width_c1}
  \end{subfigure}%
  \begin{subfigure}[b]{0.25\linewidth}
    \includegraphics[width=0.95\linewidth]{../images/output_pdf/exp_1d/gaussian/1e-06_0.1/width/CRR-loo/width_gaussian_1e-06_0,1_100_CRR-loo.pdf}
    \caption{} \label{fig:gaussian_1d_low_noise_width_c2}
  \end{subfigure}%
  \begin{subfigure}[b]{0.25\linewidth}
    \includegraphics[width=0.95\linewidth]{../images/output_pdf/exp_1d/gaussian/1e-06_1e-06/width/CRR-loo/width_gaussian_1e-06_1e-06_auto_CRR-loo.pdf}
    \caption{} \label{fig:gaussian_1d_low_noise_width_c3}
  \end{subfigure}%
  \begin{subfigure}[b]{0.25\linewidth}
    \includegraphics[width=0.95\linewidth]{../images/output_pdf/exp_1d/gaussian/1e-06_0.1/width/CRR-loo/width_gaussian_1e-06_0,1_auto_CRR-loo.pdf}
    \caption{} \label{fig:gaussian_1d_low_noise_width_c4}
  \end{subfigure}%
  \caption{Asymptotic width of the confidence regions in the fully Gaussian low-noise case
  $\gamma=10^{-6}$ for different $(\theta, \lambda)$. Upward triangles indicate the
  $5\%$ sample quantile across the whole test sample of confidence regions' widths,
  whereas downward triangles indicate the maximal width. The median width is drawn
  with a slightly thicker line. The colouring matches the colours of respective confidence
  levels as in fig.~\ref{fig:gaussian_1d_low_noise}.}
  \label{fig:gaussian_1d_low_noise_width}
\end{figure}

\begin{figure}%[b, width=0.5\textwidth]
  \centering
  \begin{subfigure}[b]{0.25\linewidth}
    \includegraphics[width=0.95\linewidth]{../images/output_pdf/exp_1d/gaussian/1e-06_1e-06/coverage/GPR-f/coverage_gaussian_1e-06_1e-06_10_GPR-f.pdf}
  \end{subfigure}%
  \begin{subfigure}[b]{0.25\linewidth}
    \includegraphics[width=0.95\linewidth]{../images/output_pdf/exp_1d/gaussian/1e-06_0.1/coverage/GPR-f/coverage_gaussian_1e-06_0,1_10_GPR-f.pdf}
  \end{subfigure}%
  \begin{subfigure}[b]{0.25\linewidth}
    \includegraphics[width=0.95\linewidth]{../images/output_pdf/exp_1d/gaussian/1e-06_1e-06/coverage/GPR-f/coverage_gaussian_1e-06_1e-06_1000_GPR-f.pdf}
  \end{subfigure}%
  \begin{subfigure}[b]{0.25\linewidth}
    \includegraphics[width=0.95\linewidth]{../images/output_pdf/exp_1d/gaussian/1e-06_0.1/coverage/GPR-f/coverage_gaussian_1e-06_0,1_1000_GPR-f.pdf}
  \end{subfigure}\\
  \begin{subfigure}[b]{0.25\linewidth}
    \includegraphics[width=0.95\linewidth]{../images/output_pdf/exp_1d/gaussian/1e-06_1e-06/coverage/RRCM/coverage_gaussian_1e-06_1e-06_10_RRCM.pdf}
  \end{subfigure}%
  \begin{subfigure}[b]{0.25\linewidth}
    \includegraphics[width=0.95\linewidth]{../images/output_pdf/exp_1d/gaussian/1e-06_0.1/coverage/RRCM/coverage_gaussian_1e-06_0,1_10_RRCM.pdf}
  \end{subfigure}%
  \begin{subfigure}[b]{0.25\linewidth}
    \includegraphics[width=0.95\linewidth]{../images/output_pdf/exp_1d/gaussian/1e-06_1e-06/coverage/RRCM/coverage_gaussian_1e-06_1e-06_1000_RRCM.pdf}
  \end{subfigure}%
  \begin{subfigure}[b]{0.25\linewidth}
    \includegraphics[width=0.95\linewidth]{../images/output_pdf/exp_1d/gaussian/1e-06_0.1/coverage/RRCM/coverage_gaussian_1e-06_0,1_1000_RRCM.pdf}
  \end{subfigure}\\
  \begin{subfigure}[b]{0.25\linewidth}
    \includegraphics[width=0.95\linewidth]{../images/output_pdf/exp_1d/gaussian/1e-06_1e-06/coverage/RRCM-loo/coverage_gaussian_1e-06_1e-06_10_RRCM-loo.pdf}
  \end{subfigure}%
  \begin{subfigure}[b]{0.25\linewidth}
    \includegraphics[width=0.95\linewidth]{../images/output_pdf/exp_1d/gaussian/1e-06_0.1/coverage/RRCM-loo/coverage_gaussian_1e-06_0,1_10_RRCM-loo.pdf}
  \end{subfigure}%
  \begin{subfigure}[b]{0.25\linewidth}
    \includegraphics[width=0.95\linewidth]{../images/output_pdf/exp_1d/gaussian/1e-06_1e-06/coverage/RRCM-loo/coverage_gaussian_1e-06_1e-06_1000_RRCM-loo.pdf}
  \end{subfigure}%
  \begin{subfigure}[b]{0.25\linewidth}
    \includegraphics[width=0.95\linewidth]{../images/output_pdf/exp_1d/gaussian/1e-06_0.1/coverage/RRCM-loo/coverage_gaussian_1e-06_0,1_1000_RRCM-loo.pdf}
  \end{subfigure}\\
  \begin{subfigure}[b]{0.25\linewidth}
    \includegraphics[width=0.95\linewidth]{../images/output_pdf/exp_1d/gaussian/1e-06_1e-06/coverage/CRR/coverage_gaussian_1e-06_1e-06_10_CRR.pdf}
  \end{subfigure}%
  \begin{subfigure}[b]{0.25\linewidth}
    \includegraphics[width=0.95\linewidth]{../images/output_pdf/exp_1d/gaussian/1e-06_0.1/coverage/CRR/coverage_gaussian_1e-06_0,1_10_CRR.pdf}
  \end{subfigure}%
  \begin{subfigure}[b]{0.25\linewidth}
    \includegraphics[width=0.95\linewidth]{../images/output_pdf/exp_1d/gaussian/1e-06_1e-06/coverage/CRR/coverage_gaussian_1e-06_1e-06_1000_CRR.pdf}
  \end{subfigure}%
  \begin{subfigure}[b]{0.25\linewidth}
    \includegraphics[width=0.95\linewidth]{../images/output_pdf/exp_1d/gaussian/1e-06_0.1/coverage/CRR/coverage_gaussian_1e-06_0,1_1000_CRR.pdf}
  \end{subfigure}\\
  \begin{subfigure}[b]{0.25\linewidth}
    \includegraphics[width=0.95\linewidth]{../images/output_pdf/exp_1d/gaussian/1e-06_1e-06/coverage/CRR-loo/coverage_gaussian_1e-06_1e-06_10_CRR-loo.pdf}
  \end{subfigure}%
  \begin{subfigure}[b]{0.25\linewidth}
    \includegraphics[width=0.95\linewidth]{../images/output_pdf/exp_1d/gaussian/1e-06_0.1/coverage/CRR-loo/coverage_gaussian_1e-06_0,1_10_CRR-loo.pdf}
  \end{subfigure}%
  \begin{subfigure}[b]{0.25\linewidth}
    \includegraphics[width=0.95\linewidth]{../images/output_pdf/exp_1d/gaussian/1e-06_1e-06/coverage/CRR-loo/coverage_gaussian_1e-06_1e-06_1000_CRR-loo.pdf}
  \end{subfigure}%
  \begin{subfigure}[b]{0.25\linewidth}
    \includegraphics[width=0.95\linewidth]{../images/output_pdf/exp_1d/gaussian/1e-06_0.1/coverage/CRR-loo/coverage_gaussian_1e-06_0,1_1000_CRR-loo.pdf}
  \end{subfigure}\\
  \begin{subfigure}[b]{0.25\linewidth}
    \includegraphics[width=0.95\linewidth]{../images/output_pdf/exp_1d/gaussian/1e-06_1e-06/gaussian_1e-06_1e-06_10_ratio.pdf}
    \caption{} \label{fig:gaussian_1d_low_noise_arb_c1}
  \end{subfigure}%
  \begin{subfigure}[b]{0.25\linewidth}
    \includegraphics[width=0.95\linewidth]{../images/output_pdf/exp_1d/gaussian/1e-06_0.1/gaussian_1e-06_0,1_10_ratio.pdf}
    \caption{} \label{fig:gaussian_1d_low_noise_arb_c2}
  \end{subfigure}%
  \begin{subfigure}[b]{0.25\linewidth}
    \includegraphics[width=0.95\linewidth]{../images/output_pdf/exp_1d/gaussian/1e-06_1e-06/gaussian_1e-06_1e-06_1000_ratio.pdf}
    \caption{} \label{fig:gaussian_1d_low_noise_arb_c3}
  \end{subfigure}%
  \begin{subfigure}[b]{0.25\linewidth}
    \includegraphics[width=0.95\linewidth]{../images/output_pdf/exp_1d/gaussian/1e-06_0.1/gaussian_1e-06_0,1_1000_ratio.pdf}
    \caption{} \label{fig:gaussian_1d_low_noise_arb_c4}
  \end{subfigure}%
  \caption{The coverage rate dependence on $n$ in the fully Gaussian low-noise case $\gamma=10^{-6}$
  for arbitrary $\theta$:
  (\subref{fig:gaussian_1d_low_noise_arb_c1}, \subref{fig:gaussian_1d_low_noise_arb_c2})~--~$\theta=10$,
  and (\subref{fig:gaussian_1d_low_noise_arb_c3}, \subref{fig:gaussian_1d_low_noise_arb_c4})~--~$\theta=1000$.
  For description refer to fig.~\ref{fig:gaussian_1d_low_noise}.}
  \label{fig:gaussian_1d_low_noise_arb}
\end{figure}

\begin{figure}%[b, width=0.5\textwidth]
  \centering
  \begin{subfigure}[b]{0.25\linewidth}
    \includegraphics[width=0.95\linewidth]{../images/output_pdf/exp_1d/gaussian/1e-06_1e-06/width/GPR-f/width_gaussian_1e-06_1e-06_10_GPR-f.pdf}
  \end{subfigure}%
  \begin{subfigure}[b]{0.25\linewidth}
    \includegraphics[width=0.95\linewidth]{../images/output_pdf/exp_1d/gaussian/1e-06_0.1/width/GPR-f/width_gaussian_1e-06_0,1_10_GPR-f.pdf}
  \end{subfigure}%
  \begin{subfigure}[b]{0.25\linewidth}
    \includegraphics[width=0.95\linewidth]{../images/output_pdf/exp_1d/gaussian/1e-06_1e-06/width/GPR-f/width_gaussian_1e-06_1e-06_1000_GPR-f.pdf}
  \end{subfigure}%
  \begin{subfigure}[b]{0.25\linewidth}
    \includegraphics[width=0.95\linewidth]{../images/output_pdf/exp_1d/gaussian/1e-06_0.1/width/GPR-f/width_gaussian_1e-06_0,1_1000_GPR-f.pdf}
  \end{subfigure}\\
  \begin{subfigure}[b]{0.25\linewidth}
    \includegraphics[width=0.95\linewidth]{../images/output_pdf/exp_1d/gaussian/1e-06_1e-06/width/RRCM/width_gaussian_1e-06_1e-06_10_RRCM.pdf}
  \end{subfigure}%
  \begin{subfigure}[b]{0.25\linewidth}
    \includegraphics[width=0.95\linewidth]{../images/output_pdf/exp_1d/gaussian/1e-06_0.1/width/RRCM/width_gaussian_1e-06_0,1_10_RRCM.pdf}
  \end{subfigure}%
  \begin{subfigure}[b]{0.25\linewidth}
    \includegraphics[width=0.95\linewidth]{../images/output_pdf/exp_1d/gaussian/1e-06_1e-06/width/RRCM/width_gaussian_1e-06_1e-06_1000_RRCM.pdf}
  \end{subfigure}%
  \begin{subfigure}[b]{0.25\linewidth}
    \includegraphics[width=0.95\linewidth]{../images/output_pdf/exp_1d/gaussian/1e-06_0.1/width/RRCM/width_gaussian_1e-06_0,1_1000_RRCM.pdf}
  \end{subfigure}\\
  \begin{subfigure}[b]{0.25\linewidth}
    \includegraphics[width=0.95\linewidth]{../images/output_pdf/exp_1d/gaussian/1e-06_1e-06/width/RRCM-loo/width_gaussian_1e-06_1e-06_10_RRCM-loo.pdf}
  \end{subfigure}%
  \begin{subfigure}[b]{0.25\linewidth}
    \includegraphics[width=0.95\linewidth]{../images/output_pdf/exp_1d/gaussian/1e-06_0.1/width/RRCM-loo/width_gaussian_1e-06_0,1_10_RRCM-loo.pdf}
  \end{subfigure}%
  \begin{subfigure}[b]{0.25\linewidth}
    \includegraphics[width=0.95\linewidth]{../images/output_pdf/exp_1d/gaussian/1e-06_1e-06/width/RRCM-loo/width_gaussian_1e-06_1e-06_1000_RRCM-loo.pdf}
  \end{subfigure}%
  \begin{subfigure}[b]{0.25\linewidth}
    \includegraphics[width=0.95\linewidth]{../images/output_pdf/exp_1d/gaussian/1e-06_0.1/width/RRCM-loo/width_gaussian_1e-06_0,1_1000_RRCM-loo.pdf}
  \end{subfigure}\\
  \begin{subfigure}[b]{0.25\linewidth}
    \includegraphics[width=0.95\linewidth]{../images/output_pdf/exp_1d/gaussian/1e-06_1e-06/width/CRR/width_gaussian_1e-06_1e-06_10_CRR.pdf}
  \end{subfigure}%
  \begin{subfigure}[b]{0.25\linewidth}
    \includegraphics[width=0.95\linewidth]{../images/output_pdf/exp_1d/gaussian/1e-06_0.1/width/CRR/width_gaussian_1e-06_0,1_10_CRR.pdf}
  \end{subfigure}%
  \begin{subfigure}[b]{0.25\linewidth}
    \includegraphics[width=0.95\linewidth]{../images/output_pdf/exp_1d/gaussian/1e-06_1e-06/width/CRR/width_gaussian_1e-06_1e-06_1000_CRR.pdf}
  \end{subfigure}%
  \begin{subfigure}[b]{0.25\linewidth}
    \includegraphics[width=0.95\linewidth]{../images/output_pdf/exp_1d/gaussian/1e-06_0.1/width/CRR/width_gaussian_1e-06_0,1_1000_CRR.pdf}
  \end{subfigure}\\
  \begin{subfigure}[b]{0.25\linewidth}
    \includegraphics[width=0.95\linewidth]{../images/output_pdf/exp_1d/gaussian/1e-06_1e-06/width/CRR-loo/width_gaussian_1e-06_1e-06_10_CRR-loo.pdf}
    \caption{} \label{fig:gaussian_1d_low_noise_width_arb_c1}
  \end{subfigure}%
  \begin{subfigure}[b]{0.25\linewidth}
    \includegraphics[width=0.95\linewidth]{../images/output_pdf/exp_1d/gaussian/1e-06_0.1/width/CRR-loo/width_gaussian_1e-06_0,1_10_CRR-loo.pdf}
    \caption{} \label{fig:gaussian_1d_low_noise_width_arb_c2}
  \end{subfigure}%
  \begin{subfigure}[b]{0.25\linewidth}
    \includegraphics[width=0.95\linewidth]{../images/output_pdf/exp_1d/gaussian/1e-06_1e-06/width/CRR-loo/width_gaussian_1e-06_1e-06_1000_CRR-loo.pdf}
    \caption{} \label{fig:gaussian_1d_low_noise_width_arb_c3}
  \end{subfigure}%
  \begin{subfigure}[b]{0.25\linewidth}
    \includegraphics[width=0.95\linewidth]{../images/output_pdf/exp_1d/gaussian/1e-06_0.1/width/CRR-loo/width_gaussian_1e-06_0,1_1000_CRR-loo.pdf}
    \caption{} \label{fig:gaussian_1d_low_noise_width_arb_c4}
  \end{subfigure}%
  \caption{Confidence region size dynamics for arbitrary $\theta$ in the fully Gaussian
  case (see description in fig.~\ref{fig:gaussian_1d_low_noise_arb}).}
  \label{fig:gaussian_1d_low_noise_width_arb}
\end{figure}

We now proceed to the non-Gaussian noiseless experiments. The results for the conformal
confidence intervals are not qualitatively different in the non-Gaussian setting:
coverage rate maintains its convergence to the specified confidence levels and all
conformal procedures demonstrate very similar asymptotic validity. For the Heaviside
step function typical confidence bands are shown in fig.~\ref{fig:nongauss_1d_heaviside},
and the asymptotic coverage rate of various confidence bands are presented at fig.~
\ref{fig:heaviside_1d_low_noise}, and~\ref{fig:heaviside_1d_low_noise_arb}. For another
non-Gaussian function (``f6'', fig.~\ref{fig:nongauss_1d_f6}) see fig.~\ref{fig:f6_1d_low_noise},
and~\ref{fig:f6_1d_low_noise_arb}.
\begin{figure}%[t, width=0.5\textwidth]
  \centering
  \begin{subfigure}[b]{0.33\linewidth}
    \includegraphics[width=0.9\linewidth]{../images/output_pdf/profile/heaviside/1e-06_1e-06/50/profile_heaviside_1e-06_1e-06_100_10p-GPR_50.pdf}
  \end{subfigure}%
  \begin{subfigure}[b]{0.33\linewidth}
    \includegraphics[width=0.9\linewidth]{../images/output_pdf/profile/heaviside/1e-06_1e-06/50/profile_heaviside_1e-06_1e-06_100_10p-CRR_50.pdf}
  \end{subfigure}%
  \begin{subfigure}[b]{0.33\linewidth}
    \includegraphics[width=0.9\linewidth]{../images/output_pdf/profile/heaviside/1e-06_1e-06/50/profile_heaviside_1e-06_1e-06_100_10p-RRCM_50.pdf}
  \end{subfigure}
  \caption{Sample Bayesian and conformal confidence bands for the ``Heaviside'' step
  function (train sample size $n=50$): \textit{left} -- GPR, \textit{middle} -- CRR,
  and \textit{right} -- RRCM.}
  \label{fig:nongauss_1d_heaviside}
\end{figure}
\begin{figure}%[t, width=0.5\textwidth]
  \centering
  \begin{subfigure}[b]{0.33\linewidth}
    \includegraphics[width=0.9\linewidth]{../images/output_pdf/profile/f6/1e-06_1e-06/50/profile_f6_1e-06_1e-06_100_10p-GPR_50.pdf}
  \end{subfigure}%
  \begin{subfigure}[b]{0.33\linewidth}
    \includegraphics[width=0.9\linewidth]{../images/output_pdf/profile/f6/1e-06_1e-06/50/profile_f6_1e-06_1e-06_100_10p-CRR_50.pdf}
  \end{subfigure}%
  \begin{subfigure}[b]{0.33\linewidth}
    \includegraphics[width=0.9\linewidth]{../images/output_pdf/profile/f6/1e-06_1e-06/50/profile_f6_1e-06_1e-06_100_10p-RRCM_50.pdf}
  \end{subfigure}
  \caption{Sample Bayesian and conformal confidence bands for the ``f6'' function
  (train sample size $n=50$): \textit{left} -- GPR, \textit{middle} -- CRR, and
  \textit{right} -- RRCM.}
  \label{fig:nongauss_1d_f6}
\end{figure}

In contrast to the fully Gaussian setting, the GPR confidence intervals are not
consistently conservative. As is evident from coverage rate dynamics for both functions,
for relatively high regularization $\lambda=10^{-1}$ the GPR region fails to show
conservative validity in a consistent manner, despite quite faithful approximation
by the KRR as demonstrated by the ``RMSE-std'' ratio. At the same time conformal
procedures, e.g. RRCM, show no significant departures from claimed validity.

The main conclusion we can draw from the case on negligible noise is that at least
for the Gaussian kernel the conformal confidence intervals seem to perform reasonably
well both in terms of validity in a non-Gaussian setting and efficiency in Gaussian
setting.

\begin{figure}%[b, width=0.5\textwidth]
  \centering
  \begin{subfigure}[b]{0.25\linewidth}
    \includegraphics[width=0.95\linewidth]{../images/output_pdf/exp_1d/heaviside/1e-06_1e-06/coverage/GPR-f/coverage_heaviside_1e-06_1e-06_100_GPR-f.pdf}
  \end{subfigure}%
  \begin{subfigure}[b]{0.25\linewidth}
    \includegraphics[width=0.95\linewidth]{../images/output_pdf/exp_1d/heaviside/1e-06_0.1/coverage/GPR-f/coverage_heaviside_1e-06_0,1_100_GPR-f.pdf}
  \end{subfigure}%
  \begin{subfigure}[b]{0.25\linewidth}
    \includegraphics[width=0.95\linewidth]{../images/output_pdf/exp_1d/heaviside/1e-06_1e-06/coverage/GPR-f/coverage_heaviside_1e-06_1e-06_auto_GPR-f.pdf}
  \end{subfigure}%
  \begin{subfigure}[b]{0.25\linewidth}
    \includegraphics[width=0.95\linewidth]{../images/output_pdf/exp_1d/heaviside/1e-06_0.1/coverage/GPR-f/coverage_heaviside_1e-06_0,1_auto_GPR-f.pdf}
  \end{subfigure}\\
  \begin{subfigure}[b]{0.25\linewidth}
    \includegraphics[width=0.95\linewidth]{../images/output_pdf/exp_1d/heaviside/1e-06_1e-06/coverage/RRCM/coverage_heaviside_1e-06_1e-06_100_RRCM.pdf}
  \end{subfigure}%
  \begin{subfigure}[b]{0.25\linewidth}
    \includegraphics[width=0.95\linewidth]{../images/output_pdf/exp_1d/heaviside/1e-06_0.1/coverage/RRCM/coverage_heaviside_1e-06_0,1_100_RRCM.pdf}
  \end{subfigure}%
  \begin{subfigure}[b]{0.25\linewidth}
    \includegraphics[width=0.95\linewidth]{../images/output_pdf/exp_1d/heaviside/1e-06_1e-06/coverage/RRCM/coverage_heaviside_1e-06_1e-06_auto_RRCM.pdf}
  \end{subfigure}%
  \begin{subfigure}[b]{0.25\linewidth}
    \includegraphics[width=0.95\linewidth]{../images/output_pdf/exp_1d/heaviside/1e-06_0.1/coverage/RRCM/coverage_heaviside_1e-06_0,1_auto_RRCM.pdf}
  \end{subfigure}\\
  \begin{subfigure}[b]{0.25\linewidth}
    \includegraphics[width=0.95\linewidth]{../images/output_pdf/exp_1d/heaviside/1e-06_1e-06/heaviside_1e-06_1e-06_100_ratio.pdf}
    \caption{} \label{fig:heaviside_1d_low_noise_c1}
  \end{subfigure}%
  \begin{subfigure}[b]{0.25\linewidth}
    \includegraphics[width=0.95\linewidth]{../images/output_pdf/exp_1d/heaviside/1e-06_0.1/heaviside_1e-06_0,1_100_ratio.pdf}
    \caption{} \label{fig:heaviside_1d_low_noise_c2}
  \end{subfigure}%
  \begin{subfigure}[b]{0.25\linewidth}
    \includegraphics[width=0.95\linewidth]{../images/output_pdf/exp_1d/heaviside/1e-06_1e-06/heaviside_1e-06_1e-06_auto_ratio.pdf}
    \caption{} \label{fig:heaviside_1d_low_noise_c3}
  \end{subfigure}%
  \begin{subfigure}[b]{0.25\linewidth}
    \includegraphics[width=0.95\linewidth]{../images/output_pdf/exp_1d/heaviside/1e-06_0.1/heaviside_1e-06_0,1_auto_ratio.pdf}
    \caption{} \label{fig:heaviside_1d_low_noise_c4}
  \end{subfigure}%
  \caption{Coverage dynamics for the ``Heaviside'' step function.}
  \label{fig:heaviside_1d_low_noise}
\end{figure}

\begin{figure}%[b, width=0.5\textwidth]
  \centering
  \begin{subfigure}[b]{0.25\linewidth}
    \includegraphics[width=0.95\linewidth]{../images/output_pdf/exp_1d/heaviside/1e-06_1e-06/coverage/GPR-f/coverage_heaviside_1e-06_1e-06_10_GPR-f.pdf}
  \end{subfigure}%
  \begin{subfigure}[b]{0.25\linewidth}
    \includegraphics[width=0.95\linewidth]{../images/output_pdf/exp_1d/heaviside/1e-06_0.1/coverage/GPR-f/coverage_heaviside_1e-06_0,1_10_GPR-f.pdf}
  \end{subfigure}%
  \begin{subfigure}[b]{0.25\linewidth}
    \includegraphics[width=0.95\linewidth]{../images/output_pdf/exp_1d/heaviside/1e-06_1e-06/coverage/GPR-f/coverage_heaviside_1e-06_1e-06_1000_GPR-f.pdf}
  \end{subfigure}%
  \begin{subfigure}[b]{0.25\linewidth}
    \includegraphics[width=0.95\linewidth]{../images/output_pdf/exp_1d/heaviside/1e-06_0.1/coverage/GPR-f/coverage_heaviside_1e-06_0,1_1000_GPR-f.pdf}
  \end{subfigure}\\
  \begin{subfigure}[b]{0.25\linewidth}
    \includegraphics[width=0.95\linewidth]{../images/output_pdf/exp_1d/heaviside/1e-06_1e-06/coverage/RRCM/coverage_heaviside_1e-06_1e-06_10_RRCM.pdf}
  \end{subfigure}%
  \begin{subfigure}[b]{0.25\linewidth}
    \includegraphics[width=0.95\linewidth]{../images/output_pdf/exp_1d/heaviside/1e-06_0.1/coverage/RRCM/coverage_heaviside_1e-06_0,1_10_RRCM.pdf}
  \end{subfigure}%
  \begin{subfigure}[b]{0.25\linewidth}
    \includegraphics[width=0.95\linewidth]{../images/output_pdf/exp_1d/heaviside/1e-06_1e-06/coverage/RRCM/coverage_heaviside_1e-06_1e-06_1000_RRCM.pdf}
  \end{subfigure}%
  \begin{subfigure}[b]{0.25\linewidth}
    \includegraphics[width=0.95\linewidth]{../images/output_pdf/exp_1d/heaviside/1e-06_0.1/coverage/RRCM/coverage_heaviside_1e-06_0,1_1000_RRCM.pdf}
  \end{subfigure}\\
  \begin{subfigure}[b]{0.25\linewidth}
    \includegraphics[width=0.95\linewidth]{../images/output_pdf/exp_1d/heaviside/1e-06_1e-06/heaviside_1e-06_1e-06_10_ratio.pdf}
    \caption{} \label{fig:heaviside_1d_low_noise_arb_c1}
  \end{subfigure}%
  \begin{subfigure}[b]{0.25\linewidth}
    \includegraphics[width=0.95\linewidth]{../images/output_pdf/exp_1d/heaviside/1e-06_0.1/heaviside_1e-06_0,1_10_ratio.pdf}
    \caption{} \label{fig:heaviside_1d_low_noise_arb_c2}
  \end{subfigure}%
  \begin{subfigure}[b]{0.25\linewidth}
    \includegraphics[width=0.95\linewidth]{../images/output_pdf/exp_1d/heaviside/1e-06_1e-06/heaviside_1e-06_1e-06_1000_ratio.pdf}
    \caption{} \label{fig:heaviside_1d_low_noise_arb_c3}
  \end{subfigure}%
  \begin{subfigure}[b]{0.25\linewidth}
    \includegraphics[width=0.95\linewidth]{../images/output_pdf/exp_1d/heaviside/1e-06_0.1/heaviside_1e-06_0,1_1000_ratio.pdf}
    \caption{} \label{fig:heaviside_1d_low_noise_arb_c4}
  \end{subfigure}%
  \caption{Coverage dynamics for the ``Heaviside'' step function(c.f. fig.~\ref{fig:heaviside_1d_low_noise}).}
  \label{fig:heaviside_1d_low_noise_arb}
\end{figure}

% \begin{figure}%[b, width=0.5\textwidth]
%   \centering
%   \begin{subfigure}[b]{0.25\linewidth}
%     \includegraphics[width=0.95\linewidth]{../images/output_pdf/exp_1d/pressure2/1e-06_1e-06/coverage/GPR-f/coverage_pressure2_1e-06_1e-06_100_GPR-f.pdf}
%   \end{subfigure}%
%   \begin{subfigure}[b]{0.25\linewidth}
%     \includegraphics[width=0.95\linewidth]{../images/output_pdf/exp_1d/pressure2/1e-06_0.1/coverage/GPR-f/coverage_pressure2_1e-06_0,1_100_GPR-f.pdf}
%   \end{subfigure}%
%   \begin{subfigure}[b]{0.25\linewidth}
%     \includegraphics[width=0.95\linewidth]{../images/output_pdf/exp_1d/pressure2/1e-06_1e-06/coverage/GPR-f/coverage_pressure2_1e-06_1e-06_auto_GPR-f.pdf}
%   \end{subfigure}%
%   \begin{subfigure}[b]{0.25\linewidth}
%     \includegraphics[width=0.95\linewidth]{../images/output_pdf/exp_1d/pressure2/1e-06_0.1/coverage/GPR-f/coverage_pressure2_1e-06_0,1_auto_GPR-f.pdf}
%   \end{subfigure}\\
%   \begin{subfigure}[b]{0.25\linewidth}
%     \includegraphics[width=0.95\linewidth]{../images/output_pdf/exp_1d/pressure2/1e-06_1e-06/coverage/RRCM/coverage_pressure2_1e-06_1e-06_100_RRCM.pdf}
%   \end{subfigure}%
%   \begin{subfigure}[b]{0.25\linewidth}
%     \includegraphics[width=0.95\linewidth]{../images/output_pdf/exp_1d/pressure2/1e-06_0.1/coverage/RRCM/coverage_pressure2_1e-06_0,1_100_RRCM.pdf}
%   \end{subfigure}%
%   \begin{subfigure}[b]{0.25\linewidth}
%     \includegraphics[width=0.95\linewidth]{../images/output_pdf/exp_1d/pressure2/1e-06_1e-06/coverage/RRCM/coverage_pressure2_1e-06_1e-06_auto_RRCM.pdf}
%   \end{subfigure}%
%   \begin{subfigure}[b]{0.25\linewidth}
%     \includegraphics[width=0.95\linewidth]{../images/output_pdf/exp_1d/pressure2/1e-06_0.1/coverage/RRCM/coverage_pressure2_1e-06_0,1_auto_RRCM.pdf}
%   \end{subfigure}\\
%   \begin{subfigure}[b]{0.25\linewidth}
%     \includegraphics[width=0.95\linewidth]{../images/output_pdf/exp_1d/pressure2/1e-06_1e-06/pressure2_1e-06_1e-06_100_ratio.pdf}
%     \caption{} \label{fig:pressure2_1d_low_noise_c1}
%   \end{subfigure}%
%   \begin{subfigure}[b]{0.25\linewidth}
%     \includegraphics[width=0.95\linewidth]{../images/output_pdf/exp_1d/pressure2/1e-06_0.1/pressure2_1e-06_0,1_100_ratio.pdf}
%     \caption{} \label{fig:pressure2_1d_low_noise_c2}
%   \end{subfigure}%
%   \begin{subfigure}[b]{0.25\linewidth}
%     \includegraphics[width=0.95\linewidth]{../images/output_pdf/exp_1d/pressure2/1e-06_1e-06/pressure2_1e-06_1e-06_auto_ratio.pdf}
%     \caption{} \label{fig:pressure2_1d_low_noise_c3}
%   \end{subfigure}%
%   \begin{subfigure}[b]{0.25\linewidth}
%     \includegraphics[width=0.95\linewidth]{../images/output_pdf/exp_1d/pressure2/1e-06_0.1/pressure2_1e-06_0,1_auto_ratio.pdf}
%     \caption{} \label{fig:pressure2_1d_low_noise_c4}
%   \end{subfigure}%
%   \caption{Coverage dynamics for the ``pressure2'' function.}
%   \label{fig:pressure2_1d_low_noise}
% \end{figure}

% \begin{figure}%[b, width=0.5\textwidth]
%   \centering
%   \begin{subfigure}[b]{0.25\linewidth}
%     \includegraphics[width=0.95\linewidth]{../images/output_pdf/exp_1d/pressure2/1e-06_1e-06/coverage/GPR-f/coverage_pressure2_1e-06_1e-06_10_GPR-f.pdf}
%   \end{subfigure}%
%   \begin{subfigure}[b]{0.25\linewidth}
%     \includegraphics[width=0.95\linewidth]{../images/output_pdf/exp_1d/pressure2/1e-06_0.1/coverage/GPR-f/coverage_pressure2_1e-06_0,1_10_GPR-f.pdf}
%   \end{subfigure}%
%   \begin{subfigure}[b]{0.25\linewidth}
%     \includegraphics[width=0.95\linewidth]{../images/output_pdf/exp_1d/pressure2/1e-06_1e-06/coverage/GPR-f/coverage_pressure2_1e-06_1e-06_1000_GPR-f.pdf}
%   \end{subfigure}%
%   \begin{subfigure}[b]{0.25\linewidth}
%     \includegraphics[width=0.95\linewidth]{../images/output_pdf/exp_1d/pressure2/1e-06_0.1/coverage/GPR-f/coverage_pressure2_1e-06_0,1_1000_GPR-f.pdf}
%   \end{subfigure}\\
%   \begin{subfigure}[b]{0.25\linewidth}
%     \includegraphics[width=0.95\linewidth]{../images/output_pdf/exp_1d/pressure2/1e-06_1e-06/coverage/RRCM/coverage_pressure2_1e-06_1e-06_10_RRCM.pdf}
%   \end{subfigure}%
%   \begin{subfigure}[b]{0.25\linewidth}
%     \includegraphics[width=0.95\linewidth]{../images/output_pdf/exp_1d/pressure2/1e-06_0.1/coverage/RRCM/coverage_pressure2_1e-06_0,1_10_RRCM.pdf}
%   \end{subfigure}%
%   \begin{subfigure}[b]{0.25\linewidth}
%     \includegraphics[width=0.95\linewidth]{../images/output_pdf/exp_1d/pressure2/1e-06_1e-06/coverage/RRCM/coverage_pressure2_1e-06_1e-06_1000_RRCM.pdf}
%   \end{subfigure}%
%   \begin{subfigure}[b]{0.25\linewidth}
%     \includegraphics[width=0.95\linewidth]{../images/output_pdf/exp_1d/pressure2/1e-06_0.1/coverage/RRCM/coverage_pressure2_1e-06_0,1_1000_RRCM.pdf}
%   \end{subfigure}\\
%   \begin{subfigure}[b]{0.25\linewidth}
%     \includegraphics[width=0.95\linewidth]{../images/output_pdf/exp_1d/pressure2/1e-06_1e-06/pressure2_1e-06_1e-06_10_ratio.pdf}
%     \caption{} \label{fig:pressure2_1d_low_noise_arb_c1}
%   \end{subfigure}%
%   \begin{subfigure}[b]{0.25\linewidth}
%     \includegraphics[width=0.95\linewidth]{../images/output_pdf/exp_1d/pressure2/1e-06_0.1/pressure2_1e-06_0,1_10_ratio.pdf}
%     \caption{} \label{fig:pressure2_1d_low_noise_arb_c2}
%   \end{subfigure}%
%   \begin{subfigure}[b]{0.25\linewidth}
%     \includegraphics[width=0.95\linewidth]{../images/output_pdf/exp_1d/pressure2/1e-06_1e-06/pressure2_1e-06_1e-06_1000_ratio.pdf}
%     \caption{} \label{fig:pressure2_1d_low_noise_arb_c3}
%   \end{subfigure}%
%   \begin{subfigure}[b]{0.25\linewidth}
%     \includegraphics[width=0.95\linewidth]{../images/output_pdf/exp_1d/pressure2/1e-06_0.1/pressure2_1e-06_0,1_1000_ratio.pdf}
%     \caption{} \label{fig:pressure2_1d_low_noise_arb_c4}
%   \end{subfigure}%
%   \caption{Coverage dynamics for the ``pressure2'' function(c.f. fig.~\ref{fig:pressure2_1d_low_noise}).}
%   \label{fig:pressure2_1d_low_noise_arb}
% \end{figure}

\begin{figure}%[b, width=0.5\textwidth]
  \centering
  \begin{subfigure}[b]{0.25\linewidth}
    \includegraphics[width=0.95\linewidth]{../images/output_pdf/exp_1d/f6/1e-06_1e-06/coverage/GPR-f/coverage_f6_1e-06_1e-06_100_GPR-f.pdf}
  \end{subfigure}%
  \begin{subfigure}[b]{0.25\linewidth}
    \includegraphics[width=0.95\linewidth]{../images/output_pdf/exp_1d/f6/1e-06_0.1/coverage/GPR-f/coverage_f6_1e-06_0,1_100_GPR-f.pdf}
  \end{subfigure}%
  \begin{subfigure}[b]{0.25\linewidth}
    \includegraphics[width=0.95\linewidth]{../images/output_pdf/exp_1d/f6/1e-06_1e-06/coverage/GPR-f/coverage_f6_1e-06_1e-06_auto_GPR-f.pdf}
  \end{subfigure}%
  \begin{subfigure}[b]{0.25\linewidth}
    \includegraphics[width=0.95\linewidth]{../images/output_pdf/exp_1d/f6/1e-06_0.1/coverage/GPR-f/coverage_f6_1e-06_0,1_auto_GPR-f.pdf}
  \end{subfigure}\\
  \begin{subfigure}[b]{0.25\linewidth}
    \includegraphics[width=0.95\linewidth]{../images/output_pdf/exp_1d/f6/1e-06_1e-06/coverage/RRCM/coverage_f6_1e-06_1e-06_100_RRCM.pdf}
  \end{subfigure}%
  \begin{subfigure}[b]{0.25\linewidth}
    \includegraphics[width=0.95\linewidth]{../images/output_pdf/exp_1d/f6/1e-06_0.1/coverage/RRCM/coverage_f6_1e-06_0,1_100_RRCM.pdf}
  \end{subfigure}%
  \begin{subfigure}[b]{0.25\linewidth}
    \includegraphics[width=0.95\linewidth]{../images/output_pdf/exp_1d/f6/1e-06_1e-06/coverage/RRCM/coverage_f6_1e-06_1e-06_auto_RRCM.pdf}
  \end{subfigure}%
  \begin{subfigure}[b]{0.25\linewidth}
    \includegraphics[width=0.95\linewidth]{../images/output_pdf/exp_1d/f6/1e-06_0.1/coverage/RRCM/coverage_f6_1e-06_0,1_auto_RRCM.pdf}
  \end{subfigure}\\
  \begin{subfigure}[b]{0.25\linewidth}
    \includegraphics[width=0.95\linewidth]{../images/output_pdf/exp_1d/f6/1e-06_1e-06/f6_1e-06_1e-06_100_ratio.pdf}
    \caption{} \label{fig:f6_1d_low_noise_c1}
  \end{subfigure}%
  \begin{subfigure}[b]{0.25\linewidth}
    \includegraphics[width=0.95\linewidth]{../images/output_pdf/exp_1d/f6/1e-06_0.1/f6_1e-06_0,1_100_ratio.pdf}
    \caption{} \label{fig:f6_1d_low_noise_c2}
  \end{subfigure}%
  \begin{subfigure}[b]{0.25\linewidth}
    \includegraphics[width=0.95\linewidth]{../images/output_pdf/exp_1d/f6/1e-06_1e-06/f6_1e-06_1e-06_auto_ratio.pdf}
    \caption{} \label{fig:f6_1d_low_noise_c3}
  \end{subfigure}%
  \begin{subfigure}[b]{0.25\linewidth}
    \includegraphics[width=0.95\linewidth]{../images/output_pdf/exp_1d/f6/1e-06_0.1/f6_1e-06_0,1_auto_ratio.pdf}
    \caption{} \label{fig:f6_1d_low_noise_c4}
  \end{subfigure}%
  \caption{Coverage dynamics and ``RMSE-std'' ratio for the ``f6'' function.}
  \label{fig:f6_1d_low_noise}
\end{figure}

\begin{figure}%[b, width=0.5\textwidth]
  \centering
  \begin{subfigure}[b]{0.25\linewidth}
    \includegraphics[width=0.95\linewidth]{../images/output_pdf/exp_1d/f6/1e-06_1e-06/coverage/GPR-f/coverage_f6_1e-06_1e-06_10_GPR-f.pdf}
  \end{subfigure}%
  \begin{subfigure}[b]{0.25\linewidth}
    \includegraphics[width=0.95\linewidth]{../images/output_pdf/exp_1d/f6/1e-06_0.1/coverage/GPR-f/coverage_f6_1e-06_0,1_10_GPR-f.pdf}
  \end{subfigure}%
  \begin{subfigure}[b]{0.25\linewidth}
    \includegraphics[width=0.95\linewidth]{../images/output_pdf/exp_1d/f6/1e-06_1e-06/coverage/GPR-f/coverage_f6_1e-06_1e-06_1000_GPR-f.pdf}
  \end{subfigure}%
  \begin{subfigure}[b]{0.25\linewidth}
    \includegraphics[width=0.95\linewidth]{../images/output_pdf/exp_1d/f6/1e-06_0.1/coverage/GPR-f/coverage_f6_1e-06_0,1_1000_GPR-f.pdf}
  \end{subfigure}\\
  \begin{subfigure}[b]{0.25\linewidth}
    \includegraphics[width=0.95\linewidth]{../images/output_pdf/exp_1d/f6/1e-06_1e-06/coverage/RRCM/coverage_f6_1e-06_1e-06_10_RRCM.pdf}
  \end{subfigure}%
  \begin{subfigure}[b]{0.25\linewidth}
    \includegraphics[width=0.95\linewidth]{../images/output_pdf/exp_1d/f6/1e-06_0.1/coverage/RRCM/coverage_f6_1e-06_0,1_10_RRCM.pdf}
  \end{subfigure}%
  \begin{subfigure}[b]{0.25\linewidth}
    \includegraphics[width=0.95\linewidth]{../images/output_pdf/exp_1d/f6/1e-06_1e-06/coverage/RRCM/coverage_f6_1e-06_1e-06_1000_RRCM.pdf}
  \end{subfigure}%
  \begin{subfigure}[b]{0.25\linewidth}
    \includegraphics[width=0.95\linewidth]{../images/output_pdf/exp_1d/f6/1e-06_0.1/coverage/RRCM/coverage_f6_1e-06_0,1_1000_RRCM.pdf}
  \end{subfigure}\\
  \begin{subfigure}[b]{0.25\linewidth}
    \includegraphics[width=0.95\linewidth]{../images/output_pdf/exp_1d/f6/1e-06_1e-06/f6_1e-06_1e-06_10_ratio.pdf}
    \caption{} \label{fig:f6_1d_low_noise_arb_c1}
  \end{subfigure}%
  \begin{subfigure}[b]{0.25\linewidth}
    \includegraphics[width=0.95\linewidth]{../images/output_pdf/exp_1d/f6/1e-06_0.1/f6_1e-06_0,1_10_ratio.pdf}
    \caption{} \label{fig:f6_1d_low_noise_arb_c2}
  \end{subfigure}%
  \begin{subfigure}[b]{0.25\linewidth}
    \includegraphics[width=0.95\linewidth]{../images/output_pdf/exp_1d/f6/1e-06_1e-06/f6_1e-06_1e-06_1000_ratio.pdf}
    \caption{} \label{fig:f6_1d_low_noise_arb_c3}
  \end{subfigure}%
  \begin{subfigure}[b]{0.25\linewidth}
    \includegraphics[width=0.95\linewidth]{../images/output_pdf/exp_1d/f6/1e-06_0.1/f6_1e-06_0,1_1000_ratio.pdf}
    \caption{} \label{fig:f6_1d_low_noise_arb_c4}
  \end{subfigure}%
  \caption{Coverage dynamics and ``RMSE-std'' ratio for the ``f6'' function (see
  the description of fig.~\ref{fig:f6_1d_low_noise}).}
  \label{fig:f6_1d_low_noise_arb}
\end{figure}

Now let's assess the performance of the confidence regions in noisy setting. We
study the effects of a moderate noise-to-signal $\gamma=10^{-1}$ ratio, relative
to the variability of the studied test functions.

Firstly, we consider the case when the $\theta$ hyper-parameter is exactly equal
to $\theta_0$. In this setting the sample GPR and KRR RRCM confidence bands are
depicted in fig.~\ref{fig:gauss_1d_prof_conf}, and fig.~\ref{fig:gauss_1d_prof_gpr}.
Qualitatively there is no difference to the negligible nose case, except in the current
setting the band are wider do to higher observation noise. Despite the overall worse
fit, however, both conformal and Bayesian confidence regions show approximately correct
asymptotic coverage rates (fig.~\ref{fig:gaussian_1d_high_noise_c1} and~\ref{fig:gaussian_1d_high_noise_c2}).
Regarding the efficiency, GPR confidence intervals are narrower that the conformal
regions, but for sufficiently large train samples $n$ the width of the latter gradually
shrinks to that of the Bayesian interval (fig.~\ref{fig:gaussian_1d_high_noise_width_c1}
and~\ref{fig:gaussian_1d_high_noise_width_c2}. Thus, in this setting, as in the
previous one, we found evidence supporting the conjectured properties of conformal
procedures over KRR with the Gaussian kernel.

In the setting, when the kernel precision $\theta$ no equal to the theoretical $\theta_0$,
but is either estimated with the MLE $\hat{\theta}_\text{MLE}$, provided by the GPR
likelihood, eq.~\ref{eq:bkrr_likelihood}, (fig.~\ref{fig:gaussian_1d_high_noise_c3}
and~\ref{fig:gaussian_1d_high_noise_c4}), or too low $\theta=10$ (fig.~\ref{fig:gaussian_1d_high_noise_arb_c1}
and~\ref{fig:gaussian_1d_high_noise_arb_c1}), or too high $\theta=10^3$ (fig.~\ref{fig:gaussian_1d_high_noise_arb_c3}
and~\ref{fig:gaussian_1d_high_noise_arb_c4}) we arrive at similar conclusions: confidence
intervals constructed with an NCM based on KRR with kernel eq.~\ref{eq:gauss_kenrel}
provide at least conservatively the required level of validity regardless of the
parameters ($\theta$, $\lambda$) at the core of the non-conformity measure.

From this particular experiment in this setting, can be concluded that there is evidence
in favour of the conjectured asymptotic conservative validity of the conformal confidence
regions for all considered NCMs and residual types as well as their asymptotic efficiency.
As expected, the GPR confidence predictions uphold their theoretical guarantees in
this case.

\begin{figure}%[b, width=0.5\textwidth]
  \centering
  \begin{subfigure}[b]{0.25\linewidth}
    \includegraphics[width=0.95\linewidth]{../images/output_pdf/exp_1d/gaussian/0.1_1e-06/coverage/GPR-f/coverage_gaussian_0,1_1e-06_100_GPR-f.pdf}
  \end{subfigure}%
  \begin{subfigure}[b]{0.25\linewidth}
    \includegraphics[width=0.95\linewidth]{../images/output_pdf/exp_1d/gaussian/0.1_0.1/coverage/GPR-f/coverage_gaussian_0,1_0,1_100_GPR-f.pdf}
  \end{subfigure}%
  \begin{subfigure}[b]{0.25\linewidth}
    \includegraphics[width=0.95\linewidth]{../images/output_pdf/exp_1d/gaussian/0.1_1e-06/coverage/GPR-f/coverage_gaussian_0,1_1e-06_auto_GPR-f.pdf}
  \end{subfigure}%
  \begin{subfigure}[b]{0.25\linewidth}
    \includegraphics[width=0.95\linewidth]{../images/output_pdf/exp_1d/gaussian/0.1_0.1/coverage/GPR-f/coverage_gaussian_0,1_0,1_auto_GPR-f.pdf}
  \end{subfigure}\\
  \begin{subfigure}[b]{0.25\linewidth}
    \includegraphics[width=0.95\linewidth]{../images/output_pdf/exp_1d/gaussian/0.1_1e-06/coverage/RRCM/coverage_gaussian_0,1_1e-06_100_RRCM.pdf}
  \end{subfigure}%
  \begin{subfigure}[b]{0.25\linewidth}
    \includegraphics[width=0.95\linewidth]{../images/output_pdf/exp_1d/gaussian/0.1_0.1/coverage/RRCM/coverage_gaussian_0,1_0,1_100_RRCM.pdf}
  \end{subfigure}%
  \begin{subfigure}[b]{0.25\linewidth}
    \includegraphics[width=0.95\linewidth]{../images/output_pdf/exp_1d/gaussian/0.1_1e-06/coverage/RRCM/coverage_gaussian_0,1_1e-06_auto_RRCM.pdf}
  \end{subfigure}%
  \begin{subfigure}[b]{0.25\linewidth}
    \includegraphics[width=0.95\linewidth]{../images/output_pdf/exp_1d/gaussian/0.1_0.1/coverage/RRCM/coverage_gaussian_0,1_0,1_auto_RRCM.pdf}
  \end{subfigure}\\
  \begin{subfigure}[b]{0.25\linewidth}
    \includegraphics[width=0.95\linewidth]{../images/output_pdf/exp_1d/gaussian/0.1_1e-06/coverage/RRCM-loo/coverage_gaussian_0,1_1e-06_100_RRCM-loo.pdf}
  \end{subfigure}%
  \begin{subfigure}[b]{0.25\linewidth}
    \includegraphics[width=0.95\linewidth]{../images/output_pdf/exp_1d/gaussian/0.1_0.1/coverage/RRCM-loo/coverage_gaussian_0,1_0,1_100_RRCM-loo.pdf}
  \end{subfigure}%
  \begin{subfigure}[b]{0.25\linewidth}
    \includegraphics[width=0.95\linewidth]{../images/output_pdf/exp_1d/gaussian/0.1_1e-06/coverage/RRCM-loo/coverage_gaussian_0,1_1e-06_auto_RRCM-loo.pdf}
  \end{subfigure}%
  \begin{subfigure}[b]{0.25\linewidth}
    \includegraphics[width=0.95\linewidth]{../images/output_pdf/exp_1d/gaussian/0.1_0.1/coverage/RRCM-loo/coverage_gaussian_0,1_0,1_auto_RRCM-loo.pdf}
  \end{subfigure}\\
  \begin{subfigure}[b]{0.25\linewidth}
    \includegraphics[width=0.95\linewidth]{../images/output_pdf/exp_1d/gaussian/0.1_1e-06/coverage/CRR/coverage_gaussian_0,1_1e-06_100_CRR.pdf}
  \end{subfigure}%
  \begin{subfigure}[b]{0.25\linewidth}
    \includegraphics[width=0.95\linewidth]{../images/output_pdf/exp_1d/gaussian/0.1_0.1/coverage/CRR/coverage_gaussian_0,1_0,1_100_CRR.pdf}
  \end{subfigure}%
  \begin{subfigure}[b]{0.25\linewidth}
    \includegraphics[width=0.95\linewidth]{../images/output_pdf/exp_1d/gaussian/0.1_1e-06/coverage/CRR/coverage_gaussian_0,1_1e-06_auto_CRR.pdf}
  \end{subfigure}%
  \begin{subfigure}[b]{0.25\linewidth}
    \includegraphics[width=0.95\linewidth]{../images/output_pdf/exp_1d/gaussian/0.1_0.1/coverage/CRR/coverage_gaussian_0,1_0,1_auto_CRR.pdf}
  \end{subfigure}\\
  \begin{subfigure}[b]{0.25\linewidth}
    \includegraphics[width=0.95\linewidth]{../images/output_pdf/exp_1d/gaussian/0.1_1e-06/coverage/CRR-loo/coverage_gaussian_0,1_1e-06_100_CRR-loo.pdf}
  \end{subfigure}%
  \begin{subfigure}[b]{0.25\linewidth}
    \includegraphics[width=0.95\linewidth]{../images/output_pdf/exp_1d/gaussian/0.1_0.1/coverage/CRR-loo/coverage_gaussian_0,1_0,1_100_CRR-loo.pdf}
  \end{subfigure}%
  \begin{subfigure}[b]{0.25\linewidth}
    \includegraphics[width=0.95\linewidth]{../images/output_pdf/exp_1d/gaussian/0.1_1e-06/coverage/CRR-loo/coverage_gaussian_0,1_1e-06_auto_CRR-loo.pdf}
  \end{subfigure}%
  \begin{subfigure}[b]{0.25\linewidth}
    \includegraphics[width=0.95\linewidth]{../images/output_pdf/exp_1d/gaussian/0.1_0.1/coverage/CRR-loo/coverage_gaussian_0,1_0,1_auto_CRR-loo.pdf}
  \end{subfigure}\\
  \begin{subfigure}[b]{0.25\linewidth}
    \includegraphics[width=0.95\linewidth]{../images/output_pdf/exp_1d/gaussian/0.1_1e-06/gaussian_0,1_1e-06_100_ratio.pdf}
    \caption{} \label{fig:gaussian_1d_high_noise_c1}
  \end{subfigure}%
  \begin{subfigure}[b]{0.25\linewidth}
    \includegraphics[width=0.95\linewidth]{../images/output_pdf/exp_1d/gaussian/0.1_0.1/gaussian_0,1_0,1_100_ratio.pdf}
    \caption{} \label{fig:gaussian_1d_high_noise_c2}
  \end{subfigure}%
  \begin{subfigure}[b]{0.25\linewidth}
    \includegraphics[width=0.95\linewidth]{../images/output_pdf/exp_1d/gaussian/0.1_1e-06/gaussian_0,1_1e-06_auto_ratio.pdf}
    \caption{} \label{fig:gaussian_1d_high_noise_c3}
  \end{subfigure}%
  \begin{subfigure}[b]{0.25\linewidth}
    \includegraphics[width=0.95\linewidth]{../images/output_pdf/exp_1d/gaussian/0.1_0.1/gaussian_0,1_0,1_auto_ratio.pdf}
    \caption{} \label{fig:gaussian_1d_high_noise_c4}
  \end{subfigure}%
  \caption{Coverage rate dynamics in the noisy fully Gaussian case $\gamma=10^{-1}$
  for different $(\theta, \lambda)$:
  \subref{fig:gaussian_1d_high_noise_c1}~--~$(\theta_0, 10^{-6})$,
  \subref{fig:gaussian_1d_high_noise_c2}~--~$(\theta_0, 10^{-1})$,
  \subref{fig:gaussian_1d_high_noise_c3}~--~$(\hat{\theta}_\text{ML}, 10^{-6})$,
  \subref{fig:gaussian_1d_high_noise_c4}~--~$(\hat{\theta}_\text{ML}, 10^{-1})$.
  Rows from \textit{top} to \textit{bottom}: ``GPR-f'', ``RRCM'', ``RRCM-loo'',
  ``CRR'', ``CRR-loo'', and the last depicting ``RMSE-std'' ratio.}
  \label{fig:gaussian_1d_high_noise}
\end{figure}

\begin{figure}%[b, width=0.5\textwidth]
  \centering
  \begin{subfigure}[b]{0.25\linewidth}
    \includegraphics[width=0.95\linewidth]{../images/output_pdf/exp_1d/gaussian/0.1_1e-06/width/GPR-f/width_gaussian_0,1_1e-06_100_GPR-f.pdf}
  \end{subfigure}%
  \begin{subfigure}[b]{0.25\linewidth}
    \includegraphics[width=0.95\linewidth]{../images/output_pdf/exp_1d/gaussian/0.1_0.1/width/GPR-f/width_gaussian_0,1_0,1_100_GPR-f.pdf}
  \end{subfigure}%
  \begin{subfigure}[b]{0.25\linewidth}
    \includegraphics[width=0.95\linewidth]{../images/output_pdf/exp_1d/gaussian/0.1_1e-06/width/GPR-f/width_gaussian_0,1_1e-06_auto_GPR-f.pdf}
  \end{subfigure}%
  \begin{subfigure}[b]{0.25\linewidth}
    \includegraphics[width=0.95\linewidth]{../images/output_pdf/exp_1d/gaussian/0.1_0.1/width/GPR-f/width_gaussian_0,1_0,1_auto_GPR-f.pdf}
  \end{subfigure}\\
  \begin{subfigure}[b]{0.25\linewidth}
    \includegraphics[width=0.95\linewidth]{../images/output_pdf/exp_1d/gaussian/0.1_1e-06/width/RRCM/width_gaussian_0,1_1e-06_100_RRCM.pdf}
  \end{subfigure}%
  \begin{subfigure}[b]{0.25\linewidth}
    \includegraphics[width=0.95\linewidth]{../images/output_pdf/exp_1d/gaussian/0.1_0.1/width/RRCM/width_gaussian_0,1_0,1_100_RRCM.pdf}
  \end{subfigure}%
  \begin{subfigure}[b]{0.25\linewidth}
    \includegraphics[width=0.95\linewidth]{../images/output_pdf/exp_1d/gaussian/0.1_1e-06/width/RRCM/width_gaussian_0,1_1e-06_auto_RRCM.pdf}
  \end{subfigure}%
  \begin{subfigure}[b]{0.25\linewidth}
    \includegraphics[width=0.95\linewidth]{../images/output_pdf/exp_1d/gaussian/0.1_0.1/width/RRCM/width_gaussian_0,1_0,1_auto_RRCM.pdf}
  \end{subfigure}\\
  \begin{subfigure}[b]{0.25\linewidth}
    \includegraphics[width=0.95\linewidth]{../images/output_pdf/exp_1d/gaussian/0.1_1e-06/width/RRCM-loo/width_gaussian_0,1_1e-06_100_RRCM-loo.pdf}
  \end{subfigure}%
  \begin{subfigure}[b]{0.25\linewidth}
    \includegraphics[width=0.95\linewidth]{../images/output_pdf/exp_1d/gaussian/0.1_0.1/width/RRCM-loo/width_gaussian_0,1_0,1_100_RRCM-loo.pdf}
  \end{subfigure}%
  \begin{subfigure}[b]{0.25\linewidth}
    \includegraphics[width=0.95\linewidth]{../images/output_pdf/exp_1d/gaussian/0.1_1e-06/width/RRCM-loo/width_gaussian_0,1_1e-06_auto_RRCM-loo.pdf}
  \end{subfigure}%
  \begin{subfigure}[b]{0.25\linewidth}
    \includegraphics[width=0.95\linewidth]{../images/output_pdf/exp_1d/gaussian/0.1_0.1/width/RRCM-loo/width_gaussian_0,1_0,1_auto_RRCM-loo.pdf}
  \end{subfigure}\\
  \begin{subfigure}[b]{0.25\linewidth}
    \includegraphics[width=0.95\linewidth]{../images/output_pdf/exp_1d/gaussian/0.1_1e-06/width/CRR/width_gaussian_0,1_1e-06_100_CRR.pdf}
  \end{subfigure}%
  \begin{subfigure}[b]{0.25\linewidth}
    \includegraphics[width=0.95\linewidth]{../images/output_pdf/exp_1d/gaussian/0.1_0.1/width/CRR/width_gaussian_0,1_0,1_100_CRR.pdf}
  \end{subfigure}%
  \begin{subfigure}[b]{0.25\linewidth}
    \includegraphics[width=0.95\linewidth]{../images/output_pdf/exp_1d/gaussian/0.1_1e-06/width/CRR/width_gaussian_0,1_1e-06_auto_CRR.pdf}
  \end{subfigure}%
  \begin{subfigure}[b]{0.25\linewidth}
    \includegraphics[width=0.95\linewidth]{../images/output_pdf/exp_1d/gaussian/0.1_0.1/width/CRR/width_gaussian_0,1_0,1_auto_CRR.pdf}
  \end{subfigure}\\
  \begin{subfigure}[b]{0.25\linewidth}
    \includegraphics[width=0.95\linewidth]{../images/output_pdf/exp_1d/gaussian/0.1_1e-06/width/CRR-loo/width_gaussian_0,1_1e-06_100_CRR-loo.pdf}
    \caption{} \label{fig:gaussian_1d_high_noise_width_c1}
  \end{subfigure}%
  \begin{subfigure}[b]{0.25\linewidth}
    \includegraphics[width=0.95\linewidth]{../images/output_pdf/exp_1d/gaussian/0.1_0.1/width/CRR-loo/width_gaussian_0,1_0,1_100_CRR-loo.pdf}
    \caption{} \label{fig:gaussian_1d_high_noise_width_c2}
  \end{subfigure}%
  \begin{subfigure}[b]{0.25\linewidth}
    \includegraphics[width=0.95\linewidth]{../images/output_pdf/exp_1d/gaussian/0.1_1e-06/width/CRR-loo/width_gaussian_0,1_1e-06_auto_CRR-loo.pdf}
    \caption{} \label{fig:gaussian_1d_high_noise_width_c3}
  \end{subfigure}%
  \begin{subfigure}[b]{0.25\linewidth}
    \includegraphics[width=0.95\linewidth]{../images/output_pdf/exp_1d/gaussian/0.1_0.1/width/CRR-loo/width_gaussian_0,1_0,1_auto_CRR-loo.pdf}
    \caption{} \label{fig:gaussian_1d_high_noise_width_c4}
  \end{subfigure}
  \caption{Confidence region width as a function of the train sample size $n$ for
  different $(\theta, \lambda)$ (see fig.~\ref{fig:gaussian_1d_high_noise}).}
  \label{fig:gaussian_1d_high_noise_width}
\end{figure}

\begin{figure}%[b, width=0.5\textwidth]
  \centering
  \begin{subfigure}[b]{0.25\linewidth}
    \includegraphics[width=0.95\linewidth]{../images/output_pdf/exp_1d/gaussian/0.1_1e-06/coverage/GPR-f/coverage_gaussian_0,1_1e-06_10_GPR-f.pdf}
  \end{subfigure}%
  \begin{subfigure}[b]{0.25\linewidth}
    \includegraphics[width=0.95\linewidth]{../images/output_pdf/exp_1d/gaussian/0.1_0.1/coverage/GPR-f/coverage_gaussian_0,1_0,1_10_GPR-f.pdf}
  \end{subfigure}%
  \begin{subfigure}[b]{0.25\linewidth}
    \includegraphics[width=0.95\linewidth]{../images/output_pdf/exp_1d/gaussian/0.1_1e-06/coverage/GPR-f/coverage_gaussian_0,1_1e-06_1000_GPR-f.pdf}
  \end{subfigure}%
  \begin{subfigure}[b]{0.25\linewidth}
    \includegraphics[width=0.95\linewidth]{../images/output_pdf/exp_1d/gaussian/0.1_0.1/coverage/GPR-f/coverage_gaussian_0,1_0,1_1000_GPR-f.pdf}
  \end{subfigure}\\
  \begin{subfigure}[b]{0.25\linewidth}
    \includegraphics[width=0.95\linewidth]{../images/output_pdf/exp_1d/gaussian/0.1_1e-06/coverage/RRCM/coverage_gaussian_0,1_1e-06_10_RRCM.pdf}
  \end{subfigure}%
  \begin{subfigure}[b]{0.25\linewidth}
    \includegraphics[width=0.95\linewidth]{../images/output_pdf/exp_1d/gaussian/0.1_0.1/coverage/RRCM/coverage_gaussian_0,1_0,1_10_RRCM.pdf}
  \end{subfigure}%
  \begin{subfigure}[b]{0.25\linewidth}
    \includegraphics[width=0.95\linewidth]{../images/output_pdf/exp_1d/gaussian/0.1_1e-06/coverage/RRCM/coverage_gaussian_0,1_1e-06_1000_RRCM.pdf}
  \end{subfigure}%
  \begin{subfigure}[b]{0.25\linewidth}
    \includegraphics[width=0.95\linewidth]{../images/output_pdf/exp_1d/gaussian/0.1_0.1/coverage/RRCM/coverage_gaussian_0,1_0,1_1000_RRCM.pdf}
  \end{subfigure}\\
  \begin{subfigure}[b]{0.25\linewidth}
    \includegraphics[width=0.95\linewidth]{../images/output_pdf/exp_1d/gaussian/0.1_1e-06/coverage/RRCM-loo/coverage_gaussian_0,1_1e-06_10_RRCM-loo.pdf}
  \end{subfigure}%
  \begin{subfigure}[b]{0.25\linewidth}
    \includegraphics[width=0.95\linewidth]{../images/output_pdf/exp_1d/gaussian/0.1_0.1/coverage/RRCM-loo/coverage_gaussian_0,1_0,1_10_RRCM-loo.pdf}
  \end{subfigure}%
  \begin{subfigure}[b]{0.25\linewidth}
    \includegraphics[width=0.95\linewidth]{../images/output_pdf/exp_1d/gaussian/0.1_1e-06/coverage/RRCM-loo/coverage_gaussian_0,1_1e-06_1000_RRCM-loo.pdf}
  \end{subfigure}%
  \begin{subfigure}[b]{0.25\linewidth}
    \includegraphics[width=0.95\linewidth]{../images/output_pdf/exp_1d/gaussian/0.1_0.1/coverage/RRCM-loo/coverage_gaussian_0,1_0,1_1000_RRCM-loo.pdf}
  \end{subfigure}\\
  \begin{subfigure}[b]{0.25\linewidth}
    \includegraphics[width=0.95\linewidth]{../images/output_pdf/exp_1d/gaussian/0.1_1e-06/coverage/CRR/coverage_gaussian_0,1_1e-06_10_CRR.pdf}
  \end{subfigure}%
  \begin{subfigure}[b]{0.25\linewidth}
    \includegraphics[width=0.95\linewidth]{../images/output_pdf/exp_1d/gaussian/0.1_0.1/coverage/CRR/coverage_gaussian_0,1_0,1_10_CRR.pdf}
  \end{subfigure}%
  \begin{subfigure}[b]{0.25\linewidth}
    \includegraphics[width=0.95\linewidth]{../images/output_pdf/exp_1d/gaussian/0.1_1e-06/coverage/CRR/coverage_gaussian_0,1_1e-06_1000_CRR.pdf}
  \end{subfigure}%
  \begin{subfigure}[b]{0.25\linewidth}
    \includegraphics[width=0.95\linewidth]{../images/output_pdf/exp_1d/gaussian/0.1_0.1/coverage/CRR/coverage_gaussian_0,1_0,1_1000_CRR.pdf}
  \end{subfigure}\\
  \begin{subfigure}[b]{0.25\linewidth}
    \includegraphics[width=0.95\linewidth]{../images/output_pdf/exp_1d/gaussian/0.1_1e-06/coverage/CRR-loo/coverage_gaussian_0,1_1e-06_10_CRR-loo.pdf}
  \end{subfigure}%
  \begin{subfigure}[b]{0.25\linewidth}
    \includegraphics[width=0.95\linewidth]{../images/output_pdf/exp_1d/gaussian/0.1_0.1/coverage/CRR-loo/coverage_gaussian_0,1_0,1_10_CRR-loo.pdf}
  \end{subfigure}%
  \begin{subfigure}[b]{0.25\linewidth}
    \includegraphics[width=0.95\linewidth]{../images/output_pdf/exp_1d/gaussian/0.1_1e-06/coverage/CRR-loo/coverage_gaussian_0,1_1e-06_1000_CRR-loo.pdf}
  \end{subfigure}%
  \begin{subfigure}[b]{0.25\linewidth}
    \includegraphics[width=0.95\linewidth]{../images/output_pdf/exp_1d/gaussian/0.1_0.1/coverage/CRR-loo/coverage_gaussian_0,1_0,1_1000_CRR-loo.pdf}
  \end{subfigure}\\
  \begin{subfigure}[b]{0.25\linewidth}
    \includegraphics[width=0.95\linewidth]{../images/output_pdf/exp_1d/gaussian/0.1_1e-06/gaussian_0,1_1e-06_10_ratio.pdf}
    \caption{} \label{fig:gaussian_1d_high_noise_arb_c1}
  \end{subfigure}%
  \begin{subfigure}[b]{0.25\linewidth}
    \includegraphics[width=0.95\linewidth]{../images/output_pdf/exp_1d/gaussian/0.1_0.1/gaussian_0,1_0,1_10_ratio.pdf}
    \caption{} \label{fig:gaussian_1d_high_noise_arb_c2}
  \end{subfigure}%
  \begin{subfigure}[b]{0.25\linewidth}
    \includegraphics[width=0.95\linewidth]{../images/output_pdf/exp_1d/gaussian/0.1_1e-06/gaussian_0,1_1e-06_1000_ratio.pdf}
    \caption{} \label{fig:gaussian_1d_high_noise_arb_c3}
  \end{subfigure}%
  \begin{subfigure}[b]{0.25\linewidth}
    \includegraphics[width=0.95\linewidth]{../images/output_pdf/exp_1d/gaussian/0.1_0.1/gaussian_0,1_0,1_1000_ratio.pdf}
    \caption{} \label{fig:gaussian_1d_high_noise_arb_c4}
  \end{subfigure}%
  \caption{Coverage rate dynamics in the noisy fully Gaussian case $\gamma=10^{-1}$
  for different $(\theta, \lambda)$ (see fig.~\ref{fig:gaussian_1d_high_noise}).}
  \label{fig:gaussian_1d_high_noise_arb}
\end{figure}

\begin{figure}%[b, width=0.5\textwidth]
  \centering
  \begin{subfigure}[b]{0.25\linewidth}
    \includegraphics[width=0.95\linewidth]{../images/output_pdf/exp_1d/gaussian/0.1_1e-06/width/GPR-f/width_gaussian_0,1_1e-06_10_GPR-f.pdf}
  \end{subfigure}%
  \begin{subfigure}[b]{0.25\linewidth}
    \includegraphics[width=0.95\linewidth]{../images/output_pdf/exp_1d/gaussian/0.1_0.1/width/GPR-f/width_gaussian_0,1_0,1_10_GPR-f.pdf}
  \end{subfigure}%
  \begin{subfigure}[b]{0.25\linewidth}
    \includegraphics[width=0.95\linewidth]{../images/output_pdf/exp_1d/gaussian/0.1_1e-06/width/GPR-f/width_gaussian_0,1_1e-06_1000_GPR-f.pdf}
  \end{subfigure}%
  \begin{subfigure}[b]{0.25\linewidth}
    \includegraphics[width=0.95\linewidth]{../images/output_pdf/exp_1d/gaussian/0.1_0.1/width/GPR-f/width_gaussian_0,1_0,1_1000_GPR-f.pdf}
  \end{subfigure}\\
  \begin{subfigure}[b]{0.25\linewidth}
    \includegraphics[width=0.95\linewidth]{../images/output_pdf/exp_1d/gaussian/0.1_1e-06/width/RRCM/width_gaussian_0,1_1e-06_10_RRCM.pdf}
  \end{subfigure}%
  \begin{subfigure}[b]{0.25\linewidth}
    \includegraphics[width=0.95\linewidth]{../images/output_pdf/exp_1d/gaussian/0.1_0.1/width/RRCM/width_gaussian_0,1_0,1_10_RRCM.pdf}
  \end{subfigure}%
  \begin{subfigure}[b]{0.25\linewidth}
    \includegraphics[width=0.95\linewidth]{../images/output_pdf/exp_1d/gaussian/0.1_1e-06/width/RRCM/width_gaussian_0,1_1e-06_1000_RRCM.pdf}
  \end{subfigure}%
  \begin{subfigure}[b]{0.25\linewidth}
    \includegraphics[width=0.95\linewidth]{../images/output_pdf/exp_1d/gaussian/0.1_0.1/width/RRCM/width_gaussian_0,1_0,1_1000_RRCM.pdf}
  \end{subfigure}\\
  \begin{subfigure}[b]{0.25\linewidth}
    \includegraphics[width=0.95\linewidth]{../images/output_pdf/exp_1d/gaussian/0.1_1e-06/width/RRCM-loo/width_gaussian_0,1_1e-06_10_RRCM-loo.pdf}
  \end{subfigure}%
  \begin{subfigure}[b]{0.25\linewidth}
    \includegraphics[width=0.95\linewidth]{../images/output_pdf/exp_1d/gaussian/0.1_0.1/width/RRCM-loo/width_gaussian_0,1_0,1_10_RRCM-loo.pdf}
  \end{subfigure}%
  \begin{subfigure}[b]{0.25\linewidth}
    \includegraphics[width=0.95\linewidth]{../images/output_pdf/exp_1d/gaussian/0.1_1e-06/width/RRCM-loo/width_gaussian_0,1_1e-06_1000_RRCM-loo.pdf}
  \end{subfigure}%
  \begin{subfigure}[b]{0.25\linewidth}
    \includegraphics[width=0.95\linewidth]{../images/output_pdf/exp_1d/gaussian/0.1_0.1/width/RRCM-loo/width_gaussian_0,1_0,1_1000_RRCM-loo.pdf}
  \end{subfigure}\\
  \begin{subfigure}[b]{0.25\linewidth}
    \includegraphics[width=0.95\linewidth]{../images/output_pdf/exp_1d/gaussian/0.1_1e-06/width/CRR/width_gaussian_0,1_1e-06_10_CRR.pdf}
  \end{subfigure}%
  \begin{subfigure}[b]{0.25\linewidth}
    \includegraphics[width=0.95\linewidth]{../images/output_pdf/exp_1d/gaussian/0.1_0.1/width/CRR/width_gaussian_0,1_0,1_10_CRR.pdf}
  \end{subfigure}%
  \begin{subfigure}[b]{0.25\linewidth}
    \includegraphics[width=0.95\linewidth]{../images/output_pdf/exp_1d/gaussian/0.1_1e-06/width/CRR/width_gaussian_0,1_1e-06_1000_CRR.pdf}
  \end{subfigure}%
  \begin{subfigure}[b]{0.25\linewidth}
    \includegraphics[width=0.95\linewidth]{../images/output_pdf/exp_1d/gaussian/0.1_0.1/width/CRR/width_gaussian_0,1_0,1_1000_CRR.pdf}
  \end{subfigure}\\
  \begin{subfigure}[b]{0.25\linewidth}
    \includegraphics[width=0.95\linewidth]{../images/output_pdf/exp_1d/gaussian/0.1_1e-06/width/CRR-loo/width_gaussian_0,1_1e-06_10_CRR-loo.pdf}
    \caption{} \label{fig:gaussian_1d_high_noise_arb_width_c1}
  \end{subfigure}%
  \begin{subfigure}[b]{0.25\linewidth}
    \includegraphics[width=0.95\linewidth]{../images/output_pdf/exp_1d/gaussian/0.1_0.1/width/CRR-loo/width_gaussian_0,1_0,1_10_CRR-loo.pdf}
    \caption{} \label{fig:gaussian_1d_high_noise_arb_width_c2}
  \end{subfigure}%
  \begin{subfigure}[b]{0.25\linewidth}
    \includegraphics[width=0.95\linewidth]{../images/output_pdf/exp_1d/gaussian/0.1_1e-06/width/CRR-loo/width_gaussian_0,1_1e-06_1000_CRR-loo.pdf}
    \caption{} \label{fig:gaussian_1d_high_noise_arb_width_c3}
  \end{subfigure}%
  \begin{subfigure}[b]{0.25\linewidth}
    \includegraphics[width=0.95\linewidth]{../images/output_pdf/exp_1d/gaussian/0.1_0.1/width/CRR-loo/width_gaussian_0,1_0,1_1000_CRR-loo.pdf}
    \caption{} \label{fig:gaussian_1d_high_noise_arb_width_c4}
  \end{subfigure}
  \caption{Confidence region width as a function of the train sample size $n$ for
  different $(\theta, \lambda)$ (see fig.~\ref{fig:gaussian_1d_high_noise_arb}).}
  \label{fig:gaussian_1d_high_noise_arb_width}
\end{figure}

In the non-Gaussian setting with noise-to-signal ratio $\gamma=10^{-1}$, all experiments
yielded results similar to the non-Gaussian setting with negligible noise: all conformal
procedures produced asymptotically valid confidence sets. As for the Gaussian Process
Regression confidence intervals, the results were similar but were less extreme.
In the case of the ``Heaviside'' step test function, despite the fact that KRR achieved
adequate fit to the test data on average, GPR confidence intervals do not provide
conservative validity on all significance levels consistently for kernel precision
parameters $\theta \leq 10^3$, with the ML estimate being $\hat{\theta}\approx 2000$
for $n=1600$ (fig.~\ref{fig:heaviside_1d_high_noise} and~\ref{fig:heaviside_1d_high_noise_arb}).
This is due to the fact that the test function is discontinuous, and of relative
low precision setting, the induced basis functions are more spread out, which results
in worse fit in the vicinity of the discontinuity.

In contrast, GPR confidence intervals exhibit more adequate validity in the case
of ``f6'' test function: which is a pair of polynomials continuously pasted at $0.5$.
That is why the performance of the GPR confidence intervals in noticeably better
(fig.~\ref{fig:f6_1d_high_noise} and.~\ref{fig:f6_1d_high_noise_arb}). The widths
of the Bayesian and conformal intervals converge as the train sample grows larger
(fig.~\ref{fig:f6_1d_high_noise_width}).

The conducted experiments in the $1$-d setting yielded empirical support for the
following conclusions:
\begin{itemize}
  \item the kernel ridge regression conformal procedure seems to be asymptotically
  conservatively valid in all setups, despite being applied in the off-line learning
  setting;
  \item  the confidence intervals of the GPR fails in the non-Gaussian settings,
  especially with low noise-to-signal ratio;
  \item in those experiments, where the GPR intervals were approximately valid, the
  size of conformal confidence regions converged to the width of the Bayesian confidence
  set.
\end{itemize}

%%
\begin{figure}%[b, width=0.5\textwidth]
  \centering
  \begin{subfigure}[b]{0.25\linewidth}
    \includegraphics[width=0.95\linewidth]{../images/output_pdf/exp_1d/heaviside/0.1_1e-06/coverage/GPR-f/coverage_heaviside_0,1_1e-06_100_GPR-f.pdf}
  \end{subfigure}%
  \begin{subfigure}[b]{0.25\linewidth}
    \includegraphics[width=0.95\linewidth]{../images/output_pdf/exp_1d/heaviside/0.1_0.1/coverage/GPR-f/coverage_heaviside_0,1_0,1_100_GPR-f.pdf}
  \end{subfigure}%
  \begin{subfigure}[b]{0.25\linewidth}
    \includegraphics[width=0.95\linewidth]{../images/output_pdf/exp_1d/heaviside/0.1_1e-06/coverage/GPR-f/coverage_heaviside_0,1_1e-06_auto_GPR-f.pdf}
  \end{subfigure}%
  \begin{subfigure}[b]{0.25\linewidth}
    \includegraphics[width=0.95\linewidth]{../images/output_pdf/exp_1d/heaviside/0.1_0.1/coverage/GPR-f/coverage_heaviside_0,1_0,1_auto_GPR-f.pdf}
  \end{subfigure}\\
  \begin{subfigure}[b]{0.25\linewidth}
    \includegraphics[width=0.95\linewidth]{../images/output_pdf/exp_1d/heaviside/0.1_1e-06/coverage/RRCM/coverage_heaviside_0,1_1e-06_100_RRCM.pdf}
  \end{subfigure}%
  \begin{subfigure}[b]{0.25\linewidth}
    \includegraphics[width=0.95\linewidth]{../images/output_pdf/exp_1d/heaviside/0.1_0.1/coverage/RRCM/coverage_heaviside_0,1_0,1_100_RRCM.pdf}
  \end{subfigure}%
  \begin{subfigure}[b]{0.25\linewidth}
    \includegraphics[width=0.95\linewidth]{../images/output_pdf/exp_1d/heaviside/0.1_1e-06/coverage/RRCM/coverage_heaviside_0,1_1e-06_auto_RRCM.pdf}
  \end{subfigure}%
  \begin{subfigure}[b]{0.25\linewidth}
    \includegraphics[width=0.95\linewidth]{../images/output_pdf/exp_1d/heaviside/0.1_0.1/coverage/RRCM/coverage_heaviside_0,1_0,1_auto_RRCM.pdf}
  \end{subfigure}\\
  \begin{subfigure}[b]{0.25\linewidth}
    \includegraphics[width=0.95\linewidth]{../images/output_pdf/exp_1d/heaviside/0.1_1e-06/heaviside_0,1_1e-06_100_ratio.pdf}
    \caption{} \label{fig:heaviside_1d_high_noise_c1}
  \end{subfigure}%
  \begin{subfigure}[b]{0.25\linewidth}
    \includegraphics[width=0.95\linewidth]{../images/output_pdf/exp_1d/heaviside/0.1_0.1/heaviside_0,1_0,1_100_ratio.pdf}
    \caption{} \label{fig:heaviside_1d_high_noise_c2}
  \end{subfigure}%
  \begin{subfigure}[b]{0.25\linewidth}
    \includegraphics[width=0.95\linewidth]{../images/output_pdf/exp_1d/heaviside/0.1_1e-06/heaviside_0,1_1e-06_auto_ratio.pdf}
    \caption{} \label{fig:heaviside_1d_high_noise_c3}
  \end{subfigure}%
  \begin{subfigure}[b]{0.25\linewidth}
    \includegraphics[width=0.95\linewidth]{../images/output_pdf/exp_1d/heaviside/0.1_0.1/heaviside_0,1_0,1_auto_ratio.pdf}
    \caption{} \label{fig:heaviside_1d_high_noise_c4}
  \end{subfigure}%
  \caption{Coverage dynamics for the ``Heaviside'' step function.}
  \label{fig:heaviside_1d_high_noise}
\end{figure}

\begin{figure}%[b, width=0.5\textwidth]
  \centering
  \begin{subfigure}[b]{0.25\linewidth}
    \includegraphics[width=0.95\linewidth]{../images/output_pdf/exp_1d/heaviside/0.1_1e-06/coverage/GPR-f/coverage_heaviside_0,1_1e-06_10_GPR-f.pdf}
  \end{subfigure}%
  \begin{subfigure}[b]{0.25\linewidth}
    \includegraphics[width=0.95\linewidth]{../images/output_pdf/exp_1d/heaviside/0.1_0.1/coverage/GPR-f/coverage_heaviside_0,1_0,1_10_GPR-f.pdf}
  \end{subfigure}%
  \begin{subfigure}[b]{0.25\linewidth}
    \includegraphics[width=0.95\linewidth]{../images/output_pdf/exp_1d/heaviside/0.1_1e-06/coverage/GPR-f/coverage_heaviside_0,1_1e-06_1000_GPR-f.pdf}
  \end{subfigure}%
  \begin{subfigure}[b]{0.25\linewidth}
    \includegraphics[width=0.95\linewidth]{../images/output_pdf/exp_1d/heaviside/0.1_0.1/coverage/GPR-f/coverage_heaviside_0,1_0,1_1000_GPR-f.pdf}
  \end{subfigure}\\
  \begin{subfigure}[b]{0.25\linewidth}
    \includegraphics[width=0.95\linewidth]{../images/output_pdf/exp_1d/heaviside/0.1_1e-06/coverage/RRCM/coverage_heaviside_0,1_1e-06_10_RRCM.pdf}
  \end{subfigure}%
  \begin{subfigure}[b]{0.25\linewidth}
    \includegraphics[width=0.95\linewidth]{../images/output_pdf/exp_1d/heaviside/0.1_0.1/coverage/RRCM/coverage_heaviside_0,1_0,1_10_RRCM.pdf}
  \end{subfigure}%
  \begin{subfigure}[b]{0.25\linewidth}
    \includegraphics[width=0.95\linewidth]{../images/output_pdf/exp_1d/heaviside/0.1_1e-06/coverage/RRCM/coverage_heaviside_0,1_1e-06_1000_RRCM.pdf}
  \end{subfigure}%
  \begin{subfigure}[b]{0.25\linewidth}
    \includegraphics[width=0.95\linewidth]{../images/output_pdf/exp_1d/heaviside/0.1_0.1/coverage/RRCM/coverage_heaviside_0,1_0,1_1000_RRCM.pdf}
  \end{subfigure}\\
  \begin{subfigure}[b]{0.25\linewidth}
    \includegraphics[width=0.95\linewidth]{../images/output_pdf/exp_1d/heaviside/0.1_1e-06/heaviside_0,1_1e-06_10_ratio.pdf}
    \caption{} \label{fig:heaviside_1d_high_noise_arb_c1}
  \end{subfigure}%
  \begin{subfigure}[b]{0.25\linewidth}
    \includegraphics[width=0.95\linewidth]{../images/output_pdf/exp_1d/heaviside/0.1_0.1/heaviside_0,1_0,1_10_ratio.pdf}
    \caption{} \label{fig:heaviside_1d_high_noise_arb_c2}
  \end{subfigure}%
  \begin{subfigure}[b]{0.25\linewidth}
    \includegraphics[width=0.95\linewidth]{../images/output_pdf/exp_1d/heaviside/0.1_1e-06/heaviside_0,1_1e-06_1000_ratio.pdf}
    \caption{} \label{fig:heaviside_1d_high_noise_arb_c3}
  \end{subfigure}%
  \begin{subfigure}[b]{0.25\linewidth}
    \includegraphics[width=0.95\linewidth]{../images/output_pdf/exp_1d/heaviside/0.1_0.1/heaviside_0,1_0,1_1000_ratio.pdf}
    \caption{} \label{fig:heaviside_1d_high_noise_arb_c4}
  \end{subfigure}%
  \caption{Coverage dynamics for the ``Heaviside'' step function (c.f. fig.~\ref{fig:heaviside_1d_high_noise}).}
  \label{fig:heaviside_1d_high_noise_arb}
\end{figure}

% \begin{figure}%[b, width=0.5\textwidth]
%   \centering
%   \begin{subfigure}[b]{0.25\linewidth}
%     \includegraphics[width=0.95\linewidth]{../images/output_pdf/exp_1d/pressure2/0.1_1e-06/coverage/GPR-f/coverage_pressure2_0,1_1e-06_100_GPR-f.pdf}
%   \end{subfigure}%
%   \begin{subfigure}[b]{0.25\linewidth}
%     \includegraphics[width=0.95\linewidth]{../images/output_pdf/exp_1d/pressure2/0.1_0.1/coverage/GPR-f/coverage_pressure2_0,1_0,1_100_GPR-f.pdf}
%   \end{subfigure}%
%   \begin{subfigure}[b]{0.25\linewidth}
%     \includegraphics[width=0.95\linewidth]{../images/output_pdf/exp_1d/pressure2/0.1_1e-06/coverage/GPR-f/coverage_pressure2_0,1_1e-06_auto_GPR-f.pdf}
%   \end{subfigure}%
%   \begin{subfigure}[b]{0.25\linewidth}
%     \includegraphics[width=0.95\linewidth]{../images/output_pdf/exp_1d/pressure2/0.1_0.1/coverage/GPR-f/coverage_pressure2_0,1_0,1_auto_GPR-f.pdf}
%   \end{subfigure}\\
%   \begin{subfigure}[b]{0.25\linewidth}
%     \includegraphics[width=0.95\linewidth]{../images/output_pdf/exp_1d/pressure2/0.1_1e-06/coverage/RRCM/coverage_pressure2_0,1_1e-06_100_RRCM.pdf}
%   \end{subfigure}%
%   \begin{subfigure}[b]{0.25\linewidth}
%     \includegraphics[width=0.95\linewidth]{../images/output_pdf/exp_1d/pressure2/0.1_0.1/coverage/RRCM/coverage_pressure2_0,1_0,1_100_RRCM.pdf}
%   \end{subfigure}%
%   \begin{subfigure}[b]{0.25\linewidth}
%     \includegraphics[width=0.95\linewidth]{../images/output_pdf/exp_1d/pressure2/0.1_1e-06/coverage/RRCM/coverage_pressure2_0,1_1e-06_auto_RRCM.pdf}
%   \end{subfigure}%
%   \begin{subfigure}[b]{0.25\linewidth}
%     \includegraphics[width=0.95\linewidth]{../images/output_pdf/exp_1d/pressure2/0.1_0.1/coverage/RRCM/coverage_pressure2_0,1_0,1_auto_RRCM.pdf}
%   \end{subfigure}\\
%   \begin{subfigure}[b]{0.25\linewidth}
%     \includegraphics[width=0.95\linewidth]{../images/output_pdf/exp_1d/pressure2/0.1_1e-06/pressure2_0,1_1e-06_100_ratio.pdf}
%     \caption{} \label{fig:pressure2_1d_high_noise_c1}
%   \end{subfigure}%
%   \begin{subfigure}[b]{0.25\linewidth}
%     \includegraphics[width=0.95\linewidth]{../images/output_pdf/exp_1d/pressure2/0.1_0.1/pressure2_0,1_0,1_100_ratio.pdf}
%     \caption{} \label{fig:pressure2_1d_high_noise_c2}
%   \end{subfigure}%
%   \begin{subfigure}[b]{0.25\linewidth}
%     \includegraphics[width=0.95\linewidth]{../images/output_pdf/exp_1d/pressure2/0.1_1e-06/pressure2_0,1_1e-06_auto_ratio.pdf}
%     \caption{} \label{fig:pressure2_1d_high_noise_c3}
%   \end{subfigure}%
%   \begin{subfigure}[b]{0.25\linewidth}
%     \includegraphics[width=0.95\linewidth]{../images/output_pdf/exp_1d/pressure2/0.1_0.1/pressure2_0,1_0,1_auto_ratio.pdf}
%     \caption{} \label{fig:pressure2_1d_high_noise_c4}
%   \end{subfigure}%
%   \caption{Coverage dynamics for the ``pressure2'' function.}
%   \label{fig:pressure2_1d_high_noise}
% \end{figure}

% \begin{figure}%[b, width=0.5\textwidth]
%   \centering
%   \begin{subfigure}[b]{0.25\linewidth}
%     \includegraphics[width=0.95\linewidth]{../images/output_pdf/exp_1d/pressure2/0.1_1e-06/coverage/GPR-f/coverage_pressure2_0,1_1e-06_10_GPR-f.pdf}
%   \end{subfigure}%
%   \begin{subfigure}[b]{0.25\linewidth}
%     \includegraphics[width=0.95\linewidth]{../images/output_pdf/exp_1d/pressure2/0.1_0.1/coverage/GPR-f/coverage_pressure2_0,1_0,1_10_GPR-f.pdf}
%   \end{subfigure}%
%   \begin{subfigure}[b]{0.25\linewidth}
%     \includegraphics[width=0.95\linewidth]{../images/output_pdf/exp_1d/pressure2/0.1_1e-06/coverage/GPR-f/coverage_pressure2_0,1_1e-06_1000_GPR-f.pdf}
%   \end{subfigure}%
%   \begin{subfigure}[b]{0.25\linewidth}
%     \includegraphics[width=0.95\linewidth]{../images/output_pdf/exp_1d/pressure2/0.1_0.1/coverage/GPR-f/coverage_pressure2_0,1_0,1_1000_GPR-f.pdf}
%   \end{subfigure}\\
%   \begin{subfigure}[b]{0.25\linewidth}
%     \includegraphics[width=0.95\linewidth]{../images/output_pdf/exp_1d/pressure2/0.1_1e-06/coverage/RRCM/coverage_pressure2_0,1_1e-06_10_RRCM.pdf}
%   \end{subfigure}%
%   \begin{subfigure}[b]{0.25\linewidth}
%     \includegraphics[width=0.95\linewidth]{../images/output_pdf/exp_1d/pressure2/0.1_0.1/coverage/RRCM/coverage_pressure2_0,1_0,1_10_RRCM.pdf}
%   \end{subfigure}%
%   \begin{subfigure}[b]{0.25\linewidth}
%     \includegraphics[width=0.95\linewidth]{../images/output_pdf/exp_1d/pressure2/0.1_1e-06/coverage/RRCM/coverage_pressure2_0,1_1e-06_1000_RRCM.pdf}
%   \end{subfigure}%
%   \begin{subfigure}[b]{0.25\linewidth}
%     \includegraphics[width=0.95\linewidth]{../images/output_pdf/exp_1d/pressure2/0.1_0.1/coverage/RRCM/coverage_pressure2_0,1_0,1_1000_RRCM.pdf}
%   \end{subfigure}\\
%   \begin{subfigure}[b]{0.25\linewidth}
%     \includegraphics[width=0.95\linewidth]{../images/output_pdf/exp_1d/pressure2/0.1_1e-06/pressure2_0,1_1e-06_10_ratio.pdf}
%     \caption{} \label{fig:pressure2_1d_high_noise_arb_c1}
%   \end{subfigure}%
%   \begin{subfigure}[b]{0.25\linewidth}
%     \includegraphics[width=0.95\linewidth]{../images/output_pdf/exp_1d/pressure2/0.1_0.1/pressure2_0,1_0,1_10_ratio.pdf}
%     \caption{} \label{fig:pressure2_1d_high_noise_arb_c2}
%   \end{subfigure}%
%   \begin{subfigure}[b]{0.25\linewidth}
%     \includegraphics[width=0.95\linewidth]{../images/output_pdf/exp_1d/pressure2/0.1_1e-06/pressure2_0,1_1e-06_1000_ratio.pdf}
%     \caption{} \label{fig:pressure2_1d_high_noise_arb_c3}
%   \end{subfigure}%
%   \begin{subfigure}[b]{0.25\linewidth}
%     \includegraphics[width=0.95\linewidth]{../images/output_pdf/exp_1d/pressure2/0.1_0.1/pressure2_0,1_0,1_1000_ratio.pdf}
%     \caption{} \label{fig:pressure2_1d_high_noise_arb_c4}
%   \end{subfigure}%
%   \caption{Coverage dynamics for the ``pressure2'' function(c.f. fig.~\ref{fig:pressure2_1d_high_noise}).}
%   \label{fig:pressure2_1d_high_noise_arb}
% \end{figure}

\begin{figure}%[b, width=0.5\textwidth]
  \centering
  \begin{subfigure}[b]{0.25\linewidth}
    \includegraphics[width=0.95\linewidth]{../images/output_pdf/exp_1d/f6/0.1_1e-06/coverage/GPR-f/coverage_f6_0,1_1e-06_100_GPR-f.pdf}
  \end{subfigure}%
  \begin{subfigure}[b]{0.25\linewidth}
    \includegraphics[width=0.95\linewidth]{../images/output_pdf/exp_1d/f6/0.1_0.1/coverage/GPR-f/coverage_f6_0,1_0,1_100_GPR-f.pdf}
  \end{subfigure}%
  \begin{subfigure}[b]{0.25\linewidth}
    \includegraphics[width=0.95\linewidth]{../images/output_pdf/exp_1d/f6/0.1_1e-06/coverage/GPR-f/coverage_f6_0,1_1e-06_auto_GPR-f.pdf}
  \end{subfigure}%
  \begin{subfigure}[b]{0.25\linewidth}
    \includegraphics[width=0.95\linewidth]{../images/output_pdf/exp_1d/f6/0.1_0.1/coverage/GPR-f/coverage_f6_0,1_0,1_auto_GPR-f.pdf}
  \end{subfigure}\\
  \begin{subfigure}[b]{0.25\linewidth}
    \includegraphics[width=0.95\linewidth]{../images/output_pdf/exp_1d/f6/0.1_1e-06/coverage/RRCM/coverage_f6_0,1_1e-06_100_RRCM.pdf}
  \end{subfigure}%
  \begin{subfigure}[b]{0.25\linewidth}
    \includegraphics[width=0.95\linewidth]{../images/output_pdf/exp_1d/f6/0.1_0.1/coverage/RRCM/coverage_f6_0,1_0,1_100_RRCM.pdf}
  \end{subfigure}%
  \begin{subfigure}[b]{0.25\linewidth}
    \includegraphics[width=0.95\linewidth]{../images/output_pdf/exp_1d/f6/0.1_1e-06/coverage/RRCM/coverage_f6_0,1_1e-06_auto_RRCM.pdf}
  \end{subfigure}%
  \begin{subfigure}[b]{0.25\linewidth}
    \includegraphics[width=0.95\linewidth]{../images/output_pdf/exp_1d/f6/0.1_0.1/coverage/RRCM/coverage_f6_0,1_0,1_auto_RRCM.pdf}
  \end{subfigure}\\
  \begin{subfigure}[b]{0.25\linewidth}
    \includegraphics[width=0.95\linewidth]{../images/output_pdf/exp_1d/f6/0.1_1e-06/f6_0,1_1e-06_100_ratio.pdf}
    \caption{} \label{fig:f6_1d_high_noise_c1}
  \end{subfigure}%
  \begin{subfigure}[b]{0.25\linewidth}
    \includegraphics[width=0.95\linewidth]{../images/output_pdf/exp_1d/f6/0.1_0.1/f6_0,1_0,1_100_ratio.pdf}
    \caption{} \label{fig:f6_1d_high_noise_c2}
  \end{subfigure}%
  \begin{subfigure}[b]{0.25\linewidth}
    \includegraphics[width=0.95\linewidth]{../images/output_pdf/exp_1d/f6/0.1_1e-06/f6_0,1_1e-06_auto_ratio.pdf}
    \caption{} \label{fig:f6_1d_high_noise_c3}
  \end{subfigure}%
  \begin{subfigure}[b]{0.25\linewidth}
    \includegraphics[width=0.95\linewidth]{../images/output_pdf/exp_1d/f6/0.1_0.1/f6_0,1_0,1_auto_ratio.pdf}
    \caption{} \label{fig:f6_1d_high_noise_c4}
  \end{subfigure}%
  \caption{Coverage dynamics and ``RMSE-std'' ratio for the ``f6'' function.}
  \label{fig:f6_1d_high_noise}
\end{figure}

\begin{figure}%[b, width=0.5\textwidth]
  \centering
  \begin{subfigure}[b]{0.25\linewidth}
    \includegraphics[width=0.95\linewidth]{../images/output_pdf/exp_1d/f6/0.1_1e-06/coverage/GPR-f/coverage_f6_0,1_1e-06_10_GPR-f.pdf}
  \end{subfigure}%
  \begin{subfigure}[b]{0.25\linewidth}
    \includegraphics[width=0.95\linewidth]{../images/output_pdf/exp_1d/f6/0.1_0.1/coverage/GPR-f/coverage_f6_0,1_0,1_10_GPR-f.pdf}
  \end{subfigure}%
  \begin{subfigure}[b]{0.25\linewidth}
    \includegraphics[width=0.95\linewidth]{../images/output_pdf/exp_1d/f6/0.1_1e-06/coverage/GPR-f/coverage_f6_0,1_1e-06_1000_GPR-f.pdf}
  \end{subfigure}%
  \begin{subfigure}[b]{0.25\linewidth}
    \includegraphics[width=0.95\linewidth]{../images/output_pdf/exp_1d/f6/0.1_0.1/coverage/GPR-f/coverage_f6_0,1_0,1_1000_GPR-f.pdf}
  \end{subfigure}\\
  \begin{subfigure}[b]{0.25\linewidth}
    \includegraphics[width=0.95\linewidth]{../images/output_pdf/exp_1d/f6/0.1_1e-06/coverage/RRCM/coverage_f6_0,1_1e-06_10_RRCM.pdf}
  \end{subfigure}%
  \begin{subfigure}[b]{0.25\linewidth}
    \includegraphics[width=0.95\linewidth]{../images/output_pdf/exp_1d/f6/0.1_0.1/coverage/RRCM/coverage_f6_0,1_0,1_10_RRCM.pdf}
  \end{subfigure}%
  \begin{subfigure}[b]{0.25\linewidth}
    \includegraphics[width=0.95\linewidth]{../images/output_pdf/exp_1d/f6/0.1_1e-06/coverage/RRCM/coverage_f6_0,1_1e-06_1000_RRCM.pdf}
  \end{subfigure}%
  \begin{subfigure}[b]{0.25\linewidth}
    \includegraphics[width=0.95\linewidth]{../images/output_pdf/exp_1d/f6/0.1_0.1/coverage/RRCM/coverage_f6_0,1_0,1_1000_RRCM.pdf}
  \end{subfigure}\\
  \begin{subfigure}[b]{0.25\linewidth}
    \includegraphics[width=0.95\linewidth]{../images/output_pdf/exp_1d/f6/0.1_1e-06/f6_0,1_1e-06_10_ratio.pdf}
    \caption{} \label{fig:f6_1d_high_noise_arb_c1}
  \end{subfigure}%
  \begin{subfigure}[b]{0.25\linewidth}
    \includegraphics[width=0.95\linewidth]{../images/output_pdf/exp_1d/f6/0.1_0.1/f6_0,1_0,1_10_ratio.pdf}
    \caption{} \label{fig:f6_1d_high_noise_arb_c2}
  \end{subfigure}%
  \begin{subfigure}[b]{0.25\linewidth}
    \includegraphics[width=0.95\linewidth]{../images/output_pdf/exp_1d/f6/0.1_1e-06/f6_0,1_1e-06_1000_ratio.pdf}
    \caption{} \label{fig:f6_1d_high_noise_arb_c3}
  \end{subfigure}%
  \begin{subfigure}[b]{0.25\linewidth}
    \includegraphics[width=0.95\linewidth]{../images/output_pdf/exp_1d/f6/0.1_0.1/f6_0,1_0,1_1000_ratio.pdf}
    \caption{} \label{fig:f6_1d_high_noise_arb_c4}
  \end{subfigure}%
  \caption{Coverage dynamics and ``RMSE-std'' ratio for the ``f6'' function (see
  the description of fig.~\ref{fig:f6_1d_high_noise}).}
  \label{fig:f6_1d_high_noise_arb}
\end{figure}

\begin{figure}%[b, width=0.5\textwidth]
  \centering
  \begin{subfigure}[b]{0.25\linewidth}
    \includegraphics[width=0.95\linewidth]{../images/output_pdf/exp_1d/f6/0.1_1e-06/width/GPR-f/width_f6_0,1_1e-06_auto_GPR-f.pdf}
  \end{subfigure}%
  \begin{subfigure}[b]{0.25\linewidth}
    \includegraphics[width=0.95\linewidth]{../images/output_pdf/exp_1d/f6/0.1_1e-06/width/RRCM/width_f6_0,1_1e-06_auto_RRCM.pdf}
  \end{subfigure}\\
  \begin{subfigure}[b]{0.25\linewidth}
    \includegraphics[width=0.95\linewidth]{../images/output_pdf/exp_1d/f6/0.1_0.1/width/GPR-f/width_f6_0,1_0,1_auto_GPR-f.pdf}
    \caption{} \label{fig:f6_1d_high_noise_width_c1}
  \end{subfigure}%
  \begin{subfigure}[b]{0.25\linewidth}
    \includegraphics[width=0.95\linewidth]{../images/output_pdf/exp_1d/f6/0.1_0.1/width/RRCM/width_f6_0,1_0,1_auto_RRCM.pdf}
    \caption{} \label{fig:f6_1d_high_noise_width_c2}
  \end{subfigure}
  \caption{Width dynamics of the KRR and GPR confidence sets for the ML estimate
  of $\theta$ for the values of the ``f6'' test function, distorted by Gaussian noise
  with $\gamma=10^{-1}$.}
  \label{fig:f6_1d_high_noise_width}
\end{figure}

% subsection results_1_d (end)

\subsection{Results: $2$-d} % (fold)
\label{sub:results_2_d}

In this section we study the performance in the $2$-d setting. We begin by testing
the conformal procedure in a Gaussian process setting, where the Bayesian GPR intervals
are provably correct: we compare the validity and assess the efficiency. We test
the effects of using ML estimate of the kernel precision parameter $\theta$ and
fixing it to deliberately large or small values. Finally, we use specific test functions,
fig.~\ref{fig:nongauss_2d_profile}, observations of which are contaminated by moderate
Gaussian noise.

\begin{figure}%[t, width=0.5\textwidth]
  \centering
  \begin{subfigure}[b]{0.5\linewidth}
    \includegraphics[width=\linewidth]{../images/output_pdf/exp_2d/f2/0.1_1e-06/1500/profile/f2_0,1_1e-06_1000_test.pdf}
  \end{subfigure}%~
  \begin{subfigure}[b]{0.5\linewidth}
    \includegraphics[width=\linewidth]{../images/output_pdf/exp_2d/f5/0.1_1e-06/1500/profile/f5_0,1_1e-06_1000_test.pdf}
  \end{subfigure}
  \caption{A sample path of a $2$-d non-Gaussian function, contaminated by moderate
  independent $\Ncal(0, \gamma)$ noise. Function: \textit{left:} ``f2''; \textit{right:}
  ``f5''.}
  \label{fig:nongauss_2d_profile}
\end{figure}

For the first part we generated realizations of a $GP(0, K(x,x'))$ process on $\Xcal=[0,1]^2$
with kernel defined in eq.~\ref{eq:gauss_kenrel} and precision $\theta_0 = 10^2$.
The typical sample path is depicted in fig.~\ref{fig:gauss_2d_profile}. As in the
$1$-d experiments we synthesized one sample path of $GP$ and then for each set of
hyper-parameters, excluding $\gamma$, we drew $L=25$ random samples from $\Xcal$
to average the effects of random sampling on the KRR and the derived confidence regions.
The goodness-of-fit ratios are presented in table~\ref{tab:gaussian_2d_rmse_std}.
For the true precision $\theta=\theta_0$ and small sample $n=150$ we see overfitting
in the case of high noise in the observations ($\gamma$) and low regularization
parameter ($\lambda$).

\begin{figure}%[t, width=0.5\textwidth]
  \centering
  \begin{subfigure}[b]{0.5\linewidth}
    \includegraphics[width=\linewidth]{../images/output_pdf/exp_2d/gaussian/1e-06_1e-06/1500/profile/gaussian_1e-06_1e-06_1000_test.pdf}
  \end{subfigure}%~
  \begin{subfigure}[b]{0.5\linewidth}
    \includegraphics[width=\linewidth]{../images/output_pdf/exp_2d/gaussian/0.1_1e-06/1500/profile/gaussian_0,1_1e-06_1000_test.pdf}
  \end{subfigure}
  \caption{A sample path of a $2$-d Gaussian process with kernel $\gamma \delta_{x,x'} + K(x,x')$:
  \textit{left:} $\gamma=10^{-6}$; \textit{right:} $\gamma=10^{-1}$.}
  \label{fig:gauss_2d_profile}
\end{figure}

Table~\ref{tab:gaussian_2d_cov_gpr} shows the error rates ($y^*\notin\Bcal(x^*)$) of
the GPR confidence intervals on a fixed test sample, the inputs $X^*\subseteq \Xcal$
of which are given by
\begin{equation*}
  X^* = \bigl\{(N^{-1} i, N^{-1} j)\,:\,i,j=0\ldots, N\bigr\} \,,
\end{equation*}
for $N=50$. Columns 1 and 4 show that the regions are approximately valid in the
cases when the KRR parameters are set to the true values used to generate the data
and with larger train sample the coverage improves. As in the $1$-d case the intervals
are more conservative for the case of low noise in the data $\gamma=10^{-6}$ and
high regularization $\lambda=10^{-1}$. Table~\ref{tab:gaussian_2d_cov_gpr_neq} show
that in current setting, the validity of the GPR confidence intervals is sensitive
to misspecification of the kernel precision.

\begin{table}
  \centering
  \caption{Goodness of fit of the KRR of simulated $2$-d Gaussian process.}
  \label{tab:gaussian_2d_rmse_std}
  \begin{tabular}{ll||rr|rr}
  \toprule
       & $\gamma$ & $10^{-6}$ &          & $10^{-1}$ &          \\\cline{2-2}
       & $\lambda$ & $10^{-6}$ & $10^{-1}$ & $10^{-6}$ & $10^{-1}$ \\\cline{2-2}
  $n$ & $\theta$ &          &          &          &          \\
  \midrule
  $150$  & $10^2$ &    \textbf{0.404} &    0.465 &    1.874 &    \textbf{0.503} \\
       & $\hat{\theta}_\text{ML}$ &    0.404 &    0.469 &    0.707 &    0.499 \\\cline{2-6}
       & $10^1$ &    1.606 &    0.795 &    2.010 &    0.774 \\
       & $10^3$ &    0.783 &    0.795 &    0.799 &    0.798 \\%\cline{2-6}
  \midrule
  $1500$ & $10^2$ &    \textbf{0.004} &    0.081 &    0.677 &    \textbf{0.291} \\
       & $\hat{\theta}_\text{ML}$ &    0.005 &    0.056 &    0.706 &    0.291 \\\cline{2-6}
       & $10^1$ &    0.342 &    0.640 &    0.373 &    0.606 \\
       & $10^3$ &    0.144 &    0.188 &    2.239 &    0.377 \\%\cline{2-6}
  \bottomrule
  \end{tabular}
\end{table}
\begin{table}
\centering
  \caption{The empirical error rate ($\%$) of the GPR confidence interval for simulated
  $2$-d Gaussian process.}
  \label{tab:gaussian_2d_cov_gpr}
  \begin{tabular}{lll||rr|rr}
  \toprule
       &      & $\gamma$ & $10^{-6}$ &          & $10^{-1}$ &          \\\cline{3-3}
       &      & $\lambda$ & $10^{-6}$ & $10^{-1}$ & $10^{-6}$ & $10^{-1}$ \\\cline{3-3}
  $n$ & $\theta$ & $\alpha(\%)$ &          &          &          &          \\
  \midrule
  $150$  & $10^2$ & $1$ &     1.3 &     0.7 &    14.6 &     0.8 \\
       &      & $5$ &     6.7 &     3.2 &    20.1 &     4.2 \\
       &      & $10$ &    12.8 &     6.1 &    24.3 &     8.9 \\
       &      & $25$ &    28.9 &    15.9 &    33.5 &    23.3 \\\cline{2-7}
       & $\hat{\theta}_\text{ML}$ & $1$ &     0.8 &     0.7 &     3.2 &     0.7 \\
       &      & $5$ &     4.8 &     3.1 &     6.5 &     4.0 \\
       &      & $10$ &    10.0 &     6.3 &    10.2 &     8.6 \\
       &      & $25$ &    25.1 &    16.5 &    22.2 &    23.2 \\
  \midrule
  $1500$ & $10^2$ & $1$ &     0.9 &     0.1 &     4.1 &     0.7 \\
       &      & $5$ &     4.5 &     0.5 &    12.0 &     4.4 \\
       &      & $10$ &     9.2 &     1.0 &    19.2 &     9.3 \\
       &      & $25$ &    23.8 &     3.3 &    36.1 &    24.5 \\\cline{2-7}
       & $\hat{\theta}_\text{ML}$ & $1$ &     0.8 &     0.1 &     2.3 &     0.8 \\
       &      & $5$ &     4.4 &     0.5 &     4.9 &     4.5 \\
       &      & $10$ &     9.0 &     0.9 &     8.0 &     9.4 \\
       &      & $25$ &    23.6 &     2.3 &    18.9 &    24.7 \\
  \bottomrule
  \end{tabular}
\end{table}
\begin{table}
  \centering
  \caption{The empirical error rate ($\%$) of the GPR confidence interval for simulated
  $2$-d Gaussian process for misspecified $\theta$.}
  \label{tab:gaussian_2d_cov_gpr_neq}
  \begin{tabular}{lll||rr|rr}
  \toprule
       &      & $\gamma$ & $10^{-6}$ &          & $10^{-1}$ &          \\\cline{3-3}
       &      & $\lambda$ & $10^{-6}$ & $10^{-1}$ & $10^{-6}$ & $10^{-1}$ \\\cline{3-3}
  $n$ & $\theta$ & $\alpha(\%)$ &          &          &          &          \\
  \midrule
  $150$  & $10^1$ & $1$ &      7.2 &      1.9 &      6.6 &      1.7 \\
       &      & $5$ &     15.1 &      6.2 &     15.6 &      6.2 \\
       &      & $10$ &     21.6 &     10.7 &     23.1 &     11.6 \\
       &      & $25$ &     36.7 &     24.3 &     39.8 &     26.2 \\\cline{2-7}
       & $10^3$ & $1$ &      1.4 &      1.4 &      1.5 &      0.7 \\
       &      & $5$ &      5.4 &      5.3 &      4.5 &      3.7 \\
       &      & $10$ &      8.9 &      8.8 &      8.1 &      7.4 \\
       &      & $25$ &     18.6 &     18.6 &     20.8 &     20.6 \\
  \midrule
  $1500$ & $10^1$ & $1$ &      1.6 &      0.6 &      1.0 &      0.6 \\
       &      & $5$ &      5.3 &      4.2 &      5.2 &      3.7 \\
       &      & $10$ &      9.6 &      8.6 &     10.4 &      8.2 \\
       &      & $25$ &     22.5 &     23.4 &     26.3 &     22.6 \\\cline{2-7}
       & $10^3$ & $1$ &      0.4 &      0.4 &      9.1 &      1.1 \\
       &      & $5$ &      1.2 &      1.2 &     13.1 &      4.9 \\
       &      & $10$ &      2.0 &      2.0 &     16.1 &      9.6 \\
       &      & $25$ &      4.7 &      4.5 &     24.0 &     24.0 \\
  \bottomrule
  \end{tabular}
\end{table}

In contrast to the GPR confidence intervals, the conformal region show remarkable
insensitivity to both the goodness of fit and the parameter misspecification
as demonstrated in table.~\ref{tab:gaussian_2d_cov_conf}, where we show the maximal
absolute deviation of the interval error rate from the stated rate $\alpha$
across all studied significance levels (eq.~\ref{eq:mad_alpha}).
\begin{equation} \label{eq:mad_alpha}
  \mathtt{MAD}(\Gamma, A; \Theta)
  = \max_{\alpha\in A}\Bigl\lvert
    m^{-1}\#\{j\,:\, y^*_j\notin \Gamma^\alpha_n(X^*_j; \Theta)\} - \alpha
  \Bigr\rvert
    \,,
\end{equation}
where $\Theta$ is the vector of hyper-parameters of the experiment (excluding $\gamma$),
$A = \{1\%, 5\%, 10\%, 25\%\}$, $(X^*_j, y^*_j)_{j=1}^{|X^*|}$ -- the test sample, and
$\Gamma$ is the conformal confidence region.

Conclusions regarding the conformal procedure in the $2$-d Gaussian setting are
similar to the respective $1$-d setting. In the case of low observation noise (interpolation
case) the intervals have more coverage for the same reasons: the procedure itself
allows for some uncertainty due to flexibility of the KRR with the Gaussian kernel.

\begin{table}
  \centering
  \caption{The maximal absolute deviation $\mathtt{MAD}(\Gamma, A; \Theta)$ ($\%$)
  of the empirical error rate from the theoretical significance level of conformal
  confidence regions for simulated $2$-d Gaussian process.}
  \label{tab:gaussian_2d_cov_conf}
  \begin{tabular}{ll||rrrr|rrrr}
  \toprule
       & $n$ &    $150$  &          &          &          &     $1500$ &          &          &          \\\cline{2-2}
       & $\gamma$ & $10^{-6}$ &          & $10^{-1}$ &          & $10^{-6}$ &          & $10^{-1}$ &          \\\cline{2-2}
       & $\lambda$ & $10^{-6}$ & $10^{-1}$ & $10^{-6}$ & $10^{-1}$ & $10^{-6}$ & $10^{-1}$ & $10^{-6}$ & $10^{-1}$ \\\cline{2-2}
  type & $\theta$ &          &          &          &          &          &          &          &          \\
  \midrule
  RRCM & $10^1$ &      1.5 &      0.7 &      1.6 &      1.1 &      0.3 &      0.2 &      0.8 &      0.4 \\
       & $10^2$ &      1.3 &      1.0 &      1.9 &      0.5 &      1.7 &      1.3 &      1.1 &      0.7 \\
       & $10^3$ &      1.4 &      0.5 &      1.5 &      0.9 &      1.1 &      2.6 &      1.2 &      0.5 \\
       & $\hat{\theta}_\text{ML}$ &      0.8 &      0.9 &      0.6 &      1.5 &      1.7 &      2.2 &      0.1 &      0.5 \\
  \midrule
  RRCM-loo & $10^1$ &      1.6 &      0.7 &      2.1 &      1.4 &      1.3 &      0.3 &      2.0 &      0.4 \\
       & $10^2$ &      0.9 &      0.3 &      1.1 &      1.2 &      2.4 &      1.7 &      2.0 &      0.4 \\
       & $10^3$ &      1.1 &      0.5 &      0.7 &      1.1 &      2.9 &      2.6 &      0.1 &      0.6 \\
       & $\hat{\theta}_\text{ML}$ &      1.1 &      0.7 &      1.2 &      0.5 &      2.6 &      2.6 &      0.8 &      0.6 \\
  \midrule
  CRR & $10^1$ &      2.4 &      1.2 &      2.3 &      1.0 &      0.3 &      0.1 &      0.8 &      0.3 \\
       & $10^2$ &      1.9 &      1.4 &      2.5 &      1.1 &      1.6 &      1.2 &      1.2 &      0.8 \\
       & $10^3$ &      1.0 &      1.2 &      2.0 &      2.1 &      0.8 &      2.2 &      1.3 &      0.5 \\
       & $\hat{\theta}_\text{ML}$ &      1.3 &      1.0 &      1.3 &      1.9 &      1.8 &      2.1 &      0.1 &      0.4 \\
  \midrule
  CRR-loo & $10^1$ &      1.0 &      1.0 &      1.4 &      1.0 &      1.2 &      0.3 &      2.0 &      0.2 \\
       & $10^2$ &      1.0 &      1.2 &      1.0 &      1.0 &      2.4 &      1.7 &      2.1 &      0.5 \\
       & $10^3$ &      1.1 &      1.0 &      1.0 &      1.6 &      2.6 &      2.4 &      0.3 &      0.5 \\
       & $\hat{\theta}_\text{ML}$ &      1.0 &      1.0 &      1.7 &      1.4 &      2.6 &      2.4 &      0.8 &      0.6 \\
  \bottomrule
  \end{tabular}
\end{table}

We proceed to the non-Gaussian case, with sample paths of the test functions with
domain $\Xcal=[-1,1]^2$ and plotted fig.~\ref{fig:nongauss_2d_profile} (p.~\pageref{fig:nongauss_2d_profile}).
Table~\ref{tab:nongaussian_2d_gpr_fit} shows the goodness-of-fit on the test sample
of the KRR for various hyper-parameters. Except for obvious overfitting in the case
of $\theta=10^3$ for $n=150$ and occasionally for low regularization $\lambda=10^{-6}$,
the overall fit is adequate. The input of the test sample in question is $X^*\subseteq \Xcal$
constructed as
\begin{equation*}
  X^* = \bigl\{(\frac{2}{N} i, \frac{2}{N} j)\,:\,i,j = -\frac{N}{2}, \ldots, \frac{N}{2}\bigr\} \,,
\end{equation*}
for $N=50$.
\begin{table}
  \centering
  \caption{``RMSE-std'' ratio averaged across $L=25$ replications of the train sample choice.}
  \label{tab:nongaussian_2d_gpr_fit}
  \begin{tabular}{ll||rrrr|rrrr}
  \toprule
       & $f(\cdot)$ &       ``f2'' &          &          &          &       ``f5'' &          &          &          \\\cline{2-2}
       & $\gamma$ & $10^{-6}$ &          & $10^{-1}$ &          & $10^{-6}$ &          & $10^{-1}$ &          \\\cline{2-2}
       & $\lambda$ & $10^{-6}$ & $10^{-1}$ & $10^{-6}$ & $10^{-1}$ & $10^{-6}$ & $10^{-1}$ & $10^{-6}$ & $10^{-1}$ \\\cline{2-2}
  $n$ & $\theta$ &          &          &          &          &          &          &          &          \\
  \midrule
  $150$  & $10^1$ &     0.23 &     0.18 &     2.81 &     0.34 &     4.09 &     0.47 &     3.47 &     0.50 \\
       & $10^2$ &     0.55 &     0.57 &     0.68 &     0.62 &     0.75 &     0.76 &     0.80 &     0.76 \\
       & $10^3$ &     1.00 &     1.01 &     1.00 &     1.01 &     1.18 &     1.19 &     1.18 &     1.19 \\
       & $\hat{\theta}_\text{ML}$ &     0.14 &     0.20 &     0.66 &     0.38 &     0.65 &     0.48 &     0.73 &     0.50 \\
  \midrule
  $1500$ & $10^1$ &     0.02 &     0.07 &     0.31 &     0.26 &     0.31 &     0.34 &     0.36 &     0.37 \\
       & $10^2$ &     0.02 &     0.05 &     3.11 &     0.29 &     1.67 &     0.27 &     2.54 &     0.31 \\
       & $10^3$ &     0.53 &     0.56 &     0.71 &     0.60 &     0.66 &     0.70 &     0.72 &     0.70 \\
       & $\hat{\theta}_\text{ML}$ &     0.01 &     0.03 &     0.72 &     0.26 &     0.46 &     0.27 &     0.67 &     0.31 \\
  \bottomrule
  \end{tabular}
\end{table}

Tables~\ref{tab:nongaussian_f2_2d_cov_gpr} and~\ref{tab:nongaussian_f5_2d_cov_gpr}
show the empirical error rate of the GPR Bayesian confidence interval on a fixed
test sample. For large train size $n$ the ML estimate of the kernel precision parameter
$\theta$ yields conservatively valid GPR prediction confidence regions for both
``f2'' and ``f5'' test functions. Note that for ``f2'' test function with high noise
$\gamma=10^{-1}$ the empirical error rate of GPR confidence intervals for the ML
estimate of $\theta$ becomes closer to the requires significance level. This is
due to the smoothness of the used function. In contrast, the ``f5'' function is
discontinuous, and thus cannot be approximated well (as seen in tab.~\ref{tab:nongaussian_2d_gpr_fit}).
Therefore, the GPR confidence intervals are more conservative for large $n$.
\begin{table}
  \centering
  \caption{The empirical error rate ($\%$) of the GPR confidence interval for the
  ``f2'' test function.}
  \label{tab:nongaussian_f2_2d_cov_gpr}
  \begin{tabular}{ll||rrrr|rrrr}
  \toprule
       & $n$ & $150$ &          &           &          & $1500$ &          &           &          \\\cline{2-2}
       & $\gamma$ & $10^{-6}$ &          & $10^{-1}$ &          & $10^{-6}$ &          & $10^{-1}$ &          \\\cline{2-2}
       & $\lambda$ & $10^{-6}$ & $10^{-1}$ & $10^{-6}$ & $10^{-1}$ & $10^{-6}$ & $10^{-1}$ & $10^{-6}$ & $10^{-1}$ \\\cline{2-2}
  $\theta$ & $\alpha(\%)$ &          &          &          &          &          &          &          &          \\
  \midrule
   $10^1$ & $1\%$ &     12.2 &      1.9 &     20.8 &      1.6 &      2.3 &      1.9 &      2.2 &      1.1 \\
        & $5\%$ &     16.9 &      3.0 &     29.7 &      4.8 &      3.2 &      3.0 &      7.9 &      4.9 \\
        & $10\%$ &     20.6 &      3.8 &     35.9 &      8.9 &      4.0 &      3.7 &     13.8 &      9.7 \\
        & $25\%$ &     29.6 &      5.5 &     48.5 &     22.4 &      5.8 &      5.6 &     29.9 &     24.3 \\
  \midrule
   $10^2$ & $1\%$ &      0.3 &      0.2 &      2.8 &      0.2 &      0.3 &      0.0 &     19.1 &      1.3 \\
        & $5\%$ &      2.3 &      2.2 &      5.0 &      2.2 &      0.6 &      0.1 &     28.6 &      5.9 \\
        & $10\%$ &      4.6 &      4.8 &      7.9 &      5.2 &      0.9 &      0.2 &     35.0 &     11.1 \\
        & $25\%$ &     14.1 &     13.7 &     17.6 &     16.7 &      2.6 &      1.2 &     48.3 &     26.5 \\
  \midrule
   $10^3$ & $1\%$ &      0.0 &      0.0 &      0.2 &      0.1 &      0.1 &      0.1 &      2.4 &      0.1 \\
        & $5\%$ &      4.1 &      4.4 &      4.1 &      3.8 &      1.9 &      2.1 &      4.0 &      1.9 \\
        & $10\%$ &     10.3 &     10.7 &     10.1 &      9.8 &      4.1 &      4.5 &      6.0 &      5.0 \\
        & $25\%$ &     26.1 &     26.5 &     25.9 &     25.5 &     12.9 &     13.4 &     13.9 &     16.4 \\
  \midrule
   $\hat{\theta}_\text{ML}$ & $1\%$ &      5.1 &      1.9 &      4.1 &      2.3 &      3.4 &      0.6 &      2.0 &      1.1 \\
        & $5\%$ &      7.1 &      3.0 &      7.1 &      5.3 &      4.4 &      1.1 &      3.9 &      5.1 \\
        & $10\%$ &      8.7 &      3.9 &     10.5 &      9.1 &      5.2 &      1.4 &      6.9 &     10.0 \\
        & $25\%$ &     13.1 &      6.2 &     21.6 &     21.4 &      7.1 &      2.4 &     18.1 &     24.9 \\
  \bottomrule
  \end{tabular}
\end{table}
\begin{table}
  \centering
  \caption{The empirical error rate ($\%$) of the GPR confidence interval for the
  ``f5'' test function.}
  \label{tab:nongaussian_f5_2d_cov_gpr}
  \begin{tabular}{ll||rrrr|rrrr}
  \toprule
       & $n$ & $150$ &          &        &          & $1500$ &          &        &          \\\cline{2-2}
       & $\gamma$ & $10^{-6}$ &          & $10^{-1}$ &          & $10^{-6}$ &          & $10^{-1}$ &          \\\cline{2-2}
       & $\lambda$ & $10^{-6}$ & $10^{-1}$ & $10^{-6}$ & $10^{-1}$ & $10^{-6}$ & $10^{-1}$ & $10^{-6}$ & $10^{-1}$ \\\cline{2-2}
  $\theta$ & $\alpha(\%)$ &          &          &          &          &          &          &          &          \\
  \midrule
  $10^1$ & $1\%$ &     19.2 &      4.2 &     23.8 &      4.1 &      4.5 &      4.1 &      3.8 &      3.7 \\
       & $5\%$ &     25.7 &      7.2 &     32.6 &      7.5 &      6.9 &      6.9 &      6.3 &      6.5 \\
       & $10\%$ &     30.4 &      9.7 &     38.8 &     10.1 &      8.7 &      9.2 &      8.4 &      8.7 \\
       & $25\%$ &     40.9 &     15.4 &     51.4 &     17.3 &     12.9 &     13.9 &     15.4 &     14.4 \\
  \midrule
  $10^2$ & $1\%$ &      1.6 &      0.8 &      2.4 &      0.8 &     10.2 &      3.2 &     15.7 &      2.7 \\
       & $5\%$ &      3.9 &      2.9 &      5.1 &      2.9 &     13.8 &      4.7 &     23.9 &      4.3 \\
       & $10\%$ &      6.4 &      5.9 &      7.9 &      5.5 &     16.4 &      5.9 &     30.0 &      5.9 \\
       & $25\%$ &     15.4 &     15.8 &     17.8 &     15.7 &     22.2 &      8.7 &     42.7 &     12.3 \\
  \midrule
  $10^3$ & $1\%$ &      0.0 &      0.0 &      0.1 &      0.0 &      0.6 &      0.4 &      1.6 &      0.4 \\
       & $5\%$ &      2.2 &      2.0 &      2.3 &      2.6 &      2.0 &      2.3 &      3.3 &      2.0 \\
       & $10\%$ &      7.7 &      7.4 &      7.6 &      8.2 &      4.4 &      4.8 &      5.7 &      4.6 \\
       & $25\%$ &     26.3 &     25.7 &     26.3 &     27.2 &     13.3 &     13.9 &     14.9 &     14.0 \\
  \midrule
  $\hat{\theta}_\text{ML}$ & $1\%$ &      5.1 &      4.5 &      5.0 &      3.8 &      3.2 &      3.7 &      2.5 &      3.3 \\
       & $5\%$ &      8.0 &      7.7 &      8.5 &      7.2 &      4.6 &      5.7 &      4.4 &      5.4 \\
       & $10\%$ &     10.7 &     10.3 &     12.0 &     10.0 &      5.9 &      7.1 &      6.6 &      7.0 \\
       & $25\%$ &     18.4 &     16.6 &     22.6 &     17.0 &      9.7 &     10.2 &     15.3 &     12.6 \\
  \bottomrule
  \end{tabular}
\end{table}

The performance of the conformal confidence predictions are summarized in tables~\ref{tab:nongaussian_f2_2d_cov_conf}
and~\ref{tab:nongaussian_f5_2d_cov_conf} using the MAD metric introduced earlier.
Overall the error rates do not sway too far from the stated levels, and the conformal
procedure yields intervals less sensitive to the hyper-parameters related to the KRR,
which is expected, since the conservative validity of the conformal prediction should
not be affected by the internals of the used non-conformity measure.

\begin{table}
  \centering
  \caption{The maximal absolute deviation $\mathtt{MAD}(\Gamma, A; \Theta)$ ($\%$)
  of the empirical error rate from the theoretical significance level of conformal
  confidence regions for the ``f2'' test function.}
  \label{tab:nongaussian_f2_2d_cov_conf}
  \begin{tabular}{ll||rrrr|rrrr}
  \toprule
       & $n$ &     $150$  &          &          &          &     $1500$ &          &          &          \\\cline{2-2}
       & $\gamma$ & $10^{-6}$ &          & $10^{-1}$ &          & $10^{-6}$ &          & $10^{-1}$ &          \\\cline{2-2}
       & $\lambda$ & $10^{-6}$ & $10^{-1}$ & $10^{-6}$ & $10^{-1}$ & $10^{-6}$ & $10^{-1}$ & $10^{-6}$ & $10^{-1}$ \\\cline{2-2}
  type & $\theta$ &          &          &          &          &          &          &          &          \\
  \midrule
  CRR & $10^1$ &      1.9 &      1.3 &      2.0 &      1.6 &      1.0 &      1.3 &      1.2 &      0.2 \\
       & $10^2$ &      1.1 &      1.0 &      1.9 &      2.9 &      0.7 &      2.7 &      1.4 &      0.6 \\
       & $10^3$ &      1.4 &      1.2 &      1.1 &      1.0 &      0.3 &      1.1 &      1.0 &      0.1 \\
       & $\hat{\theta}_\text{ML}$ &      2.4 &      1.0 &      3.4 &      1.0 &      1.4 &      2.2 &      0.6 &      0.5 \\
  \midrule
  CRR-loo & $10^1$ &      1.0 &      1.4 &      1.0 &      1.4 &      0.8 &      1.4 &      1.3 &      0.2 \\
       & $10^2$ &      1.0 &      1.0 &      1.4 &      1.0 &      3.0 &      3.2 &      1.9 &      0.5 \\
       & $10^3$ &      1.5 &      1.3 &      1.2 &      1.1 &      1.0 &      1.0 &      0.2 &      0.2 \\
       & $\hat{\theta}_\text{ML}$ &      1.4 &      1.0 &      1.2 &      1.5 &      2.6 &      2.6 &      0.5 &      0.2 \\
  \midrule
  RRCM & $10^1$ &      1.2 &      0.9 &      1.3 &      1.0 &      0.9 &      1.4 &      1.2 &      0.2 \\
       & $10^2$ &      1.3 &      1.5 &      1.5 &      2.4 &      0.7 &      2.6 &      1.3 &      0.6 \\
       & $10^3$ &      1.9 &      0.9 &      0.4 &      0.6 &      0.4 &      0.5 &      0.9 &      0.3 \\
       & $\hat{\theta}_\text{ML}$ &      1.7 &      0.8 &      2.9 &      1.2 &      1.3 &      2.3 &      0.8 &      0.4 \\
  \midrule
  RRCM-loo & $10^1$ &      1.5 &      0.5 &      0.4 &      1.0 &      0.8 &      1.6 &      1.2 &      0.3 \\
       & $10^2$ &      0.8 &      0.5 &      0.7 &      0.5 &      3.0 &      3.2 &      2.0 &      0.6 \\
       & $10^3$ &      1.9 &      0.8 &      0.7 &      0.9 &      0.9 &      0.6 &      0.2 &      0.2 \\
       & $\hat{\theta}_\text{ML}$ &      0.6 &      1.4 &      1.0 &      2.1 &      2.6 &      2.6 &      0.7 &      0.1 \\
  \bottomrule
  \end{tabular}
\end{table}

\begin{table}
  \centering
  \caption{The maximal absolute deviation $\mathtt{MAD}(\Gamma, A; \Theta)$ ($\%$)
  of the empirical error rate from the theoretical significance level of conformal
  confidence regions for the ``f5'' test function.}
  \label{tab:nongaussian_f5_2d_cov_conf}
  \begin{tabular}{ll||rrrr|rrrr}
  \toprule
       & $n$ &     $150$  &          &          &          &     $1500$ &          &          &          \\\cline{2-2}
       & $\gamma$ & $10^{-6}$ &          & $10^{-1}$ &          & $10^{-6}$ &          & $10^{-1}$ &          \\\cline{2-2}
       & $\lambda$ & $10^{-6}$ & $10^{-1}$ & $10^{-6}$ & $10^{-1}$ & $10^{-6}$ & $10^{-1}$ & $10^{-6}$ & $10^{-1}$ \\\cline{2-2}
  type & $\theta$ &          &          &          &          &          &          &          &          \\
  \midrule
  CRR & $10^1$ &      2.3 &      2.5 &      1.0 &      1.6 &      1.2 &      0.6 &      1.3 &      0.8 \\
       & $10^2$ &      1.0 &      1.0 &      1.0 &      1.0 &      1.1 &      1.3 &      1.2 &      0.4 \\
       & $10^3$ &      1.0 &      1.1 &      1.3 &      1.1 &      0.2 &      0.8 &      0.6 &      0.3 \\
       & $\hat{\theta}_\text{ML}$ &      2.4 &      1.0 &      1.1 &      1.4 &      0.3 &      0.3 &      1.0 &      0.8 \\
  \midrule
  CRR-loo & $10^1$ &      2.2 &      1.4 &      1.0 &      1.0 &      0.1 &      0.5 &      0.6 &      0.5 \\
       & $10^2$ &      1.0 &      1.0 &      1.0 &      1.2 &      0.3 &      1.8 &      1.7 &      1.2 \\
       & $10^3$ &      1.0 &      1.2 &      1.4 &      1.0 &      0.4 &      0.8 &      0.3 &      0.5 \\
       & $\hat{\theta}_\text{ML}$ &      1.0 &      1.0 &      1.0 &      1.0 &      1.0 &      0.9 &      0.1 &      0.3 \\
  \midrule
  RRCM & $10^1$ &      1.8 &      1.7 &      0.3 &      1.2 &      1.2 &      0.6 &      1.2 &      0.7 \\
       & $10^2$ &      0.8 &      0.6 &      1.4 &      0.6 &      1.0 &      1.6 &      1.1 &      0.3 \\
       & $10^3$ &      0.5 &      1.1 &      0.3 &      0.3 &      0.2 &      1.0 &      0.5 &      0.5 \\
       & $\hat{\theta}_\text{ML}$ &      1.6 &      0.5 &      0.5 &      0.8 &      0.4 &      0.5 &      0.9 &      0.8 \\
  \midrule
  RRCM-loo & $10^1$ &      1.5 &      0.5 &      0.6 &      0.3 &      0.2 &      0.5 &      0.6 &      0.5 \\
       & $10^2$ &      1.4 &      1.1 &      1.3 &      0.6 &      0.3 &      2.1 &      1.6 &      1.1 \\
       & $10^3$ &      0.2 &      1.1 &      0.5 &      1.1 &      0.8 &      1.3 &      0.4 &      0.6 \\
       & $\hat{\theta}_\text{ML}$ &      1.5 &      0.5 &      2.0 &      0.4 &      0.9 &      1.0 &      0.3 &      0.2 \\
  \bottomrule
  \end{tabular}
\end{table}

Based on the empirical evidence from the $2$-d experiments we arrive at similar
conclusions: \begin{itemize}
  \item the Gaussian Process Regression offers conservatively valid confidence regions
  in the Gaussian case, provided the hyper-parameters are not severely misspecified;
  \item in the non-Gaussian case experimental results regarding the validity of the
  GPR confidence sets are less favourable;
  \item there is empirical evidence in favour of conformal predictions based on
  the kernel ridge regression with Gaussian kernel being asymptotically valid in
  the offline setting;
  \item numerical experiments support the  conjecture that conformal procedures
  are insensitive to the choice of the core NCM: the hyper-parameters of the KRR
  weakly affected the validity of the regions (as measure by MAD metric) in contrast
  to the GPR confidence prediction.
\end{itemize}

% subsection results_2_d (end)

% section numerical_study (end)

\section{Conclusion and further work} % (fold)
\label{sec:conclusion_and_further_work}

In conclusion we summarize the results obtained in section~\ref{sec:numerical_study},
and outline further research on the applicability of the conformal procedures based
on the kernel ridge regression in batch learning setting.

In the non-Gaussian experiments we found empirical evidence suggesting that the GPR
confidence regions are not consistently valid if the Gaussianity assumptions fail
to hold, or if the KRR hyper-parameters are misspecified. Identical experiments on
the conformal procedures demonstrated that indeed, there is empirical evidence in
support of the conjectured conservative validity in the batch setting. At the same
time, confidence intervals of the conformalized kernel ridge regression seem to
perform not worse than the confidence regions of the Gaussian Process regression,
when the Gaussianity assumptions hold.

Further work on the validity of conformal procedures in batch setting, with applications
to anomaly detection, shall include: \begin{itemize}
  \item establishing theoretical foundations for the proposed conjectures for the
  kernel ridge regression with Gaussian kernel, or isolating special cases when it
  holds, and studying the cases when it fails;
  \item obtaining a generalization to the asymptotic efficiency result in \cite{burnaevV14},
  in for a kernel ridge regression with general kernels;
\end{itemize}

However, future research should not be limited to the items, listed above. We would
like to outline two potentially interesting research topics: in sec.~\ref{sub:conformalized_kernel_embeddings}
we state the possible use of conformal prediction in constructing more direct unsupervised
anomaly detection method, and in sec.~\ref{sub:conformal_kernel_precision_selection}
we propose a method for selection of kernel precision based on conformal confidence
sets.

\subsection{Conformalized kernel embeddings} % (fold)
\label{sub:conformalized_kernel_embeddings}

It might be fruitful to research conformal procedures the field of kernel embedding
of probability distributions, \cite{smola2007}. Basically in a suitable RKHS many non-
pathological distributions on metric spaces can be identified with a unique element
of said RKHS. In particular, if $K$ is a kernel that induces a universal RKHS $\Hcal$
of functions $\Xcal\mapsto\Real$, then for any probability distribution $P$ on $\Xcal$
there is a unique $\mu_P\in \Hcal$ whenever $\ex_{x\sim P} \sqrt{K(x,x)}$ is finite.
For instance, the Gaussian kernel in eq.~\ref{eq:gauss_kenrel} satisfies this property
and its induced canonical RKHS is universal. Now, the boundedness of $K$ in expectation
implies that $f\mapsto \ex_P f$ is a bounded linear functional in $\Hcal$, whence by
Riesz representation theorem there exists a unique $\mu_P\in \Hcal$ with $\ex_P f
= \langle \mu_P, f\rangle$ for all $f\in \Hcal$. In particular, for $f = k(x, \cdot)$
the reproducing property of $K$ implies that
\begin{equation*}
  \ex_{y\sim P} K(x, y) = \langle \mu_P, K(x, \cdot) \rangle = \mu_P(x) \,,
\end{equation*}
which provides a closed formula for the embedding of $P$.

In \cite{gretton2012}, a powerful two-sample distribution test was constructed, based
on this elegant idea. Essentially, for a compact metric space $\Xcal$ the distributions
$P$ and $Q$ coincide if and only if $\ex_P f = \ex_Q f$ for all continuous and bounded
$f:\Xcal\mapsto \Real$. However, because the RKHS is universal it is enough to check
the equality only in the class of functions from the Hilbert space $\Hcal$. This
reduces the problem of testing to studying the norm between the embeddings $\mu_P$
and $\mu_Q$ in $\Hcal$.

The proposed test statistic, \cite{gretton2012}, the \textit{maximum mean discrepancy}
(MDD), measures the RKHS norm of the deviation of the embeddings of the empirical
distributions. This MDD measure can readily be used as a non-conformity measure
for the following anomaly detection problem: given a sample $Z_{:n-1} = (z_i)_{i=1}^{n-1}$
decide if a new example $z_n$ is anomalous or not by measuring how well it conforms
to $Z$. If the full sample is denoted by $Z_{:n} = (z_i)_{i=1}^n$, then the proposed
conformal procedure would be based on the following NCM:
\begin{equation*}
  A(Z_{-i:n}, z_i) = \| \mu_{\hat{P}_{-i:n}} - \mu_{\hat{P}_{:n}} \|_{\Hcal} \,,
\end{equation*}
where $\hat{P}_{-i:n}$ and $\hat{P}_{:n}$ are empirical distributions over $Z_{-i:n}$
and $Z_{:n}$, respectively, with $Z_{:n-1}$ being the sample $Z_{:n}$ without the
$i$-th observation. Note, that it would be necessary to consider only vector-matrix
multiplications, since for any sub-sample $S$ of $Z_{:n}$ the empirical distribution
embedding is given by
\begin{equation*}
  \mu_{\hat{P}_S}(\cdot) = (\one_m'\one_m)^{-1} \one_m' k_S(\cdot) \,,
\end{equation*}
where $m = |S|$, $k_S(\cdot) = (K(z_i, \cdot))_{i\in S} \in \Hcal^{m\times 1}$ is
the sample evaluation vector of the kernel $K$, and $\one_m \in \Real^{m\times 1}$
is the vector of ones. Since the empirical embeddings are elements of the data-driven
pre-Hilbert space, their inner products can be computed from the sample Gram matrix
of $K$. Indeed, using the ordinary formula for sequential update of the sample average
we get
\begin{align*}
  % (n-1)\bigl(\mu_{\hat{P}_{-i:n}} - \mu_{\hat{P}_{:n}}\bigr)
  %   % &= (n-1)^{-1}\sum_{j\neq i} K(z_j, \cdot) - n^{-1}\sum_j K(z_j, \cdot) \\
  %   &= n^{-1} \sum_j K(z_j, \cdot) - K(z_i, \cdot) \,,
  (n-1)^2 \bigl\|\mu_{\hat{P}_{-i:n}} - \mu_{\hat{P}_{:n}} \bigr\|^2_\Hcal
    &= \bigl\| n^{-1} \sum_j K(z_j, \cdot) - K(z_i, \cdot) \bigr\|^2_\Hcal \\
    % &= \bigl\| e_i' (I_n - \one(\one'\one)^{-1}\one') k_Z(\cdot) \bigr\|^2_\Hcal \\
    &= e_i'K_{ZZ} e_i + \frac{1}{n^2} \one' K_{ZZ} \one - \frac{2}{n} \one' K_{ZZ} e_i \,,
    % &= e_i'(I_n - \one(\one'\one)^{-1}\one')K_{ZZ}(I_n - \one(\one'\one)^{-1}\one') e_i \,,
\end{align*}
where $\one$ is the vector of $1$'s and $e_i$ is the unit vector both of appropriate
dimensions. This is basically the distance of the $i$-th canonical feature map from
the centre of dataset in $\Hcal$.

Conformalization of this procedure would most likely proceed in the direction of
using eq.~\ref{eq:conf_p_value} as a measure of the degree of abnormality of new
examples. Furthermore, it seems that in low dimensional cases there might exist an
efficient approximation procedure that could yield multivariate quantiles or even
confidence sets based on inverting hypothesis test in the conformal prediction.

% subsection conformalized_kernel_embeddings (end)

\subsection{Conformal kernel precision selection} % (fold)
\label{sub:conformal_kernel_precision_selection}

Yet another research direction, which is intimately connected to both the conformalized
ridge regression and kernel embeddings, is concerned with designing a non-parametric
procedure, that could yield estimates of the kernel precision parameter based on
conformal confidence intervals, for example, similar in spirit to the central idea
of \cite{goldenshluger1997}. The authors consider the problem of restoring a continuous
$f:[0,1] \mapsto\Real$ in eq.~\ref{eq:signal_model} from observations contaminated
by an i.i.d sub-Gaussian noise, and propose a method to select a locally optimal
kernel bandwidth (the reciprocal of the precision $\theta$) based on approximate
minimization of the size of the confidence region for a true value of $f$ with fixed
coverage rate. They propose to approximate the optimal bandwidth with the largest
bandwidth, for which all confidence regions corresponding to lower bandwidths overlap.

In particular they show that in their setting the confidence regions for $f(x_0)$
have error rate at most $e^{- c t^2}$ for $c = \mathcal{O}(n)$, $n$ is the train
sample size, and are of the form
\begin{equation*}
  \bigl| f(x_0) - \hat{f}_\delta(x_0) \bigr|
    \leq \omega_f(x_0, \delta) + \frac{t}{\sqrt{\delta}} \,,
\end{equation*}
where $\omega_f(x_0, \delta) = \sup_{x\in[0,1]} |f(x)-f(x_0)|$ is the ``deterministic
dynamic error'', $\frac{t}{\sqrt{\delta}}$ is an upper bound for the ``stochastic error'',
and $\delta$ is the bandwidth of the kernel estimator $\hat{f}_\delta(\cdot)$. If
$f$ were known, the optimal $\delta^*$ would have been given by
\begin{equation*}
  \delta^*
    = \sup\bigl\{ \delta > 0 \, : \,
        \omega_f(x_0, \delta) \leq \frac{t}{\sqrt{\delta}} \bigr\}\,,
\end{equation*}
but, since $f$ is unknown, authors propose the ``intersection'' estimate:
\begin{equation*}
  \delta^+
    = \sup\{ \eta > 0 \, : \, \emptyset \neq 
        \cap_{\delta\leq \eta} [\hat{f}_\delta(x_0) - 2 \frac{t}{\sqrt{\delta}},
                                \hat{f}_\delta(x_0) + 2 \frac{t}{\sqrt{\delta}}] \} \,,
\end{equation*}
which they show makes the approximation error lie within acceptable bounds with high
probability: at least $ 1 - n e^{-c(n) t^2}$.

The main obstacle in their method is that the deterministic dynamic error depends
on the unknown $f$, which is why direct minimization of the width of the confidence
interval was not feasible. In contrast to their situation, conformal confidence
intervals have a clear and known width and fixed coverage guarantees, which means
that we can directly minimize the width to select the optimal bandwidth. Possible
challenge in this research might be selecting and justifying the aggregation method,
used to move from local bandwidth estimates to global ones.

% subsection conformal_kernel_precision_selection (end)

% section conclusion_and_further_work (end)

% \bibliographystyle{amsplain}
\clearpage
\bibliographystyle{ugost2008ls}
\bibliography{../references}

\end{document}

